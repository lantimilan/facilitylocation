\documentclass{article}

\usepackage{fullpage,amsmath}
\title{LP-rounding algorithms for Fault-Tolerant Facility Placement}
\author{lyan, marek}
\begin{document}

\section{Introduction}
Review STA97, Chudak98, Svi02 and Byrka07,10.

\section{Demand Reduction}
Justify assumption that $\max_j r_j \leq |F|$.

\section{Partition}
Given an optimal fractional solution $(x^\ast, y^\ast)$ to
the LP, we look for a way to partition the values to demands
of each client. That is, $\sum_{\mu \in i} \bar y_{\mu} =
y_i^\ast$ and $\sum_{\mu \in i, \nu \in j} \bar x_{\mu\nu} =
x_{ij}^\ast$. This allows us to bound our integral solution
cost using $(\bar x, \bar y)$, which in turn implies a bound
on $(x^\ast, y^\ast)$, and $\text{LP}^\ast$ is a lower bound
on $\text{OPT}$.

If this is all we need for such a partition, it would have
been trivial to construct the partition. However, all
partitions are not equal. To apply the rounding approach
similar to UFL algorithms, so as to obtain an integral
solution, that is, a set of facilities to open and a set of
connections between facilities and demands so that each
client has all its demand satisfied, we need certain
properties to hold for feasibility and estimating the cost
using $(\bar x, \bar y)$.

Assume we have split each client $j$ into $r_j$ demand
points, or simply demands already, and we have identified a
set of primary demands as well as an assignment of each
demand to a primary demand. Now we need the following
properties:
\begin{itemize}
\item For feasibility we need $N(\nu_1)$ and $N(\nu_2)$ be
  disjoint for sibling $\nu_1$ and $\nu_2$.
\item For feasibility we need $N(\nu)$ and $N(\kappa')$ be
  disjoint where $\kappa'$ is the primary demand of a
  sibling of $\nu$, denoted as $\nu'$.

\item For estimating the redirect distance, we need $N(\nu)$
  and $N(\kappa)$ have nonempty intersection. (For Byrka's
  algorithm we need N close neighborhood.)
\item For estimating the redirect distance, we need a
  primary demand to be ``best'' among all demands assigned
  to it.
\end{itemize}

\end{document}
