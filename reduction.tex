\section{Reduction to Polynomial Demands}
\label{sec: polynomial demands}

This section presents a \emph{demand reduction} trick that
reduces the problem for arbitrary demands to a special case
where demands are bounded by $|\sitesset|$, the number of
sites.  (The formal statement is a little more technical --
see Theorem~\ref{thm: reduction to polynomial}.)  Our
algorithms in the sections that follow process individual
demands of each client one by one, and thus they critically
rely on the demands being bounded polynomially in terms of
$|\sitesset|$ and $|\clientset|$ to keep the overall running time polynomial.

The reduction is based on an optimal fractional solution
$(\bfx^\ast,\bfy^\ast)$ of LP~(\ref{eqn:fac_primal}). From the
optimality of this solution, we can also assume that
$\sum_{i\in\sitesset} x^\ast_{ij} = r_j$ for all
$j\in\clientset$.  As explained in Section~\ref{sec: the lp
  formulation}, we can assume that $(\bfx^\ast,\bfy^\ast)$
is complete, that is $x^\ast_{ij} > 0$ implies $x^\ast_{ij}
= y^\ast_i$ for all $i,j$.  We split this solution into two
parts, namely $(\bfx^\ast,\bfy^\ast) = (\hatbfx,\hatbfy)+
(\dotbfx,\dotbfy)$, where
%
\begin{align*}
\haty_i &\;\assign\; \floor{y_i^\ast}, \quad
			\hatx_{ij} \;\assign\; \floor{x_{ij}^\ast} \quad\textrm{and}
			\\
\doty_i &\;\assign\; y_i^\ast - \floor{y_i^\ast}, \quad
 	\dotx_{ij} \;\assign\; x_{ij}^\ast -  \floor{x_{ij}^\ast}
\end{align*}
%
for all $i,j$. Now we construct two
FTFP instances $\hatcalI$ and $\dotcalI$ with the same
parameters as the original instance, except that the demand of each client $j$ is
$\hatr_j = \sum_{i\in\sitesset} \hatx_{ij}$ in instance $\hatcalI$ and
$\dotr_j = \sum_{i\in\sitesset} \dotx_{ij} = r_j - \hatr_j$ in instance $\dotcalI$. 
It is obvious that if we have integral solutions to both $\hatcalI$
and $\dotcalI$ then, when added together, they form an integral
solution to the original instance.  Moreover, we have the
following lemma.

%%%%%%%%%%

\begin{lemma}\label{lem: polynomial demands partition}
{\rm (i)}
  $(\hatbfx, \hatbfy)$ is a feasible integral solution to
  instance $\hatcalI$.

\noindent
{\rm (ii)}
  $(\dotbfx, \dotbfy)$ is a feasible fractional
  solution to instance $\dotcalI$.

\noindent
{\rm (iii)}
$\dotr_j\leq |\sitesset|$ for every client $j$.

\end{lemma}

\begin{proof}
(i) For feasibility, we need to verify that the constraints of LP~(\ref{eqn:fac_primal})
are satisfied. Directly from the definition, we have $\hatr_j = \sum_{i\in\sitesset} \hatx_{ij}$.
For any $i$ and $j$, by the feasibility of $(\bfx^\ast,\bfy^\ast)$ we have
$\hatx_{ij} = \floor{x_{ij}^\ast} \le \floor{y^\ast_i} = \haty_i$.

(ii) From the definition, we have  $\dotr_j = \sum_{i\in\sitesset} \dotx_{ij}$.
It remains to show that $\doty_i \geq \dotx_{ij}$ for all $i,j$. 
If $x_{ij}^\ast=0$, then $\dotx_{ij}=0$ and we are done. 
Otherwise, by completeness, we have $x_{ij}^\ast=y_i^\ast$. 
Then  $\doty_i = y_i^\ast - \floor{y_i^\ast} = x_{ij}^\ast - \floor{x_{ij}^\ast} =\dotx_{ij}$. 

(iii) From the definition of $\dotx_{ij}$ we have
  $\dotx_{ij} < 1$.  Then the bound follows from the definition of $\dotr_j$.
\end{proof}

Notice that our construction relies on the completeness assumption; in fact, it is
easy to give an example where $(\dotbfx, \dotbfy)$ would not be feasible if we
used a non-complete optimal solution $(\bfx^\ast,\bfy^\ast)$.
Note also that the solutions $(\hatbfx,\hatbfy)$ and $(\dotbfx, \dotbfy)$ are in fact
optimal for their corresponding instances, for if a better solution to $\hatcalI$ or
$\dotcalI$ existed, it could
give us a solution to $\calI$ with a smaller objective value.

%%%%%%%%%%%%%%%

\begin{theorem}\label{thm: reduction to polynomial}
  Suppose that there is a polynomial-time algorithm $\calA$
  that, for any instance of {\FTFP} with maximum demand
  bounded by $|\sitesset|$, computes an integral solution
  that approximates the fractional optimum of this instance
  within factor $\rho\geq 1$.  Then there is a
  $\rho$-approximation algorithm $\calA'$ for {\FTFP}.
\end{theorem}

%%%%%%%%%%%%%%%

\begin{proof}
  Given an {\FTFP} instance with arbitrary demands, Algorithm~$\calA'$ works
as follows: it solves the LP~(\ref{eqn:fac_primal}) to obtain a
  fractional optimal solution $(\bfx^\ast,\bfy^\ast)$, then it constructs
  instances $\hatcalI$ and $\dotcalI$ described above,  applies
  algorithm~$\calA$ to $\dotcalI$, and finally combines (by adding
  the values) the integral solution $(\hatbfx, \hatbfy)$ of
  $\hatcalI$ and the integral solution of $\dotcalI$ produced
  by $\calA$. This clearly produces a feasible integral
  solution for the original instance $\calI$.
The solution produced by $\calA$ has cost at most
$\rho\cdot\cost(\dotbfx,\dotbfy)$, because $(\dotbfx,\dotbfy)$
is feasible for $\dotcalI$. Thus the cost of $\calA'$ is at most
% 
 \begin{align*}
 \cost(\hatbfx, \hatbfy) + \rho\cdot\cost(\dotbfx,\dotbfy)
	\le
 \rho(\cost(\hatbfx, \hatbfy) + \cost(\dotbfx,\dotbfy))
		= \rho\cdot\LP^\ast \le \rho\cdot\OPT,
  \end{align*}
%
where the first inequality follows from $\rho\geq 1$. This completes
the proof.
\end{proof}
