\documentclass{article}

\usepackage{fullpage, amsmath}

\newcommand{\fac}{\mathcal{F}}
\newcommand{\cli}{\mathcal{C}}
\newcommand{\PP}{\textsf{P}}
\newcommand{\NP}{\textsf{NP}}
\newcommand{\DTIME}{\textsf{DTIME}}

\title{On the Approximation Algorithms for the Fault-tolerant Facility Location Problem\\(Research Proposal towards PhD Defense)}
\author{Li Yan\\Computer Science\\U of California Riverside}

\begin{document}
\maketitle

%%%%%%%%%%%%%%%%%%%%%%%%%%%%%%%%%%%%%%%%%%%%%%%%%%%%%%%%%%%%%%%%%%%%%
\section{Introduction}
The Facility Location problem is a wellknown problem in theoretical
computer science and operations research. The classic problem is the
uncapacilitated facility location problem (UFL). In the problem, we
are given a set of facilities $\fac$ and a set of clients $\cli$. Each
facility in $\fac$ has an opening cost $f_i$ and the connection cost
between a facility $i\in \fac$ and a client $j\in \cli$ is
$d_{ij}$. An algorithm needs to find a subset of $\fac$ to open and
connect every client to one of the open facilities.

A generalization of the UFL problem, is the Fault-tolerant Facility
Placement problem (FTFP), in which each client $j$ has demand $r_j$
and we now have sites, the set $\fac$ on which we can build
facilities. To open one facility at a site $i\in \fac$ incurs a cost
of $f_i$, and to connect one unit of demand from a client $j$ to a
facility $i$ incurs the connection cost $d_{ij}$. An algorithm needs
to open a number of facilities, possibly zero, on each site and
connect each of the $r_j$ demands of client $j$ to distinct
facilities. Facilities on the same site are considered different.

In the following we first review the related work in UFL, then we
discuss our result for FTFP, lastly we briefly describe some on-going
work and a plan towards the final defense.

%%%%%%%%%%%%%%%%%%%%%%%%%%%%%%%%%%%%%%%%%%%%%%%%%%%%%%%%%%%%%%%%%%%%%
\section{Review of Known Results for the Uncapacitated Facility
  Location problem}
It is easy to see that the UFL problem contains the Set Cover problem
as a special case. hence it is NP-hard and does not admit a polynomial
time algorithm with approximation ratio better than $(1-\epsilon)\ln
n$ for any constant $\epsilon >0$, unless $\NP \subseteq
\DTIME(n^{O(\log\log n)})$. On the other hand, Hochbaum showed that
the greedy algorithm which repeatedly picks the most cost-effective
star, consisting of one facility and a set of clients, achieved a
performance guarantee of $H_n = \ln + 1$. This shows that the lower
bound and upper bound of approximation ratio match for general UFL. In
the following we assume the connection cost, or the distances $d_{ij}$
satisfy the triangle inequality, that is, for any two facilities $i_1,
i_2$ and two clients $j_1, j_2$, we have $d_{i_1 j_2} \leq d_{i_1 j_1}
+ d_{i_2 j_1} + d_{i_2 j_2}$.

The metric UFL problem has been shown to be APX-hard and unless $\PP
= \NP$, there is no polynomial time algorithm giving approximation
ratio better than $1.463$ (add the equation here).

There are a number of approximation algorithms progressing towards
this $1.463$ lower bound. The current best known approximation ratio
is $1.488$.  The algorithms can be classified into two categories:
LP-based and combinatorial. The LP-based algorithms require to solve
the LP to obtain a fractional optimal solution and then round it to an
integral solution. Combinatorial algorithms include primal-dual,
greedy analyzed with dual-fitting, local search.

%%%%%%%%%%%%%%%%%%%%%%%%%%%%%%%%%%%%%%%%%%%%%%%%%%%%%%%%%%%%%%%%%%%%%
\section{Preliminary Results for the Fault-tolerant Facility Placement
  Problem}

%%%%%%%%%%%%%%%%%%%%%%%%%%%%%%%%%%%%%%%%%%%%%%%%%%%%%%%%%%%%%%%%%%%%%
\section{On-going Work and Expected Results}

\end{document}
