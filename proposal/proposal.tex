\documentclass{article}

\usepackage{fullpage, amsmath}

\newcommand{\fac}{\mathcal{F}}
\newcommand{\cli}{\mathcal{C}}

\title{On the Approximation Algorithms for the Fault-tolerant Facility Location Problem\\(Research Proposal towards PhD Defense)}
\author{Li Yan\\Computer Science\\U of California Riverside}

\begin{document}
\maketitle

\section{Introduction}
The Facility Location problem is a wellknown problem in theoretical
computer science and operations research. The classic problem is the
uncapacilitated facility location problem (UFL). In the problem, we
are given a set of facilities $\fac$ and a set of clients $\cli$. Each
facility in $\fac$ has an opening cost $f_i$ and the connection cost
between a facility $i\in \fac$ and a client $j\in \cli$ is
$d_{ij}$. An algorithm needs to find a subset of $\fac$ to open and
connect every client to one of the open facilities.

A generalization of the UFL problem, is the Fault-tolerant Facility
Placement problem (FTFP), in which each client $j$ has demand $r_j$
and we now have sites, the set $\fac$ on which we can build
facilities. To open one facility at a site $i\in \fac$ incurs a cost
of $f_i$, and to connect one unit of demand from a client $j$ to a
facility $i$ incurs the connection cost $d_{ij}$. An algorithm needs
to open a number of facilities, possibly zero, on each site and
connect each of the $r_j$ demands of client $j$ to distinct
facilities. Facilities on the same site are considered different.

In the following we first review the related work in UFL, then we
discuss our result for FTFP, lastly we briefly describe some on-going
work and a plan towards the final defense.

\section{Review of Known Results for the Uncapacitated Facility
  Location problem}
\section{Preliminary Results for the Fault-tolerant Facility Placement Problem}
\section{On-going Work and Expected Results}

\end{document}
