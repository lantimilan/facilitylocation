\documentclass{article}

\usepackage{fullpage, amsmath, amssymb}

\newcommand{\fac}{\mathcal{F}}
\newcommand{\cli}{\mathcal{C}}
\newcommand{\PP}{\textsf{P}}
\newcommand{\NP}{\textsf{NP}}
\newcommand{\LP}{\text{LP}}
\newcommand{\DTIME}{\textsf{DTIME}}
\newcommand{\suchthat}{:}
\newcommand{\mydef}{\text{def}}
\newcommand{\wbar}{\overline}

\title{On the Approximation Algorithms for the Fault-tolerant Facility Location Problem\\(Research Proposal towards PhD Defense)}
\author{Li Yan\\Computer Science\\U of California Riverside}

\begin{document}
\maketitle

%%%%%%%%%%%%%%%%%%%%%%%%%%%%%%%%%%%%%%%%%%%%%%%%%%%%%%%%%%%%%%%%%%%%%
\section{Introduction}
The Facility Location problem is a wellknown problem in theoretical
computer science and operations research. The classic problem is the
uncapacilitated facility location problem (UFL). In the problem, we
are given a set of facilities $\fac$ and a set of clients $\cli$. Each
facility in $\fac$ has an opening cost $f_i$ and the connection cost
between a facility $i\in \fac$ and a client $j\in \cli$ is
$d_{ij}$. An algorithm needs to find a subset of $\fac$ to open and
connect every client to one of the open facilities.

A generalization of the UFL problem, is the Fault-tolerant Facility
Placement problem (FTFP), in which each client $j$ has demand $r_j$
and we now have sites, the set $\fac$ on which we can build
facilities. To open one facility at a site $i\in \fac$ incurs a cost
of $f_i$, and to connect one unit of demand from a client $j$ to a
facility $i$ incurs the connection cost $d_{ij}$. An algorithm needs
to open a number of facilities, possibly zero, on each site and
connect each of the $r_j$ demands of client $j$ to distinct
facilities. Facilities on the same site are considered different.

In the following we first review the related work in UFL, then we
discuss our result for FTFP, lastly we briefly describe some on-going
work and a plan towards the final defense.

%%%%%%%%%%%%%%%%%%%%%%%%%%%%%%%%%%%%%%%%%%%%%%%%%%%%%%%%%%%%%%%%%%%%%
\section{Review of Known Results for the Uncapacitated Facility
  Location problem}
It is easy to see that the UFL problem contains the Set Cover problem
as a special case. hence it is NP-hard and does not admit a polynomial
time algorithm with approximation ratio better than $(1-\epsilon)\ln
n$ for any constant $\epsilon >0$, unless $\NP \subseteq
\DTIME(n^{O(\log\log n)})$. On the other hand, Hochbaum showed that
the greedy algorithm which repeatedly picks the most cost-effective
star, consisting of one facility and a set of clients, achieved a
performance guarantee of $H_n = \ln + 1$. This shows that the lower
bound and upper bound of approximation ratio match for general UFL. In
the following we assume the connection cost, or the distances $d_{ij}$
satisfy the triangle inequality, that is, for any two facilities $i_1,
i_2$ and two clients $j_1, j_2$, we have $d_{i_1 j_2} \leq d_{i_1 j_1}
+ d_{i_2 j_1} + d_{i_2 j_2}$.

The metric UFL problem has been shown to be APX-hard and unless $\PP
= \NP$, there is no polynomial time algorithm giving approximation
ratio better than $1.463$ (add the equation here).

There are a number of approximation algorithms progressing towards
this $1.463$ lower bound. The current best known approximation ratio
is $1.488$.  The algorithms can be classified into two categories:
LP-based and combinatorial. The LP-based algorithms require to solve
the LP to obtain a fractional optimal solution and then round it to an
integral solution. Combinatorial algorithms include primal-dual,
greedy analyzed with dual-fitting, local search.

\subsection{The LP-rounding Algorithms for UFL}
STA97 is the first paper achieving a $O(1)$-approximation ratio for
the UFL problem, and the clustering framework established in the same
paper underlies all subsequent LP-rounding algorithms.

Given a fractional optimal solution $(x_{ij}^\ast, y_i^\ast)$ to the
LP, if all neighborhoods $N(j) = \{i\in\fac \suchthat x_{ij}^\ast >
0\}$ are disjoint, then we can open exactly one facility in each
$N(j)$, with each $i\in N(j)$ chosen with probability $y_i^\ast$. This
will give us expected facility cost $F^\ast$ and connection cost
$C^\ast$.  However, in general those neighborhoods will overlap and we
cannot guarantee each client $j$ has one of its neighbor open if we
open each $i$ independently with probability $y_i$. Now the triangle
inequality helps. If we guarantee a subset of clients each having an
open facility in its neighborhood, and for every client $j$ outside
this subset, it can find a client $j'$ in the subset such that the
distance $d_{j j'}$ is not too large, then we can estimate the
connection cost of client $j$ by the sum of $d_{i' j'} + d_{j j'}$
where $i'$ is a facility open in $N(j')$. Moreover, we would like to
have all $j'$ in the subset with disjoint $N(j')$ so that we could
have a handle on the facility cost.

The Shmoys, Tardos and Aardal's algorithm now becomes natural. Given a
fractional optimal solution $(x_{ij}^\ast, y_i^\ast)$, the algorithm
first associate each client $j$ with a value $g(j)$, which is
essentially an upper bound on the connection cost in the integral
solution. Now the algorithm works in iterations. In one iteration it
picks a non-clustered client with minimum $g(j)$, call it $p$, then
all remaining clients $j'$ with $N(j')\cap N(p)$ non-empty will be
added to the cluster centered at $p$. The iterations completes when
all clients are clustered. The next step is to open the cheapest
facility in each center's neighborhood and all clients assigned to
that cluster will connect to that facility.

To estimate the cost, it is easy to see the facility cost being
bounded by $F^\ast$, since neighborhood of centers are disjoint and we
effectively replacing a neighborhood with a minimum cost facility,
whose cost must be no more than the average cost of that
neighborhood. For connection cost, each center $p$ has connection cost
bounded by $g(p)$, while each non-center client $j$ has connection
cost bounded by the three edges $d_{i' p} + d_{ip} + d_{ij}$, where
$i'$ is the facility chosen in $N(p)$ and $i$ is any facility in
$N(p)\cap N(j)$. Because we cap the distance in each $N(j)$ by $g(j)$,
we have that $d_{i' p} \leq g(p), d_{ip} \leq g(p)$ and $d_{ij} \leq
g(j)$. So for a non-center client, the distance is bounded by $g(p) +
g(p) + g(j) = 2g(p) + g(j) \leq 3g(j)$ since we choose a center to
minimize $g(p)$. Now we need to get a handle on $g(j)$. To relate
$g(j)$ to $C_j^\ast \stackrel{\mydef}{=} \sum_{i\in \fac}
d_{ij}x_{ij}^\ast$, we need to cut the neighborhood by some
fraction. Here are two ways, the STA97 way and my way.

The STA97 does this by cutting at a point with total connection value
$x_{ij}^\ast$ accumulates to a preselected constant $1/\alpha$. Let
the resulted neighborhood be $\wbar N(j)$. Now the chosen part has
total $x_{ij}^\ast$ at least $1/\alpha$, which implies the other part
has total $x_{ij}^\ast$ at most $1-1/\alpha$. It follows that
$g(j)*(1-1/\alpha) \leq \sum_{i\in N(j)} d_{ij} x_{ij}^\ast =
C_j^\ast$. So we have a bound on $g(j)$ in terms of $C_j^\ast$
now. The rest is essentially adding up all clients so that the
connection cost is bounded by $\sum_{j\in\cli} 3g(j) =
3C^\ast/(1-1/\alpha)$. To compensate for the cutting by $1/\alpha$, we
need to scale up all $(x_{ij}^\ast, y_i^\ast)$ by $\alpha$. So our
final ratio is $\max\{\alpha, 3/(1-1/\alpha)\}$. Pick $\alpha=4$ gives
a $4$-approximation algorithm.

A second way to cut is to use Markov bound. Instead of looking at
cumulative connection values, that is, $x_{ij}^\ast$'s, we look
directly at the maximum distance from a sub-neighborhood. Now we cut
at $g(j) \leq C_j^\ast \alpha$ for every client $j$, so our connection
cost is now at most $\sum_{j\in\cli} 3g(j) \leq 3 \alpha C^\ast$. We
need to estimate the total connection value captured by that
sub-neighborhood. And here is where Markov bound comes in handy.

For random variable $X$ with mean $\mu$, and a number $t\geq 1$, we have
\begin{equation*}
  \text{Pr}(x \geq t\mu) \leq 1/t
\end{equation*}
Now plug in $g(j)$ for $x$, $C_j^\ast$ for $\mu$, and $\alpha$ for
$t$, we know that the total connection value outside the
sub-neighborhood is no more than $1/alpha$, so the total connection
value inside the sub-neighborhood is at least $1-1/\alpha$.

Again we obtain the approximation ratio being $\max\{1/(1-1/\alpha),
3\alpha\}$, and take $\alpha = 4/3$ gives ratio $4$.

From there, the improvements by Chudak, Sviridenko and Byrka are
mostly on a better estimate on the connection cost of a non-primary
client, using randomized rounding.

Chudak noticed that $\alpha_j^\ast$, the dual solution, provided an
upper bound on $\max_{j\in N(j)} d_{ij}$. With a similar clustering,
the rounding process will open exactly one facility in each $N(p)$ but
randomly select a facility in $N(p)$ with probability
$y_{i}^\ast$. The facilities not in any $N(p)$ are then opened
independently with probability $y_i^\ast$. Now we can estimate the
distance by a provably worse process, which was then upper bounded by
$d_1 y_1 + d_2 y_2 (1-y_1) + \ldots + d_k y_k (1-y_1)\ldots(1-y_{k-1})
+ d_{k+1}(1-y_1)\ldots(1-y_k)$. This can be shown to imply a
connection cost no more than $C^\ast + 2/e \LP^\ast$. Together with
the expected facility cost being bounded by $F^\ast$, the total cost
is no more than $(1+2/e)\LP^\ast$.

Srividenko's improvement is to scale up all $y_i^\ast$ by some
constant $\alpha \geq 1$, then group clients with overlapping
$N_\alpha(j)$. A concave function was applied to estimate the
connection cost for each client and the fractional solution was
rounded by pairing up two fractional values and make at least one of
them being integral, 0 or 1. As the sum of those functions provides an
upper bound on the total cost and the value of that sum can only
decrease during the rounding process, it is sufficient to prove an
upper bound on the function value of the initial vector $(\bar y_i)$,
which was done by choosing carefully a distribution for $\alpha$ and
integrating the functions.

Byrka further developed Sviridenko and Chudak's idea with a better
estimate on the distance when all $N(j)$ are closed for a non-primary
client $j$. The exact derivation is a bit complicated but this allows
the ratio to be improved from $1.736$ to $1.575$. Following
Sviridenko's work, it is natural to expect that randomizing the
scaling factor $\alpha$ might lead to better approximation ratios, and
this has been done by Li (Princeton).

%%%%%%%%%%%%%%%%%%%%%%%%%%%%%%%%%%%%%%%%%%%%%%%%%%%%%%%%%%%%%%%%%%%%%
\section{Preliminary Results for the Fault-tolerant Facility Placement
  Problem}

%%%%%%%%%%%%%%%%%%%%%%%%%%%%%%%%%%%%%%%%%%%%%%%%%%%%%%%%%%%%%%%%%%%%%
\section{On-going Work and Expected Results}

\end{document}
