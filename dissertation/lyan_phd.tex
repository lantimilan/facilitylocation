%% lyan doctoral dissertation
%% init @ 03/26/2013

%% the preamble below is from the template
%%
%% uctest.tex 11/3/94
%% Copyright (C) 1988-2004 Daniel Gildea, BBF, Ethan Munson.
%
% This work may be distributed and/or modified under the
% conditions of the LaTeX Project Public License, either version 1.3
% of this license or (at your option) any later version.
% The latest version of this license is in
%   http://www.latex-project.org/lppl.txt
% and version 1.3 or later is part of all distributions of LaTeX
% version 2003/12/01 or later.
%
% This work has the LPPL maintenance status "maintained".
% 
% The Current Maintainer of this work is Daniel Gildea.
%
% 2007/08/01
% LaTeX Package "ucr" is modified from LaTeX package "ucthesis."
% This modification is therefore under to the conditions of 
% the LaTeX Project Public License.
% Its formality is suitable for the dissertation of Universty of
% California, Riverside.
% This test document is for the convenience of all students of
% Universty of California, Riverside.
% Contact Charles Yang at chcyang@yahoo.com if you like.
% Charles Yang has nothing to do with the original author's sarcasm.
%
% \documentclass[11pt]{ucthesis}
% \documentclass[11pt]{ucr}
\documentclass[oneside,final]{ucr}
\usepackage{amssymb}
%%%%%%%%%%%%%%%%%%%%%%%%%%%%%%%%%%%%%%%%%%%%%%%%%%%%%%%%%%%%%%%%%%%%%%%%%%%%%%%%%%%%%%%%%%%%%%%%%%%%
\usepackage{bm}
\usepackage{amsmath}
\usepackage{mathrsfs}
\usepackage[dvips]{graphicx}
\usepackage{graphics}
\usepackage{subfigure}
\usepackage{flafter}
\usepackage{sw20uctd}

%TCIDATA{OutputFilter=LATEX.DLL}
%TCIDATA{Created=Saturday, April 29, 2006 22:07:22}
%TCIDATA{LastRevised=Tuesday, July 17, 2007 22:48:56}
%TCIDATA{<META NAME="GraphicsSave" CONTENT="32">}
%TCIDATA{<META NAME="DocumentShell" CONTENT="Other Documents\SW\Thesis - University of California Thesis">}
%TCIDATA{Language=American English}
%TCIDATA{CSTFile=ucr.cst}

\newtheorem{theorem}{Theorem}
\newtheorem{acknowledgement}[theorem]{Acknowledgement}
\newtheorem{algorithm}[theorem]{Algorithm}
\newtheorem{axiom}[theorem]{Axiom}
\newtheorem{case}[theorem]{Case}
\newtheorem{claim}[theorem]{Claim}
\newtheorem{conclusion}[theorem]{Conclusion}
\newtheorem{condition}[theorem]{Condition}
\newtheorem{conjecture}[theorem]{Conjecture}
\newtheorem{corollary}[theorem]{Corollary}
\newtheorem{criterion}[theorem]{Criterion}
\newtheorem{definition}[theorem]{Definition}
\newtheorem{example}[theorem]{Example}
\newtheorem{exercise}[theorem]{Exercise}
\newtheorem{lemma}[theorem]{Lemma}
\newtheorem{notation}[theorem]{Notation}
\newtheorem{problem}[theorem]{Problem}
\newtheorem{proposition}[theorem]{Proposition}
\newtheorem{remark}[theorem]{Remark}
\newtheorem{solution}[theorem]{Solution}
\newtheorem{summary}[theorem]{Summary}
\newenvironment{proof}[1][Proof]{\textbf{#1.} }{\ \rule{0.5em}{0.5em}}
\def\dsp{\def\baselinestretch{2.0}\large\normalsize}
\dsp
%% tcilatex is a package from Scientific Workplace.
%% The user may remove the following line without serious damage.
%% \input{tcilatex}
%% The user must use \textheight and \topmargin to control to button margin.
\textheight = 8.25in
\topmargin = 0.750in

\begin{document}

% Declarations for Front Matter

\title{Approximation Algorithms for The Fault-tolerant Facility Placement Problem}
\author{Li Yan}
\degreemonth{June}
\degreeyear{2013}
\degree{Doctor of Philosophy}
\chair{Professor Marek Chrobak}
\othermembers{Professor Tao Jiang\\
Professor Stefano Lonardi\\
Professor Neal Young}
\numberofmembers{4}
\field{Computer Science}
\campus{Riverside}

\maketitle
\copyrightpage{}
\approvalpage{}

\degreesemester{Summer}

\begin{frontmatter}

\begin{acknowledgements}
  I would thank my advisor, Marek Chrobak, for bringing me into the
  PhD program of U of California Riverside, and for his guidance and
  patience on my study and research in the past five years. I am also
  grateful for the committee for helpful discussion and helpful
  comments on my research and the dissertation.

  The support and encouragement from my wife and my parents is always
  a source of morale.
\end{acknowledgements}

\begin{dedication}
\null\vfil
{\large
\begin{center}
  To my parents, who always have faith on my endeavor.
\end{center}}
\vfil\null
\end{dedication}

\begin{abstract}
  The dissertation concerns the fault-tolerant facility placement
  problem (FTFP), a variant of the well-known uncapacitated facility
  location problem (UFL). The result was mostly on the approximation
  algorithms and their performance guarantee. It is easily seen that
  FTFP is a generalization of UFL. In this thesis we show that several
  techniques that have been applied in the UFL problem can be
  generalized to the FTFP problem with good approximation results.
\end{abstract}

\tableofcontents
\listoffigures
\listoftables

\end{frontmatter}

%\part{First Part}

%% ch1 intro
\chapter{Introduction} \label{ch: intro}

\section{The Problem and the Background}

\section{The Notation and Definition}

\section{The Notion of P vs NP, Approximation}


\textbf{****THIS MARKS THE TEXT OF THE TEMPLATE****}

Every dissertation should have an introduction.  You might not realize
it, but the introduction should introduce the concepts, backgrouand,
and goals of the dissertation.

\begin{table}
\begin{tabular}{|l|r|}
  \hline 
Title & Author \\
\hline
War And Peace & Leo Tolstoy \\
The Great Gatsby & F. Scott Fitzgerald \\ \hline
\end{tabular}
\caption{A normalsize table.  There has been a complaint that table
captions are not single-spaced.  This is odd because the code
indicates that they should be.}
\end{table}

\begin{table}
\caption{A small table.}
\begin{scriptsizetabular}{|l|r|}
  \hline 
Title & Author \\
\hline
War And Peace & Leo Tolstoy \\
The Great Gatsby & F. Scott Fitzgerald \\ \hline
\end{scriptsizetabular}
\end{table}

%% ch2 related work and results summary
\chapter{Related Work and Known Results} \label{ch: related_work}

\section{Related Problems}

\subsection{UFL}
Upper and lower bound.

\subsection{FTFL}
Upper and lower bound.

\subsection{Our Problem: FTFP}
Upper and lower bound.


%% ch3 techniques
\chapter{Techniques} \label{ch: techniques}

We employ two techniques to obtain approximation results on the FTFP
problem.

\section{Demand Reduction}

\section{Adaptive Partition}

%% ch4 LP-rounding results
\chapter{LP-rounding Algorithms} \label{ch: lp-rounding}

\section{3-approximation}

\section{1.736-approximation}

\section{1.575-approximation}

%% ch5 primal-dual results
\chapter{Primal-dual Algorithms} \label{ch: primal-dual}

\section{The Greedy algorithm with $O(\log n)$ Ratio}

\section{An Example Showing the Difficulty in Obtaining $O(1)$ Ratio}

%% ch6 conclusion
\chapter{Conclusion} \label{ch: conclusion}

\bibliographystyle{plain}
\bibliography{facility}

\appendix
\chapter{Some Ancillary Stuff}

Ancillary material should be put in appendices, which appear after the
bibliography. 

\end{document}
