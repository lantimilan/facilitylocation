%% lyan doctoral dissertation
%% init @ 03/26/2013
%% TODO: make a better template

%% the preamble below is from the template
%%
%% uctest.tex 11/3/94
%% Copyright (C) 1988-2004 Daniel Gildea, BBF, Ethan Munson.
%
% This work may be distributed and/or modified under the
% conditions of the LaTeX Project Public License, either version 1.3
% of this license or (at your option) any later version.
% The latest version of this license is in
%   http://www.latex-project.org/lppl.txt
% and version 1.3 or later is part of all distributions of LaTeX
% version 2003/12/01 or later.
%
% This work has the LPPL maintenance status "maintained".
% 
% The Current Maintainer of this work is Daniel Gildea.
%
% 2007/08/01
% LaTeX Package "ucr" is modified from LaTeX package "ucthesis."
% This modification is therefore under to the conditions of 
% the LaTeX Project Public License.
% Its formality is suitable for the dissertation of Universty of
% California, Riverside.
% This test document is for the convenience of all students of
% Universty of California, Riverside.
% Contact Charles Yang at chcyang@yahoo.com if you like.
% Charles Yang has nothing to do with the original author's sarcasm.
%
% \documentclass[11pt]{ucthesis}
% \documentclass[11pt]{ucr}
\documentclass[oneside,final]{ucr}
\usepackage{amssymb}
%%%%%%%%%%%%%%%%%%%%%%%%%%%%%%%%%%%%%%%%%%%%%%%%%%%%%%%%%%%%%%%%%%%%%%%%%%%%%%%%%%%%%%%%%%%%%%%%%%%%
\usepackage{bm}
\usepackage{amsmath}
%\usepackage[dvips]{graphicx}
%\usepackage{graphics}
\usepackage{graphicx}
\usepackage{caption,subcaption,enumerate}
\usepackage{flafter}
\usepackage{sw20uctd}

\usepackage{tikz}
\usetikzlibrary{shapes}
%% pseudocode related
\usepackage[nothing]{algorithm}
\usepackage{algorithmicx}
\usepackage[noend]{algpseudocode}
\usepackage{array}

\floatname{algorithm}{Pseudocode}
\renewcommand{\algorithmicrequire}{\textbf{Input:}}
\renewcommand{\algorithmicensure}{\textbf{Output:}}


\newtheorem{theorem}{Theorem}
\newtheorem{acknowledgement}[theorem]{Acknowledgement}
%%\newtheorem{algorithm}[theorem]{Algorithm}
\newtheorem{axiom}[theorem]{Axiom}
\newtheorem{case}[theorem]{Case}
\newtheorem{claim}[theorem]{Claim}
\newtheorem{conclusion}[theorem]{Conclusion}
\newtheorem{condition}[theorem]{Condition}
\newtheorem{conjecture}[theorem]{Conjecture}
\newtheorem{corollary}[theorem]{Corollary}
\newtheorem{observation}[theorem]{Observation}
\newtheorem{criterion}[theorem]{Criterion}
\newtheorem{definition}[theorem]{Definition}
\newtheorem{example}[theorem]{Example}
\newtheorem{exercise}[theorem]{Exercise}
\newtheorem{fact}[theorem]{Fact}
\newtheorem{lemma}[theorem]{Lemma}
\newtheorem{notation}[theorem]{Notation}
\newtheorem{problem}[theorem]{Problem}
\newtheorem{proposition}[theorem]{Proposition}
\newtheorem{remark}[theorem]{Remark}
\newtheorem{solution}[theorem]{Solution}
\newtheorem{summary}[theorem]{Summary}
\newenvironment{proof}[1][Proof]{\textbf{#1.} }{\ \rule{0.5em}{0.5em}}
\def\dsp{\def\baselinestretch{2.0}\large\normalsize}
\dsp

%% The user must use \textheight and \topmargin to control to button margin.
\textheight = 8.25in
\topmargin = 0.750in


%%%%%%%%%%%%%%%%%%%%%%%%%%%%%%%%%%%%%%%%%%%%%%%%%%%%%%%%%%

% non-math stuff

\newcommand{\myparagraph}[1]{{\medskip\noindent{\bf #1}}}
\newcommand{\emparagraph}[1]{{\medskip\noindent{\it #1}}}
\newcommand{\etal}{{\it et al.}}
\newcommand{\myif}{{\mbox{\rm\ if \ }}}
\newcommand{\mycase}[1]{\mbox{{\underline{Case #1}}:\/}}

\newcommand{\margincomment}[1]%
    {{%
      \marginpar{{\tiny\begin{minipage}{0.5in}
                       \begin{flushleft}
                          {#1}
                       \end{flushleft}
                       \end{minipage}
                }}
    }}


%%%%%%%%%%%%%%%%%%%%%%%%%%%%%%%%%%%%%%%%%%%%%%%%%%%%%%%%%%

% various letters

\newcommand{\hatc}{{\hat c}}
\newcommand{\hatC}{{\hat C}}
\newcommand{\hatr}{{\hat r}}
\newcommand{\hatx}{{\hat x}}
\newcommand{\haty}{{\hat y}}
\newcommand{\dotx}{{\dot x}}
\newcommand{\doty}{{\dot y}}
\newcommand{\dotr}{{\dot r}}
\newcommand{\boldx}{{\mathbf x}}

\newcommand{\doubledone}{{\bar 1}}
\newcommand{\doubledtwo}{{\bar 2}}
\newcommand{\barc}{{\bar c}}
\newcommand{\bart}{{\bar t}}

\newcommand{\barx}{{\bar x}}
\newcommand{\bary}{{\bar y}}
\newcommand{\barz}{{\bar z}}
\newcommand{\barr}{{\bar r}}
\newcommand{\barX}{{\bar X}}
\newcommand{\barY}{{\bar Y}}
\newcommand{\barZ}{{\bar Z}}
\newcommand{\bara}{{\bar a}}
\newcommand{\bard}{{\bar d}}
\newcommand{\barm}{{\bar m}}
\newcommand{\barA}{{\bar A}}
\newcommand{\barB}{{\bar B}}
\newcommand{\barC}{{\bar C}}
\newcommand{\barG}{{\bar G}}
\newcommand{\barE}{{\bar E}}
\newcommand{\barV}{{\bar V}}

\newcommand{\wbarC}{{\overline{C}}}
\newcommand{\wbarD}{{\overline{D}}}
\newcommand{\wbarN}{{\overline{N}}}
\newcommand{\wbarX}{{\overline{X}}}


\newcommand{\barbeta}{{\bar\beta}}
\newcommand{\bargamma}{{\bar\gamma}}
\newcommand{\apomega}{{\bar\omega}}

\newcommand{\bfr}{\boldsymbol{r}}
\newcommand{\bfv}{{\bf v}}
\newcommand{\bfx}{\boldsymbol{x}}
\newcommand{\bfy}{\boldsymbol{y}}
\newcommand{\bfz}{{\bf z}}
\newcommand{\bfQ}{{\bf Q}}
\newcommand{\bfR}{{\bf R}}
\newcommand{\bfS}{{\bf S}}
\newcommand{\bfT}{{\bf T}}
\newcommand{\bfV}{{\bf V}}
\newcommand{\bfone}{{\bf 1}}
\newcommand{\bfalpha}{\boldsymbol{\alpha}}
\newcommand{\bfbeta}{\boldsymbol{\beta}}

\newcommand{\calA}{{\cal A}}
\newcommand{\calB}{{\cal B}}
\newcommand{\calC}{{\cal C}}
\newcommand{\calD}{{\cal D}}
\newcommand{\calE}{{\cal E}}
\newcommand{\calG}{{\cal G}}
\newcommand{\calH}{{\cal H}}
\newcommand{\calJ}{{\cal J}}
\newcommand{\calK}{{\cal K}}
\newcommand{\calL}{{\cal L}}
\newcommand{\calM}{{\cal M}}
\newcommand{\calN}{{\cal N}}
\newcommand{\calS}{{\cal S}}
\newcommand{\calU}{{\cal U}}
\newcommand{\calX}{{\cal X}}
\newcommand{\calT}{{\cal T}}

\newcommand{\hatcalI}{{\hat{\cal I}}}
\newcommand{\barcalI}{{\bar{\cal I}}}
\newcommand{\dotcalI}{{\dot{\cal I}}}

\newcommand{\vecS}{{\bar S}}
\newcommand{\vecT}{{\bar T}}
\newcommand{\vecone}{{\bf 1}}
\newcommand{\tildec}{{\tilde c}}
\newcommand{\tilded}{{\tilde d}}
\newcommand{\tildeD}{{\tilde D}}
\newcommand{\tildeC}{{\widetilde C}}
\newcommand{\tildeZ}{{\tilde Z}}
\newcommand{\tilder}{{\widetilde r}}
\newcommand{\tildex}{{\widetilde x}}
\newcommand{\wtildeN}{{\widetilde N}}
\newcommand{\tildebfr}{\widetilde{\boldsymbol{r}}}
\newcommand{\tildebfx}{\widetilde{\boldsymbol{x}}}
\newcommand{\tildebfy}{\widetilde{\boldsymbol{y}}}

\newcommand{\barbfx}{\bar{\boldsymbol{x}}}
\newcommand{\barbfy}{\bar{\boldsymbol{y}}}
\newcommand{\hatbfx}{\hat{\boldsymbol{x}}}
\newcommand{\hatbfy}{\hat{\boldsymbol{y}}}
\newcommand{\dotbfx}{\dot{\boldsymbol{x}}}
\newcommand{\dotbfy}{\dot{\boldsymbol{y}}}

\newcommand{\wbarcalC}{{\overline{\calC}}}
\newcommand{\wbarcalD}{{\overline{\calD}}}
\newcommand{\eps}{{\epsilon}}

%%%%%%%%%%%%%%%%%%%%%%%%%%%%%%%%%%%%%%%%%%%%%%%%%%%%%%%%%%

\newcommand{\half}{{\mbox{$\frac{1}{2}$}}}
\newcommand{\threehalfs}{{\mbox{$\frac{3}{2}$}}}
\newcommand{\threefourths}{{\mbox{$\frac{3}{4}$}}}
\newcommand{\fivehalfs}{{\mbox{$\frac{5}{2}$}}}
\newcommand{\onethird}{{\mbox{$\frac{1}{3}$}}}
\newcommand{\twothirds}{{\mbox{$\frac{2}{3}$}}}
\newcommand{\fourthirds}{{\mbox{$\frac{4}{3}$}}}
\newcommand{\fivethirds}{{\mbox{$\frac{5}{3}$}}}
\newcommand{\fivefourths}{{\mbox{$\frac{5}{4}$}}}
\newcommand{\onefourth}{{\mbox{$\frac{1}{4}$}}}
\newcommand{\onefifth}{{\mbox{$\frac{1}{5}$}}}
\newcommand{\twofifths}{{\mbox{$\frac{2}{5}$}}}
\newcommand{\threefifths}{{\mbox{$\frac{3}{5}$}}}
\newcommand{\fourfifths}{{\mbox{$\frac{4}{5}$}}}
\newcommand{\ninefifths}{{\mbox{$\frac{9}{5}$}}}
\newcommand{\sevensixths}{{\mbox{$\frac{7}{6}$}}}
\newcommand{\oneeighth}{{\mbox{$\frac{1}{8}$}}}
\newcommand{\threeeighths}{{\mbox{$\frac{3}{8}$}}}
\newcommand{\fiveeighths}{{\mbox{$\frac{5}{8}$}}}
\newcommand{\seveneighths}{{\mbox{$\frac{7}{8}$}}}
\newcommand{\onetenth}{{\mbox{$\frac{1}{10}$}}}
\newcommand{\seventenths}{{\mbox{$\frac{7}{10}$}}}
\newcommand{\ninetenths}{{\mbox{$\frac{9}{10}$}}}
\newcommand{\twonineths}{{\mbox{$\frac{2}{9}$}}}
\newcommand{\fivenineths}{{\mbox{$\frac{5}{9}$}}}
\newcommand{\elevennineths}{{\mbox{$\frac{11}{9}$}}}
\newcommand{\threetwentieths}{{\mbox{$\frac{3}{20}$}}}
\newcommand{\twentyfivenineteenths}{{\mbox{$\frac{25}{19}$}}}

\newcommand{\sqrttwo}{\sqrt{2}}

%%%%%%%%%%%%%%%%%%%%%%%%%%%%%%%%%%%%%%%%%%%%%%%%%%%%%%%%%%

% various delimiters

\newcommand{\braced}[1]{{ \left\{ #1 \right\} }}
\newcommand{\angled}[1]{{ \left\langle #1 \right\rangle }}
\newcommand{\brackd}[1]{{ \left[ #1 \right] }}
\newcommand{\parend}[1]{{ \left( #1 \right) }}
\newcommand{\barred}[1]{{ \left| #1 \right| }}
\newcommand{\dbarred}[1]{{ \left\| #1 \right\| }}
\newcommand{\floor}[1]{{ \lfloor #1 \rfloor }}
\newcommand{\ceiling}[1]{{ \lceil #1 \rceil }}

%%%%%%%%%%%%%%%%%%%%%%%%%%%%%%%%%%%%%%%%%%%%%%%%%%%%%%%%%%

% some math symbols

\newcommand{\set}{\,{\leftarrow}\,}
\newcommand{\suchthat}{{\,:\,}}
\newcommand{\cost}{{\it cost}}
\newcommand{\yield}{{\it yield}}
\newcommand{\opt}{{\it opt}}

\newcommand{\algA}{{\bf A}}
\newcommand{\LHS}{{\rm LHS}}
\newcommand{\RHS}{{\rm RHS}}
\newcommand{\reals}{{\bf R}}
\newcommand{\posreals}{{\bf R}^+}

\newcommand{\assign}{{\,\leftarrow\,}}

\newcommand{\absvalue}[1]{{\barred{#1}}}
\newcommand{\posvalue}[1]{{\brackd{#1}^+}}

\newcommand{\NP}{{\mbox{\sf NP}}}
\newcommand{\PP}{{\mbox{\sf P}}}
\newcommand{\DTIME}{{\mbox{\sf DTIME}}}

\newcommand{\letbox}[1]{{\makebox[11pt]{{\small {$#1$}}}}}
\newcommand{\optstring}[1]{{ \frame{\;\raisebox{0pt}[12pt][5pt]{#1}\;} }}

\newcommand{\leftend}{{\diamond}}
\newcommand{\rightend}{{\diamond}}

%\newcommand{\argmin}{{\mbox{\rm argmin}}}
\DeclareMathOperator*{\argmin}{arg\,min}

\newcommand\litem[1]{\item{\bfseries #1\enspace}}
\newcommand{\ceil}[1] {\lceil #1 \rceil}
\newcommand{\naive}{na\"{\i}ve}
\newcommand{\LP}{\mbox{\rm LP}}
\newcommand{\OPT}{\mbox{\rm OPT}}
\newcommand{\ALG}{\mbox{\rm ALG}}
\newcommand{\LPR}[1]{{\mbox{\rm LPR#1}}}
\newcommand{\smallLPR}[1]{{\mbox{\tiny\rm LPR#1}}}
% algorithm names
\newcommand{\ESTA}{\mbox{\rm ESTA}} % 4approx
\newcommand{\EGUP}{\mbox{\rm EGUP}} % 3approx
\newcommand{\ECHS}{\mbox{\rm ECHS}} % 1.736
\newcommand{\EBGS}{\mbox{\rm EBGS}} % 1.575
\newcommand{\GUP}{\mbox{\rm GUP}}
\newcommand{\smallESTA}{\mbox{\tiny\rm ESTA}}
\newcommand{\smallEGUP}{\mbox{\tiny\rm EGUP}}
\newcommand{\smallECHS}{\mbox{\tiny\rm ECHS}}
\newcommand{\smallEBGS}{\mbox{\tiny\rm EBGS}}

\newcommand{\SOL}[1]{{{\mbox{\rm SOL}}_{#1}}}
\newcommand{\FTFP}{\mbox{\rm FTFP}}
\newcommand{\FTFL}{\mbox{\rm FTFL}}
\newcommand{\calI}{\mathcal{I}}
\newcommand{\avg}{{\mbox{\scriptsize\rm avg}}}

\newcommand{\dmax}{\text{dmax}}
\newcommand{\davg}{\text{davg}}
\newcommand{\favg}{f_{\text{avg}}}
\newcommand{\conn}{\text{conn}}
\newcommand{\cls}{\text{cls}}
\newcommand{\far}{\text{far}}

\newcommand{\sitesset}{\mathbb{F}}
\newcommand{\clientset}{\mathbb{C}}
\newcommand{\facilityset}{\overline{\sitesset}}
\newcommand{\demandset}{\overline{\clientset}}

%\newcommand{\dist}{{\mbox{dist}}}
\newcommand{\concost}{C^{\avg}}
\newcommand{\faccost}{F^{\avg}}
\newcommand{\tcc}{\mbox{\rm{tcc}}}
\newcommand{\clsdist}{C_{\cls}^{\avg}}
\newcommand{\fardist}{C_{\far}^{\avg}}
\newcommand{\clsmax}{C_{\cls}^{\max}}
\newcommand{\clsnb}{N_{\cls}}
\newcommand{\farnb}{N_{\far}}
\newcommand{\wbarclsnb}{\wbarN_{\cls}}
\newcommand{\wbarfarnb}{\wbarN_{\far}}
\newcommand{\wtildeclsnb}{\wtildeN_{\cls}}
\newcommand{\tcccls}{\mbox{\rm{tcc}}_{\cls}}
\newcommand{\dmaxcls}{\mbox{\rm{dmax}}_{\cls}}

\newcommand{\Exp}{\mbox{\rm Exp}}

\newcommand{\FacilityDistSort}{{\textsc{FacilityDistSort}}}
\newcommand{\NearestUnitChunk}{{\textsc{NearestUnitChunk}}}
\newcommand{\AugmentToUnit}{{\textsc{AugmentToUnit}}}
\newcommand{\connsum}{{\textrm{conn}}}

%%%%%%%%%%%%%%%%%%%%%%%%%%%%%%%%%%%%%%%%%%%%%%%%%%%%%%%%%%

% theorem and such

\newtheorem{theorem}{Theorem}
\newtheorem{definition}[theorem]{Definition}
\newtheorem{corollary}[theorem]{Corollary}
\newtheorem{lemma}[theorem]{Lemma}
\newtheorem{fact}[theorem]{Fact}
\newtheorem{claim}[theorem]{Claim}
\newtheorem{conjecture}[theorem]{Conjecture}
\newtheorem{observation}[theorem]{Observation}

%%%%%%%%%%%%%%%%%%%%%%%%%%%%%%%%%%%%%%%%%%%%%%%%%%%%%%%%%%

\newcommand{\ignore}[1]{}

% for \cal definition
\makeatletter
\DeclareRobustCommand*\cal{\@fontswitch\relax\mathcal}
\makeatother
\setlength\headsep{-0.5in} 
\begin{document}

% Declarations for Front Matter

\title{Approximation Algorithms for the Fault-Tolerant Facility Placement Problem}
\author{Li Yan}
\degreemonth{June}
\degreeyear{2013}
\degree{Doctor of Philosophy}
\chair{Professor Marek Chrobak}
\othermembers{Professor Tao Jiang\\
Professor Stefano Lonardi\\
Professor Neal Young}
\numberofmembers{4}
\field{Computer Science}
\campus{Riverside}

\maketitle
\copyrightpage{}
\approvalpage{}

\degreesemester{Summer}

\begin{frontmatter}

\begin{acknowledgements}
  I would thank my advisor, Marek Chrobak, for bringing me
  into the PhD program of University of California
  Riverside, and for his guidance and patience on my study
  and research in the past five years. I am also grateful
  for the committee, Tao Jiang, Stefano Lonardi, and Neal
  Young for helpful discussions and comments on my research
  and this dissertatqion.

  The supportive environment of the algorithm lab and
  computer science department has made PhD study here a
  pleasant experience and I am grateful for Claire Huang,
  Wei Li and the algorithm lab for helpful discussions and
  stimulation of ideas.
\end{acknowledgements}

\begin{dedication}
\null\vfil
{\large
\begin{center}
  To my parents, who always have faith on my endeavor.
\end{center}}
\vfil\null
\end{dedication}

\begin{abstract}
  The dissertation concerns the fault-tolerant facility
  placement problem (FTFP), a variant of the well-known
  uncapacitated facility location problem (UFL). In the FTFP
  problem, we have a set of sites where we can open
  facilities and a set of clients each with an integral
  demand. To satisfy their demands, clients need to be
  connected to open facilities in sites. The goal is to
  satisfy all clients' demand while minimizing the total
  cost, that is the cost of opening facilities and the cost
  of connecting clients to facilities. The problem is shown
  to be NP-hard and hence we study the approximation
  algorithms and their performance guarantee. Approximation
  algorithms are algorithms that run in polynomial time with
  provable performance when compared to optimal solutions.

  In this thesis we give two techniques that lead to several
  LP-rounding algorithms with progressively improved
  approximation ratio. The best ratio we have is 1.575. This
  ratio matches the best LP-based approximation ratio for
  the more restricted problem, namely UFL. We have also
  studied the applicability of primal-dual approaches to
  FTFP. In particular, we show that a natural greedy
  algorithm analyzed using dual-fitting technique gives an
  upper bound of O(logn) for approximation ratio. On the
  negative side, under a natural assumption, we give an
  example showing the dual-fitting analysis cannot give a
  ratio better than Omega(logn/loglogn).
\end{abstract}

\tableofcontents
\listoffigures
\listoftables

\end{frontmatter}

%% ch1 intro
\chapter{Introduction} \label{ch: intro}

\section{The Problem and the Background}
The facility location problems (FL) are about selecting a
set of candidate places to build (or open) facilities and
connecting clients to the facilities to satisfy their
demands. They model real world scenarios like setting up
warehouses to deliver commodities to retailers, building
power suppliers to serve the needs of a district of
residents, placing content servers in a network to send
files to clients. This simple model has been studied
extensively since 1960s (see books by Mirchandani and
Francis~\cite{Francis90}) due to its practical significance
and nice properties that allow a multitude of approaches to
be applicable. The problems we study in this thesis fall
under the category of the so-called \emph{discrete} facility
location problems. In the discrete facility location
problems, we are given a set of facilities that could be
opened, and a set of clients each with some demand, and the
distance (or connection cost) between a client and a
facility. A client needs to be serviced by getting connected
to as many open facilities as its demand. We would like to
service all clients while minimizing the total cost of
opening facilities and connecting clients to facilities.

The facility location problems have taken a central place in
both operations research and theoretical computer science
since the 60's, and a number of approaches have been
proposed, including heuristic solutions, branch and bound,
probabilistic methods, and more recently, approximation
algorithms. Interestingly, some of the heuristic solutions,
although are not concerned with the performance on the
hardest instance, have later been shown to offer a provably
worst-case performance guarantee.

As many optimization problems arising in practical
applications, it comes without much surprise that the
facility location problems is \NP-hard, and thus precludes a
promise for polynomial time algorithms that solve the
problems exactly. Nonetheless there are polynomial time
algorithms that deliver a solution with cost only a small
percentage off from the cost of optimal solutions. These
type of algorithms, known as approximation algorithms and
their performance analysis, is the subject of this thesis.

There are over a dozen different variation in the problem
formulation, and the uncapacitated facility location problem
(UFL) concerns the simpliest model where each candidate
facility point has an opening cost and the connection cost
between a client and a facility is the distance between the
two. Each client needs to be connected to one open
facility. A solution consists of a set of facilities to open
and a specification of connections for each client to an
open facility. The cost of the solution is simply the
facility cost, defined as the sum of opening cost for the
facilities chosen to open, and the connection cost, defined
as the cost to connect every client to some open
facility. The UFL problem asks for a solution that satisfies
all clients and with minimum total cost. The UFL problem
with general distances has an algorithm with approximation
ratio $O(\log n)$ where $n$ is the number of clients, due to
Hochbaum~\cite{Hochbaum82}. A matching lower bound of
$O(\log n)$ is immediate, as the UFL problem contains the
well-known Set-Cover problem as a special case. In the
Set-Cover problem, we are given a universe $\calU =
\{e_1,\ldots,e_n\}$ and a collection of sets $\calS =
\{S_1,\ldots,S_m\}$ such that every $S_i, i=1,\ldots,m$ is a
subset of $\calU$. The problem asks for the smallest number
of sets from $\calS$ whose union is $\calU$. More on the
relation of the two problems, UFL and Set-Cover in
Section~\ref{sec:hardness}.

Renewed interest in the UFL problem has been made possible
after Shmoys, Tardos and Aardal~\cite{ShmoysTA97} showed
that, when distances form a metric, there is an algorithm
with $O(1)$ approximation ratio. Improved algorithms with
more sophisticated ideas have been proposed. The past two
decades has observed a sequence of improved approximation
ratio, from the original
$4$-approximation~\cite{ShmoysTA97}, to the
$1.488$-approximation by Li~\cite{Li11}, the best known
approximation ratio to date.

The problem studied in this thesis is a generalization of
the UFL problem, in that each client may specify a demand
and the client then needs to be connected to several
different facilities with the number of connections equal to
its demand.  A solution consists of a specification of the
number of facilities to open in each site, and the number of
connections between clients and sites. Open multiple
facilities in the same site incurs a cost of the opening
cost for this site multiplied by the number of facilities
opened. The connection cost between a site and a client is
the number of connections times the distance between the
two, with the constaint that the number of connections
cannot exceed the number of facilities opened in that
site. The FTFP problem asks for a solution with minimum
total cost, that is, the sum of facility cost and connection
cost.

\section{Notation and Definition}
We denote the set of sites as $\sitesset$ and the set of
clients as $\clientset$. Each client $j \in \clientset$ has
a demand $r_j$, meaning the client $j$ needs to be connected
to $r_j$ different facilities. To open one facility at site
$i$ incurs a cost of $f_i$. To make one connection from
client $j$ to a facility at site $i$ incurs a cost of
$d_{ij}$. The problem asks for a vector of $(\bfx, \bfy)$
such that $x_{ij} \in \{0, 1, 2, \ldots\}$ denotes the
number of connections between site $i$ and client $j$, and
$y_i \in \{0, 1, 2\ldots\}$ denotes the number of facilities
opened at site $i$. We then seek a solution such that $y_i
\geq x_{ij}$ for every $i \in \sitesset, j \in \clientset$
and $\sum_{i\in\sitesset} x_{ij} \geq r_j$ for all clients
$j \in \clientset$, and we are to minimize the total cost of
the solution, which is $\sum_{i \in \sitesset} f_i y_i +
\sum_{i \in \sitesset, j \in \clientset} d_{ij} x_{ij}$. We
call $\sum_{i \in \sitesset} f_i y_i$ the facility cost of a
solution and $\sum_{i \in \sitesset, j \in \clientset}
d_{ij} x_{ij}$ the connection cost of a solution $(\bfx,
\bfy)$.

\section{The Notion of P vs NP, Approximation}
All problems unders study in this thesis are optimization
problems. An optimization problem is specified by a set of
parameters and constraints, which describes an instance. An
optimization problem is either a minimization problem or a
maximization problem.

A feasible solution is a solution that satisfies all the
constraints. There is also a cost function evalutes every
feasible solution to a numerical cost. For a minimization
problem, an optimal solution is a feasible solution with a
cost that is no more than the cost of any feasible solution.

An algorithm is said to solve an optimization problem
exactly if the algorithm always computes an optimal solution
given an instance with a nonempty solution space. For
problems that are \NP-hard, such an exact algorithm that
runs in polynomial time is not possible unless $\PP =
\NP$. Therefore we resort to polynomial time algorithms that
computes a solution whose cost can be proved to be within
some factor away from the cost of an optimal solution.

\section{Hardness Results on Approximating UFL}
\label{sec:hardness}

Since FTFP contains UFL as a special case, any hardness
result obtained on UFL remains applicable to FTFP. In the
following, we review some well-known hardness results on
UFL, with the implication that the same claims hold for FTFP
as well.

The UFL problem is easily seen to be NP-hard, as they
contain the Set Cover problem as a special case. The Set
Cover problem is that, given a universe $\calU = \{e_1,
\ldots, e_n\}$ and a collection of subsets $\calS = \{S_1,
\ldots, S_m\}$ such that $S_l \leq U$ for $l=1,\ldots,m$,
find a minimum number of sets in $\calS$ to cover all
elements in $\calU$. It is well-known that the Set-Cover
problem is NP-hard.

%%% NP-optimization problems:
%%% - can verify instance validity in polynomial time
%%% - can verify solution feasibility in polynomial time
%%% - can compute solution cost in polynomial time
%%%
%%% Turing reduction (polynomial time, then Cook reduction)
%%% from problem A to problem B
%%% given an algorithm for B, then can solve A by constructing
%%% an oracle machine with oracle for B.
\begin{proposition}\label{prop:UFLNP}
  The general UFL problem is NP-hard.
\end{proposition}
\begin{proof}
  Reduction from the Set Cover problem. In the Set Cover
  problem, we have a universe of elements, that is $\calU =
  \{e_1, \ldots, e_n\}$, and a collection of sets $\calS =
  \{S_1, \ldots, S_m\}$ such that $S_i \subseteq \calU$ for
  $i=1,\ldots,m$. We construct a general UFL instance like
  this: for each $e_j, j=1,\ldots,n$ we have a client $j$,
  and for each set $S_i, i=1,\ldots,m$ we have a facility
  $i$. The facility cost $f_i=1$ for every facility
  $i=1,\ldots,m$~\footnote{Actually any value of $f_i > 0$
    will work, for example, we can set $f_i=100$ for every
    facility $i$.} and the distance $d_{ij} = 1$ if $e_j \in
  S_i$ and $d_{ij} = \infty$ if $e_j \notin S_i$. Clearly an
  optimal solution for the UFL instance can only use edges
  with $d_{ij} = 1$. It is easy to see that given any
  optimal solution of the Set Cover instance, we can
  construct an optimal solution for the UFL instance, by
  simply taking the facilities whose corresponding sets are
  chosen in the set cover. On the other hand, given an
  optimal solution to the UFL instance, we can only have
  $d_{ij}=1$ connections, that implies for every client $j$,
  the corresponding element $e_j$ is covered by some set
  $S_i$, whose corresponding facility $i$ is chosen in the
  UFL solution. Let $I$ be the set of facilities chosen in
  the UFL solution, it is easily seen that the corresponding
  sets in the Set Cover instance form a set cover.
\end{proof}

\begin{proposition} \label{prop:metricNP}
  The metric UFL problem is NP-hard.
\end{proposition}
\begin{proof}
  The reduction is also from the Set Cover problem. Unlike
  the general UFL problem, we can no longer have edges with
  length $1$ and $\infty$ now, as the distances are
  constrained by the triangle inequality. We still have sets
  $S_i$ in the Set Cover instance correspond to the
  facilities $i$ in the metric UFL instance, and elements
  $e_j$ in the Set Cover instance correspond to the clients
  $j$ in the metric UFL instance. Our facility cost $f_i =
  \epsilon$ for some small number $\epsilon > 0$, and every
  facility $i$ has the same facility cost. Our distance
  $d_{ij}$ is now $1$ if $e_j \in S_i$ and $3$ if $e_j
  \notin S_i$ in the Set Cover instance.

  Given the construction, it is clear that any optimal
  solution for the metric UFL instance cannot use an edge of
  distance $3$, as there exists a solution that beats such a
  solution with lower cost, namely a solution that opens all
  facilities with total cost $m\epsilon + n\cdot 1 =
  m\epsilon + n$, as we can choose $\epsilon = 1/m^2$. It
  follows that any optimal solution for the metric UFL
  instance must have all clients connected at distance of
  $1$. Such a solution would have a facility set corresponds
  to a set cover for the corresponding Set Cover instance,
  as for every client $j$, the corresponding element $e_j$
  is covered by some set $S_i$ corresponding to some
  facility $i$.
\end{proof}

%%% MaxSNP, a technically difficult class of problems
%%% includes: Max2SAT, B-Max3SAT, B-VertexCover, B-IndependentSet, MaxCut
%%% Assume P != NP, MaxSNP hard problems cannot have PTAS.
Now we show the {\MaxSNP}-hardness of the metric UFL
problem. This implies that there exists some constant $c$
such that the metric UFL problem cannot be approximated to
be better than $c$-approximation. As a consequence, the
metric UFL problem cannot have polynomial approximation
scheme (PTAS), that is, a family of algorithms that computes
a solution with cost no more than $(1+\epsilon)$ from the
optimal, for any given constant $\epsilon > 0$, and the
running time is polynomial in the input size with $\epsilon$
being constant.

\begin{proposition}\label{prop:maxsnp}
  The metric UFL problem is {\MaxSNP}-hard~{\mbox{\rm \cite{GuhaK98}}}.
\end{proposition}
\begin{proof}
  The proof is by a reduction from the B-Vertex Cover
  problem. In the B-Vertex Cover problem, we are given a
  graph $G=(V,E)$, and a constant $B$, such that every
  vertex $u\in V$ has degree no more than $B$. And the
  problem asks for a vertex cover with minimum size. That
  is, we are to find a minimum set $V' \subseteq V$ such
  that every edge $e \in E$ has at least one endpoint in
  $V'$.

  The idea is to show that, for any given constant $\epsilon
  < 1$, given an algorithm for the metric UFL problem with
  approximation ratio $1+\epsilon$ for any constant
  $\epsilon < 1$, we are able to find an algorithm for the
  B-Vertex Cover problem with approximation ratio
  $1+\epsilon'$ such that $\epsilon'$ is a constant
  depending on $\epsilon$ and possibly $B$, and $\epsilon'$
  approaches $0$ as $\epsilon$ approaches $0$. In the
  following we shall see that we can set $\epsilon' =
  (1+B)\epsilon$ for our purpose.

  Given an instance of B-Vertex Cover, we construct an
  instance of metric UFL. For every vertex $u \in V$ we have
  a facility $i$ and for every edge $e \in E$ we have a city
  $j$. $c_{ij} = 1$ if the corresponding edge $e$ of client
  $j$ is incident on vertex $u$, which corresponds to the
  facility $i$, $d_{ij} = 3$ otherwise. We postpone defining
  the facility cost $f_i$ but remark that all facilities $i$
  have the same facility opening cost, that is $f_i$ equal
  for all facility $i$, so in the following we use $f$ to
  denote the facility opening cost for a single facility.

  Given an instance of B-Vertex Cover, let $k$ be the size
  of an optimal vertex cover. The reason we need the
  knowledge of $k$ is that we shall set our $f$ as a
  function of $k$. On the other hand, we really do not know
  $k$, as it is \NP-hard to compute $k$, the optimal
  solution value for B-Vertex Cover. However, we can still
  proceed assuming knowing $k$, because we can perform the
  following steps for every possible value of
  $k=1,2,\ldots,n$ and if our proof goes through for every
  choice of $k$, our claim must hold on the (unknown) value
  of $k$ as well. From now on we assume the knowledge of
  $k$. The same trick is used in the proof of
  Theorem~\ref{thm:1463} as well.

  We are to show that we can use an
  $(1+\epsilon)$-approximation algorithm $\calA_{\smallUFL}$
  for metric UFL to construct an
  $(1+\epsilon')$-approximation algorithm
  $\calA_{\smallBVC}$ for B-Vertex Cover. First we run
  algorithm $\calA_{\smallUFL}$ on the UFL instance, and let
  there be $\beta k$ facilities open and $\gamma n$ cities
  connect with $d_{ij}=1$ and the rest with $d_{ij} =
  3$. Then the cost of $\calA_{\smallUFL}$ is
  \begin{equation*}
    \ALG_{\smallUFL} = \beta k f + \gamma n + 3 (1 - \gamma) n
  \end{equation*}

  Since $\calA_{\smallUFL}$ is a
  $(1+\epsilon)$-approximation algorithm, we have that
  $\ALG_{\smallUFL} \leq (1+\epsilon) \OPT_{\smallUFL}$. To
  get a handle on $\OPT_{\smallUFL}$, we use a feasible
  solution to this $\UFL$ instance. One possible choice is
  to use the solution with $k$-facilities and all clients
  are connected at distance $1$. Notice that we only need to
  know such a solution exists. This solution has cost $kf +
  n$. Therefore, we have
  \begin{equation*}
    \beta k f + \gamma n + 3 (1 - \gamma) n \leq (1 + \epsilon) (kf + n)
  \end{equation*}
  Cancelling $n$ from both sides, we get
  \begin{equation}
    \label{eq:APX:UFL}
    \beta k f + 2(1-\gamma) n \leq (1+\epsilon)kf + \epsilon n
  \end{equation}

  Now we look for a solution to the B-Vertex Cover
  instance. The solution to {\UFL} shows that we can use
  $\beta k$ facilities to connect to $\gamma n$ clients with
  distance $1$, for the others at distance $3$, we need at
  most one facility each to connect them at distance $1$,
  that is, we need at most $(1-\gamma)n$ vertices to cover
  the remaining edges. So we have a vertex cover with size
  $\beta k + (1 - \gamma) n$, we hope to show that
  \begin{equation}
    \label{eq:APX:BVC}
    \beta k + (1 - \gamma) n \leq (1 + \epsilon') k,
  \end{equation}
  where $\epsilon'$ is a constant depend on $\epsilon$ and possibly
  $B$, with the property that $\epsilon'$ approaches $0$ as $\epsilon$
  approaches $0$.

  Compare Eqn.~(\ref{eq:APX:UFL}) and
  Eqn.~(\ref{eq:APX:BVC}), we need to have some way to clean
  up the variables to get simpler inequalities so that we
  can use (\ref{eq:APX:UFL}) to deduce
  (\ref{eq:APX:BVC}). Since we have the flexibility to
  choose $f$, we use that to simplify (\ref{eq:APX:UFL}) by
  setting $f$ to be such that $n/(kf) = B$, then
  (\ref{eq:APX:UFL}) becomes
  \begin{equation}
    \label{eq:APX:UFL2}
    \beta + 2(1-\gamma)B \leq (1+\epsilon) + \epsilon B.
  \end{equation}
  This looks very similar to the left hand side of
  (\ref{eq:APX:BVC}) now, if we can get rid of $n$ in that
  left hand side. We now use the fact that every vertex in
  the B-Vertex Cover instance has degree at most $B$, so the
  $k$ vertices in an optimal solution to this B-Vertex Cover
  instance have sum of degree at most $kB$, which is no less
  than $n$, the number of edges covered, since the $k$
  vertices form a vertex cover. So we have $kB \geq n$,
  thus, to obtain (\ref{eq:APX:BVC}), it suffices to show
  \begin{equation}
    \label{eq:APX:BVC2}
    \beta k + (1 - \gamma) kB \leq (1 + \epsilon') k.
  \end{equation}
  Dividing $k$ from both sides, our goal now is to show
  \begin{equation}
    \label{eq:APX:BVC3}
    \beta + (1 - \gamma) B \leq (1 + \epsilon').
  \end{equation}
  Recall that in (\ref{eq:APX:UFL2}) we have $\beta +
  2(1-\gamma)B \leq (1+\epsilon) + \epsilon B$, if we set
  $\epsilon' = \epsilon (1 + B)$, we shall have the
  following
  \begin{equation*}
    \beta + (1-\gamma)B \leq \beta + 2(1-\gamma)B \leq (1+\epsilon) +
    \epsilon B = 1 + \epsilon (1 + B) = 1 + \epsilon',
  \end{equation*}
  where the first inequality is due to $\gamma \geq 1$, and
  the second is from (\ref{eq:APX:UFL2}).  It is easy to see
  that $\epsilon'$ approaches $0$ as $\epsilon$ approaches
  $0$ and we have the desired inequality
  (\ref{eq:APX:BVC3}). We have thus found a $(1+\epsilon')$
  approximation algorithm for the B-Vertex Cover problem,
  given a $(1+\epsilon)$-approximation algorithm for the
  metric {\UFL} problem, with the property that as
  $\epsilon$ approaches $0$, $\epsilon'$ approaches $0$ as
  well.

  Since B-Vertex Cover has \MaxSNP-hard, we conclude metric UFL is
  \MaxSNP-hard as well.
\end{proof}

After we have shown the metric UFL problem is MaxSNP-hard,
we focus on the metric version from now on. In the follow,
we mention UFL and FTFP without explicitly specifying that
their distances form a metric, that is, $d_{ij}$'s are
symmetric and satisfy the triangle inequality.

We briefly mention the last piece of hardness result, which
is the Guha-Khuller theorem.
\begin{theorem}\label{thm:1463}
  {\UFL} cannot be approximated to less than $1.463$ unless $\NP
  \subseteq \DTIME(n^{O\log\log n})$.
\end{theorem}

\begin{proof}
  The proof is by contradiction. In other words, we show
  that if metric UFL can be solved by a polynomial time
  algorithm with approximation ratio $\alpha$ that is less
  than $1.463$, then we have a polynomial time algorithm
  with approximation ratio $1/\rho \ln n$ for some constant
  $\rho > 1$ for Set Cover, where $n$ is the number of
  elements in the universe in the Set Cover instance. Using
  a result by Feige~\cite{Feige98}, the existence of an
  $1/rho \ln n$ approximation algorithm for some constant
  $\rho > 1$ implies $\NP \subseteq \DTIME(n^{O(\log\log
    n)})$.

  Given a Set Cover instance with a universe $\calU = \{e_j
  \suchthat j=1,\ldots, n\}$ of elements and a family of
  sets $\calS = \{S_i, i=1,\ldots,m\}$ with every set $S_i
  \in \calS$ being a subset of $\calU$. Then the proof
  proceeds in iterations. In each iteration we construct a
  metric UFL instance with the set of facilities $\sitesset$
  corresponding to the set $\calS$, and we have one client
  $j$ for each uncovered element $e_j$ in $\calU$. The
  distances are defined as $1$ if $e_j \in S_i$, and $3$
  otherwise. We then run the supposed $\rho$-approximation
  algorithm for the constructed UFL instance. Our
  construction ensures that, if the UFL solution does not
  cover a large portion of the clients, then the ratio
  $\alpha$ between the UFL solution and an optimal integral
  solution for the UFL instance must be at least
  $1.463$. The other case is that in all iterations we have
  an UFL solution that covers a large portion of clients,
  and this gives us a solution for the given Set Cover
  instance with no more than $1/\rho \ln n$ times of an
  optimal solution for some $\rho > 1$, which then implies
  $\NP \subseteq \DTIME(n^{O(\log\log n)})$.

  Let $k$ be the number of sets in an optimal solution for
  the Set Cover instance. Note that we can run the following
  for every $k=1,\ldots,n$ so we can assume we know $k$.  We
  now give the construction of the UFL instance. In
  iteration $t$, suppose we begin with $n_t$ elements
  uncovered. Then we have a UFL instance with $|\sitesset| =
  m, |\clientset| = n_t$, $d_{ij} = 1 \text{ or } 3$ and
  $f_i$ to be specified but all $i\in\sitesset$ have equal
  $f_i$, although $f_i$ is set to a different value in each
  iteration. Now suppose the UFL algorithm chooses $\beta_t
  k$ facilities and covers $\gamma_t n_t$ clients, the rest
  $(1-\gamma_t) n_t$ clients are then servied at a distance
  of $3$. The cost of this UFL solution is
\begin{equation*}
  \ALG_{\smallUFL} = \beta_t k f_i + \gamma_t n_t \cdot 1 + (1 -
  \gamma_t) n_t \cdot 3.
\end{equation*}
We set $f_i = c\,n_t / k$ for some constant $c$. Notice that
the same constant $c$ is used for all iterations and we
shall use $c$ to optimize our analysis on the lower bound of
approximation ratio of any UFL algorithms. The cost of the
UFL solution now becomes
\begin{equation*}
  \beta_t c n_t + \gamma_t n_t + 3 (1-\gamma_t) n_t = n_t
  (c\beta_t + 3 - 2\gamma_t).
\end{equation*}

On the other hand, we know that there exists a solution with
$k$ facilities that covers all clients at distance $1$,
which corresponds to an optimal Set Cover solution. So an
optimal UFL solution has cost no more than
\begin{equation*}
  \OPT_{\smallUFL} \leq k f_i + n_t = c n_t + n_t = (1+c)n_t.
\end{equation*}

Since we are running an $\alpha$-approximation algorithm for
the UFL instance, we have
\begin{equation*}
  \alpha \geq \frac{\ALG_{\smallUFL}}{\OPT_{\smallUFL}} \geq \frac{n_t
  (c\beta_t + 3 - 2\gamma_t)}{(1+c)n_t} = \frac{\beta_t + 3 -
2\gamma_t}{1+c}.
\end{equation*}

Let $\rho > 1$ be some fixed constant. Now we have two
cases:

\mycase{1} There exists some iteration $t$ such that $\gamma_t <
1 - e^{-\rho\beta_t}$. Then we have
\begin{equation*}
  \alpha > \frac{c\beta_t + 3 - 2 (1 - e^{-\rho\beta_t})}{1+c} =
  \frac{c\beta_t + 1 + 2 e^{-\rho\beta_t}}{1+c}.
\end{equation*}
Because we have no control over $\beta_t$, we set $\beta_t$
so that the right hand side is minimized, and this would
surely be a lower bound on $\alpha$. Fix $c$ and $\rho$, and
view the right hand side as a function of a single variable
of $\beta_t$. We then choose $\beta_t$ to minimize the right
hand side. By setting the derivative with respect to
$\beta_t$ to zero, we have $\beta_t = (1/\rho)\ln
(2\rho/c)$. Therefore
\begin{equation*}
  \alpha > \frac{c/\rho \ln (2\rho/c) + 1 + c/\rho}{1 +
    c}.
\end{equation*}
Now we deal with $\rho$. For any fixed $c > 0$, we observe
that the right hand side is a decreasing function of $\rho >
1$. So the lower bound on $\alpha$ is tightest when $\rho$
is close to $1$. By taking $\rho$ arbitrarily close to $1$
(although $\rho$ remains a constant strictly greater than
1), we have $\alpha > \frac{c\ln (2/c) + 1 + c}{1+c}$. Since
we have the choice of $c$, we choose $c$ to maximize the
right hand side of the above, by taking the derivative of
the right hand side with respect to $c$ and set it to
zero. This will give us as good a lower bound on $\alpha$ as
possible. We thus have $c$ being the solution of the
equation $\ln (2/c) = 1+c$ and solve for
$c=0.463$. Substitute the value $c$ back, we have $\alpha >
1.463$ in this case.

\mycase{2} The other case is that we have $\gamma_t \geq
1-e^{-\rho\beta_t}$ for every iteration $t$. Suppose we have $l$
iterations, we then have
\begin{equation*}
  n (1-\gamma_1) (1-\gamma_2) \ldots (1-\gamma_l) = 1.
\end{equation*}
Applying the assumed inequality $\gamma_t \geq 1 - e^{-\rho\beta_t}$,
we have
\begin{equation*}
  n \prod_{t=1}^l e^{-\rho\beta_t} \geq 1,
\end{equation*}
which is
\begin{equation*}
  e^{-\sum_{t=1}^l \rho\beta_t} \geq 1/n,
\end{equation*}
which is
\begin{equation*}
  \sum_{t=1}^l \rho\beta_t \leq \ln n.
\end{equation*}
Notice that we have a Set Cover solution with $(\sum_{t=1}^l
\beta_t)k$ sets, so this Set Cover solution has cost no more
than $\sum_{t=1}^l \beta_t$ times $k$, where $k$ is the
number of sets in an optimal Set Cover solution. So we have
a Set Cover algorithm with approximation ratio no more than
$\sum_{t=1}^l \beta_t k / k \leq 1/\rho \ln n$ for some
$\rho > 1$. Using Feige's result we have that metric UFL
cannot have approximation algorithm with ratio less than
$1.463$ unless $\NP \subseteq \DTIME(n^{O(\log\log n)})$.
\end{proof}


Using an observation by Sviridenko, the underlying
assumption can be strengthened to $\PP \neq \NP$. That is,
metric UFL cannot have a polynomial time algorithm with
ratio less than $1.463$ unless $\PP = \NP$. With that, we
conclude the discussion on the hardness results for the
Uncapaciated Facility Location problem ({\UFL}), with the
implication that all these hardness results apply to our
problem, the Fault-tolerant Facility Placement problem
(\FTFP) as well.
%% end of ch1

%% ch2 related work and results summary
\chapter{Related Work and Known Results} \label{ch: related_work}

In designing algorithms for the facility location problems
(FL), which encompass all three problems addressed in the
following, the Uncapacitated Facility Location problem
(UFL), the Fault-tolerant Facility Location problem (FTFL),
and the Fault-tolerant Facility Placement problem (FTFP), we
have two competing goals: On one hand we want to open as few
facilities as possible so that our facility cost is small;
on the other hand we need as many facilities as possible so
that every client can connect to a nearby facility.

Some simple observations about the UFL problem, and apply to
other FL problems with minor adjustments:
\begin{observation}
  If all facility opening cost are zero, then an optimal
  solution is to open all facilities and connect each client
  to the nearest facility.
\end{observation}

\begin{observation}
  If all distances are zero, that is, all facilities and
  clients are colocated at the same point, then an optimal
  solution is to open the cheapest facility and connect all
  clients to that facility.
\end{observation}

\begin{observation}
  For a fixed set of open facilities, there is a polynomial
  algorithm to find the optimal assignment from clients to
  those facilities.
\end{observation}
\begin{proof}
  For UFL, it is a simple matter of finding the nearest open
  facility for each client and a brute force search only
  takes time $O(|\sitesset|\cdot |\clientset|)$. Even for
  more complicated variants where the facilities each has a
  capacity, the optimal assignment can be found by using the
  mincost flow algorithm.
\end{proof}

Next we review the known algorithms for UFL and FTFL, since
these two are well studied problems in literature. In
particular, the LP-rounding algorithms for UFL inspired our
approach to the FTFP problem.

\section{Related Work on UFL}
The Uncapacitated Facility Location problem (UFL) is the
simpliest variant of the Facility Location problems (FL),
and has received the most attention in research
community. One thing that is surprising about UFL is that
almost all known approximation algorithm design techniques
can be applied to UFL with a good approximation ratio. The
first $O(1)$-approximation algorithm was obtained by Shmoys,
Tardos and Aardal~\cite{ShmoysTA97}, with a ratio of $3.16$,
using LP-rounding. Following that, more sophisticated
rounding algorithms have been proposed with improved
approximation ratios. Significant improvement comes from
randomized rounding and Chudak and Shmoys~\cite{ChudakS04},
and Sviridenko~\cite{Svi02} achieved a raio of $1.736$ and
$1.582$ respectively. The LP-rounding algorithms require
solving the LP. More efficient algorithms that are
combinatorial have been proposed as well. Jain and
Vazirani~\cite{JainV01} introduced a primal-dual algorithm
which grows a feasible dual solution and updates a primal
solution accordingly until the primal solution is
feasible. The approximation ratio is obtained via a relaxed
version of the complementary slackness conditions. Two
greedy algorithms that repeatedly picking the star with
minimum average cost, where a star is a facility and a
subset of clients, has been analyzed by Jain, Markakis,
Mahdian, Saberi and Vazirani~\cite{JainMMSV03} using the
dual-fitting technique. They showed that the cost of the
integral solution is equal to the sum of dual variable
values produced by the greedy algorithm, and the dual
variable values, after shrinking by a common factor, is
feasible to the dual program. That common factor is then the
desired approximation ratio. They showed their algorithms
have approximation ratio $1.861$ and $1.61$
respectively. Another approach is local search, in which we
start with a feasible integral solution and make local moves
to improve the solution, and stop at some local
optimum. Arya {\etal}~\cite{AryaGKMMP01} showed that local
search gives a ratio of $3$ for the UFL problem.

Given that the total cost of a solution to the {\UFL}
problem consists of two parts, the facility cost and the
connection cost, a notion of bifactor approximation was
introduced in the literation. An algorithm with facility
cost $F_{\smallALG}$ and connection cost $C_{\smallALG}$, is
said to be $(\gamma_f,\gamma_c)$ approximation if, for every
feasible solution {\SOL}, with facility cost $F_{\smallSOL}$ and
connection cost $C_{\smallSOL}$, we have
\begin{equation*}
  F_{\smallALG} + C_{\smallALG} \leq \gamma_f F_{\smallSOL} +
  \gamma_c C_{\smallSOL}.
\end{equation*}
In particular, for the optimal fractional solution
$(\bfx^\ast,\bfy^\ast)$ with facility cost $F^\ast =
\sum_{i\in\sitesset} f_i y_i^\ast$ and connection cost
$C^\ast = \sum_{j\in\clientset} d_{ij} x_{ij}^\ast$, we have
\begin{equation*}
  F_{\smallALG} + C_{\smallALG} \leq \gamma_f F^\ast +
  \gamma_c C^\ast.
\end{equation*}

The notion of bifactor approximation is helpful when an
algorithm has imbalanced factors $\gamma_f$ and
$\gamma_c$. It is easy to see such an algorithm has
approximation ratio $\max\{\gamma_f, \gamma_c\}$. However,
more can be said, as there are techniques like cost scaling
and greedy augmentation to balance the two factors, that is,
to decrease one at the expense of increasing the other, and
achieve a better overall approximation ratio. The techniques
are introduced by Guha, Khuller and Charikar~\cite{GuhaK98,
  CharikarG05}.

We now present a more detailed description on the
LP-rounding approaches, as our results on {\FTFP} are built
on the work of LP-rounding for {\UFL}.

Every LP-rounding algorithm for {\UFL} starts with solving
the LP to obtain an optimal fraction solution
$(\bfx^\ast,\bfy^\ast)$. Then we need to round the
fractional solution to an integral solution
$(\hat\bfx,\hat\bfy)$ without increasing the cost by
much. As hinted in the above discussion, an integral
solution with small cost would have each client connect to a
nearby facility and few facilities open. Consider a client
$j$, to get a handle on the connection cost, we would like
$j$ to connect to some neighboring facility $i\in N(j)$,
where $N(j) = \{i\in\sitesset \suchthat x_{ij}^\ast >
0\}$. However, it is in general not possible to have every
client connect to a neighboring facility, or we would open
too many facilities, and thus incur a high facility cost. An
alternative is to select a subset of clients, denoted as $C'
\subseteq \clientset$ and only require clients in $C'$ have
a neighoring facility open. Clients outside $C'$ are then
connected to a facility via some clients in $C'$. The
connection cost for clients in $\clientset \setminus C'$
are then bounded using the triangle inequality.

Suppose we have chosen a subset $C'$ of client, we then open
one facility for each client $j'$ in $C'$ and $j'$ would
connect to that facility. Denote by $\phi(j)$ the facility
that client $j$ connects to. To bound the facility cost
$F_{\smallALG}$, it suffices to require the clients $j'$ in
$C'$ have disjoint neighborhood $N(j')$, and $f_{\phi(j')}$
be bounded by the average facility cost of $N(j')$, that is,
$f_{\phi(j')} \leq \sum_{i\in N(j')} f_i y_i^\ast$. Call two
clients $j_1$ and $j_2$ \emph{related} if their neighborhood
overlap, that is $N(j_1) \cap N(j_2) \neq \emptyset$. It is
easy to see this relation defines an equivalence class. The
requirement of disjoint neighborhood on the set $C'$
immediately implies that from each equivalance class we can
select at most one client into set $C'$. The chosen client
is then called the \emph{representative} of that class, and
we say other clients in that class are \emph{assigned} to
that representative.

Suppose we do select exactly one representative client from
each equivalance class, and we open one facility, which is
$\phi(j')$ for each representative client $j'$ in $C'$ in a
way that $f_{\phi(j')}$ is no more than the average cost of
facilities in $N(j')$, that is $f_{\phi(j')} \leq \sum_{i\in
  N(j')} f_i y_{i}$. Then we know that the facility cost is
bounded by $\sum_{i\in \sitesset} f_i y_i^\ast = F^\ast$
because the neighborhood of clients in $C'$ are disjoint.

For connection cost, each client $j'$ in $C'$ connects to
the only facility open in $N(j')$ and we have mentioned that
there is a way to bound the distance from $\phi(j')$ to $j'$
when $\phi(j') \in N(j')$. For clients $j$ not in $C'$, we
need to have an open facility to connect to, in case none of
its neighbor opens. One possibility is to use $\phi(j')$, if
$j$ is assigned to $j'$. When that happens, $d_{\phi(j'), j}$
can be bounded by the triangle inequality as
\begin{equation*}
  d_{\phi(j),j} = d_{\phi(j'), j} \leq d_{\phi(j'), j'} + d_{j'j} \leq
  d_{\phi(j'), j'} + d_{i,j'} + d_{i,j} \text{ for all } i
  \in \sitesset.
\end{equation*}
In the above $d_{jj'}$ denotes the shortest distance between
client $j$ and $j'$, that is $d_{jj'} = \min_{i\in\sitesset}
d_{ij'} + d_{ij}$. Now if we can show a bound on both
$d_{\phi(j'),j'}$ and $d_{j'j}$ in terms of some quantity of
client $j$, we can expect to have some bound on
$d_{\phi(j'),j}$.

Since the above holds for any facility $i \in \sitesset$, it
would certainly hold for some facility $i'$ in $N(j') \cap
N(j)$. Such facility $i'$ exists because the way the
relation between clients are defined. We then have a bound
on all three quantities, namely $d_{\phi(j'), j'}, d_{i'j'}$
and $d_{i'j}$ because the corresponding facilities are all
in the clients' neighborhood. Now we only need $d_{\phi(j')
  j'}$ and $d_{i'j'}$ to be small in terms of some bound on
$d_{ij}$. As we have a choice on which representative we
pick in each equivalence class, we pick the representative
to minimize some measure on that client. Let $g(j)$ be some
measure on client $j$ such that for any facility $i$ in
$N'(j)$, we have $d_{ij} \leq g(j)$, where $N'(j)$ is a
subset of facilities that depends on the defition of
$g(j)$. Then our integral solution consists of one facility
from each $N'(j')$. Given $g(j)$ for all clients
$j\in\clientset$, we can choose the representative $j'$ as
the one that minimizes the measure $g(j)$ in its equivalance
class. One consequence is that any client $j$ outside $C'$,
can now have its connection cost being bounded as:
\begin{equation*}
  d_{\phi(j), j} \leq d_{\phi(j'), j'} + d_{i'j'} + d_{i'j}
  \leq g(j') + g(j') + g(j) \leq g(j) + g(j) + g(j) = 3g(j).
\end{equation*}
In other words, given the function $g(j)$ and its associated
subsets of facilities $N'(j)$ for each client $j\in
\clientset$, we are able to round the fractional solution
$(\bfx^\ast, \bfy^\ast)$ into an integral solution such that
$F_{\smallALG} \leq F^\ast$ and $d_{\phi(j), j} \leq 3
g(j)$.

Next we discuss approaches to derive the function
$g(j)$. Two ways are possible: One is to use the dual
solution $(\bfalpha^\ast, \bfbeta^\ast)$, and then the
complementary slackness conditions tell us that $d_{ij} \leq
\alpha_j^\ast$ for every $i \in N(j) = \{i\in\sitesset
\suchthat x_{ij}^\ast > 0\}$. This defines $g(j) =
\alpha_j^\ast$ and $N'(j) = N(j)$. The other way is to trim
the neighborhood of a client by throwing away most distant
neighbors so that the farthest facility in the trimmed
neighborhood is at a distance within some factor of the
average distance $C_j^\ast = \sum_{i\in N(j)} d_{ij}
x_{ij}^\ast$. We now open one facility in $N'(j')$ for each
representative $j' \in C'$ in a way that $f_{\phi(j')}$ is
no more than the average facility cost of facilities in
$N'(j')$, that is $\sum_{i\in N'(j')} f_i y_i / \sum_{i \in
  N'(j')} y_i$. Now we have a bound on $d_{\phi(j),j}$ in
terms of $C_j^\ast$. However, our facility cost is now
bounded by
\begin{equation*}
  F_{\smallALG} \leq \sum_{j'\in C'} \left(\sum_{i\in N'(j')} f_i y_i^\ast /
    \sum_{i\in N'(j')} y_i^\ast\right).
\end{equation*}
To estimate this cost, we assume a lower bound $\gamma$ on
$\sum_{i\in N'(j')} y_i^\ast$ for every client $j' \in C'$.
Our facility cost can then be bounded as
\begin{align*}
  F_{\smallALG} &\leq \sum_{j'\in C'} \left(\sum_{i\in
      N'(j')} f_i y_i^\ast /
    \sum_{i\in N'(j')} y_i^\ast\right) \qquad (\text{choice
    of } \phi(j'))\\
  &\leq \sum_{j'\in C'}
  \sum_{i \in N'(j')} f_i y_i^\ast /\gamma \qquad
  (\text{assumption } \sum_{i\in N'(j')} y_i^\ast \leq \gamma)\\
  &= \frac{1}{\gamma} \sum_{j' \in C'} \sum_{i \in N'(j')}
  f_i y_i^\ast\\
  &\leq \frac{1}{\gamma} \sum_{j' \in C'} \sum_{i \in N(j')}
  f_i y_i^\ast \qquad (\text{replace } N'(j) \text{ with } N(j))\\
  &=
  \frac{1}{\gamma} \sum_{i \in \sitesset} f_i y_i^\ast = F^\ast.\\
\end{align*}
We note that the factor that bounds $g(j)$ by $C_j^\ast$ is
related to $\gamma$ and we denote that factor as
$f(\gamma)$. Then we have $g(j) \leq f(\gamma) C_j^\ast$. It
follows that we can obtain an integral solution with
facility cost no more than $\gamma F^\ast$ and connection
cost no more than $3 \sum_{j\in \clientset} g(j) = 3
\sum_{j\in \clientset} f(\gamma) C_j^\ast = 3 f(\gamma)
C^\ast$. Recall that $F^\ast = \sum_{i\in \sitesset} f_i
y_i^\ast$ and $C^\ast = \sum_{i\in\sitesset, j\in\clientset}
d_{ij} x_{ij}^\ast$ and we also define $\LP^\ast = F^\ast +
C^\ast$. Our rounded solution then has approximation ratio
$\max\{\gamma, 3 f(\gamma)\}$.

For concreteness we now describe a simple rounding by
Chudak~\cite{Chudak98}, and the original approach by Shmoys,
Tardos and Aardal~\cite{ShmoysTA97}, followed by improvement
by Chudak and Shmoys~\cite{ChudakS04} and
Sviridenko~\cite{Svi02}, Byrka~\cite{Byrka07} and
Li~\cite{Li11}.

We start with an result by Chudak which gives an easy
$4$-approximation. The algorithm starts with fractional
optimal solution for the primal $(\bfx^\ast,\bfy^\ast)$ and
the dual $(\bfalpha^\ast, \bfbeta^\ast)$. The algorithm then
repeatedly pick an unconnnected client $j$ with minimum
$\alpha_j^\ast$, open a cheapest facility $i$ in $N(j)$, and
then connect remaining clients to $i$ with a neighborhood
overlapping $N(j)$. Given the above discussion, clearly the
facility cost is no more than $F^\ast$. For connection cost,
we can use $g(j) = \alpha_j^\ast$, and client $j$ is
connected to a facility with distance no more than $3 g(j) =
3 \alpha_j^\ast$. Summarizing we have total cost no more
than $F^\ast + \sum_{j\in\clientset} \alpha_j^\ast = F^\ast
+ 3\,\LP^\ast \leq 4\,\LP^\ast$.

Another $4$-approximation, which is the original approach by
Shmoys, Tardos and Aardal uses the other way to obtain
$g(j)$, by cutting the neighborhood. More precisely, for
each client $j$, they cut at a distance $g(j)$ such that the
accumulated fractional value is at least some constant
$\gamma$, that is
\begin{equation*}
  \sum_{i\in N(j)\suchthat d_{ij} \leq g(j)} x_{ij}^\ast \geq \gamma.
\end{equation*}
An argument similar to Markov's inequality would then show that the
trimmed neighborhood, call it $N'(j)$, has the farthest facility
within a distance of $g(j)$. To bound $g(j)$, we have
\begin{equation*}
  C_j^\ast \stackrel{\text{def}}{=} \sum_{i\in N(j)} d_{ij} x_{ij}^\ast \geq \sum_{i\in
    N(j)\setminus N'(j)} d_{ij} x_{ij}^\ast \geq \sum_{i \in N(j)
    \setminus N'(j)} g(j) x_{ij}^\ast \geq g(j) (1 - \gamma).
\end{equation*}
As a result $g(j) \leq C_j^\ast / (1 - \gamma)$ and notice
that $\sum_{j\in \clientset} C_j^\ast = C^\ast$. As
discussed above, we have total connection cost no more than
$3C^\ast/(1-\gamma)$. On the facility cost, since we have
limited ourselves only facilities in $N'(j)$, as opposed to
$N(j)$. If we open the cheapest facility in $N'(j)$ for each
chosen client $j$, all we can say now is that the facility
cost is not more than
\begin{equation*}
  \sum_{j \in P} \sum_{i\in N'(j)} f_i y_i^\ast / \sum_{i\in N'(j)} y_i^\ast.
\end{equation*}
We are blessed since we have an lower bound of $\gamma$ on
$\sum_{i\in N'(j)} y_i^\ast$ for every client $j$. As shown
earlier, the above sum can be upper bounded by
\begin{equation*}
  \sum_{j \in P} \sum_{i\in N'(j)} f_i y_i^\ast / \gamma \leq
  \frac{1}{\gamma} \sum_{j \in P} \sum_{i \in N(j)} f_i y_i^\ast \leq
  \frac{1}{\gamma} \sum_{i \in sitesset} f_i y_i^\ast =
  \frac{1}{\gamma} F^\ast.
\end{equation*}
So our approximation ratio is $\max\{3/(1-\gamma),
1/\gamma\}$. Pick $\gamma=1/4$ and we have a
$4$-approximation.

The simpliest rounding described so far is clearly not the
best possible and there are several improvements proposed
afterwards, nonetheless, they all follow the same
equivalence class partition and opens exactly one facility
for each representative. To save on connection cost, one
observes that a not chosen client can use a neighboring
facility provided that the neighbor is open. For the
interest of facility cost, it is not necessary to open the
cheapest, by opening each facility in a neighborhood with
probability proportional to $y_i^\ast$, we would have an
expectation of facility cost bounded by $F^\ast$ as
well. The analysis, however, is much more involved and
requires one to estimate the probabilities of a not chosen
client connects to its neighbor and to the facility opened
by its assigned representative client. We shall not
elaborate here as our rounding algorithms for the FTFP
problem in the main part will rephrase and prove similar
theorems later. This concludes our review on the known
approximation results on the Uncapacitated Facility Location
problem (UFL).


\section{Related Work on FTFL}
The fault-tolerant facility location problem (\FTFL), was
first introduced by Jain and Vazirani~\cite{JainV03} and
they adapted their primal-dual algorithm for UFL to {\FTFL}
to obtain a ratio of $3\ln R$ where $R=\max_j r_j$ is the
maximum demand among all clients. The first constant
approximation algorithm was given by Guha, Meyerson and
Munagala~\cite{GuhaMM03}, using LP-rounding similar to the
Shmoys' \etal's approach for the {\UFL} problem. Subsequent
improvement were made by Swamy and Shmoys~\cite{SwamyS08}
using pipage rounding with a ratio of $2.076$. The current
best known approximation ratio is $1.7245$, by Byrka,
Srinivasan and Swamy~\cite{ByrkaSS10}, using dependent
rounding with a laminar clustering structure.

We note that all the known $O(1)$-approximation algorithms
for {\FTFL} are LP-rounding algorithms and they need to
solve the LP as a first step. Given the success of
primal-dual based approaches for {\UFL}, it is natural to
ask whether such algorithms could be adapted to {\FTFL} with
a good ratio. To the best of the author's knowledge, it is
not known whether there is a primal-dual algorithm for
{\FTFL} with a sub-logarithmic approximation ratio. We shall
have more to say about this in Chapter~\ref{ch:
  primal-dual}.

\section{Our Problem: FTFP}
Our problem, the fault-tolerant facility placement problem
(\FTFP), was introduced by Xu and
Shen~\cite{XuS09}~\footnote{In their paper they call the
  problem the fault-tolerant facility allocation problem, or
  FTFA.}. The study of {\FTFP} was partly motivated to
obtain a better understanding of the implication of
fault-tolerant requirement on facility location
problems. The Xu and Shen's results seem to be valid only
for a special case of {\FTFP} and we later adapted the
Chudak's 4-approximation algorithm for {\UFL} to {\FTFP},
thus obtaining the first $O(1)$-approximation algorithm for
this problem. The algorithm was based on LP-rounding. Our
subsequent work improved the algorithm and analysis on
LP-rounding and our most recent results demonstrated that
LP-rounding algorithms for {\FTFP} can achieve an
approximation ratio that matches that for {\UFL}. For the
applicability of primal-dual techniques on {\FTFP}, we
provide an argument of possible difficulty in obtaining
sub-logarithmic ratio using dual-fitting approach. Again we
refer the reader to Chapter~\ref{ch: primal-dual} for more
discussion on results using primal-dual approaches on
{\FTFP}.

%% linear program
\chapter{Linear Program} \label{ch:lp} 

In all our algorithms, we use the standard linear
program~\ref{eqn:fac_primal} and its
dual~\ref{eqn:fac_dual}, first formulated by Balinski, to
develop our algorithms and estimate the optimal solution
value. For readers unfamiliar with Linear Programming (LP)
and Integer Programming (IP), we included a short
introductory section in the appendix.

The FTFP problem has a natural Integer Programming (IP)
formulation. Let $y_i$ represent the number of facilities
built at site $i$ and let $x_{ij}$ represent the number of
connections from client $j$ to facilities at site $i$. If we
relax the integrality constraints, we obtain the following LP:

%%%%%%%%%%%
\begin{alignat}{3}
  \textrm{minimize} \quad \cost(\bfx,\bfy) &= \textstyle{\sum_{i\in \sitesset}f_iy_i 
								+ \sum_{i\in \sitesset, j\in \clientset}d_{ij}x_{ij}}\label{eqn:fac_primal}\hspace{-1.5in}&&
									\\ \notag
  \textrm{subject to}\quad y_i - x_{ij} &\geq 0 			&\quad 		&\forall i\in \sitesset, j\in \clientset 
									\\ \notag
     \textstyle{\sum_{i\in \sitesset} x_{ij}} &\geq r_j  &			&\forall j\in \clientset
 									\\ \notag
  	  x_{ij} \geq 0, y_i &\geq 0 						& 			&\forall i\in \sitesset, j\in \clientset 
  									\\ \notag
\end{alignat}

%%%%%%%%%%%%

\noindent
The dual program is:

\begin{alignat}{3}
  \textrm{maximize}\quad \textstyle{\sum_{j\in \clientset}} r_j\alpha_j&\label{eqn:fac_dual}  
     						\\ \notag
  \textrm{subject to} \quad \textstyle{
    \sum_{j\in \clientset}\beta_{ij}} &\leq f_i  &\quad\quad			&\forall i \in \sitesset  
							\\ \notag
  \alpha_{j} - \beta_{ij} 	&\leq  d_{ij}       &                 & \forall i\in \sitesset, j\in \clientset 
							\\ \notag
  \alpha_j \geq 0, \beta_{ij} &\geq 0           &            & \forall i\in \sitesset, j\in \clientset
  							\\ \notag
\end{alignat}

In each of our algorithms we will fix some optimal
solutions of the LPs (\ref{eqn:fac_primal}) and (\ref{eqn:fac_dual})
that we will denote by $(\bfx^\ast, \bfy^\ast)$ and
$(\bfalpha^\ast,\bfbeta^\ast)$, respectively.

With $(\bfx^\ast, \bfy^\ast)$ fixed, we can define the
optimal facility cost as $F^\ast=\sum_{i\in\sitesset} f_i
y_i^\ast$ and the optimal connection cost as $C^\ast =
\sum_{i\in\sitesset,j\in\clientset} d_{ij}x_{ij}^\ast$.
Then $\LP^\ast = \cost(\bfx^\ast,\bfy^\ast) = F^\ast+C^\ast$
is the joint optimal value of (\ref{eqn:fac_primal}) and
(\ref{eqn:fac_dual}).  We can also associate with each
client $j$ its fractional connection cost $C^\ast_j =
\sum_{i\in\sitesset} d_{ij}x_{ij}^\ast$.  Clearly, $C^\ast =
\sum_{j\in\clientset} C^\ast_j$.  Throughout the paper we
will use notation $\OPT$ for the optimal integral solution
of (\ref{eqn:fac_primal}).  $\OPT$ is the value we wish to
approximate, but, since $\OPT\ge\LP^\ast$, we can instead use
$\LP^\ast$ to estimate the approximation ratio of our
algorithms.

%%%%%%%%%

\paragraph{Completeness and facility splitting.}
Define $(\bfx^\ast, \bfy^\ast)$ to be \emph{complete} if
$x_{ij}^\ast>0$ implies that $x_{ij}^\ast=y_i^\ast$ for all $i,j$. In
other words, each connection either uses a site fully or not at all.
As shown by Chudak and Shmoys~\cite{ChudakS04}, we can modify the
given instance by adding at most $|\clientset|$ sites to obtain an
equivalent instance that has a complete optimal solution, where
``equivalent" means that the values of $F^\ast$, $C^\ast$ and
$\LP^\ast$, as well as $\OPT$, are not affected. Roughly, the argument
is this: We notice that, without loss of generality, for each client
$k$ there exists at most one site $i$ such that $0 < x_{ik}^\ast <
y_i^\ast$.  We can then perform the following \emph{facility
  splitting} operation on $i$: introduce a new site $i'$, let
$y^\ast_{i'} = y^\ast_i - x^\ast_{ik}$, redefine $y^\ast_i$ to be
$x^\ast_{ik}$, and then for each client $j$ redistribute $x^\ast_{ij}$
so that $i$ retains as much connection value as possible and $i'$
receives the rest. Specifically, we set
%
\begin{align*}
  &y^\ast_{i'} \;\assign\; y^\ast_i - x^\ast_{ik},\;   y^\ast_{i} \;\assign\; x^\ast_{ik}, \quad \text{ and }\\
  &x^\ast_{i'j} \;\assign\;\max( x^\ast_{ij} - x^\ast_{ik}, 0 ),\;	 x^\ast_{ij} \;\assign\; \min( x^\ast_{ij} , x^\ast_{ik}) 
			\quad	\textrm{for all}\ j \neq k.
\end{align*}
%
This operation eliminates the partial connection between $k$
and $i$ and does not create any new partial
connections. Each client can split at most one site and
hence we shall have at most $|\clientset|$ more sites.

By the above paragraph,  without loss of generality we can
assume that the optimal fractional solution $(\bfx^\ast, \bfy^\ast)$
is complete. This assumption will in fact greatly simplify some of
the arguments in the paper. Additionally, we will frequently use the facility
splitting operation described above in our algorithms to obtain fractional solutions with
desirable properties.


%% ch3 techniques
\chapter{Techniques} \label{ch: techniques} 

After obtaining an optimal fractional solution to
LP(\ref{eqn:fac_primal}) and (\ref{eqn:fac_dual}), we employ
two techniques to obtain approximation results on the FTFP
problem. Our first technique, which we call \emph{demand
  reduction}, allow us to restrict our attention to a
restricted version of the FTFP problem, in which all demands
$r_j$ are not too large. This restriction then allow the
application of our next technique, \emph{adaptive
  partition}, so that we obtain an FTFP instance with
facilities and unit demand points. For this FTFP instance,
each facility can be either open or close, and each unit
demand point connects to one of the open facilities. We
would like to point out that we still need to cater the
fault-tolerant requirement, that is, unit demands originated
from the same client must connect to different
facilities. We shall see that our adaptive partitioning step
takes care of the fault-tolerant requirement smoothly.

\section{Demand Reduction}
\label{sec: polynomial demands}

This section presents a \emph{demand reduction} trick that
reduces the problem for arbitrary demands to a special case
where demands are bounded by $|\sitesset|$, the number of
sites.  (The formal statement is a little more technical --
see Theorem~\ref{thm: reduction to polynomial}.)  Our
algorithms in the sections that follow process individual
demands of each client one by one, and thus they critically
rely on the demands being bounded polynomially in terms of
$|\sitesset|$ and $|\clientset|$ to keep the overall running time polynomial.

The reduction is based on an optimal fractional solution
$(\bfx^\ast,\bfy^\ast)$ of LP~(\ref{eqn:fac_primal}). From
the optimality of this solution, we can also assume that
$\sum_{i\in\sitesset} x^\ast_{ij} = r_j$ for all
$j\in\clientset$.  As explained in Section~\ref{ch:lp}, we
can assume that $(\bfx^\ast,\bfy^\ast)$ is complete, that is
$x^\ast_{ij} > 0$ implies $x^\ast_{ij} = y^\ast_i$ for all
$i,j$.  We split this solution into two parts, namely
$(\bfx^\ast,\bfy^\ast) = (\hatbfx,\hatbfy)+
(\dotbfx,\dotbfy)$, where
%
\begin{align*}
\haty_i &\;\assign\; \floor{y_i^\ast}, \quad
			\hatx_{ij} \;\assign\; \floor{x_{ij}^\ast} \quad\textrm{and}
			\\
\doty_i &\;\assign\; y_i^\ast - \floor{y_i^\ast}, \quad
 	\dotx_{ij} \;\assign\; x_{ij}^\ast -  \floor{x_{ij}^\ast}
\end{align*}
%
for all $i,j$. Now we construct two
FTFP instances $\hatcalI$ and $\dotcalI$ with the same
parameters as the original instance, except that the demand of each client $j$ is
$\hatr_j = \sum_{i\in\sitesset} \hatx_{ij}$ in instance $\hatcalI$ and
$\dotr_j = \sum_{i\in\sitesset} \dotx_{ij} = r_j - \hatr_j$ in instance $\dotcalI$. 
It is obvious that if we have integral solutions to both $\hatcalI$
and $\dotcalI$ then, when added together, they form an integral
solution to the original instance.  Moreover, we have the
following lemma.

%%%%%%%%%%

\begin{lemma}\label{lem: polynomial demands partition}
{\rm (i)}
  $(\hatbfx, \hatbfy)$ is a feasible integral solution to
  instance $\hatcalI$.

\noindent
{\rm (ii)}
  $(\dotbfx, \dotbfy)$ is a feasible fractional
  solution to instance $\dotcalI$.

\noindent
{\rm (iii)}
$\dotr_j\leq |\sitesset|$ for every client $j$.

\end{lemma}

\begin{proof}
(i) For feasibility, we need to verify that the constraints of LP~(\ref{eqn:fac_primal})
are satisfied. Directly from the definition, we have $\hatr_j = \sum_{i\in\sitesset} \hatx_{ij}$.
For any $i$ and $j$, by the feasibility of $(\bfx^\ast,\bfy^\ast)$ we have
$\hatx_{ij} = \floor{x_{ij}^\ast} \le \floor{y^\ast_i} = \haty_i$.

(ii) From the definition, we have  $\dotr_j = \sum_{i\in\sitesset} \dotx_{ij}$.
It remains to show that $\doty_i \geq \dotx_{ij}$ for all $i,j$. 
If $x_{ij}^\ast=0$, then $\dotx_{ij}=0$ and we are done. 
Otherwise, by completeness, we have $x_{ij}^\ast=y_i^\ast$. 
Then  $\doty_i = y_i^\ast - \floor{y_i^\ast} = x_{ij}^\ast - \floor{x_{ij}^\ast} =\dotx_{ij}$. 

(iii) From the definition of $\dotx_{ij}$ we have
  $\dotx_{ij} < 1$.  Then the bound follows from the definition of $\dotr_j$.
\end{proof}

Notice that our construction relies on the completeness assumption; in fact, it is
easy to give an example where $(\dotbfx, \dotbfy)$ would not be feasible if we
used a non-complete optimal solution $(\bfx^\ast,\bfy^\ast)$.
Note also that the solutions $(\hatbfx,\hatbfy)$ and $(\dotbfx, \dotbfy)$ are in fact
optimal for their corresponding instances, for if a better solution to $\hatcalI$ or
$\dotcalI$ existed, it could
give us a solution to $\calI$ with a smaller objective value.

%%%%%%%%%%%%%%%

\begin{theorem}\label{thm: reduction to polynomial}
  Suppose that there is a polynomial-time algorithm $\calA$
  that, for any instance of {\FTFP} with maximum demand
  bounded by $|\sitesset|$, computes an integral solution
  that approximates the fractional optimum of this instance
  within factor $\rho\geq 1$.  Then there is a
  $\rho$-approximation algorithm $\calA'$ for {\FTFP}.
\end{theorem}

%%%%%%%%%%%%%%%

\begin{proof}
  Given an {\FTFP} instance with arbitrary demands, Algorithm~$\calA'$ works
as follows: it solves the LP~(\ref{eqn:fac_primal}) to obtain a
  fractional optimal solution $(\bfx^\ast,\bfy^\ast)$, then it constructs
  instances $\hatcalI$ and $\dotcalI$ described above,  applies
  algorithm~$\calA$ to $\dotcalI$, and finally combines (by adding
  the values) the integral solution $(\hatbfx, \hatbfy)$ of
  $\hatcalI$ and the integral solution of $\dotcalI$ produced
  by $\calA$. This clearly produces a feasible integral
  solution for the original instance $\calI$.
The solution produced by $\calA$ has cost at most
$\rho\cdot\cost(\dotbfx,\dotbfy)$, because $(\dotbfx,\dotbfy)$
is feasible for $\dotcalI$. Thus the cost of $\calA'$ is at most
% 
 \begin{align*}
 \cost(\hatbfx, \hatbfy) + \rho\cdot\cost(\dotbfx,\dotbfy)
	\le
 \rho(\cost(\hatbfx, \hatbfy) + \cost(\dotbfx,\dotbfy))
		= \rho\cdot\LP^\ast \le \rho\cdot\OPT,
  \end{align*}
%
where the first inequality follows from $\rho\geq 1$. This completes
the proof.
\end{proof}

%%%%%%%%%%%%%%%% TODO: expand this %%%%%%%%%%%%%%%%%%
Two nice consequences are immediate, given demand reduction.
\subsection{Reduction from FTFP to FTFL}
Given demand reduction, we may assume that we are working
with a restricted version of FTFP where every demand $r_j$
is no more than $|\sitesset|$. In this case we can reduce
this version of FTFP into FTFL. The reduction simply creates
$|\sitesset|$ facilities at each site, and every such
facility may be open or close later. Then we have an FTFL
instance where every client have a demand $r_j$ and every
facility could be open or close. Then any FTFL rounding
algorithm can be applied to solve this FTFL instance, and
the solution trivially maps into a solution for the
corresponding FTFP instance. Moreover, the approximation
ratio for FTFL is preserved for FTFP. Given that FTFL has a
$1.7245$-approximation algorithm by Byrka, Srinivasan and
Swamy~\cite{ByrkaSS10}, it is easy to see that FTFP has an
approximation algorithm with the same ratio.

\subsection{Asymptotic Approximation Ratio for Large
  Demands}
When all demands are large, say $R_{\min} =
\omega(|\sitesset|)$, where $R_{\min} \defeq
\min_{j\in\clientset} r_j$ is the minimum demand among all
clients, then after we apply demand reduction, the integral
part has a cost that dominates the overall cost. Since we
have a ratio $1$ for the integral part, we have a ratio
asymptotically close to $1$ in the overall cost.


\section{Adaptive Partition}
\label{sec: adaptive partitioning}

In this section we develop our second technique, which we
call \emph{adaptive partitioning}. Given an FTFP instance
and an optimal fractional solution $(\bfx^\ast, \bfy^\ast)$
to LP~(\ref{eqn:fac_primal}), we split each client $j$ into
$r_j$ individual \emph{unit demand points} (or just
\emph{demands}), and we split each site $i$ into no more
than $|\sitesset|+2R|\clientset|^2$ \emph{facility points} (or
\emph{facilities}), where $R=\max_{j\in\clientset}r_j$. We
denote the demand set by $\demandset$ and the facility set
by $\facilityset$, respectively.  We will also partition
$(\bfx^\ast,\bfy^\ast)$ into a fractional solution
$(\barbfx,\barbfy)$ for the split instance.  We will
typically use symbols $\nu$ and $\mu$ to index demands and
facilities respectively, that is $\barbfx =
(\barx_{\mu\nu})$ and $\barbfy = (\bary_{\mu})$.  As before,
the \emph{neighborhood of a demand} $\nu$ is
$\wbarN(\nu)=\braced{\mu\in\facilityset \suchthat
  \barx_{\mu\nu}>0}$.  We will use notation $\nu\in j$ to
mean that $\nu$ is a demand of client $j$; similarly,
$\mu\in i$ means that facility $\mu$ is on site
$i$. Different demands of the same client (that is,
$\nu,\nu'\in j$) are called \emph{siblings}.  Further, we
use the convention that $f_\mu = f_i$ for $\mu\in i$,
$\alpha_\nu^\ast = \alpha_j^\ast$ for $\nu\in j$ and
$d_{\mu\nu} = d_{\mu j} = d_{ij}$ for $\mu\in i$ and $\nu\in
j$.  We define $\concost_{\nu}
=\sum_{\mu\in\wbarN(\nu)}d_{\mu\nu}\barx_{\mu\nu} =
\sum_{\mu\in\facilityset}d_{\mu\nu}\barx_{\mu\nu}$. 
One can think of $\concost_{\nu}$ as the
average connection cost of demand $\nu$, if we chose a
connection to facility $\mu$ with probability
$\barx_{\mu\nu}$. In our partitioned fractional solution we
guarantee for every $\nu$ that $\sum_{\mu\in\facilityset}
\barx_{\mu\nu}=1$.

Some demands in $\demandset$ will be designated as
\emph{primary demands} and the set of primary demands will
be denoted by $P$. By definition we have $P\subseteq \demandset$.
 In addition, we will use the overlap
structure between demand neighborhoods to define a mapping
that assigns each demand $\nu\in\demandset$ to some primary
demand $\kappa\in P$. As shown in the rounding algorithms in
later sections, for each primary demand we guarantee exactly
one open facility in its neighborhood, while for a
non-primary demand, there is constant probability that none
of its neighbors open. In this case we estimate its
connection cost by the distance to the facility opened in
its assigned primary demand's neighborhood. For this reason
the connection cost of a primary demand must be ``small''
compared to the non-primary demands assigned to it. We also
need sibling demands assigned to different primary demands to satisfy
the fault-tolerance requirement. Specifically, this
partitioning will be constructed to satisfy a number of
properties that are detailed below.
%
\begin{description}
	
      \renewcommand{\theenumii}{(\alph{enumii})}
      \renewcommand{\labelenumii}{\theenumii}

\item{(PS)} \emph{Partitioned solution}.
Vector $(\barbfx,\barbfy)$ is a partition of $(\bfx^\ast,\bfy^\ast)$, with unit-value
  demands, that is:

	\begin{enumerate}
		%
	\item \label{PS:one} 
          $\sum_{\mu\in\facilityset} \barx_{\mu\nu} = 1$ for each demand $\nu\in\demandset$. 
		%
	\item \label{PS:xij} $\sum_{\mu\in i, \nu\in j} \barx_{\mu\nu}
          = x^\ast_{ij}$ for each site $i\in\sitesset$ and client $j\in\clientset$.
		%
	\item \label{PS:yi}
          $\sum_{\mu\in i} \bary_{\mu} = y^\ast_i$ for each site $i\in\sitesset$.
		%
	\end{enumerate}
		
\item{(CO)} \emph{Completeness.}
	Solution   $(\barbfx,\barbfy)$ is complete, that is $\barx_{\mu\nu}\neq 0$ implies
				$\barx_{\mu\nu} = \bary_{\mu}$, for all $\mu\in\facilityset, \nu\in\demandset$.

\item{(PD)} \emph{Primary demands.}
	Primary demands satisfy the following conditions:

	\begin{enumerate}
		
	\item\label{PD:disjoint}  For any two different primary demands $\kappa,\kappa'\in P$ we have
				$\wbarN(\kappa)\cap \wbarN(\kappa') = \emptyset$.

	\item \label{PD:yi} For each site $i\in\sitesset$, 
		$ \sum_{\mu\in i}\sum_{\kappa\in P}\barx_{\mu\kappa} \leq y_i^\ast$.
		
	\item \label{PD:assign} Each demand $\nu\in\demandset$ is assigned
        to one primary demand $\kappa\in P$ such that

  			\begin{enumerate}
	
				\item \label{PD:assign:overlap} $\wbarN(\nu) \cap \wbarN(\kappa) \neq \emptyset$, and
				%
				\item \label{PD:assign:cost} $\concost_{\nu}+\alpha_{\nu}^\ast \geq
        			\concost_{\kappa}+\alpha_{\kappa}^\ast$.

			\end{enumerate}

	\end{enumerate}
	
\item{(SI)} \emph{Siblings}. For any pair $\nu,\nu'$ of different siblings we have
  \begin{enumerate}

	\item \label{SI:siblings disjoint}
		  $\wbarN(\nu)\cap \wbarN(\nu') = \emptyset$.
		
	\item \label{SI:primary disjoint} If $\nu$ is assigned to a primary demand $\kappa$ then
 		$\wbarN(\nu')\cap \wbarN(\kappa) = \emptyset$. In particular, by Property~(PD.\ref{PD:assign:overlap}),
		this implies that different sibling demands are assigned to different primary demands.

	\end{enumerate}
	
\end{description}

As we shall demonstrate in later sections, these properties allow us
to extend known UFL rounding algorithms to obtain an integral solution
to our FTFP problem with a matching approximation ratio. Our
partitioning is ``adaptive" in the sense that it is constructed one
demand at a time, and the connection values for the demands of a
client depend on the choice of earlier demands, of this or other
clients, and their connection values. We would like to point out that
the adaptive partitioning process for the $1.575$-approximation
algorithm (Section~\ref{sec: 1.575-approximation}) is more subtle than that for 
the $3$-apprximation (Section~\ref{sec: 3-approximation}) and the
$1.736$-approximation algorithms (Section~\ref{sec:
  1.736-approximation}), due to the introduction of close and far
neighborhood.

%%%%%%%%%%%%%%%%

\paragraph{Implementation of Adaptive Partitioning.}
We now describe an algorithm for partitioning the instance
and the fractional solution so that the properties (PS),
(CO), (PD), and (SI) are satisfied.  Recall that
$\facilityset$ and $\demandset$, respectively, denote the
sets of facilities and demands that will be created in this
stage, and $(\barbfx,\barbfy)$ is the partitioned solution
to be computed. 

The adaptive partitioning algorithm consists of two phases:
Phase 1 is called the partitioning phase and Phase 2 is called
the augmenting phase. Phase 1 is done in iterations, where
in each iteration we find the ``best'' client $j$ and create a
new demand $\nu$ out of it. This demand either becomes a
primary demand itself, or it is assigned to some existing
primary demand. We call a client $j$ \emph{exhausted} when
all its $r_j$ demands have been created and assigned to some
primary demands. Phase 1 completes when all clients are
exhausted. In Phase 2 we ensure that every demand has a
total connection values $\barx_{\mu\nu}$ equal to $1$, that is condition (PS.\ref{PS:one}).

For each site $i$ we will initially create one ``big" facility $\mu$
with initial value $\bary_\mu = y^\ast_i$.  While we partition the
instance, creating new demands and connections, this facility may end
up being split into more facilities to preserve completeness of the
fractional solution. Also, we will gradually decrease the fractional
connection vector for each client $j$, to account for the demands
already created for $j$ and their connection values.  These decreased
connection values will be stored in an auxiliary vector
$\tildebfx$. The intuition is that $\tildebfx$ represents the part of
$\bfx^\ast$ that still has not been allocated to existing demands and
future demands can use $\tildebfx$ for their connections. For
technical reasons, $\tildebfx$ will be indexed by facilities (rather
than sites) and clients, that is $\tildebfx = (\tildex_{\mu j})$.  At
the beginning, we set $\tildex_{\mu j}\assign x_{ij}^\ast$ for each
$j\in\clientset$, where $\mu\in i$ is the single facility created
initially at site $i$.  At each step, whenever we create a new demand
$\nu$ for a client $j$, we will define its values $\barx_{\mu\nu}$ and
appropriately reduce the values $\tildex_{\mu j}$, for all facilities
$\mu$. We will deal with two types of neighborhoods, with respect to
$\tildebfx$ and $\barbfx$, that is $\wtildeN(j)=\{\mu\in\facilityset
\suchthat\tildex_{\mu j} > 0\}$ for $j\in\clientset$ and
$\wbarN(\nu)=\{\mu\in\facilityset \suchthat \barx_{\mu\nu} >0\}$ for
$\nu\in\demandset$.  During this process we preserve the completeness
(CO) of the fractional solutions $\tildebfx$ and $\barbfx$. More
precisely, the following properties will hold for every facility $\mu$
after every iteration:
%
\begin{description}
	
	\item{(c1)} For each demand $\nu$ either $\barx_{\mu\nu}=0$ or
			$\barx_{\mu\nu}=\bary_{\mu}$. This is the same
      condition as condition (CO), yet we repeat it here as
      (c1) needs to hold after every iteration, while
      condition (CO) only applies to the final partitioned
      fractional solution $(\barbfx, \barbfy)$.

	\item{(c2)} For each client $j$,
			either $\tildex_{\mu j}=0$ or $\tildex_{\mu j}=\bary_{\mu}$.
			
\end{description}

A full description of the algorithm is given in
Pseudocode~\ref{alg:lpr2}.  Initially, the set $U$ of
non-exhausted clients contains all clients, the set
$\demandset$ of demands is empty, the set $\facilityset$ of
facilities consists of one facility $\mu$ on each site $i$
with $\bary_\mu = y^\ast_i$, and the set $P$ of primary
demands is empty (Lines 1--4).  In one iteration of the
while loop (Lines 5--8), for each client $j$ we
compute a quantity called $\tcc(j)$ (tentative connection
cost), that represents the average distance from $j$ to the
set $\wtildeN_1(j)$ of the nearest facilities $\mu$ whose
total connection value to $j$ (the sum of $\tildex_{\mu
  j}$'s) equals $1$.  This set is computed by Procedure
$\NearestUnitChunk()$ (see Pseudocode~\ref{alg:helper},
Lines~1--9), which adds facilities to $\wtildeN_1(j)$ in
order of nondecreasing distance, until the total connection
value is exactly $1$. (The procedure actually uses the
$\bary_\mu$ values, which are equal to the connection values,
by the completeness condition (c2).)  This may require splitting the last added
facility and adjusting the connection values so that
conditions (c1) and (c2) are preserved.

%%%%%%%%%%%

\begin{algorithm}[ht]
  \caption{Algorithm: Adaptive Partitioning}
  \label{alg:lpr2}
  \begin{algorithmic}[1]
    \Require $\sitesset$, $\clientset$, $(\bfx^\ast,\bfy^\ast)$
    \Ensure  $\facilityset$,  $\demandset$, $(\barbfx, \barbfy)$ 
    \Comment Unspecified $\barx_{\mu \nu}$'s and $\tildex_{\mu j}$'s are assumed to be $0$

    \State $\tildebfr \assign \bfr, U\assign \clientset, \facilityset\assign \emptyset,
    \demandset\assign \emptyset, P\assign \emptyset$
    \Comment{Phase 1}

    \For{each site $i\in\sitesset$} 
    \State create a facility $\mu$ at $i$ and add $\mu$ to $\facilityset$
    \State $\bary_\mu \assign y_i^\ast$ and $\tildex_{\mu j}\assign
    x_{ij}^\ast$ for each $j\in\clientset$ 
    \EndFor

    \While{$U\neq \emptyset$}
    \For{each $j\in U$}
    \State $\wtildeN_1(j) \assign {\NearestUnitChunk}(j, \facilityset, \tildebfx, \barbfx, \barbfy)$ \Comment see Pseudocode~\ref{alg:helper}
    \State $\tcc(j)\assign \sum_{\mu\in \wtildeN_1(j)} d_{{\mu}j}\cdot \tildex_{\mu j}$
    \EndFor
 
    \State $p \assign {\argmin}_{j\in U}\{ \tcc(j)+\alpha_j^\ast \}$
    \State create a new demand $\nu$ for client $p$

    \If{$\wtildeN_1 (p)\cap \wbarN(\kappa) \neq \emptyset$
      for some primary demand $\kappa\in P$}
    \State assign $\nu$ to $\kappa$
    \State $\barx_{\mu \nu}\assign \tildex_{\mu p}$ and $\tildex_{\mu p}\assign 0$ for each $\mu \in \wtildeN(p) \cap \wbarN(\kappa)$
    \Else 
    \State make $\nu$ primary, $P \assign P \cup \{\nu\}$, assign $\nu$ to itself
    \State set $\barx_{\mu\nu} \assign \tildex_{\mu p}$ and $\tildex_{\mu p}\assign 0$ for each $\mu\in \wtildeN_1(p)$

    \EndIf
    \State $\demandset\assign \demandset\cup \{\nu\},
    \tilder_p \assign \tilder_p -1$
	\State \textbf{if} {$\tilder_p=0$} \textbf{then} $U\assign U \setminus \{p\}$
    \EndWhile

    \For{each client $j\in\clientset$} \Comment{Phase 2}
    \For{each demand $\nu\in j$}    \Comment{each client $j$ has $r_j$ demands}
    \State \textbf{if} $\sum_{\mu\in \wbarN(\nu)}\barx_{\mu\nu}<1$
    \textbf{then} $\AugmentToUnit(\nu, j, \facilityset, \tildebfx, \barbfx, \barbfy)$ \Comment see Pseudocode~\ref{alg:helper}
    \EndFor
    \EndFor
  \end{algorithmic}
\end{algorithm}
%%%%%%%%%%%%%%%%%%%%%%%%%%%%
%% subroutine: NearestUnitChunk and AugmentToUnit
%%%%%%%%%%%%%%%%%%%%%%%%%%%%%
\begin{algorithm}[ht]
  \caption{Helper functions used in Pseudocode~\ref{alg:lpr2}}
  \label{alg:helper}
  \begin{algorithmic}[1]
    \Function{\NearestUnitChunk}{$j, \facilityset, \tildebfx, \barbfx,\barbfy$}		
						\Comment upon return, $\sum_{\mu\in\wtildeN_1(j)} \tildex_{\mu j} = 1$
    \State Let $\wtildeN(j) = \{\mu_1,...,\mu_{q}\}$ where $d_{\mu_1 j} \leq d_{\mu_2 j} \leq \ldots \leq d_{\mu_{q j}}$
    \State Let $l$ be such that $\sum_{k=1}^{l} \bary_{\mu_k} \geq 1$ and $\sum_{k=1}^{l -1} \bary_{\mu_{k}} < 1$
    \State Create a new facility $\sigma$ at the same site as $\mu_l$ and add it to $\facilityset$
			\Comment split $\mu_l$
    \State Set $\bary_{\sigma}\assign \sum_{k=1}^{l} \bary_{\mu_{k}}-1$
					and $\bary_{\mu_l} \assign \bary_{\mu_l} - \bary_{\sigma}$
    \State For each $\nu\in\demandset$ with $\barx_{\mu_{l}\nu}>0$
 			set $\barx_{\mu_{l}\nu} \assign \bary_{\mu_l}$ and $\barx_{\sigma \nu} \assign \bary_{\sigma}$
    \State For each $j'\in\clientset$ with $\tildex_{\mu_{l} j'}>0$ (including $j$)
			set $\tildex_{\mu_l j'} \assign \bary_{\mu_l}$ and $\tildex_{\sigma j'} \assign \bary_\sigma$
	\State (All other new connection values are set to $0$)
    \State \Return $\wtildeN_1(j) = \{\mu_{1},\ldots,\mu_{l-1}, \mu_{l}\}$    				
    \EndFunction

    \Function{\AugmentToUnit}{$\nu, j, \facilityset, \tildebfx, \barbfx, \barbfy$}
    					\Comment $\nu$ is a demand of client $j$
    \While{$\sum_{\mu\in \facilityset} \barx_{\mu\nu} <1$}
    					\Comment upon return, $\sum_{\mu\in\wbarN(\nu)} \barx_{\mu\nu} = 1$
    \State Let $\eta$ be any facility such that $\tildex_{\eta j} > 0$
    \If{$1-\sum_{\mu\in \facilityset} \barx_{\mu\nu} \geq \tildex_{\eta j}$}
    \State $\barx_{\eta\nu} \assign \tildex_{\eta j}, \tildex_{\eta j} \assign 0$
    \Else
    \State Create a new facility $\sigma$ at the same site as $\eta$ and add it to $\facilityset$
    					\Comment split $\eta$
    \State Let $\bary_\sigma \assign 1-\sum_{\mu\in \facilityset} \barx_{\mu\nu}, \bary_{\eta} \assign \bary_{\eta} - \bary_{\sigma}$
    \State Set $\barx_{\sigma\nu}\assign \bary_{\sigma},\; \barx_{\eta \nu} \assign  0,\; \tildex_{\eta j} \assign \bary_{\eta}, \; \tildex_{\sigma j} \assign 0$
    \State For each $\nu' \neq \nu$ with $\barx_{\eta \nu'}>0$, set $\barx_{\eta \nu'} \assign \bary_{\eta},\; \barx_{\sigma \nu'} \assign \bary_{\sigma}$
    \State For each $j' \neq j$ with $\tildex_{\eta j'}>0$, set $\tildex_{\eta j'} \assign \bary_{\eta}, \tildex_{\sigma j'} \assign \bary_{\sigma}$
	\State  (All other new connection values are set to $0$)
    \EndIf
    \EndWhile
    \EndFunction
  \end{algorithmic}
\end{algorithm}

%%%%%%%%%%%%%%


The next step is to pick a client $p$ with minimum
$\tcc(p)+\alpha_p^\ast$ and create a demand $\nu$ for $p$
(Lines~9--10). If $\wtildeN_1(p)$ overlaps the neighborhood
of some existing primary demand $\kappa$ (if there are
multiple such $\kappa$'s, pick any of them), we assign $\nu$
to $\kappa$, and $\nu$ acquires all the connection values
$\tildex_{\mu p}$ between client $p$ and facility $\mu$ in
$\wtildeN(p)\cap \wbarN(\kappa)$ (Lines~11--13). Note that
although we check for overlap with $\wtildeN_1(p)$, we then
move all facilities in the intersection with $\wtildeN(p)$,
a bigger set, into $\wbarN(\nu)$.  The other case is when
$\wtildeN_1(p)$ is disjoint from the neighborhoods of all
existing primary demands. Then, in Lines~15--16, $\nu$
becomes itself a primary demand and we assign $\nu$ to
itself. It also inherits the connection values to all
facilities $\mu\in\wtildeN_1(p)$ from $p$ (recall that
$\tildex_{\mu p} = \bary_{\mu}$), with all other
$\barx_{\mu\nu}$ values set to $0$.

At this point all primary demands satisfy
Property~(PS.\ref{PS:one}), but this may not be true for
non-primary demands. For those demands we still may need to
adjust the $\barx_{\mu\nu}$ values so that the total
connection value for $\nu$, that is $\connsum(\nu) \stackrel{\mathrm{def}}{=}
\sum_{\mu\in\facilityset}\barx_{\mu \nu}$, is equal $1$. This
is accomplished by Procedure $\AugmentToUnit()$ (definition
in Pseudocode~\ref{alg:helper}, Lines~10--21) that allocates
to $\nu\in j$ some of the remaining connection values
$\tildex_{\mu j}$ of client $j$ (Lines 19--21).
$\AugmentToUnit()$ will repeatedly pick any facility $\eta$ with
$\tildex_{\eta j} >0$.  If $\tildex_{\eta j} \leq
1-\connsum(\nu)$, then the connection value $\tildex_{\eta
  j}$ is reassigned to $\nu$. 
Otherwise, $\tildex_{\eta j} >
1-\connsum(\nu)$, in which case we split $\eta$ so that
connecting $\nu$ to one of the created copies of $\eta$ will
make $\connsum(\nu)$ equal $1$, and we'll be done.


\smallskip

Notice that we start with $|\sitesset|$ facilities and in
each iteration of the while loop in Line~5 (Pseudocode~\ref{alg:lpr2}) each client causes at most one split.
 We have a total of no more than $R|\clientset|$ iterations as in
each iteration we create one demand. (Recall that $R =
\max_jr_j$.) In Phase 2 we do an augment step for each
demand $\nu$ and this creates no more than $R|\clientset|$
new facilities.  So the total number of facilities we
created will be at most $|\sitesset|+ R|\clientset|^2 +
R|\clientset| \leq |\sitesset| + 2R|\clientset|^2$, which is
polynomial in $|\sitesset|+|\clientset|$ due to our earlier
bound on $R$.

%%% example for adaptive partition
\paragraph{Example.}
We now illustrate our partitioning algorithm with an example, where the FTFP instance
has four sites and four clients. The demands are $r_1=1$ and $r_2=r_3=r_4=2$.
The facility costs are $f_i = 1$ for all $i$. The distances are defined as follows: 
$d_{ii} = 3$ for $i=1,2,3,4$ and $d_{ij} = 1$ for all $i\neq j$. 
Solving the LP(\ref{eqn:fac_primal}), we obtain the fractional solution given in
Table~\ref{tbl:example_opt}.
%
{
\small
\begin{table}[ht]
%
\hfill
\setlength{\extrarowheight}{4pt}
\begin{subtable}{0.2\textwidth}
  \centering
  \begin{tabular}{c | c c c c | c }
    $x_{ij}^\ast$ & $1$ & $2$ & $3$ & $4$ & $y_{i}^\ast$\\
    \hline
    $1$ & 0 & $\fourthirds$ & $\fourthirds$ & $\fourthirds$ & $\fourthirds$ \\
    $2$ & $\onethird$ & 0 & $\onethird$ & $\onethird$ & $\onethird$ \\
    $3$ & $\onethird$ & $\onethird$ & 0 & $\onethird$ & $\onethird$ \\
    $4$ & $\onethird$ & $\onethird$ & $\onethird$ & 0 & $\onethird$ \\
  \end{tabular}
  \subcaption{}
  \label{tbl:example_opt}
\end{subtable}
%
\hspace{0.8in}
%
\begin{subtable}{0.4\textwidth}
  \centering
  \begin{tabular}{c | c c c c c c c | c} % seven demands, five facilities
    $\barx_{\mu\nu}$ & $1'$ & $2'$ & $2''$ & $3'$ & $3''$ & $4'$ & $4''$ & $\bary_{\mu}$ \\
    \hline
    $\dot{1}$ & 0 & 1 & 0 & 1 & 0 & 1 & 0 & 1\\
    $\ddot{1}$ & 0 & 0 & $\onethird$ & 0 & $\onethird$ & 0 & $\onethird$ & $\onethird$ \\
    $\dot{2}$ & $\onethird$ & 0 & 0 & 0 & $\onethird$ & 0 & $\onethird$  & $\onethird$ \\
    $\dot{3}$ & $\onethird$ & 0 & $\onethird$ & 0 & 0 & 0 & $\onethird$  & $\onethird$ \\
    $\dot{4}$ & $\onethird$ & 0 & $\onethird$ & 0 & $\onethird$ & 0 & 0 & $\onethird$ \\
  \end{tabular}
  \subcaption{}
  \label{tbl:example_part}
\end{subtable}
\hfill{\ }
%
\caption{
  An example of an execution of the partitioning algorithm.
  (a) An optimal fractional solution $x^\ast,y^\ast$.
  (b) The partitioned solution. $j'$ and $j''$ denote the first and second demand of a client $j$, 
	and $\dot{\imath}$ and $\ddot{\imath}$ denote the first and second facility at site $i$.}
%
\end{table}
}

It is easily seen that the fractional solution in
Table~\ref{tbl:example_opt} is optimal and complete ($x_{ij}^\ast > 0$
implies $x_{ij}^\ast = y_i^\ast$). The dual optimal solution has all
$\alpha_j^\ast = 4/3$ for $j=1,2,3,4$.

Now we perform Phase 1, the adaptive partitioning, following the
description in Pseudocode~\ref{alg:lpr2}. To streamline the
presentation, we assume that all ties are broken in favor of
lower-numbered clients, demands or facilities.  First we create one
facility at each of the four sites, denoted as $\dot{1}$, $\dot{2}$,
$\dot{3}$ and $\dot{4}$ (Line~2--4, Pseudocode~\ref{alg:lpr2}).  We
then execute the ``while'' loop in Line 5
Pseudocode~\ref{alg:lpr2}. This loop will have seven iterations.
Consider the first iteration. In Line 7--8 we compute $\tcc(j)$ for
each client $j=1,2,3,4$ in $U$. When computing $\wtildeN_1(2)$,
facility $\dot{1}$ will get split into $\dot{1}$ and $\ddot{1}$ with
$\bary_{\dot{1}}=1$ and $\bary_{\ddot{1}} = 1/3$. (This will happen in
Line~4--7 of Pseudocode~\ref{alg:helper}.)  Then, in Line~9 we will
pick client $p=1$ and create a demand denoted as $1'$ (see
Table~\ref{tbl:example_part}). Since there are no primary demands yet,
we make $1'$ a primary demand with $\wbarN(1') = \wtildeN_1(1) =
\{\dot{2}, \dot{3}, \dot{4}\}$. Notice that client $1$ is exhausted
after this iteration and $U$ becomes $\{2,3,4\}$.

In the second iteration we compute $\tcc(j)$ for $j=2,3,4$ and pick
client $p=2$, from which we create a new demand $2'$. We have
$\wtildeN_1(2) = \{\dot{1}\}$, which is disjoint from $\wbarN(1')$. So
we create a demand $2'$ and make it primary, and set $\wbarN(2') =
\{\dot{1}\}$. In the third iteration we compute $\tcc(j)$ for
$j=2,3,4$ and again we pick client $p=2$. Since $\wtildeN_1(2) =
\{\ddot{1}, \dot{3}, \dot{4}\}$ overlaps with $\wbarN(1')$, we create
a demand $2''$ and assign it to $1'$. We also set $\wbarN(2'') =
\wbarN(1') \cap \wtildeN(2) = \{\dot{3}, \dot{4}\}$. After this
iteration client $2$ is exhausted and we have $U = \{3,4\}$.

In the fourth iteration we compute $\tcc(j)$ for client $j=3,4$. We
pick $p=3$ and create demand $3'$. Since $\wtildeN_1(3) = \{\dot{1}\}$
overlaps $\wbarN(2')$, we assign $3'$ to $2'$ and set
$\wbarN(3') = \{\dot{1}\}$. In the fifth iteration we compute
$\tcc(j)$ for client $j=3,4$ and pick $p=3$ again. At this time
$\wtildeN_1(3) = \{\ddot{1},\dot{2},\dot{4}\}$, which overlaps with
$\wbarN(1')$. So we create a demand $3''$ and assign it to $1'$, as
well as set $\wbarN(3'') = \{\dot{2}, \dot{4}\}$.

In the last two iterations we will pick client $p=4$ twice and
create demands $4'$ and $4''$. For $4'$ we have $\wtildeN_1(4) =
\{\dot{1}\}$ so we assign $4'$ to $2'$ and set $\wbarN(4') =
\{\dot{1}\}$. For $4''$ we have $\wtildeN_1(4) = \{\ddot{1}, \dot{2},
\dot{3}\}$ and we assign it to $1'$, as well as set $\wbarN(4'') =
\{\dot{2}, \dot{3}\}$.

Now that all clients are exhausted we perform Phase 2, the augmenting
phase, to construct a fractional solution in which all demands have
total connection value equal to $1$.  We iterate through each of the
seven demands created, that is $1',2',2'',3',3'',4',4''$.  $1'$ and $2'$
already have neighborhoods with total connection value of $1$, so
nothing will change in the first two iterations.
$2''$ has $\dot{3},\dot{4}$ in its neighborhood, with total connection value of
$2/3$, and $\wtildeN(2) = \{\ddot{1}\}$ at this time, so we add
$\ddot{1}$ into $\wbarN(2'')$ to make $\wbarN(2'') = \{\ddot{1},
\dot{3}, \dot{4}\}$ and now $2''$ has total connection value of
$1$. Similarly, $3''$ and $4''$ each get $\ddot{1}$ added to their
neighborhood and end up with total connection value of $1$. The other
two demands, namely $3'$ and $4'$, each have $\dot{1}$ in its
neighborhood so each of them has already its total connection value
equal $1$. This completes Phase 2.

The final partitioned fractional solution is given in
Table~\ref{tbl:example_part}. We have created a total of five
facilities $\dot{1}, \ddot{1}, \dot{2}, \dot{3}, \dot{4}$, and seven
demands, $1',2',2'',3',3'',4',4''$. It can be verified that all the
stated properties are satisfied.

%%%% end example %%%

%%%%%%

\medskip

\emparagraph{Correctness.}  We now show that all the
required properties (PS), (CO), (PD) and (SI) are satisfied
by the above construction.

Properties~(PS) and (CO) follow directly from the
algorithm. (CO) is implied by the completeness condition
(c1) that the algorithm maintains after each
iteration. Condition~(PS.\ref{PS:one}) is a result of
calling Procedure~$\AugmentToUnit()$ in Line~21. To see that
(PS.\ref{PS:xij}) holds, note that
at each step the algorithm maintains the
invariant that, for every $i\in\sitesset$ and
$j\in\clientset$, we have $\sum_{\mu\in i}\sum_{\nu \in j}
\barx_{\mu \nu} + \sum_{\mu\in i} \tildex_{\mu j} =
x_{ij}^\ast$. In the end, we will create $r_j$ demands for
each client $j$, with each demand $\nu\in j$ satisfying
(PS.\ref{PS:one}), and thus $\sum_{\nu\in
  j}\sum_{\mu\in\facilityset}\barx_{\mu\nu}=r_j$.  This
implies that $\tildex_{\mu j}=0$ for every facility
$\mu\in\facilityset$, and (PS.\ref{PS:xij}) follows.
(PS.\ref{PS:yi}) holds because every time we split a
facility $\mu$ into $\mu'$ and $\mu''$, the sum of
$\bary_{\mu'}$ and $\bary_{\mu''}$ is equal to the old value of
$\bary_{\mu}$.

Now we deal with properties in group (PD).  First,
(PD.\ref{PD:disjoint}) follows directly from the algorithm,
Pseudocode~\ref{alg:lpr2} (Lines 14--16), since every
primary demand has its neighborhood fixed when created, and
that neighborhood is disjoint from those of the existing primary
demands.

Property (PD.\ref{PD:yi}) follows from (PD.\ref{PD:disjoint}), (CO) and
(PS.\ref{PS:yi}). In more detail, it can be justified as
follows. By (PD.\ref{PD:disjoint}), for each $\mu\in i$ there
is at most one $\kappa\in P$ with $\barx_{\mu\kappa} > 0$
and we have $\barx_{\mu\kappa} = \bary_{\mu}$ due do (CO).
Let $K\subseteq i$ be the set of those $\mu$'s for which
such $\kappa\in P$ exists, and denote this $\kappa$ by
$\kappa_\mu$. Then, using conditions (CO) and
(PS.\ref{PS:yi}), we have $ \sum_{\mu\in i}\sum_{\kappa\in
  P}\barx_{\mu\kappa} = \sum_{\mu\in K}\barx_{\mu\kappa_\mu}
= \sum_{\mu\in K}\bary_{\mu} \leq \sum_{\mu\in i}
\bary_{\mu} = y_i^\ast$.

Property (PD.\ref{PD:assign:overlap}) follows from the way the algorithm
assigns primary demands.  When demand $\nu$ of
client $p$ is assigned to a primary demand $\kappa$ in
Lines~11--13 of Pseudocode~\ref{alg:lpr2}, we move all
facilities in $\wtildeN(p)\cap \wbarN(\kappa)$ (the
intersection is nonempty) into $\wbarN(\nu)$, and we never
remove a facility from $\wbarN(\nu)$.  We postpone the proof 
for (PD.\ref{PD:assign:cost}) to Lemma~\ref{lem: PD:assign:cost holds}.

Finally we argue that the properties in group (SI)
hold. (SI.\ref{SI:siblings disjoint}) is easy, since for any client
$j$, each facility $\mu$ is added to the neighborhood of at most one
demand $\nu\in j$, by setting $\barx_{\mu\nu}$ to $\bary_\mu$, while
other siblings $\nu'$ of $\nu$ have $\barx_{\mu\nu'}=0$. Note that
right after a demand $\nu\in p$ is created, its neighborhood is
disjoint from the neighborhood of $p$, that is $\wbarN(\nu)\cap
\wtildeN(p) = \emptyset$, by Lines~11--13 of the algorithm. Thus all
demands of $p$ created later will have neighborhoods disjoint from the
set $\wbarN(\nu)$ before the augmenting phase 2. Furthermore,
Procedure~$\AugmentToUnit()$ preserves this property, because when it
adds a facility to $\wbarN(\nu)$ then it removes it from
$\wtildeN(p)$, and in case of splitting, one resulting facility is
added to $\wbarN(\nu)$ and the other to $\wtildeN(p)$. Property
(SI.\ref{SI:primary disjoint}) is shown below in Lemma~\ref{lem:
  property SI:primary disjoint holds}.

It remains to show Properties~(PD.\ref{PD:assign:cost}) and
(SI.\ref{SI:primary disjoint}). We show them in the lemmas
below, thus completing the description of our adaptive
partition process.

%%%%%%%

\begin{lemma}\label{lem: property SI:primary disjoint holds}
  Property~(SI.\ref{SI:primary disjoint}) holds after the
  Adaptive Partitioning stage.
\end{lemma}

\begin{proof}
  Let $\nu_1,\ldots,\nu_{r_j}$ be the demands of a client
  $j\in\clientset$, listed in the order of creation, and, for each
  $q=1,2,\ldots,r_j$, denote by $\kappa_q$ the primary demand that
  $\nu_q$ is assigned to. After the completion of Phase~1 of
  Pseudocode~\ref{alg:lpr2} (Lines 5--18), we have
  $\wbarN(\nu_s)\subseteq \wbarN(\kappa_s)$ for  $s=1,\ldots,r_j$. 
Since any two primary demands have disjoint
  neighborhoods, we have $\wbarN(\nu_s) \cap \wbarN(\kappa_q) =
  \emptyset$ for any $s\neq q$, that is
	Property~(SI.\ref{SI:primary disjoint}) holds right after Phase~1.

        After Phase~1 all neighborhoods $\wbarN(\kappa_s),
        s=1,\ldots,r_j$ have already been fixed and they do not change
        in Phase~2.  None of the facilities in $\wtildeN(j)$ appear in
        any of $\wbarN(\kappa_s)$ for $s=1,\ldots,r_j$, by the way we
        allocate facilities in Lines~13 and 16.  Therefore during the
        augmentation process in Phase~2, when we add facilities from
        $\wtildeN(j)$ to $\wbarN(\nu)$, for some $\nu\in j$
        (Line~19--21 of Pseudocode~\ref{alg:lpr2}), all the required
        disjointness conditions will be preserved.
\end{proof}

%%%%%%%

We need one more lemma before proving our last property
(PD.\ref{PD:assign:cost}).  For a client $j$ and a demand
$\nu$, we use notation $\tcc^{\nu}(j)$ for the value of
$\tcc(j)$ at the time when $\nu$ was created. (It is not
necessary that $\nu\in j$ but we assume that $j$ is not
exhausted at that time.)


\begin{lemma}\label{lem: tcc optimal}
  Let $\eta$ and $\nu$ be two demands, with $\eta$ created
  no later than $\nu$, and let $j\in\clientset$ be a client
  that is not exhausted when $\nu$ is created. Then we have
\begin{description}
	\item{(a)} $\mbox{\tcc}^\eta(j) \le \mbox{\tcc}^{\nu}(j)$, and 
	\item{(b)} if $\nu\in j$ then $\mbox{\tcc}^\eta(j) \le \concost_{\nu}$.
\end{description}
\end{lemma}

\begin{proof}
  We focus first on the time when demand $\eta$ is about to be created,
  right after the call to $\NearestUnitChunk()$ in
  Pseudocode~\ref{alg:lpr2}, Line~7.  Let $\wtildeN(j) =
  \{\mu_1,...,\mu_q\}$ with all facilities $\mu_s$ ordered
  according to nondecreasing distance from $j$.  Consider
  the following linear program:
%
\begin{alignat*}{1}
	\textrm{minimize} \quad & \sum_s d_{\mu_s j}z_s
			\\
	\textrm{subject to} \quad & \sum_s z_s  \ge 1
			\\
 	0 &\le z_s \le \tildex_{\mu_s j} \quad \textrm{for all}\ s
\end{alignat*}
%
  This is a fractional
  minimum knapsack covering problem (with knapsack size equal $1$) and its optimal fractional
  solution is the greedy solution, whose value is exactly
  $\tcc^\eta(j)$.  

On the other hand, we claim that
  $\tcc^{\nu}(j)$ can be thought of as the value of some feasible
  solution to this linear program, and that the same is true for $\concost_{\nu}$ if $\nu\in j$.
  Indeed, each of these
  quantities involves some later values $\tildex_{\mu j}$,
  where $\mu$ could be one of the facilities $\mu_s$ or a
  new facility obtained from splitting. For each $s$,
  however, the sum of all values $\tildex_{\mu j}$,
  over the facilities $\mu$ that were split from $\mu_s$, cannot exceed
 the value $\tildex_{\mu_s j}$ at the time when
  $\eta$ was created, because splitting facilities preserves this sum and
 creating new demands for $j$ can only decrease it.
Therefore both quantities
  $\tcc^{\nu}(j)$ and $\concost_{\nu}$ (for $\nu\in j$) correspond to some
  choice of the $z_s$ variables (adding up to $1$), and the
  lemma follows.
\end{proof}

%%%%%%%

\begin{lemma}\label{lem: PD:assign:cost holds}
Property~(PD.\ref{PD:assign:cost}) holds after the Adaptive Partitioning stage.
\end{lemma}

\begin{proof}
Suppose that demand $\nu\in j$ is assigned to some primary demand $\kappa\in p$.
Then
%
\begin{eqnarray*}
 \concost_{\kappa} + \alpha_{\kappa}^\ast \;=\; \tcc^{\kappa}(p) + \alpha^\ast_p
 					\;\le\; \tcc^{\kappa}(j) + \alpha^\ast_j   
					\;\le\; \concost_{\nu} + \alpha^\ast_\nu.
\end{eqnarray*}
%
We now justify this derivation. By definition we have
$\alpha_{\kappa}^\ast = \alpha^\ast_p$.  Further, by the
algorithm, if $\kappa$ is a primary demand of client $p$,
then $\concost_{\kappa}$ is equal to $\tcc(p)$ computed when
$\kappa$ is created, which is exactly $\tcc^{\kappa}(p)$. Thus
the first equation is true. The first inequality follows
from the choice of $p$ in Line~9 in
Pseudocode~\ref{alg:lpr2}. The last inequality holds
because $\alpha^\ast_j = \alpha^\ast_\nu$ (due to $\nu\in
j$), and because $\tcc^{\kappa}(j) \le \concost_{\nu}$, which
follows from Lemma~\ref{lem: tcc optimal}.
\end{proof}

We have thus proved that all properties (PS), (CO), (PD) and (SI) hold
for our partitioned fractional solution $(\barbfx,\barbfy)$. In the
following sections we show how to use these properties to round the
fractional solution to an approximate integral solution. For the
$3$-approximation algorithm (Section~\ref{sec: 3-approximation}) and
the $1.736$-approximation algorithm (Section~\ref{sec:
  1.736-approximation}), the first phase of the algorithm is exactly
the same partition process as described above. However, the
$1.575$-approximation algorithm (Section~\ref{sec:
  1.575-approximation}) demands a more sophisticated partitioning
process as the interplay between close and far neighborhood of sibling
demands result in more delicate properties that our partitioned
fractional solution must satisfy.

%% ch4 LP-rounding results
\chapter{LP-rounding Algorithms} \label{ch: lp-rounding}

%% EGUP 3
\section{Algorithm~{\EGUP} with Ratio $3$}
\label{sec: 3-approximation}

With the partitioned FTFP instance and its associated fractional
solution in place, we now begin to introduce our rounding algorithms.
The algorithm we describe in this section achieves ratio $3$. Although
this is still quite far from our best ratio $1.575$ that we derive
later, we include this algorithm in the paper to illustrate, in a
relatively simple setting, how the properties of our partitioned
fractional solution are used in rounding it to an integral solution
with cost not too far away from an optimal solution.  The rounding
approach we use here is an extension of the corresponding method for
UFL described in~\cite{gupta08}.

\paragraph{Algorithm~{\EGUP.}}
At a high level, we would open exactly one facility for each
primary demand $\kappa$, and each non-primary demand is
connected to the facility opened for the primary demand it
was assigned to.

More precisely, we apply a rounding process, guided by the
fractional values $(\bary_{\mu})$ and $(\barx_{\mu\nu})$,
that produces an integral solution. This integral solution
is obtained by choosing a subset of facilities in
$\facilityset$ to open, and for each demand in $\demandset$,
specifying an open facility that this demand will be
connected to.  For each primary demand $\kappa\in P$, we
want to open one facility $\phi(\kappa) \in
\wbarN(\kappa)$. To this end, we use randomization: for each
$\mu\in\wbarN(\kappa)$, we choose $\phi(\kappa) = \mu$ with
probability $\barx_{\mu\kappa}$, ensuring that exactly one
$\mu \in \wbarN(\kappa)$ is chosen. Note that
$\sum_{\mu\in\wbarN(\kappa)}\barx_{\mu\kappa}=1$, so this
distribution is well-defined.  We open this facility
$\phi(\kappa)$ and connect to $\phi(\kappa)$ all demands
that are assigned to $\kappa$.

In our description above, the algorithm is presented as a
randomized algorithm. It can be de-randomized using the
method of conditional expectations, which is commonly used
in approximation algorithms for facility location problems
and standard enough that presenting it here would be
redundant. Readers less familiar with this field are
recommended to consult \cite{ChudakS04}, where the method of
conditional expectations is applied in a context very
similar to ours.

%%%%%%%%%

\paragraph{Analysis.}
We now bound the expected facility cost and connection cost
by establishing the two lemmas below.

%%%%%

\begin{lemma}\label{lemma:3fac}
The expectation of facility cost $F_{\smallEGUP}$ of our solution is
  at most $F^\ast$.
\end{lemma}

\begin{proof}
  By Property~(PD.\ref{PD:disjoint}), the neighborhoods of
  primary demands are disjoint. Also, for any primary demand
  $\kappa\in P$, the probability that a facility
  $\mu\in\wbarN(\kappa)$ is chosen as the open facility
  $\phi(\kappa)$ is $\barx_{\mu\kappa}$. Hence the expected
  total facility cost is
%
\begin{align*}
    \Exp[F_{\smallEGUP}]
	&= \textstyle{\sum_{\kappa\in P}\sum_{\mu\in\wbarN(\kappa)}} f_{\mu} \barx_{\mu\kappa}
	\\
	&= \textstyle{\sum_{\kappa\in P}\sum_{\mu\in\facilityset}} f_{\mu} \barx_{\mu\kappa} 
	\\
	&= \textstyle{\sum_{i\in\sitesset}} f_i \textstyle{\sum_{\mu\in i}\sum_{\kappa\in P}} \barx_{\mu\kappa} 
	\\
	&\leq \textstyle{\sum_{i\in\sitesset}} f_i y_i^\ast 
	= F^\ast,
\end{align*}
%
where the inequality follows from Property~(PD.\ref{PD:yi}).
\end{proof}

%%%%%%%

\begin{lemma}\label{lemma:3dist}
The expectation of connection cost $C_{\smallEGUP}$ of our solution
is at most  $C^\ast+2\cdot\LP^\ast$.
\end{lemma}

\begin{proof}
  For a primary demand $\kappa$, its expected connection cost is
  $C_{\kappa}^{\avg}$ because we choose facility $\mu$ with
  probability $\barx_{\mu\kappa}$.

  Consider a non-primary demand $\nu$ assigned to a primary demand
  $\kappa\in P$. Let $\mu$ be any facility in $\wbarN(\nu) \cap
  \wbarN(\kappa)$.  Since $\mu$ is in both $\wbarN(\nu)$ and
  $\wbarN(\kappa)$, we have $d_{\mu\nu} \leq \alpha_{\nu}^\ast$ and
  $d_{\mu\kappa} \leq \alpha_{\kappa}^\ast$ (This follows from the
  complementary slackness conditions since
  $\alpha_{\nu}^\ast=\beta_{\mu\nu}^\ast + d_{\mu\nu}$ for each
  $\mu\in \wbarN(\nu)$.). Thus, applying the triangle inequality, for
  any fixed choice of facility $\phi(\kappa)$ we have
%
\begin{equation*}
    d_{\phi(\kappa)\nu} \leq d_{\phi(\kappa)\kappa}+d_{\mu\kappa}+d_{\mu\nu}
    \leq d_{\phi(\kappa)\kappa} + \alpha_{\kappa}^\ast + \alpha_{\nu}^\ast.
\end{equation*}
%
Therefore the expected distance from $\nu$ to its facility $\phi(\kappa)$ is 
%
\begin{align*}
  \Exp[  d_{\phi(\kappa)\nu}   ] &\le \concost_{\kappa} + \alpha_{\kappa}^\ast + \alpha_{\nu}^\ast 
\\
  &\leq \concost_{\nu} + \alpha_{\nu}^\ast + \alpha_{\nu}^\ast
   = \concost_{\nu} + 2\alpha_{\nu}^\ast,
  \end{align*}
%
  where the second inequality follows from Property~(PD.\ref{PD:assign:cost}).  
From the definition of $\concost_{\nu}$ and Property~(PS.\ref{PS:xij}), for any $j\in \clientset$ 
we have
%
\begin{align*}
\sum_{\nu\in j} \concost_{\nu} &= \sum_{\nu\in j}\sum_{\mu\in\facilityset}d_{\mu\nu}\barx_{\mu\nu}
			\\
 			&= \sum_{i\in\sitesset} d_{ij}\sum_{\nu\in j}\sum_{\mu\in i}\barx_{\mu\nu}
			\\
			&= \sum_{i\in\sitesset} d_{ij}x^\ast_{ij} 
			= C^\ast_j.
\end{align*}
% 
Thus, summing over all demands, the expected total connection cost is
%
\begin{align*}
    \Exp[C_{\smallEGUP}] &\le 
			\textstyle{\sum_{j\in\clientset} \sum_{\nu\in j}} (\concost_{\nu} + 2\alpha_{\nu}^\ast) 
			\\
    	& = \textstyle{\sum_{j\in\clientset}} (C_j^\ast + 2r_j\alpha_j^\ast)
 		= C^\ast + 2\cdot\LP^\ast,
\end{align*}
%
completing the proof of the lemma.
\end{proof}

%%%%%%%%

\begin{theorem}
Algorithm~{\EGUP} is a $3$-approximation algorithm.
\end{theorem}

\begin{proof}
  By Property~(SI.\ref{SI:primary disjoint}), different
  demands from the same client are assigned to different
  primary demands, and by (PD.\ref{PD:disjoint}) each primary
  demand opens a different facility. This ensures that our
  solution is feasible, namely each client $j$ is connected
  to $r_j$ different facilities (some possibly located on
  the same site).  As for the total cost,
  Lemma~\ref{lemma:3fac} and Lemma~\ref{lemma:3dist} imply
  that the total cost is at most
  $F^\ast+C^\ast+2\cdot\LP^\ast = 3\cdot\LP^\ast \leq
  3\cdot\OPT$.
\end{proof}

%%%%%%%%%

%% ECHS 1.736
\section{Algorithm~{\ECHS} with Ratio $1.736$}
\label{sec: 1.736-approximation}

In this section we improve the approximation ratio to $1+2/e \approx
1.736$. The improvement comes from a slightly modified rounding
process and refined analysis.  Note that the facility opening cost of
Algorithm~{\EGUP} does not exceed that of the fractional optimum
solution, while the connection cost could be far from the optimum,
since we connect a non-primary demand to a facility in the neighborhood of
its assigned primary demand and then estimate the distance using the
triangle inequality. The basic idea to improve the estimate of the connection cost,
following the approach of Chudak and Shmoys~\cite{ChudakS04}, 
is to connect each non-primary demand to its
nearest neighbor when one is available, and to only use the facility opened by
its assigned primary demand when none of its neighbors is open.

%%%%%%%%%%

\paragraph{Algorithm~{\ECHS}.}
As before,
the algorithm starts by solving the linear program and applying the
adaptive partitioning algorithm  described in 
Section~\ref{sec: adaptive partitioning} to obtain a partitioned
solution $(\barbfx, \barbfy)$. Then we apply the rounding
process to compute an integral solution (see Pseudocode~\ref{alg:lpr3}).  

We start, as before, by opening exactly one facility $\phi(\kappa)$ in the 
neighborhood of each primary demand $\kappa$ (Line 2).  For any
non-primary demand $\nu$ assigned to $\kappa$, we refer to
$\phi(\kappa)$ as the \emph{target} facility of $\nu$.  In
Algorithm~{\EGUP}, $\nu$ was connected to $\phi(\kappa)$,
but in Algorithm~{\ECHS} we may be able to find an open
facility in $\nu$'s neighborhood and connect $\nu$ to this
facility.  Specifically, the two changes in the
algorithm are as follows:
%
\begin{description}
\item{(1)} Each facility $\mu$ that is not in the neighborhood of any
  primary demand is opened, independently, with probability
  $\bary_{\mu}$ (Lines 4--5). Notice that if $\bary_\mu>0$ then, due
  to completeness of the partitioned fractional solution, we have
  $\bary_{\mu}= \barx_{\mu\nu}$ for some demand $\nu$. This implies
  that $\bary_{\mu}\leq 1$, because $\barx_{\mu\nu}\le 1$, by
  (PS.\ref{PS:one}).
%
\item{(2)} When connecting demands to facilities, a primary demand
  $\kappa$ is connected to the only facility $\phi(\kappa)$ opened in
  its neighborhood, as before (Line 3).  For a non-primary demand
  $\nu$, if its neighborhood $\wbarN(\nu)$ has an open facility, we
  connect $\nu$ to the closest open facility in $\wbarN(\nu)$ (Line
  8). Otherwise, we connect $\nu$ to its target facility (Line 10).
%
\end{description}

%%%%%%%%%%%%%

\begin{algorithm}
  \caption{Algorithm~{\ECHS}:
    Constructing Integral Solution}
  \label{alg:lpr3}
  \begin{algorithmic}[1]
    \For{each $\kappa\in P$} 
    \State choose one $\phi(\kappa)\in \wbarN(\kappa)$,
    with each $\mu\in\wbarN(\kappa)$ chosen as $\phi(\kappa)$
    with probability $\bary_\mu$ 
    \State open $\phi(\kappa)$ and connect $\kappa$ to $\phi(\kappa)$
    \EndFor
    \For{each $\mu\in\facilityset - \bigcup_{\kappa\in P}\wbarN(\kappa)$} 
    \State open $\mu$ with probability $\bary_\mu$ (independently)
    \EndFor
    \For{each non-primary demand $\nu\in\demandset$}
    \If{any facility in $\wbarN(\nu)$ is open}
    \State{connect $\nu$ to the nearest open facility in $\wbarN(\nu)$}
    \Else
    \State connect $\nu$ to $\phi(\kappa)$ where $\kappa$ is $\nu$'s
     assigned primary demand
    \EndIf
    \EndFor
  \end{algorithmic}
\end{algorithm}

%%%%%%%%%%%%%%%%

\paragraph{Analysis.}
We shall first argue that the integral solution thus
constructed is feasible, and then we bound the total cost of
the solution. Regarding feasibility, the only constraint
that is not explicitly enforced by the algorithm is the
fault-tolerance requirement; namely that each client $j$ is
connected to $r_j$ different facilities. Let $\nu$ and
$\nu'$ be two different sibling demands of client $j$ and let
their assigned primary demands be $\kappa$ and $\kappa'$
respectively. Due to (SI.\ref{SI:primary
  disjoint}) we know $\kappa \neq \kappa'$. From
(SI.\ref{SI:siblings disjoint}) we have $\wbarN(\nu) \cap
\wbarN(\nu') = \emptyset$. From (SI.\ref{SI:primary
  disjoint}), we have $\wbarN(\nu) \cap \wbarN(\kappa') =
\emptyset$ and $\wbarN(\nu') \cap \wbarN(\kappa) =
\emptyset$. From (PD.\ref{PD:disjoint}) we have
$\wbarN(\kappa)\cap \wbarN(\kappa') = \emptyset$. It follows
that $(\wbarN(\nu) \cup \wbarN(\kappa)) \cap (\wbarN(\nu')
\cup \wbarN(\kappa')) = \emptyset$. Since the algorithm
connects $\nu$ to some facility in $\wbarN(\nu) \cup
\wbarN(\kappa)$ and $\nu'$ to some facility in $\wbarN(\nu')
\cup \wbarN(\kappa')$, $\nu$ and $\nu'$ will be connected to
different facilities.


%%%%%%%%%

\smallskip
We now show that the expected cost of the computed solution is bounded by
$(1+2/e) \cdot \LP^\ast$. By
(PD.\ref{PD:disjoint}), every facility may appear in at
most one primary demand's neighborhood, and the facilities
open in Line~4--5 of Pseudocode~\ref{alg:lpr3} do not appear
in any primary demand's neighborhood. Therefore, by
linearity of expectation, the expected facility cost of
Algorithm~{\ECHS} is 
%
\begin{equation*}
\Exp[F_{\smallECHS}] 
	= \sum_{\mu\in\facilityset} f_\mu \bary_{\mu} 
	= \sum_{i\in\sitesset} f_i\sum_{\mu\in i} \bary_{\mu} 
	= \sum_{i\in\sitesset} f_i y_i^\ast = F^\ast,
\end{equation*}
%
where the third equality follows from (PS.\ref{PS:yi}).

\smallskip

To bound the connection cost, we adapt an argument of Chudak
and Shmoys~\cite{ChudakS04}. Consider a demand $\nu$ and denote by $C_\nu$ the
random variable representing the connection cost for $\nu$.
Our goal now is to estimate $\Exp[C_\nu]$, the expected value of $C_\nu$.
Demand $\nu$ can either get connected directly to some facility in
$\wbarN(\nu)$ or indirectly to its target facility $\phi(\kappa)\in
\wbarN(\kappa)$, where $\kappa$ is the primary demand to
which $\nu$ is assigned. We will analyze these two cases separately.

In our analysis, in this section and the next one, we will use notation
%
\begin{equation*}
D(A,\sigma) {=} \sum_{\mu\in A}
d_{\mu\sigma}\bary_{\mu}/\sum_{\mu\in A} \bary_{\mu}
\end{equation*}
%
for the average distance between a demand $\sigma$ and a set $A$ of facilities.
Note that, in particular, we have $\concost_\nu = D(\wbarN(\nu),\nu)$.

We first estimate the expected cost $d_{\phi(\kappa)\nu}$ of the indirect
connection. Let $\Lambda^\nu$ denote the event that some 
facility in $\wbarN(\nu)$ is opened. Then
%
\begin{equation}
	\Exp[C_\nu \mid\neg\Lambda^\nu] 
	=   \Exp[ d_{\phi(\kappa)\nu} \mid \neg\Lambda^\nu] 
	= 	D(\wbarN(\kappa) \setminus \wbarN(\nu), \nu).
			\label{eqn: expected indirect connection}
\end{equation}
%
Note that $\neg\Lambda^\nu$ implies that $\wbarN(\kappa) \setminus
\wbarN(\nu)\neq\emptyset$, since $\wbarN(\kappa)$ contains
exactly one open facility, namely $\phi(\kappa)$.

%%%%%%%%%%

\begin{lemma}
  \label{lem:echu indirect}
  Let $\nu$ be a demand assigned to a primary demand $\kappa$, and
assume that $\wbarN(\kappa) \setminus \wbarN(\nu)\neq\emptyset$.
Then 
%
\begin{equation*}
	\Exp[ C_\nu \mid\neg\Lambda^\nu]  \leq
  		\concost_\nu+2\alpha_{\nu}^\ast.
\end{equation*}
\end{lemma}

\begin{proof}
By (\ref{eqn: expected indirect connection}), we need to show that $D(\wbarN(\kappa)
  \setminus \wbarN(\nu), \nu) \leq \concost_\nu +
  2\alpha_{\nu}^\ast$. There are two cases to consider.

\begin{description}
%	
\item{\mycase{1}}
	 There exists some $\mu'\in \wbarN(\kappa) \cap
  \wbarN(\nu)$ such that $d_{\mu' \kappa} \leq \concost_\kappa$.
In this case, for every $\mu\in \wbarN(\kappa)\setminus \wbarN(\nu)$, we have
%
\begin{equation*}
d_{\mu \nu} \leq d_{\mu \kappa} + d_{\mu' \kappa} + d_{\mu' \nu}  
 	\le  \alpha^\ast_\kappa + \concost_\kappa + \alpha^\ast_{\nu}
  \leq \concost_\nu + 2\alpha_{\nu}^\ast,
\end{equation*}
%
using the triangle inequality, complementary slackness, and (PD.\ref{PD:assign:cost}).
By summing over all $\mu\in \wbarN(\kappa) \setminus \wbarN(\nu)$, it
follows that $D(\wbarN(\kappa) \setminus \wbarN(\nu), \nu) \leq
\concost_\nu + 2\alpha_{\nu}^\ast$.

\item{\mycase{2}}
 Every $\mu'\in \wbarN(\kappa)\cap \wbarN(\nu)$
has $d_{\mu'\kappa} > \concost_\kappa$. Since $\concost_{\kappa} = D(\wbarN(\kappa),\kappa)$,
this implies that
$D(\wbarN(\kappa) \setminus \wbarN(\nu),\kappa)\leq \concost_{\kappa}$. Therefore,
choosing an arbitrary $\mu'\in \wbarN(\kappa)\cap \wbarN(\nu)$,
we obtain
%
\begin{equation*}
  D(\wbarN(\kappa) \setminus \wbarN(\nu), \nu) 
	\leq  D(\wbarN(\kappa) \setminus \wbarN(\nu), \kappa) 
			+ d_{\mu' \kappa} + d_{\mu' \nu} 
	\leq  \concost_{\kappa} +
  \alpha_{\kappa}^\ast + \alpha_{\nu}^\ast
	\leq \concost_\nu + 2\alpha_{\nu}^\ast,
\end{equation*}
%
where we again use the triangle inequality,
complementary slackness, and  (PD.\ref{PD:assign:cost}).
%
\end{description}
%
Since the lemma holds in both cases, the proof is now complete.
\end{proof}

We now continue our estimation of the connection cost.  The next step
of our analysis is to show that 
%
\begin{equation}
	\Exp[C_\nu]\le \concost_{\nu} + \frac{2}{e}\alpha^\ast_\nu.
	\label{eqn: echs bound for connection cost}
\end{equation}
%
The argument is divided into three cases. The first, easy case is when
$\nu$ is a primary demand $\kappa$. According to the algorithm
(see Pseudocode~\ref{alg:lpr3}, Line~2), we have $C_\kappa = d_{\mu\kappa}$ with probability $\bary_{\mu}$, 
for $\mu\in \wbarN(\kappa)$. Therefore $\Exp[C_\kappa] = \concost_{\kappa}$, so
(\ref{eqn: echs bound for connection cost}) holds.

Next, we consider a non-primary demand $\nu$. Let $\kappa$
be the primary demand that $\nu$ is assigned to. We first
deal with the sub-case when $\wbarN(\kappa)\setminus
\wbarN(\nu) = \emptyset$, which is the same as
$\wbarN(\kappa) \subseteq \wbarN(\nu)$. Property (CO)
implies that $\barx_{\mu\nu} = \bary_{\mu} =
\barx_{\mu\kappa}$ for every $\mu \in \wbarN(\kappa)$, so we
have $\sum_{\mu\in\wbarN(\kappa)} \barx_{\mu\nu} =
\sum_{\mu\in\wbarN(\kappa)} \barx_{\mu\kappa} = 1$, due to
(PS.\ref{PS:one}). On the other hand, we have
$\sum_{\mu\in\wbarN(\nu)} \barx_{\mu\nu} = 1$, and
$\barx_{\mu\nu} > 0$ for all $\mu\in \wbarN(\nu)$. Therefore
$\wbarN(\kappa) = \wbarN(\nu)$ and $C_\nu$ has exactly the
same distribution as $C_\kappa$.  So this case reduces to
the first case, namely we have $\Exp[C_{\nu}] =
\concost_{\nu}$, and (\ref{eqn: echs bound for connection
  cost}) holds.

The last, and only non-trivial case is when $\wbarN(\kappa)\setminus
\wbarN(\nu)\neq\emptyset$. We handle this case in the following lemma.

%%%%%%

\begin{lemma}\label{lem: echs expected C_nu}
Assume that $\wbarN(\kappa) \setminus \wbarN(\nu) \neq \emptyset$.
Then the expected connection cost of $\nu$, conditioned on the event that at least one of 
its neighbor opens, satisfies
%
\begin{equation*}
  \Exp[C_\nu \mid \Lambda^\nu] \leq \concost_{\nu}.
\end{equation*}
\end{lemma}

\begin{proof}
The proof is similar to an analogous result in~\cite{ChudakS04,ByrkaA10}. 
For the sake of completeness we sketch here a simplified argument, adapted to our
terminology and notation.
The idea is to consider a different random process that is
easier to analyze and whose expected connection cost is not better than that in
the algorithm.

We partition $\wbarN(\nu)$ into groups $G_1,...,G_k$, where two
different facilities $\mu$ and $\mu'$ are put in the same $G_s$, where
$s\in \{1,\ldots,k\}$, if they both belong to the same set
$\wbarN(\kappa)$ for some primary demand $\kappa$. If some $\mu$ is
not a neighbor of any primary demand, then it constitutes a singleton
group.  For each $s$, let $\bard_s = D(G_s,\nu)$ be the average
distance from $\nu$ to $G_s$.  Assume that $G_1,...,G_k$ are ordered
by nondecreasing average distance to $\nu$, that is $\bard_1 \le
\bard_2 \le ... \le \bard_k$.  For each group $G_s$, we select it,
independently, with probability $g_s = \sum_{\mu\in G_s}\bary_{\mu}$.
For each selected group $G_s$,  we
open exactly one facility in $G_s$, where each $\mu\in G_s$
is opened with probability $\bary_{\mu}/\sum_{\eta\in G_s}
\bary_{\eta}$.

So far, this process is the same as that in the algorithm (if restricted to $\wbarN(\nu)$).
However, we connect $\nu$ in a slightly different way, by choosing the smallest
$s$ for which $G_s$ was selected and connecting $\nu$ to the open facility in $G_s$.
This can only increase our expected connection cost, assuming that at least one
facility in $\wbarN(\nu)$ opens, so
%
\begin{align}
  \Exp[C_\nu \mid \Lambda^\nu] &\leq \frac{1}{\Prob[\Lambda^\nu]}
  \left( \bard_1 g_1 + \bard_2 g_2 (1-g_1) + \ldots + \bard_k g_k
    (1-g_1) (1-g_2) \ldots (1-g_k) \right)
			\notag
  \\
  &\leq \frac{1}{\Prob[\Lambda^\nu]}
	\cdot \sum_{s=1}^k \bard_s g_s
	\cdot
		\left(\sum_{t=1}^k g_t \prod_{z=1}^{t-1} (1-g_z)\right)
			\label{eqn: echs ineq direct cost, step 1}
  \\
  &= \sum_{s=1}^k \bard_s g_s
			\label{eqn: echs ineq direct cost, step 2}
	\\
			&= \concost_{\nu}.
				\label{eqn: echs ineq direct cost, step 3}
\end{align}
%
The proof for inequality (\ref{eqn: echs ineq direct cost, step 1}) 
is given in \ref{sec: ECHSinequality} (note that $\sum_{s=1}^k g_s = 1$),
equality (\ref{eqn: echs ineq direct cost, step 2}) follows from
$\Prob[\Lambda^\nu] = 1 - \prod_{t=1}^k (1-g_t)
					= \sum_{t=1}^k g_t
                                        \prod_{z=1}^{t-1} (1 - g_z)$,
and (\ref{eqn: echs ineq direct cost, step 3}) follows from the definition
of the distances $\bard_s$, probabilities $g_s$, and simple algebra.
\end{proof}

Next, we show an estimate on the probability that none of $\nu$'s
neighbors is opened by the algorithm.

\begin{lemma}\label{lem: probability of not Lambda^nu}
The probability that none of $\nu$'s neighbors is opened satisfies
$\Prob[\neg\Lambda^\nu] \le 1/e$.
\end{lemma}

\begin{proof}
We use the same partition of $\wbarN(\nu)$ into groups $G_1,...,G_k$ as
in the proof of Lemma~\ref{lem: echs expected C_nu}. Denoting by
$g_s$ the probability that a group $G_s$ is selected (and thus that it
has an open facility), we have
%
\begin{equation*}
\Prob[\neg\Lambda^\nu] = \prod_{s=1}^k (1 - g_s)
			\le e^{- \sum_{s=1}^k g_s}
			= e^{-\sum_{\mu \in \wbarN(\nu)} \bary_{\mu}}
			= \frac{1}{e}.
\end{equation*}
%
In this derivation, we first use that $1-x\le e^{-x}$ holds for all $x$,
the second equality follows from $\sum_{s=1}^k g_s = \sum_{\mu \in \wbarN(\nu)} \bary_{\mu}$
and the last equality follows from 
$\sum_{\mu \in \wbarN(\nu)} \bary_{\mu} = 1$.
\end{proof}

We are now ready to estimate the unconditional expected connection cost of $\nu$
(in the case when $\wbarN(\kappa)\setminus \wbarN(\nu)\neq\emptyset$)
as follows:
%
\begin{align}
  \notag
  \Exp[C_\nu] &= \Exp[C_{\nu} \mid \Lambda^\nu] \cdot \Prob[\Lambda^\nu] 
	+ \Exp[C_{\nu} \mid \neg \Lambda^\nu] \cdot	\Prob[\neg \Lambda^\nu]
  \\
  &\leq \concost_{\nu} \cdot \Prob[\Lambda^\nu] 
		+ (\concost_{\nu} + 2\alpha_{\nu}^\ast)  \cdot \Prob[\neg \Lambda^\nu]
  \label{eqn: Cnu estimate 0}
  \\
  &= \concost_{\nu} 
	+  2\alpha_{\nu}^\ast \cdot \Prob[\neg \Lambda^\nu]
		\notag
	\\
	&\le \concost_{\nu} + \frac{2}{e}\cdot\alpha_{\nu}^\ast.
	  \label{eqn: Cnu estimate last}
\end{align}
%
In the above derivation, inequality (\ref{eqn: Cnu estimate 0})
follows from Lemmas~\ref{lem:echu indirect} and \ref{lem: echs expected C_nu}, 
and inequality (\ref{eqn: Cnu estimate last}) follows from
Lemma~\ref{lem: probability of not Lambda^nu}.

\medskip

We have thus shown that the bound (\ref{eqn: echs bound for connection cost})
holds in all three cases.
Summing over all demands $\nu$ of a client $j$, we can now bound
the expected connection cost of client $j$:
%
\begin{equation*}
  \Exp[C_j] = \textstyle\sum_{\nu\in j} \Exp[C_\nu] 
\leq {\textstyle\sum_{\nu\in j} (\concost_{\nu} + \frac{2}{e}\cdot\alpha_{\nu}^\ast) }
  = { C_j^\ast + \frac{2}{e}\cdot r_j\alpha_j^\ast}.
\end{equation*}
%
Finally, summing over all clients $j$, we obtain our bound on
the expected connection cost,
%
\begin{equation*}
 \Exp[ C_{\smallECHS}] \le C^\ast + \frac{2}{e}\cdot\LP^\ast.
\end{equation*}
% 
Therefore we have established that
our algorithm constructs a feasible integral solution with
an overall expected cost 
%
\begin{equation*}
  \label{eq:chudakall}
	 \Exp[ F_{\smallECHS} + C_{\smallECHS}]
	\le
  	F^\ast + C^\ast + \frac{2}{e}\cdot \LP^\ast = (1+2/e)\cdot \LP^\ast
  \leq (1+2/e)\cdot \OPT.
\end{equation*}
%
Summarizing, we obtain the main result of this section.

\begin{theorem}\label{thm:1736}
  Algorithm~{\ECHS} is a $(1+2/e)$-approximation algorithm for \FTFP.
\end{theorem}

%% EBGS 1.575
\section{Algorithm~{\EBGS} with Ratio $1.575$}
\label{sec: 1.575-approximation}

In this section we give our main result, a $1.575$-approximation
algorithm for $\FTFP$, where $1.575$ is the value of $\min_{\gamma\geq
  1}\max\{\gamma, 1+2/e^\gamma, \frac{1/e+1/e^\gamma}{1-1/\gamma}\}$,
rounded to three decimal digits. This matches the ratio of the best
known LP-rounding algorithm for UFL by
Byrka~{\etal}~\cite{ByrkaGS10}. 

Recall that in Section~\ref{sec: 1.736-approximation} we showed how to
compute an integral solution with facility cost bounded by $F^\ast$
and connection cost bounded by $C^\ast + 2/e\cdot\LP^\ast$. Thus,
while our facility cost does not exceed the optimal fractional
facility cost, our connection cost is significantly larger than the
connection cost in the optimal fractional solution.  A natural idea is
to balance these two ratios by reducing the connection cost at the
expense of the facility cost. One way to do this would be to increase
the probability of opening facilities, from $\bary_{\mu}$ (used in
Algorithm~{\ECHS}) to, say, $\gamma\bary_{\mu}$, for some $\gamma >
1$. This increases the expected facility cost by a factor of $\gamma$
but, as it turns out, it also reduces the probability that an indirect
connection occurs for a non-primary demand to $1/e^\gamma$ (from the
previous value $1/e$ in {\ECHS}). As a consequence, for each primary
demand $\kappa$, the new algorithm will select a facility to open from
the nearest facilities $\mu$ in $\wbarN(\kappa)$ such that the
connection values $\barx_{\mu\nu}$ sum up to $1/\gamma$, instead of
$1$ as in Algorithm {\ECHS}. It is easily seen that this will improve
the estimate on connection cost for primary demands.  These two
changes, along with a more refined analysis, are the essence of the
approach in~\cite{ByrkaGS10}, expressed in our terminology.

Our approach can be thought of as a combination of the above ideas
with the techniques of demand reduction and
adaptive partitioning that we introduced earlier. However, our
adaptive partitioning technique needs to be carefully modified,
because now we will be using a more intricate neighborhood structure,
with the neighborhood of each demand divided into two disjoint parts,
and with restrictions on how parts from different demands can overlap.

We begin by describing properties that our partitioned fractional
solution $(\barbfx,\barbfy)$ needs to satisfy. Assume that $\gamma$ is
some constant such that $1 < \gamma < 2$. As mentioned earlier,
the neighborhood $\wbarN(\nu)$ of each demand $\nu$ will be divided
into two disjoint parts.  The first part, called the \emph{close
  neighborhood} and denoted $\wbarclsnb(\nu)$, contains the facilities
in $\wbarN(\nu)$ nearest to $\nu$ with the total connection value
equal $1/\gamma$, that is $\sum_{\mu\in\wbarclsnb(\nu)} \barx_{\mu\nu}
= 1/\gamma$.  The second part, called the \emph{far neighborhood} and
denoted $\wbarfarnb(\nu)$, contains the remaining facilities in
$\wbarN(\nu)$ (so $\sum_{\mu\in\wbarfarnb(\nu)} \barx_{\mu\nu} = 1-1/\gamma$).  We
restate these definitions formally below in Property~(NB).  Recall
that for any set $A$ of facilities and a demand $\nu$, by
$D(A,\nu)$ we denote the average distance between $\nu$ and the
facilities in $A$, that is $D(A,\nu) =\sum_{\mu\in A}
d_{\mu\nu}\bary_{\mu}/\sum_{\mu\in A} \bary_{\mu}$.  We will use
notations $\clsdist(\nu)=D(\wbarclsnb(\nu),\nu)$ and
$\fardist(\nu)=D(\wbarfarnb(\nu),\nu)$ for the average distances from
$\nu$ to its close and far neighborhoods, respectively.  By the
definition of these sets and the completeness property (CO), these
distances can be expressed as
%
\begin{equation*}
\clsdist(\nu)=\gamma\sum_{\mu\in\wbarclsnb(\nu)}
			d_{\mu\nu}\barx_{\mu\nu} \quad\text{and}\quad
\fardist(\nu)=\frac{\gamma}{\gamma-1}\sum_{\mu\in\wbarfarnb(\nu)}
d_{\mu\nu}\barx_{\mu\nu}. 
\end{equation*}
%
We will also use notation $\clsmax(\nu)=\max_{\mu\in\wbarclsnb(\nu)}
d_{\mu\nu}$ for the maximum distance from $\nu$ to its close
neighborhood. The average distance from a demand $\nu$ to its overall
neighborhood $\wbarN(\nu)$ is denoted as $\concost(\nu) =
D(\wbarN(\nu), \nu) = \sum_{\mu \in \wbarN(\nu)} d_{\mu\nu}
\barx_{\mu\nu}$. It is easy to see that
\begin{equation}
  \concost(\nu) = \frac{1}{\gamma} \clsdist(\nu) + \frac{\gamma -
    1}{\gamma} \fardist(\nu).
  \label{eqn:avg dist cls dist far dist}
\end{equation}

Our partitioned solution $(\barbfx,\barbfy)$ must satisfy the same
partitioning and completeness properties as before, namely properties
(PS) and (CO) in Section~\ref{sec: adaptive partitioning}.  In
addition, it must satisfy a new neighborhood property (NB) and modified
properties (PD') and (SI'), listed below.

\begin{description}
	
      \renewcommand{\theenumii}{(\alph{enumii})}
      \renewcommand{\labelenumii}{\theenumii}

\item{(NB)} \label{NB}
	\emph{Neighborhoods.}
	For each demand $\nu \in \demandset$, its neighborhood is divided into \emph{close} and
	\emph{far} neighborhood, that is $\wbarN(\nu) = \wbarclsnb(\nu) \cup \wbarfarnb(\nu)$, where
	%
	\begin{itemize}
	\item $\wbarclsnb(\nu) \cap \wbarfarnb(\nu) = \emptyset$,
	\item $\sum_{\mu\in\wbarclsnb(\nu)} \barx_{\mu\nu} =1/\gamma$, and 
	\item if $\mu\in \wbarclsnb(\nu)$ and $\mu'\in \wbarfarnb(\nu)$ 
				then $d_{\mu\nu}\le d_{\mu'\nu}$.   
	\end{itemize}
	%
	Note that the first two conditions, together with
        (PS.\ref{PS:one}), imply that $\sum_{\mu\in\wbarfarnb(\nu)}
        \barx_{\mu\nu} =1-1/\gamma$. When defining $\wbarclsnb(\nu)$,
        in case of ties, which can occur when some facilities in
        $\wbarN(\nu)$ are at the same distance from $\nu$, we use a
        tie-breaking rule that is explained in the proof of
        Lemma~\ref{lem: PD1: primary overlap} (the only place where
        the rule is needed).

\item{(PD')} \emph{Primary demands.}
	Primary demands satisfy the following conditions:

	\begin{enumerate}
		
	\item\label{PD1:disjoint}  For any two different primary demands $\kappa,\kappa'\in P$ we have
				$\wbarclsnb(\kappa)\cap \wbarclsnb(\kappa') = \emptyset$.

	\item \label{PD1:yi} For each site $i\in\sitesset$, 
		$ \sum_{\kappa\in P}\sum_{\mu\in
                  i\cap\wbarclsnb(\kappa)}\barx_{\mu\kappa} \leq
                y_i^\ast$. In the summation, as before, we overload notation $i$ to stand for the set of
						facilities created on site $i$.
		
	\item \label{PD1:assign} Each demand $\nu\in\demandset$ is assigned
        to one primary demand $\kappa\in P$ such that

  			\begin{enumerate}
	
				\item \label{PD1:assign:overlap} $\wbarclsnb(\nu) \cap \wbarclsnb(\kappa) \neq \emptyset$, and
				%
				\item \label{PD1:assign:cost}
          $\clsdist(\nu)+\clsmax(\nu) \geq
          \clsdist(\kappa)+\clsmax(\kappa)$.
          %
			\end{enumerate}

	\end{enumerate}
	
\item{(SI')} \emph{Siblings}. For any pair $\nu,\nu'\in\demandset$ of different siblings we have
  \begin{enumerate}

	\item \label{SI1:siblings disjoint}
		  $\wbarN(\nu)\cap \wbarN(\nu') = \emptyset$.
		
	\item \label{SI1:primary disjoint} If $\nu$ is assigned to a primary demand $\kappa$ then
 		$\wbarN(\nu')\cap \wbarclsnb(\kappa) = \emptyset$. In particular, by Property~(PD'.\ref{PD1:assign:overlap}),
		this implies that different sibling demands are assigned to different primary demands, since $\wbarclsnb(\nu')$ is a subset of $\wbarN(\nu')$.

	\end{enumerate}
	
\end{description}

%%%%%%%%%%%%%%%%%

\paragraph{Modified adaptive partitioning.}
To obtain a fractional solution with the above properties, we employ a
modified adaptive partitioning algorithm. As in Section~\ref{sec:
  adaptive partitioning}, we have two phases.  In Phase~1 we split
clients into demands and create facilities on sites, while in Phase~2
we augment each demand's connection values $\barx_{\mu\nu}$ so that the total connection
value of each demand $\nu$ is $1$. As the partitioning algorithm proceeds, for any demand $\nu$,
$\wbarN(\nu)$ denotes the set of facilities with $\barx_{\mu\nu} > 0$;
hence the notation $\wbarN(\nu)$ actually represents a dynamic set which gets fixed 
once the partitioning algorithm concludes both Phase 2. On the
other hand, $\wbarclsnb(\nu)$ and $\wbarfarnb(\nu)$ refer to the close
and far neighborhoods at the time when $\wbarN(\nu)$ is fixed.

Similar to the algorithm in Section~\ref{sec: adaptive partitioning},
Phase~1 runs in iterations. Fix some iteration and consider any client
$j$.  As before, $\wtildeN(j)$ is the neighborhood of $j$ with respect
to the yet unpartitioned solution, namely the set of facilities $\mu$
such that $\tildex_{\mu j}>0$. Order the facilities in this set as
$\wtildeN(j) = \braced{\mu_1,...,\mu_q}$ with non-decreasing distance
from $j$, that is $d_{\mu_1 j} \leq d_{\mu_2 j} \leq \ldots \leq
d_{\mu_q j}$. Without loss of generality,
there is an index $l$ for which $\sum_{s=1}^l \tildex_{\mu_s j} =
1/\gamma$, since we can always split one facility to achieve
this. Then we define $\wtildeclsnb(j) = \braced{\mu_1,...,\mu_l}$. 
(Unlike close neighborhoods of demands, $\wtildeclsnb(j)$ can vary over time.)
We also use notation
%
\begin{equation*}
\tcccls(j) =  D(\wtildeclsnb(j), j) = \gamma\sum_{\mu\in\wtildeclsnb(j)} d_{\mu j} \tildex_{\mu j}
			\quad\textrm{ and }\quad
 \dmaxcls(j) = \max_{\mu \in \wtildeclsnb(j)} d_{\mu j}. 
\end{equation*}
%

When the iteration starts, we first find a not-yet-exhausted client
$p$ that minimizes the value of $\tcccls(p) + \dmaxcls(p)$ and create
a new demand $\nu$ for $p$.  Now we have two cases:
%
\begin{description}
%
\item{\mycase{1}} $\wtildeclsnb(p) \cap \wbarN(\kappa)\neq\emptyset$
  for some existing primary demand $\kappa\in P$.  In this case we
  assign $\nu$ to $\kappa$. As before, if there are multiple such
  $\kappa$, we pick any of them. We also fix $\barx_{\mu \nu} \assign
  \tildex_{\mu p}$ and $\tildex_{\mu p}\assign 0$ for each $\mu \in
  \wtildeN(p)\cap \wbarN(\kappa)$. Note that although we
  check for overlap between $\wtildeclsnb(p)$ and $\wbarN(\kappa)$,
  the facilities we actually move into $\wbarN(\nu)$ include all
  facilities in the intersection of $\wtildeN(p)$, a bigger set, with
  $\wbarN(\kappa)$.

  At this time, the total connection value 
	between $\nu$ and $\mu\in \wbarN(\nu)$ is at most $1/\gamma$,
	 since $\sum_{\mu \in \wbarN(\kappa)}\bary_{\mu} = 1/\gamma$ 
	(this follows from the definition of neighborhoods for new primary demands in Case~2 below) 
	and  we have $\wbarN(\nu) \subseteq \wbarN(\kappa)$ at this point. Later
  in Phase 2 we will add additional facilities from $\wtildeN(p)$ to
  $\wbarN(\nu)$ to make $\nu$'s total connection value equal to $1$.

%
\item{\mycase{2}} $\wtildeclsnb(p) \cap \wbarN(\kappa) = \emptyset$
  for all existing primary demands $\kappa\in P$.  In this case we
  make $\nu$ a primary demand (that is, add it to $P$) and assign it
  to itself.  We then move the facilities from $\wtildeclsnb(p)$ to
  $\wbarN(\nu)$, that is for $\mu \in \wtildeclsnb(p)$ we set
  $\barx_{\mu \nu}\assign \tildex_{\mu p}$ and $\tildex_{\mu p}\set
  0$.

  It is easy to see that the total connection value of $\nu$ to
  $\wbarN(\nu)$ is now exactly $1/\gamma$, that is
	$\sum_{\mu \in \wbarN(\nu)}\bary_{\mu} = 1/\gamma$.
Moreover, facilities
  remaining in $\wtildeN(p)$ are all farther away from $\nu$ than
  those in $\wbarN(\nu)$. As we add only facilities from $\wtildeN(p)$
  to $\wbarN(\nu)$ in Phase~2, the final $\wbarclsnb(\nu)$ contains
  the same set of facilities as the current set $\wbarN(\nu)$.
  (More precisely, $\wbarclsnb(\nu)$ consists of the facilities that
	either are currently in $\wbarN(\nu)$ or were obtained from splitting
	the facilities currently in $\wbarN(\nu)$.)
%
\end{description}
%
Once all clients are exhausted, that is, each client $j$ has $r_j$
demands created, Phase~1 concludes. We then run Phase~2, the
augmenting phase, following the same steps as in Section~\ref{sec:
  adaptive partitioning}.  For each client $j$ and each demand $\nu\in
j$ with total connection value to $\wbarN(\nu)$ less than $1$
(that is, $\sum_{\mu\in\wbarN(\nu)} \barx_{\mu\nu} < 1$),
we use our $\AugmentToUnit()$
procedure to add additional facilities (possibly split, if necessary)
from $\wtildeN(j)$ to $\wbarN(\nu)$ to make the total connection value
between $\nu$ and $\wbarN(\nu)$ equal $1$.

\medskip

This completes the description of the partitioning
algorithm. Summarizing, for each client $j\in\clientset$ we 
created $r_j$ demands on the same point as $j$, and we created a number
of facilities at each site $i\in\sitesset$. Thus computed sets of
demands and facilities are denoted $\demandset$ and $\facilityset$,
respectively.  For each facility $\mu\in i$ we defined its fractional
opening value $\bary_\mu$, $0\le \bary_\mu\le 1$, and for each demand
$\nu\in j$ we defined its fractional connection value
$\barx_{\mu\nu}\in \braced{0,\bary_\mu}$.  The connections with
$\barx_{\mu\nu} > 0$ define the neighborhood $\wbarN(\nu)$. The facilities in
$\wbarN(\nu)$ that are closest to $\nu$ and have total connection value from $\nu$ equal
$1/\gamma$ form the close neighborhood $\wbarclsnb(\nu)$, while the remaining facilities
in $\wbarN(\nu)$ form the far neighborhood
$\wbarfarnb(\nu)$. It remains to show that this partitioning satisfies all the desired
properties.

%%%%%%%

\medskip
\paragraph{Correctness of partitioning.}
We now argue that our partitioned fractional solution $(\barbfx,\barbfy)$
satisfies all the stated properties. Properties~(PS), (CO) and (NB) are
directly enforced by the algorithm.

(PD'.\ref{PD1:disjoint}) holds because for each primary demand
$\kappa\in p$, $\wbarclsnb(\kappa)$ is the same set as
$\wtildeclsnb(p)$ at the time when $\kappa$ was created, and
$\wtildeclsnb(p)$ is removed from $\wtildeN(p)$ right after this
step. Further, the partitioning algorithm makes $\kappa$ a primary
demand only if $\wtildeclsnb(p)$ is disjoint from the set
$\wbarN(\kappa')$ of all existing primary demands $\kappa'$ at that
iteration, but these neighborhoods are the same as the final close
neighborhoods $\wbarclsnb(\kappa')$.

The justification of (PD'.\ref{PD1:yi}) is similar to that for
(PD.\ref{PD:yi}) from Section~\ref{sec: adaptive partitioning}. All
close neighborhoods of primary demands are disjoint, due to
(PD'.\ref{PD1:disjoint}), so each facility $\mu \in i$ can appear in
at most one $\wbarclsnb(\kappa)$, for some $\kappa\in P$. Condition
(CO) implies that $\bary_{\mu} = \barx_{\mu\kappa}$ for $\mu \in \wbarclsnb(\kappa)$.
As a result, the summation on
the left-hand side is not larger than $\sum_{\mu\in i}\bary_{\mu} = y_i^\ast$.

Regarding (PD'.\ref{PD1:assign:overlap}), at first glance this
property seems to follow directly from the algorithm, as we only
assign a demand $\nu$ to a primary demand $\kappa$ when $\wbarN(\nu)$
at that iteration overlaps with $\wbarN(\kappa)$ (which is equal to
the final value of $\wbarclsnb(\kappa)$).  However, it is a little
more subtle, as the final $\wbarclsnb(\nu)$ may contain facilities
added to $\wbarN(\nu)$ in Phase 2. Those facilities may turn out to be
closer to $\nu$ than some facilities in $\wbarN(\kappa) \cap
\wtildeN(j) $ (not $\wtildeN_{\cls}(j)$) that we added to
$\wbarN(\nu)$ in Phase 1. If the final $\wbarclsnb(\nu)$ consists only of
facilities added in Phase 2, we no longer have the desired overlap of
$\wbarclsnb(\kappa)$ and $\wbarclsnb(\nu)$. Luckily this bad scenario
never occurs. We postpone the proof of this property to
Lemma~\ref{lem: PD1: primary overlap}.  The proof of
(PD'.\ref{PD1:assign:cost}) is similar to that of Lemma~\ref{lem:
  PD:assign:cost holds}, and we defer it to Lemma~\ref{lem: PD1:
  primary optimal}.

(SI'.\ref{SI1:siblings disjoint}) follows directly from the algorithm
because for each demand $\nu\in j$, all facilities added to
$\wbarN(\nu)$ are immediately removed from $\wtildeN(j)$ and each
facility is added to $\wbarN(\nu)$ of exactly one demand $\nu \in j$.
Splitting facilities obviously preserves (SI'.\ref{SI1:siblings disjoint}).

The proof of (SI'.\ref{SI1:primary disjoint}) is similar to that of
Lemma~\ref{lem: property SI:primary disjoint holds}. If $\kappa=\nu$
then (SI'.\ref{SI1:primary disjoint}) follows from
(SI'.\ref{SI1:siblings disjoint}), so we can assume that
$\kappa\neq\nu$.  Suppose that $\nu'\in j$ is assigned to $\kappa'\in
P$ and consider the situation after Phase~1. By the way we reassign
facilities in Case~1, at this time we have $\wbarN(\nu)\subseteq
\wbarN(\kappa) = \wbarclsnb(\kappa)$ and $\wbarN(\nu')\subseteq
\wbarN(\kappa') =\wbarclsnb(\kappa')$, so $\wbarN(\nu')\cap
\wbarclsnb(\kappa) = \emptyset$, by (PD'.\ref{PD1:disjoint}).
Moreover, we have $\wtildeN(j) \cap \wbarclsnb(\kappa) = \emptyset$
after this iteration, because any facilities that were also in
$\wbarclsnb(\kappa)$ were removed from $\wtildeN(j)$ when $\nu$ was
created. In Phase~2, augmentation does not change $\wbarclsnb(\kappa)$
and all facilities added to $\wbarN(\nu')$ are from the set
$\wtildeN(j)$ at the end of Phase 1, which is a subset of the set
$\wtildeN(j)$ after this iteration, since $\wtildeN(j)$ can only shrink. 
So the condition (SI'.\ref{SI1:primary disjoint}) will
remain true.

%%%%%%%%%%%%%%%

\begin{lemma} \label{lem: PD1: primary overlap}
  Property (PD'.\ref{PD1:assign:overlap}) holds.
\end{lemma}

\begin{proof}
  Let $j$ be the client for which $\nu\in j$. We consider an iteration
  when we create $\nu$ from $j$ and assign it to $\kappa$, and
  within this proof, notation $\wtildeclsnb(j)$ and $\wtildeN(j)$
  will refer to the value of the sets at this particular time.  
% if reviewers complain, we can introduce superscript v to indicate that
% as we do elsewhere
At this time, $\wbarN(\nu)$ is initialized to $\wtildeN(j)\cap
  \wbarN(\kappa)$.  Recall that $\wbarN(\kappa)$ is now equal to the
  final $\wbarclsnb(\kappa)$ (taking into account facility splitting). We
  would like to show that the set $\wtildeclsnb(j)\cap
  \wbarclsnb(\kappa)$ (which is not empty) will be included in
  $\wbarclsnb(\nu)$ at the end. Technically speaking, this will not be
  true due to facility splitting, so we need to rephrase this claim
  and the proof in terms of the set of facilities obtained after the
  algorithm completes.

%%%%%%%%%%%%%%%%%%%%%%%%%%%%%%%%%%%%%%%%%
\begin{figure}[ht]
\begin{center}
\includegraphics[width=3.2in]{proof_of_lemma_PD'3a.pdf}
\caption{Illustration of the sets $\wbarN(\nu)$, $A$, $B$,
  $E^-$ and $E^+$ in the proof of Lemma~\ref{lem: PD1:
    primary overlap}. Let $X \Subset Y$ mean that the facility
	sets $X$ is obtained from $Y$ by splitting facilities.
	We then have $A \Subset \wtildeN(j)$, 
	$B \Subset  \wtildeclsnb(j) \cap \wbarclsnb(\kappa)$, 
	$E^- \Subset  \wtildeclsnb(j) - \wbarclsnb(\kappa)$, 
	$E^+ \Subset \wtildeN(j) - \wtildeclsnb(j)$.}
\label{fig: sets lemma PD'3a}
\end{center}
\end{figure}

  We define the sets $A$, $B$, $E^-$ and $E^+$ as the subsets of
  $\facilityset$ (the final set of facilities) that were obtained from
  splitting facilities in the sets $\wtildeN(j)$, $\wtildeclsnb(j)\cap
  \wbarclsnb(\kappa)$, $\wtildeclsnb(j) - \wbarclsnb(\kappa)$ and
  $\wtildeN(j) - \wtildeclsnb(j)$, respectively.  (See
  Figure~\ref{fig: sets lemma PD'3a}.)  We claim that at the end
  $B\subseteq \wbarclsnb(\nu)$, with the caveat that the ties in the
  definition of $\wbarclsnb(\nu)$ are broken in favor of the
  facilities in $B$.  (This is the tie-breaking rule that we mentioned
  in the definition of $\wbarclsnb(\nu)$.)  This will be sufficient to
  prove the lemma because $B\neq\emptyset$, by the algorithm.

  We now prove this claim. In this paragraph $\wbarN(\nu)$ denotes the
  final set $\wbarN(\nu)$ after both phases are completed. Thus the total
connection value of $\wbarN(\nu)$ to $\nu$ is $1$.
	Note first that
  $B\subseteq \wbarN(\nu) \subseteq A$, because we never remove
  facilities from $\wbarN(\nu)$ and we only add facilities from
  $\wtildeN(j)$.  Also, $B\cup E^-$ represents the facilities obtained
  from $\wtildeclsnb(j)$, so $\sum_{\mu\in B\cup E^-} \bary_{\mu} =
  1/\gamma$.  This and $B\subseteq \wbarN(\nu)$ implies that the total
  connection value of $B\cup (\wbarN(\nu)\cap E^-)$ to $\nu$ is at
  most $1/\gamma$. But all facilities in $B\cup (\wbarN(\nu)\cap E^-)$
  are closer to $\nu$ (taking into account our tie breaking in property (NB))
 	than those in $E^+\cap \wbarN(\nu)$. It follows
  that $B\subseteq \wbarclsnb(\nu)$, completing the proof.
\end{proof}

%%%%%%%%%%%%%%

\begin{lemma}\label{lem: PD1: primary optimal}
  Property (PD'.\ref{PD:assign:cost}) holds.
\end{lemma}

\begin{proof}
This proof is similar to that for Lemma~\ref{lem: PD:assign:cost holds}.
For a client $j$ and demand $\eta$, we will write
$\tcccls^\eta(j)$ and $\dmaxcls^\eta(j)$ to denote the values of
$\tcccls(j)$ and $\dmaxcls(j)$ at the time when $\eta$
was created. (Here $\eta$ may or may not be a demand of client $j$).

Suppose $\nu \in j$ is assigned to a primary demand $\kappa \in p$.
By the way primary demands are constructed in the partitioning
algorithm, $\wtildeclsnb(p)$ becomes $\wbarN(\kappa)$, which is equal
to the final value of $\wbarclsnb(\kappa)$. So we have
$\clsdist(\kappa) = \tcccls^\kappa (p)$ and $\clsmax(\kappa) =
\dmaxcls^\kappa(p)$. Further, since we choose $p$ to minimize
$\tcccls(p) + \dmaxcls(p)$, we have that $\tcccls^\kappa(p) +
\dmaxcls^\kappa(p) \leq \tcccls^\kappa(j) + \dmaxcls^\kappa(j)$.

Using an argument analogous to that in the proof of Lemma~\ref{lem: tcc optimal}, 
our modified partitioning algorithm guarantees that
  $\tcccls^{\kappa}(j) \leq \tcccls^{\nu}(j) \leq \clsdist(\nu)$ and
  $\dmaxcls^{\kappa}(j) \leq \dmaxcls^{\nu}(j) \leq \clsmax(\nu)$ since $\nu$ was
  created later.
  Therefore, we have
%
  \begin{align*}
    \clsdist(\kappa) + \clsmax(\kappa) &= \tcccls^{\kappa}(p) +	\dmaxcls^{\kappa}(p) 
					\\
					&\leq \tcccls^{\kappa}(j) + \dmaxcls^{\kappa}(j) 
					\leq \tcccls^{\nu}(j) + \dmaxcls^{\nu}(j) 
					\leq \clsdist(\nu) + \clsmax(\nu),
  \end{align*}
%
completing the proof.
\end{proof}

%%%%%%%%

Now we have completed the proof that the computed partitioning satisfies
all the required properties. 


\paragraph{Algorithm~{\EBGS}.}
The complete algorithm starts with solving the LP(\ref{eqn:fac_primal}) and
computing the partitioning described earlier in this section.  Given
the partitioned fractional solution $(\barbfx, \barbfy)$ with the
desired properties, we start the process of opening facilities and
making connections to obtain an integral solution. To this end, for
each primary demand $\kappa\in P$, we open exactly one facility
$\phi(\kappa)$ in $\wbarclsnb(\kappa)$, where each
$\mu\in\wbarclsnb(\kappa)$ is chosen as $\phi(\kappa)$ with
probability $\gamma\bary_{\mu}$. For all facilities
$\mu\in\facilityset - \bigcup_{\kappa\in P}\wbarclsnb(\kappa)$, we
open them independently, each with probability
$\gamma\bary_{\mu}$. 

We claim that all probabilities are well-defined, that is
$\gamma\bary_{\mu} \le 1$ for all $\mu$. Indeed, if $\bary_{\mu}>0$ then
$\bary_{\mu} = \barx_{\mu\nu}$ for some $\nu$, by Property~(CO).
If $\mu\in \wbarclsnb(\nu)$ then the definition of close
neighborhoods implies that $\barx_{\mu\nu} \le 1/\gamma$.
If $\mu\in \wbarfarnb(\nu)$ then
$\barx_{\mu\nu} \le 1-1/\gamma \le 1/\gamma$, because $\gamma < 2$.
Thus $\gamma\bary_{\mu} \le 1$, as claimed.

Next, we connect demands to facilities.  Each primary demand
$\kappa\in P$ will connect to the only open facility $\phi(\kappa)$ in
$\wbarclsnb(\kappa)$.  For each non-primary demand $\nu\in \demandset
- P$, if there is an open facility in $\wbarclsnb(\nu)$ then we
connect $\nu$ to the nearest such facility. Otherwise, we connect
$\nu$ to the nearest far facility in $\wbarfarnb(\nu)$ if one is
open. Otherwise, we connect $\nu$ to its \emph{target facility}
$\phi(\kappa)$, where $\kappa$ is the primary demand that $\nu$ is
assigned to.

%%%%%%%%%%%

\paragraph{Analysis.}
By the algorithm, for each client $j$, all its $r_j$ demands are connected to
open facilities. If two different siblings $\nu,\nu'\in j$ are assigned, respectively,
to primary demands $\kappa$, $\kappa'$ then, by
Properties~(SI'.\ref{SI1:siblings disjoint}), (SI'.\ref{SI1:primary
  disjoint}), and (PD'.\ref{PD1:disjoint}) we have
%
\begin{equation*}
( \wbarN(\nu) \cup \wbarclsnb(\kappa)) \cap (\wbarN(\nu')\cup \wbarclsnb(\kappa')) = \emptyset.
\end{equation*}
%
This condition guarantees that $\nu$ and $\nu'$ are assigned to different facilities,
regardless whether they are connected to a neighbor facility or to its target facility.
Therefore the computed solution is feasible.

\medskip

We now estimate the cost of the solution computed by Algorithm {\EBGS}. The lemma
below bounds the expected facility cost.

%%%%%%%%%%

\begin{lemma} \label{lem: EBGS facility cost}
The expectation of facility cost $F_{\smallEBGS}$ of Algorithm~{\EBGS} is at most $\gamma F^\ast$.
\end{lemma}

\begin{proof}
By the algorithm, each facility $\mu\in \facilityset$ is opened with
probability $\gamma \bary_{\mu}$, independently of whether it belongs to the
close neighborhood of a primary demand or not. Therefore, by
  linearity of expectation, we have that the expected facility cost is
%
\begin{equation*}
	\Exp[F_{\smallEBGS}] = \sum_{\mu \in \facilityset} f_\mu \gamma \bary_{\mu} 
			= \gamma \sum_{i\in \sitesset} f_i \sum_{\mu\in i} \bary_{\mu} 
			= \gamma \sum_{i \in \sitesset} f_i y_i^\ast = \gamma F^\ast,
\end{equation*}
%
where the third equality follows from (PS.\ref{PS:yi}).
\end{proof}

%%%%%%%%%%%

\medskip

In the remainder of this section we focus on the connection cost. Let $C_{\nu}$ be the
random variable representing the connection cost of a demand $\nu$. Our objective is
to show that the expectation of $\nu$ satisfies
%
\begin{equation}
\Exp[C_\nu]	\leq \concost(\nu) \cdot \max\left\{\frac{1/e+1/e^\gamma}{1-1/\gamma}, 1 + \frac{2}{e^\gamma}\right\}.
		\label{eqn: expectation of C_nu for EBGS}
\end{equation}
%
If $\nu$ is a primary demand then, due to the algorithm, we have $\Exp[C_{\nu}] =
\clsdist(\nu) \le \concost(\nu)$, so (\ref{eqn: expectation of C_nu for EBGS}) is
easily satisfied.

Thus for the rest of the argument we will focus on the case when $\nu$
is a non-primary demand.  Recall that the
algorithm connects $\nu$ to the nearest open facility in
$\wbarclsnb(\nu)$ if at least one facility in $\wbarclsnb(\nu)$ is
open. Otherwise the algorithm connects $\nu$ to the nearest open
facility in $\wbarfarnb(\nu)$, if any. In the event that no facility in
$\wbarN(\nu)$ opens, the algorithm will connect $\nu$ to its target
facility $\phi(\kappa)$, where $\kappa$ is the primary demand that
$\nu$ was assigned to, and $\phi(\kappa)$ is the only facility open in
$\wbarclsnb(\kappa)$. Let $\Lambda^\nu$ denote the event that at least
one facility in $\wbarN(\nu)$ is open and $\Lambda^\nu_{\cls}$ be the
event that at least one facility in $\wbarclsnb(\nu)$ is open.
$\neg \Lambda^\nu$ denotes the complement event of $\Lambda^\nu$, that is,
the event that none of $\nu$'s neighbors opens. 
We want to estimate the following three conditional expectations: 
%
\begin{equation*}
  \Exp[C_{\nu} \mid
  \Lambda^\nu_{\cls}],\quad \Exp[C_{\nu} \mid \Lambda^\nu \wedge \neg
  \Lambda^\nu_{\cls}], \quad\text{and}\quad \Exp[C_{\nu} \mid \neg \Lambda^\nu], 
\end{equation*}
%
and their associated probabilities.

We start with a lemma dealing with the third expectation,
$\Exp[C_\nu\mid\neg \Lambda^{\nu}] = \Exp[d_{\phi(\kappa)\nu} \mid
\Lambda^{\nu}]$. The proof of this lemma relies on
Properties~(PD'.\ref{PD1:assign:overlap}) and
(PD'.\ref{PD1:assign:cost}) of modified partitioning and follows the
reasoning in the proof of a similar lemma
in~\cite{ByrkaGS10,ByrkaA10}.

%%%%%%%

\begin{lemma}\label{lem: EBGS target connection cost}
Assuming that no facility in $\wbarN(\nu)$ opens, the expected connection
cost of $\nu$ is
%
\begin{equation}
  \Exp[C_{\nu} \mid \neg \Lambda^{\nu}] \leq
  \clsdist(\nu) + 2\fardist(\nu).
  \label{eqn: expected connection cost target facility}
\end{equation}
%
\end{lemma}

\begin{proof}
It suffices to show a stronger inequality
\begin{equation}
  \Exp[C_{\nu} \mid \neg \Lambda^{\nu}] \leq
  \clsdist(\nu) + \clsmax(\nu) + \fardist(\nu)
			\label{eqn: lemma ebgs indirect connection cost},
\end{equation}
which then implies (\ref{eqn: expected connection cost
  target facility}) because $\clsmax(\nu) \leq
\fardist(\nu)$.  The proof of (\ref{eqn: lemma ebgs indirect
  connection cost}) is similar to that in
\cite{ByrkaA10}. For the sake of completeness, we provide it
here, formulated in our terminology and notation.

Assume that the event $\neg \Lambda^{\nu}$ is true, that is Algorithm~{\EBGS}
does not open any facility in $\wbarN(\nu)$.
Let $\kappa$ be the primary demand that $\nu$ was assigned to. Also let
%K
\begin{equation*}
K = \wbarclsnb(\kappa) \setminus \wbarN(\nu), \quad
V_{\cls} = \wbarclsnb(\kappa) \cap \wbarclsnb(\nu) \quad \textrm{and}\quad 
V_{\far} = \wbarclsnb(\kappa) \cap \wbarfarnb(\nu).
\end{equation*}
% 
Then $K, V_{\cls}, V_{\far}$ form a partition of
$\wbarclsnb(\kappa)$, that is, they are disjoint and their union is $\wbarclsnb(\kappa)$.
Moreover, we have that $K$ is not empty, because Algorithm~{\EBGS}
opens some facility in $\wbarclsnb(\kappa)$ and this facility cannot be in $V_{\cls}\cup V_{\far}$,
by our assumption. 
We also have that $V_{\cls}$ is not empty due to (PD'.\ref{PD1:assign:overlap}). 

Recall that $D(A,\eta) = \sum_{\mu\in A}d_{\mu\eta}\bary_{\mu}/\sum_{\mu\in A}\bary_{\mu}$
is the average distance between a demand $\eta$ and the facilities in a set $A$. We shall show that
%
\begin{equation}
	 D(K, \nu) \leq \clsdist(\kappa)+\clsmax(\kappa) + \fardist(\nu).
				\label{eqn: bound on D(K,nu)}
\end{equation}
%
This is sufficient, because, by the algorithm, $D(K,\nu)$ is exactly 
the expected connection cost for demand $\nu$ conditioned on
the event that none of $\nu$'s neighbors 
opens, that is the left-hand side of (\ref{eqn: lemma ebgs indirect connection cost}).
Further, (PD'.\ref{PD1:assign:cost}) states that 
$\clsdist(\kappa)+\clsmax(\kappa) \le \clsdist(\nu) + \clsmax(\nu)$, and thus
(\ref{eqn: bound on D(K,nu)})  implies (\ref{eqn: lemma ebgs indirect connection cost}).

\medskip

The proof of (\ref{eqn: bound on D(K,nu)}) is by analysis of several cases.
%

\medskip
\noindent
{\mycase{1}} $D(K, \kappa) \leq \clsdist(\kappa)$. For any
facility $\mu \in V_{\cls}$ (recall that $V_{\cls}\neq\emptyset$), 
we have $d_{\mu\kappa} \leq \clsmax(\kappa)$ 
and $d_{\mu\nu} \leq \clsmax(\nu) \leq \fardist(\nu)$. Therefore, using the
case assumption, we get
	$D(K,\nu) \leq D(K,\kappa) + d_{\mu\kappa} + d_{\mu\nu} 
				\leq \clsdist(\kappa) + \clsmax(\kappa) + \fardist(\nu)$.

\medskip
\noindent
{\mycase{2}} There exists a facility $\mu\in V_{\cls}$ such that
  $d_{\mu\kappa} \leq \clsdist(\kappa)$. Since $\mu\in V_{\cls}$, we infer
  that $d_{\mu\nu} \leq \clsmax(\nu) \leq \fardist(\nu)$.  Using
  $\clsmax(\kappa)$ to bound $D(K, \kappa)$, we have $D(K, \nu)
  \leq D(K, \kappa) + d_{\mu\kappa} + d_{\mu\nu} \leq
  \clsmax(\kappa) + \clsdist(\kappa) + \fardist(\nu)$.

\medskip
\noindent
{\mycase{3}} In this case we assume that neither of Cases~1 and 2 applies, that is
 $D(K, \kappa) > \clsdist(\kappa)$ and every $\mu \in V_{\cls}$ satisfies
 $d_{\mu\kappa} >  \clsdist(\kappa)$. This implies that
$D(K\cup V_{\cls}, \kappa) > \clsdist(\kappa) = D(\wbarclsnb(\kappa), \kappa)$.
Since sets $K$, $V_{\cls}$ and $V_{\far}$ form a partition of $\wbarclsnb(\kappa)$,
we obtain that in this case $V_{\far}$ is not
empty and $D(V_{\far}, \kappa) < \clsdist(\kappa)$. 
Let $\delta = \clsdist(\kappa) - D(V_{\far}, \kappa) > 0$. 
We now have two sub-cases:
%
\begin{description}
	
\item{\mycase{3.1}} {$D(V_{\far}, \nu) \leq \fardist(\nu) + \delta$}.
  Substituting $\delta$, this implies that $D(V_{\far}, \nu) +
  D(V_{\far},\kappa) \le \clsdist(\kappa) + \fardist(\nu)$.  From the
  definition of the average distance $D(V_{\far},\kappa)$ and
  $D(V_{\far}, \nu)$, we obtain that there exists some $\mu \in
  V_{\far}$ such that $d_{\mu\kappa} + d_{\mu\nu} \leq
  \clsdist(\kappa) + \fardist(\nu)$.  Thus $D(K, \nu) \leq D(K,
  \kappa) + d_{\mu\kappa} + d_{\mu\nu} \leq \clsmax(\kappa) +
  \clsdist(\kappa) + \fardist(\nu)$.

\item{\mycase{3.2}} {$D(V_{\far}, \nu) > \fardist(\nu) + \delta$}.
  The case assumption implies that $V_{\far}$ is a proper subset of
  $\wbarfarnb(\nu)$, that is $\wbarfarnb(\nu) \setminus V_{\far}
  \neq\emptyset$.  Let $\hat{y} = \gamma \sum_{\mu\in V_{\smallfar}}
  \bary_{\mu}$.  We can express $\fardist(\nu)$ using $\hat{y}$ as
  follows
%
\begin{equation*}
\fardist(\nu) = D(V_{\far},\nu) \frac{\hat{y}}{\gamma-1} +
    D(\wbarfarnb(\nu)\setminus V_{\far}, \nu) \frac{\gamma-1-\hat{y}}{\gamma-1}.
\end{equation*}
%
Then, using the case condition and simple algebra, we have
%
  \begin{align}
    \clsmax(\nu) &\leq D(\wbarfarnb(\nu) \setminus V_{\far}, \nu) 
			\notag
		\\
		&\leq \fardist(\nu) - \frac{\hat{y}\delta}{\gamma-1-\hat{y}} 
		\leq \fardist(\nu) - \frac{\hat{y}\delta}{1-\hat{y}},
			\label{eqn: case 3, bound on C_cls^max(nu)}
  \end{align}
%
where the last step follows from $1 < \gamma < 2$. 

On the other hand, since $K$, $V_{\cls}$, and $V_{\far}$ form a partition of $\wbarclsnb(\kappa)$,
we have
$\clsdist(\kappa) = (1-\hat{y}) D(K\cup V_{\cls}, \kappa) + \hat{y} D(V_{\far}, \kappa)$.
Then using the definition of $\delta$ we obtain
%
\begin{equation}
    D(K \cup V_{\cls}, \kappa) = \clsdist(\kappa) + \frac{\hat{y}\delta}{1-\hat{y}}.
				\label{eqn: formula for D(V_cls,kappa)}
\end{equation}
%
  Now we are essentially done. If there exists some $\mu \in V_{\cls}$ such
  that $d_{\mu\kappa} \leq \clsdist(\kappa) +
  \hat{y}\delta/(1-\hat{y})$, then	we have
%
  \begin{align*}
    D(K, \nu) &\leq D(K, \kappa) + d_{\mu\kappa} + d_{\mu\nu} \\
    &\leq \clsmax(\kappa) + \clsdist(\kappa) +
    			\frac{\hat{y}\delta}{1-\hat{y}}
    + \clsmax(\nu)\\
    &\leq \clsmax(\kappa) + \clsdist(\kappa) + \fardist(\nu),
  \end{align*}
%
where we used (\ref{eqn: case 3, bound on C_cls^max(nu)}) in the last step.
  Otherwise, from (\ref{eqn: formula for D(V_cls,kappa)}),
we must have $D(K, \kappa) \leq \clsdist(\kappa) +
  \hat{y}\delta/(1-\hat{y})$. Choosing any $\mu \in V_{\cls}$, it follows that
%
  \begin{align*}
    D(K, \nu) &\leq D(K, \kappa) + d_{\mu\kappa} + d_{\mu\nu} \\
    &\leq \clsdist(\kappa) + \frac{\hat{y}\delta}{1-\hat{y}} +
    		\clsmax(\kappa)  + \clsmax(\nu)\\
    &\leq \clsdist(\kappa) + \clsmax(\kappa) + \fardist(\nu),
  \end{align*}
%
again using (\ref{eqn: case 3, bound on C_cls^max(nu)}) in the last step.

\end{description}

This concludes the proof of (\ref{eqn: expected connection
  cost target facility}).  As explained earlier,
Lemma~\ref{lem: EBGS target connection cost} follows.
\end{proof}

Next, we derive some estimates for the expected cost of direct
connections.  The next technical lemma is a generalization of
Lemma~\ref{lem: echs expected C_nu}. In Lemma~\ref{lem: echs expected
  C_nu} we bound the expected distance to the closest open facility in
$\wbarN(\nu)$, conditioned on at least one facility in $\wbarN(\nu)$
being open. The lemma below provides a similar estimate for an
arbitrary set $A$ of facilities in $\wbarN(\nu)$, conditioned on that
at least one facility in set $A$ is open.  Recall that $D(A,\nu) =
\sum_{\mu \in A} d_{\mu\nu} \bary_{\mu} / \sum_{\mu \in A}
\bary_{\mu}$ is the average distance from $\nu$ to a facility in $A$. 

%%%%%%%

\begin{lemma}\label{lem: expected distance in EBGS}
  For any non-empty set $A\subseteq \wbarN(\nu)$, let $\Lambda^\nu_A$ be
  the event that at least one facility in $A$ is opened by Algorithm
  {\EBGS}, and denote by $C_\nu(A)$ the random variable representing
  the distance from $\nu$ to the closest open facility in $A$.  Then
  the expected distance from $\nu$ to the nearest open facility in
  $A$, conditioned on at least one facility in $A$ being opened, is
%
\begin{equation*}
	\Exp[C_\nu(A) \mid \Lambda^\nu_A ] \le D(A,\nu).
\end{equation*}
\end{lemma}

\begin{proof}
  The proof follows the same reasoning as the proof of Lemma~\ref{lem:
    echs expected C_nu}, so we only sketch it here. We start with a
  similar grouping of facilities in $A$: for each primary demand
  $\kappa$, if $\wbarclsnb(\kappa)\cap A\neq\emptyset$ then
  $\wbarclsnb(\kappa)\cap A$ forms a group. Facilities in $A$ that are
  not in a neighborhood of any primary demand form singleton groups.
  We denote these groups $G_1,...,G_k$. It is clear that the groups
  are disjoint because of (PD'.\ref{PD1:disjoint}). Denoting by
  $\bard_s = D(G_s, \nu)$ the average distance from $\nu$ to a group $G_s$, we
  can assume that these groups are ordered so that $\bard_1\le ... \le
  \bard_k$.

  Each group can have at most one facility open and the events
  representing opening of any two facilities that belong to different
  groups are independent. To estimate the distance from $\nu$ to the
  nearest open facility in $A$, we use an alternative
  random process to make connections, that is easier to
  analyze. Instead of connecting $\nu$ to the nearest open facility in
  $A$, we will choose the smallest $s$ for which $G_s$ has an open
  facility and connect $\nu$ to this facility. (Thus we selected an
  open facility with respect to the minimum $\bard_s$, not the actual
  distance from $\nu$ to this facility.)  This can only increase the
  expected connection cost, thus denoting $g_s = \sum_{\mu\in G_s}
  \gamma\bary_\mu$ for all $s=1,\ldots,k$, and letting $\Prob[\Lambda^\nu_A]$
  be the probability that $A$ has at least one facility open, we have
%
\begin{align}
    \Exp[C_\nu(A) \mid \Lambda^\nu_A] &\leq \frac{1}{\Prob[\Lambda^\nu_A]} (\bard_1 g_1 +
    \bard_2 g_2 (1 - g_1) + \ldots + \bard_k  g_k(1 -
    g_1)\ldots(1-g_{k-1}))
    \label{eqn: dist set to nu 1}
    \\
    &\leq \frac{1}{\Prob[\Lambda^\nu_A]} \frac{\sum_{s=1}^k \bard_s
      g_s}{\sum_{s=1}^k  g_s} (1 - \prod_{s=1}^k (1 -  g_s))
    \label{eqn: dist set to nu 2}
    \\
    \notag
    &= \frac{\sum_{s=1}^k \bard_s g_s}{\sum_{s=1}^k g_s} =
    \frac{\sum_{\mu \in A} d_{\mu\nu} \gamma \bary_{\mu}}{\sum_{\mu
        \in A} \gamma \bary_{\mu}}
    \\
    \notag
    &= \frac{\sum_{s=1}^k d_{\mu\nu} \bary_{\mu}}{\sum_{\mu \in A}
      \bary_{\mu}} = D(A, \nu).
    \\
    \notag
\end{align}
%
Inequality (\ref{eqn: dist set to nu 2}) follows from inequality
(\ref{eq:min expected distance}) in~\ref{sec: ECHSinequality}. The rest of the
derivation follows from $\Prob[\Lambda^\nu_A] = 1 - \prod_{s=1}^k (1 -
g_s)$, and the definition of $\bard_s$, $g_s$ and $D(A,\nu)$.
\end{proof}

A consequence of Lemma~\ref{lem: expected distance in EBGS} is the
following corollary which bounds the other two expectations
of $C_\nu$, when at least one facility is opened in $\wbarclsnb(\nu)$,
and when no facility in $\wbarclsnb(\nu)$ opens but a facility in
$\wbarfarnb(\nu)$ is opened.

%%%%%%%%%

\begin{corollary} \label{coro: EBGS close and far distance} 
%
{\rm (a)} $\Exp[C_{\nu} \mid \Lambda_{\cls}^\nu] \leq \clsdist(\nu)$,
and
{\rm (b)} $\Exp[C_{\nu} \mid \Lambda^\nu \wedge \neg \Lambda_{\cls}^\nu]
    			\leq \fardist(\nu)$.
\end{corollary}

\begin{proof}
When there is an open facility in $\wbarclsnb(\nu)$, the algorithm
  connect $\nu$ to the nearest open facility in
  $\wbarclsnb(\nu)$. When no facility in $\wbarclsnb(\nu)$ opens but
  some facility in $\wbarfarnb(\nu)$ opens, the algorithm connects
  $\nu$ to the nearest open facility in $\wbarfarnb(\nu)$. The rest of
  the proof follows from Lemma~\ref{lem: expected distance in
    EBGS}. By setting the set $A$ in Lemma~\ref{lem: expected distance
    in EBGS} to $\wbarclsnb(\nu)$, we have
%
  \begin{equation*}
    \Exp[C_{\nu} \mid \Lambda_{\cls}^\nu] \leq D(\wbarclsnb(\nu), \nu),
    = \clsdist(\nu),
    \label{eqn: expected connection cost close facility}
  \end{equation*}
% 
proving part (a), and by setting the set $A$ to $\wbarfarnb(\nu)$, we have
%
  \begin{equation*}
    \Exp[C_{\nu}
    \mid \Lambda^\nu \wedge \neg \Lambda_{\cls}^\nu] \leq
    D(\wbarfarnb(\nu), \nu) = \fardist(\nu),
    \label{eqn: expected connection cost far facility}
  \end{equation*}
which proves part (b).
\end{proof}

Given the estimate on the three expected distances when $\nu$ connects
to its close facility in $\wbarclsnb(\nu)$ in (\ref{eqn: expected
  connection cost close facility}), or its far facility in
$\wbarfarnb(\nu)$ in (\ref{eqn: expected connection cost far
  facility}), or its target facility $\phi(\kappa)$ in (\ref{eqn:
  expected connection cost target facility}), the only missing pieces
are estimates on the corresponding probabilities of each event, which
we do in the next lemma. Once done, we shall put all pieces together
and proving the desired inequality on $\Exp[C_{\nu}]$, that is
(\ref{eqn: expectation of C_nu for EBGS}).

The next Lemma bounds the probabilities for events
that no facilities in $\wbarclsnb(\nu)$ and $\wbarN(\nu)$ are
opened by the algorithm.

%%%%%%%%%%%

\begin{lemma}\label{lem: close and far neighbor probability}
{\rm (a)} $\Prob[\neg\Lambda^\nu_{\cls}] \le 1/e$, and
{\rm (b)} $\Prob[\neg\Lambda^\nu] \le 1/e^\gamma$.
\end{lemma}

\begin{proof}
  (a) To estimate $\Prob[\neg\Lambda^\nu_{\cls}]$, we again consider a
  grouping of facilities in $\wbarclsnb(\nu)$, as in the proof of
  Lemma~\ref{lem: expected distance in EBGS}, according to the primary
  demand's close neighborhood that they fall in, with facilities not
  belonging to such neighborhoods forming their own singleton groups.
  As before, the groups are denoted $G_1, \ldots, G_k$. It is easy to
  see that $\sum_{s=1}^k g_s = \sum_{\mu \in \wbarclsnb(\nu)} \gamma
  \bary_{\mu} = 1$. For any group $G_s$, the probability that a
  facility in this group opens is $\sum_{\mu \in G_s} \gamma
  \bary_{\mu} = g_s$ because in the algorithm at most one facility in
  a group can be chosen and each is chosen with probability $\gamma
  \bary_{\mu}$. Therefore the probability that no facility 
  opens is $\prod_{s=1}^k (1 - g_s)$, which is
  at most $e^{-\sum_{s=1}^k g_s} = 1/e$. Therefore we have
  $\Prob[\neg\Lambda^\nu_A] \leq 1/e$.

(b)
  This proof is similar to the proof of (a). The probability $\Prob[\neg\Lambda^\nu]$ is at most
  $e^{-\sum_{s=1}^k g_s} = 1/e^\gamma$, because we now have
  $\sum_{s=1}^k g_s = \gamma \sum_{\mu \in \wbarN(\nu)} \bary_{\mu} =
  \gamma \cdot 1 = \gamma$.
\end{proof}


We are now ready to bound the overall connection cost of
Algorithm~{\EBGS}, namely inequality (\ref{eqn: expectation of C_nu for EBGS}).

%%%%%%%

\begin{lemma}\label{lem: EBGS nu's connection cost}
The expected connection of $\nu$ is
%
\begin{equation*}
\Exp[C_\nu] \le
  \concost(\nu)\cdot\max\Big\{\frac{1/e+1/e^\gamma}{1-1/\gamma}, 1+\frac{2}{e^\gamma}\Big\}.
\end{equation*}
\end{lemma}

\begin{proof}
  Recall that, to connect $\nu$, the algorithm uses the closest facility in
  $\wbarclsnb(\nu)$ if one is opened; otherwise it will try to connect $\nu$
  to the closest facility in $\wbarfarnb(\nu)$. Failing that, it will
  connect $\nu$ to $\phi(\kappa)$, the sole facility open in the
  neighborhood of $\kappa$, the primary demand $\nu$ was assigned
  to. Given that, we estimate $\Exp[C_\nu]$ as follows:
%
  \begin{align}
    \Exp[C_{\nu}] 
		\;&= \;\Exp[C_{\nu}\mid \Lambda^\nu_{\cls}] \cdot \Prob[\Lambda^\nu_{\cls}]	
				\;+\; \Exp[C_{\nu}\mid \Lambda^\nu\ \wedge\neg \Lambda^\nu_{\cls}] 
				\cdot \Prob[\Lambda^\nu\, \wedge\neg \Lambda^\nu_{\cls}]	
				\notag
		\\
		& \quad\quad\quad
				+ \; \Exp[C_{\nu}\mid \neg \Lambda^\nu] \cdot \Prob[\neg \Lambda^\nu]
				\notag
		\\
		&\leq \; \clsdist(\nu) \cdot \Prob[\Lambda^\nu_{\cls}]
			\;+\; \fardist(\nu)	
				\cdot \Prob[\Lambda^\nu\, \wedge\neg \Lambda^\nu_{\cls}]
                      \label{eqn: apply three expected dist}
						\\
                        &\quad\quad\quad
			+\; [\,\clsdist(\nu) + 2\fardist(\nu)\,] \cdot \Prob[\neg\Lambda^\nu]
		\notag
		\\
                &=\; [\,\clsdist(\nu) + \fardist(\nu)\,]\cdot \Prob[\neg\Lambda^\nu] 
						\;+\; 
							[\,\fardist(\nu)   -\clsdist(\nu)\,]
                                \cdot \Prob[\neg\Lambda^\nu_{\cls}]
                              \;+\;  \clsdist(\nu)
                                                        \notag
		\\
             &\leq\; [\,\clsdist(\nu) + \fardist(\nu)\,] \cdot \frac{1}{e^\gamma}
             \;+\; [\,\fardist(\nu) - \clsdist(\nu)\,] \cdot \frac{1}{e}
             \;+\; \clsdist(\nu)
             \label{eqn: probability estimate}
             \\
             \notag
             &=\; \Big(1 - \frac{1}{e} + \frac{1}{e^\gamma}\Big)\cdot \clsdist(\nu)
 				\;+\; \Big(\frac{1}{e} + \frac{1}{e^\gamma}\Big)\cdot\fardist(\nu).
\end{align}
%
Inequality (\ref{eqn: apply three expected dist}) follows from
Corollary~\ref{coro: EBGS close and far distance} and 
Lemma~\ref{lem: EBGS target connection cost}. 
Inequality (\ref{eqn: probability estimate}) follows from 
Lemma~\ref{lem: close and far neighbor probability} and
$\fardist(\nu) - \clsdist(\nu)\ge 0$.

Now define $\rho =\clsdist(\nu)/\concost(\nu)$. It is easy to
see that $\rho$ is between 0 and 1. Continuing the above
derivation, applying (\ref{eqn:avg dist cls dist far dist}), we get
%
\begin{align*}
\Exp[C_{\nu}]
             \;&\le\; \concost(\nu) 
			\cdot\left((1-\rho)\frac{1/e+1/e^\gamma}{1-1/\gamma} 
				+ \rho (1 + \frac{2}{e^\gamma})\right)
			\\
             &\leq \concost(\nu) 
				\cdot \max\left\{\frac{1/e+1/e^\gamma}{1-1/\gamma}, 1 + \frac{2}{e^\gamma}\right\},
\end{align*}
%
and the proof is now complete.
\end{proof}

With Lemma~\ref{lem: EBGS nu's connection cost} proven, we are now ready to bound our total connection cost.
For any client $j$ we have
%
\begin{align*}
\sum_{\nu\in j} C^{\avg}(\nu)
	&= \sum_{\nu\in j}\sum_{\mu\in\facilityset} d_{\mu\nu}\barx_{\mu\nu} 
	\\
	&= \sum_{i\in\sitesset}d_{ij}\sum_{\mu\in i}\sum_{\nu\in j} \barx_{\mu\nu}
	= \sum_{i\in\sitesset} d_{ij}x_{ij}^\ast = C_j^\ast.
\end{align*}
% 
Summing over all clients $j$ we obtain that the total expected connection cost is
%
\begin{equation*}
	\Exp[ C_{\smallEBGS} ] \le  C^\ast\max\left\{\frac{1/e+1/e^\gamma}{1-1/\gamma}, 1+\frac{2}{e^\gamma}\right\}.
\end{equation*}
%
Recall that the expected facility cost is bounded by $\gamma F^\ast$,
as argued earlier. Hence the total expected cost is bounded by $\max\{\gamma,
\frac{1/e+1/e^\gamma}{1-1/\gamma}, 1+\frac{2}{e^\gamma}\}\cdot
\LP^\ast$. Picking $\gamma=1.575$ we obtain the desired ratio.

%%%%%%

\begin{theorem}\label{thm:ebgs}
  Algorithm~{\EBGS} is a $1.575$-approximation algorithm for \FTFP.
\end{theorem}

%% ch5 primal-dual results
\chapter{Primal-dual Algorithms} 
\label{ch: primal-dual} 

In this chapter we present results and discuss combinatorial
algorithms to the FTFP problem. These approaches, although
employ the Linear Program in guiding the algorithm and
deriving approximation ratio, the use of LP is
implicit. Moreover, the algorithms do not require solving
the LP and having access to the fractional optimal
solutions. Two notable such approaches are primal-dual and
dual-fitting. In primal-dual algorithms, we start with a
feasible dual solution, usually with all dual variables set
to zero, then we raise a subset of dual variables and update
the corresponding primal variables accordingly. At any time,
we keep the dual solution feasible and we stop when the
primal solution becomes feasible. The approximation ratio is
derived by a relaxed version of the complementary slackness
conditions.

Another approach, dual-fitting, starts with an empty dual
solution as well, and raise a subset of dual variables in
each iteration, updating corresponding primal variables and
stop when the primal solution is feasible. The difference is
that in dual-fitting, the dual solution may not be feasible,
and we require the cost of the primal solution bounded by
the value of the possibly infeasible dual solution. The next
step is to find the smallest possible number $\gamma$, which
may depend on the input size, such that the dual solution,
when divided by $\gamma$, becomes feasible. It is easy to
see that $\gamma$ provides an upper bound on the
approximation ratio, because the value of a feasible dual
solution is a lower bound on the value of an optimal primal
solution.

Jain and Vazirani~\cite{JainV01} designed a primal-dual
algorithm, which we call the JV algorithm, for the {\UFL}
problem. Recall that in the {\UFL}, all demands $r_j =
1$. In the JV algorithm, every client $j$ has a number
$\alpha_j$ associated. All $\alpha_j$ start with zero. The
algorithm has two phases. The first phase runs in
iterations. In each iteration, all $\alpha_j$ that were not
temporarily connected are raise uniformly. The contribution
from a client $j$ to a facility $i$ is $\max\{0, \alpha_j -
d_{ij}\}$. Whenever a facility received enough total
contribution, that is $\sum_{j\in\clientset} (\alpha_j -
d_{ij})_+ = f_i$, then $i$ is temporarily open and all
clients with $\alpha_j \geq d_{ij}$ temporarily connect to
$i$. The facility $i$ is called the witness of the client
$j$. The first phase concludes when all clients are
temporarily connected. In the second phase, we construct an
auxilary graph with nodes being temporarily open facilities
and two nodes are connected by an edge if there exists some
client $j$ that contributes to both of them, that is,
$\alpha_j > d_{i_1 j}$ and $\alpha_j > d_{i_2 j}$. We then
pick a maximal independent set in the auxilary graph as the
set of facilities to open. For connections, if a client $j$
has an open facility $i$ with $d_{ij} \leq \alpha_j$, then
it connects to that facility. Otherwise, there exists some
temporarily open facility $i$ such that $\alpha_j \geq
d_{ij}$. Since $i$ is not open, there must exists some
facility $i'$ that is open and some client $j'$ such that
$\alpha_{j'} > d_{i'j}$ and $\alpha_{j'} > d_{i j'}$. It
follows that $\alpha_{j'} \leq \min\{t(i), t(i')\}$ where
$t(i)$ is the time that facility $i$ is temporarily
open. The reason is, if $\alpha_{j'} > t(i)$, then it would
have temporarily connected to facility $i$ earlier so its
$\alpha_{j'}$ value would have been smaller. On the other
hand, since facility $i$ is the witness of client $j$, we
have $t(i) \leq \alpha_j$~\footnote{$t(i) < \alpha_j$ is
  possible if facility $i$ is temporarily open and later $j$
  has $\alpha_j = d_{ij}$ to temporarily connect to facility
  $i$.} Therefore we have $\alpha_j \geq \alpha_{j'} \geq
d_{i'j'}, d_{i j'}$. In addition, we also have $\alpha_j
\geq d_{ij}$. Hence $d_{i'j} \leq d_{i'j'} + d_{ij'} +
d_{ij} \leq \alpha_{j'} + \alpha_{j'} + \alpha_j \leq
3\alpha_j$. If we define $\beta_{ij} = 0$ if client $j$ does
not contribute to any facility $i$ and $\beta_{ij} =
\alpha_j - d_{ij}$ if client $j$ connects directly to
facility $i$. Now we estimate the total cost of this
solution. For facility cost we have $\sum_{i,j} \beta_{ij}$,
and for connection cost, if a client $j$ is directly
connected, then its $d_{ij} \leq \alpha_j - \beta_{ij}$,
otherwise it is $d_{ij} \leq 3\alpha_j$. The total cost is
hence no more than $3\sum_{j\in\clientset} \alpha_j$. Since
$\{\alpha_j\}$ form a feasible dual solution, we have the
optimal solution value is no less than
$\sum_{j\in\clientset} \alpha_j$. Therefore, we have our
solution costs no more than $3$ times the cost of an optimal
solution.

The fault-tolerant facility location problem ({\FTFL}) was
introduced by Jain and Vazirani~\cite{JainV03} primarily to
demonstrate that their primal-dual algorithm on {\UFL} can
be applied to a more general problem, where clients could
have demand more than $1$, and each facility could be open
or close. A client $j$ with demand $r_j$ needs to be
connected to $r_j$ different facilities. The primal-dual
algorithm they provide gives a ratio of $3\ln \max_j
r_j$. Subsequent attempts on adapting either the primal-dual
approach or the dual-fitting approach to {\FTFL} with a
sub-logarithmic approximation ratio were not successful,
although for the uniform demands case, that is, when all
$r_j$ are equal, Adrian Bumb~\cite{Bumb02} demonstrated that
the JV algorithm~\cite{JainV01} for {\UFL} can be adapted to
obtain the same ratio as for {\UFL}. On a separate paper,
Swamy and Shmoys~\cite{SwamyS08} showed that a greedy
algorithm analyzed using dual-fitting can be shown to have a
ratio of $1.52$. For the non-uniform demand case, the best
known result is an $O(\log n)$-approximation.

For our problem, {\FTFP}, we have seen in earlier chapters
that it can be approximated with the same ratio as {\UFL}
when LP-rounding is used. However, the attempt to obtain a
sub-logarithmic approximation ratio on {\FTFP} using the
primal-dual algorithm or the dual-fitting algorithm were not
successful. On the positive side, we derive a weak result
that the greedy algorithm does give a $O(\log n)$ ratio for
{\FTFP}. We remark here that the $O(\log n)$ ratio does not
even use the triangle inequality. On the negative side, we
provide an example showing that the greedy algorithm with
dual-fitting analysis can at best give a ratio of $O(\log n/
\log\log n)$ under a very reasonable assumption, which we
call \emph{local-charging} assumption. Here we have $n =
|\clientset|$.

In the following we first describe the greedy algorithm and
its analysis. We show that the greedy algorithm is $O(\log
n)$ approximation using dual-fitting analysis. Then we
present our example showing a lower bound on the
dual-fitting analysis on the greedy algorithm. We conclude
this chapter with some possible approach to obtain
sub-logarithmic approximation results.

\section{The Greedy algorithm with $O(\log n)$ Ratio}
\label{sec:upp}
In this section we show that the greedy algorithm which
repeatedly picking the best star (the one with minimum
average cost) gives an approximation ratio of $H_n = \ln(n)$
where $n=|\clientset|$ is the number of clients. A star is a
site $i$ and a subset of clients $C'$. The cost of such a
star $S$ is $c(S) = f_i + \sum_{j\in C'} d_{ij}$, and the
average cost of $S$ is $c(S) / |C'|$. Call a client $j$
fully-satisfied if $j$ has made $r_j$ connections. Let $U$
be the set of not fully-satisfied clients. While not all
clients fully-satisfied, the algorithm picks a star
$S=(i,C')$ with $C' \subseteq U$, and open one facility at
site $i$. Each client in $C'$ then makes one more connection
with site $i$. The algorithm terminates when all clients are
fully-satisfied.

When we run the greedy algorithm, for every client $j$, we
associate each demand of $j$ with a number $\alpha_j^l$,
which is the average cost of the star when $l^{th}$ demand
of $j$ is connected. Now we let $\alpha_j = \alpha_j^{r_j}$,
that is, take $\alpha_j$ to be the finishing $\alpha_j^l$,
and order clients by increasing $\alpha_j$. That is,
\begin{equation*}
  \alpha_1 \leq \alpha_2 \leq \ldots \leq \alpha_n
\end{equation*}

Due to the algorithm, for every $j=1,\ldots,n$, we have
\begin{equation*}
  \sum_{l=j}^n (\alpha_j - d_{il})_+ \leq f_i
\end{equation*}
for every site $i$.  The reason is that, when the last
demand of $j$ is connected, all clients $j+1,\ldots,n$ are
still active so their total contribution cannot exceed
$f_i$.

Now we take a closer look at the numbers $\{\alpha_j\}$. We
know that the algorithm's total cost is exactly
$\sum_{j=1}^n \sum_{l=1}^{r_j} \alpha_j^l$, which is no more
than $\sum_{j=1}^n r_j \alpha_j$ since we take $\alpha_j$ to
be $\alpha_j^{r_j}$. Now if we can show that $\sum_{j=1}^n
r_j \alpha_j$ is no more than $\gamma \cdot \textrm{OPT}$,
where $\textrm{OPT}$ is the cost of an integral optimal
solution to the given FTFP instance, then we claim our
algorithm returns an integral solution within a factor of
$\gamma$.

We show that $\sum_{j=1}^n r_j \alpha_j$ is within a factor
of $\gamma$ from $\textrm{OPT}$ by showing that
$\{\alpha_j/\gamma\}$ is a feasible dual solution to the
following program, which is the dual program of the primal
LP for FTFP.
\begin{align*}
  \max\; &\sum_j r_j\alpha_j\\
  \textrm{subject to: }& \sum_{j=1}^n (\alpha_j - d_{ij})_+
  \leq f_i \textrm{ for every facility i}\\
\end{align*}

To find the minimum $\gamma$ that would shrink
$\{\alpha_j\}$ to a feasible dual solution, we need to find
a worst case instance to maximize $\gamma$, also it is clear
that the worst case instance must contain a star whose
feasibility requirement would achieves the value of
$\gamma$, and this star would be the worst star in that
instance.

As a first step we can drop the $\max\{0, \cdot\}$, because
we can always find a new star by dropping those $j$ with
$\alpha_j - d_{ij}$ term negative, and that new star would
still be a worst case star. Suppose a worst case star has
$k$ clients, and is with facility $i$, then we have
\begin{equation*}
  \sum_{j=1}^k \alpha_j - d_{ij} \leq f_i
\end{equation*}
Here we rename clients in the new star to be $1,\ldots,k$,
although among them, they are still ordered by their
$\alpha_j$.

Now our goal is to find a supremum of the following program:
\begin{align*}
  \max\; & \frac{\sum_{j=1}^k \alpha_j}{f_i + \sum_{j=1}^k d_{ij}}\\
  \textrm{subject to: } & \sum_{l=j}^k (\alpha_l -
  d_{il})_+\leq f_i \textrm{ for } j=1,\ldots,n\\
\end{align*}

Since we are dealing with a particular star, we can abstract
away $i$, to obtain the following program:
\begin{align}
  \label{eq:star}
  \max\; & \frac{\sum_{j=1}^k \alpha_j}{f + \sum_{j=1}^k
    d_j}\\ \notag
  \textrm{subject to: } & \sum_{l=j}^k (\alpha_j - d_{l})_+
  \leq f \textrm{ for } j=1,\ldots,n\\ \notag
\end{align}

Now we claim we can drop the $\max\{0, \cdot\}$ operator
because this would relax the constraint in (\ref{eq:star})
and can only make objective value larger (since we are
maximizing), so the real optimal is upper bounded by the
relaxed optimal. This allows us to work on
\begin{align}
  \label{eq:frlp}
  \max\; & \frac{\sum_{j=1}^k \alpha_j}{f + \sum_{j=1}^k
    d_j}\\ \notag
  \textrm{subject to: } & \sum_{l=j}^k (\alpha_j - d_{l})
  \leq f \textrm{ for } j=1,\ldots,n\\ \notag
\end{align}

For each $j=1,\ldots,n$, the constraint above simply can be
rewritten as
\begin{equation}
  (k-j+1) \alpha_j \leq f + \sum_{l=j}^k d_l \leq f +
  \sum_{l=1}^k d_l.
\end{equation}
The first inequality is a rewrite of the constraint in
(\ref{eq:frlp}) and the second is straightforward.

Therefore we have $\alpha_j \leq (1/(k-j+1)) (f +
\sum_{j=1}^k d_j)$, and it easily follows that
\begin{equation}
  \sum_{j=1}^n \alpha_j \leq (1/k + 1/(k-1) + \ldots + 1) =
  H_k \leq H_n = \ln(n)
\end{equation}

\section{An Example Showing the Difficulty in Obtaining
  $O(1)$ Ratio}

For FTFP, the greedy algorithm that repeatedly picks the
best star until all clients have all demands satisfied can
be implemented in polynomial time. In Section~\ref{sec:upp}
we show that this algorithm is $H_n$-approximation where
$n=|\mathcal C|$ is the number of clients. Since the same
greedy algorithm is shown to have constant approximation
ratio for UFL~\cite{MahdianMSV01}, a natural question to
ask is whether greedy can be shown to have $O(1)$
approximation ratio. Here we give an argument that hints a
negative answer.

We assume the greedy algorithm is analyzed using the
dual-fitting technique, which associates with every client
$j$ with a number $\alpha_j$, interpreted as a dual solution
to the LP~(\ref{eqn:fac_dual}). However, the dual solution
in general may not be feasible. The dual-fitting technique
aims at finding a smallest possible number $\gamma$ such
that after the dual solution $\{\alpha_j\}$ is shrinked
(divided) by $\gamma$, all dual constraints are
satisfied. That is
\begin{equation*}
\sum_{j\in \mathcal C} (\alpha_j/\gamma
- d_{ij})_+ \leq f_i \qquad \text{ for all } i\in \mathcal F. 
\end{equation*}
That $\gamma$ is taken as the approximation ratio.

In the greedy algorithm, a star with minimum average cost is
picked at each iteration and each member client of that star
then gets one more connection. It is not specified by the
algorithm how we distribute the cost of $f_i$ into member
clients, which is part of the analysis. Nonetheless we
assume that the cost of $f_i$ is distributed among members
only, and not to clients outside this star. We call this
\emph{local charging} assumption. Our second assumption is
that the proposed dual solution $\alpha_j$, is taken as the
average of individual $\alpha_j^l$ for each of the $l^{th}$
demand of client $j$, with $l=1,\ldots,r_j$. Suppose the
$l^{th}$ demand of $j$ is satisfied while $j$ is in a star
with facility $i$, then $\alpha_j^l = d_{ij} + f_i^{j,l}$,
where $f_i^{j,l}$ is the portion of $f_i$ attributed to $j$
in the analysis. Notice that taking the average implies the
$\alpha_j$ values thus computed results in
$\sum_{j\in\clientset} r_j \alpha_j$ equal to the cost of
the integral solution by the greedy algorithm.

%%%%%%%%%%%%%%%%%%% start figure %%%%%%%%%%%%%%%%%%%%%%%%%%%%%
\begin{figure}
  \centering
  \begin{tikzpicture}[auto]
    \node[draw,rectangle,minimum size=.7cm] (fac) at (-4,0) {};
    \node at (-5,0) {$f_1$};
    
    \node[draw,ellipse,minimum width=4cm,minimum
    height=1.8cm] (client1) at
    (6,0) {$n_1 = k^{k-1}, r_1$};
    \node[draw,ellipse,minimum width=4cm,minimum
    height=1.8cm] (client2) at
    (6,-3) {$n_2 = k^{k-2}, r_2$};
    \node[draw,ellipse,minimum width=4cm,minimum
    height=1.8cm] (client3) at
    (6,-6) {$n_3 = k^{k-3}, r_3$};
    \node[draw,ellipse,minimum width=4cm,minimum
    height=1.8cm] (clientk) at
    (6,-12) {$n_k = 1, r_k$};
    
    \node at (-3,-10) {\large{demands $r_1 \ll r_2 \ll \ldots \ll r_k$}};

    \draw (fac) to node {$d_1=0$} (client1);
    \draw[bend right] (fac) to node {$d_2 = d_1 + f_1 /
      n_2$}  (client2);
    \draw[bend right] (fac) to node {$d_3 = d_2 + f_1 /
      n_3$}  (client3);
    \draw[bend right] (fac) to node {$d_k = d_{k-1} + f_1 / n_k$}  (clientk);
  \end{tikzpicture}
  \caption{An example showing the greedy algorithm for FTFP,
    analyzed using dual-fitting, could give a solution with
    cost $\Omega(\log n / \log\log n)$ from the optimal
    value, assuming facility cost can only be charged to
    clients within the star.}
  \label{fig:greedy_lower_bound}
\end{figure}
%%%%%%%%%%%%%%%%% end figure %%%%%%%%%%%%%%%%%%
%%%%%%%%%%%%%%%%%%%%%%%%%%%%%%%%%%%%%%%%%%%%%%%

Our example has one site and $k$ groups of clients, a figure
is given in Figure~\ref{fig:greedy_lower_bound}. Opening
one facility at that site costs $f_1$. The first group has
$n_1$ clients each with demand $r_1$, all at distance $d_1 =
0$ from $f_1$. The other groups are listed below:
\begin{align*}
  &d_1 = 0\\
  &d_2 = \frac{f_1}{n_1}\\
  &d_3 = f_1/n_2 + d_2 = f_1/n_2 + f_1/n_1 = f_1 (\frac{1}{n_2} + \frac{1}{n_1})\\
  &\ldots\\
  &d_k = f_1/n_{k-1} + d_{k-1} = f_1 (\frac{1}{n_{k-1}} + \ldots + \frac{1}{n_1})\\
\end{align*}
For the numbers, we need $r_1 \ll r_2 \ll \ldots \ll r_k$,
and $n_1 = u^{k-1}, n_2 = u^{k-2}, \ldots, n_k = u^0 = 1$
for some number $u$ (Actually we take $u=k$, this choice may
not be the best possible).

Call a star with facility cost zero \emph{trivial}. It is
\emph{non-trivial} if the facility has non-zero cost. Now
the greedy execution goes like this: The first non-trivial
star (with $r_1$ replica) is $(f_1, n_1)$. Then we have a
trivial star of zero cost facility and all $n_2$ clients in
group $2$ for $r_1$ replica. The second non-trivial star
(with $r_2$ replica) is $(f_1, n_2)$. Notice that $r_2 \gg
r_1$. The $r_1$ replica of trivial star with group $2$
satisfy $r_1$ demand of the $n_2$ group. After that the
$n_2$ group clients each has residual demand $r_2 - r_1 =
r_2$. The process repeats until the $k^{th}$ group finishes
with $r_k$ new facilities.

According to our local charging assumption, we have
$\alpha_1 = f_1$, now defined as the total dual value of
clients in group $n_1$, regardless how the analysis would
distribute within that group. Similarly $\alpha_2 = f_1 +
n_2 d_2$, and so on. Substituing in the numbers, we have

\begin{align*}
  &\alpha_1 = f_1\\
  &\alpha_2 = f_1 + n_2 d_2 = f_1 + f_1/n_1\cdot n_2 = f_1 (1 + n_2 /
  n_1)\\
  &\alpha_3 = f_1 + n_3 d_3 = f_1 + f_1 (\frac{1}{n_2} +
  \frac{1}{n_1}) n_3 = f_1 (1 + \frac{n_3}{n_2} + \frac{n_3}{n_1})\\
  &\ldots\\
  &\alpha_k = f_1 + n_k d_k = f_1 + f_1 n_k (\frac{1}{n_{k-1}} + \ldots
  \frac{1}{n_1})\\
\end{align*}
Notice that $r_1 \ll r_2 \ll \ldots \ll r_k$ implies $\alpha_j$ is
decided by the max among $\alpha_j^l$.

Now back to the dual constraint, it requires that the shrinking factor
$\gamma$ needs to satisfy the following inequality:
\begin{equation}
  \frac{\alpha_1}{\gamma} - d_1 + \frac{\alpha_2}{\gamma} - d_2 +
  \ldots + \frac{\alpha_k}{\gamma} - d_k \leq f_1.
\end{equation}
Substitute in the $\alpha_j$ values derived above, we have
\begin{align*}
  \gamma &\geq (\sum_{j=1}^k \alpha_j) / (f_1 + \sum_{j=1}^k d_j)\\
  &\geq \frac{f_1 + n_1 d_1 + f_1 + n_2 d_2 + f_1 + n_3 d_3 + \ldots +
    f_1 + n_k
    d_k}{f_1 + n_1 d_1 + n_2 d_2 + \ldots + n_k d_k}\\
  &= 1 + (k-1)f_1 / (f_1 + n_1 d_1 + n_2 d_2 + \ldots + n_k d_k)\\
  &= 1 + (k-1)f_1 / \left(f_1 + n_2 f_1 / n_1 + \ldots + n_k f_1
    (\frac{1}{n_{k-1}} + \frac{1}{n_{k-2}} + \ldots +
    \frac{1}{n_1})\right)\\
  &= 1 + (k-1) / \left(1 + n_2 / n_1 + \ldots + n_k
    (\frac{1}{n_{k-1}} + \frac{1}{n_{k-2}} + \ldots +
    \frac{1}{n_1})\right)\\
  &= 1 + (k-1) / \left(1 + 1/u + \ldots + (\frac{1}{u} + \ldots +
    \frac{1}{u^{k-1}})\right)\\
  &= 1 + (k-1) / \left(1 + k/u + (k-1)/u^2 + \ldots +
    1/u^{k-1}\right)\\
  &\geq 1 + (k-1) / \left(1 + k/u + k/u^2 + \ldots +
    k/u^{k-1}\right)\\
  &= 1 + (k-1) / \left(1 + 1 + 1/k + \ldots + 1/k^{k-2}\right)\\
  &\approx k/2\\
\end{align*}
So for $k$ groups we can force a shrinking factor $\gamma$
as big as $k/2$. Recall that we have greedy being no more
than $H_n$-approximation. Is that a contradiction? No,
because we have the number of clients $n=k^{k-1} + k^{k-1} +
\ldots + 1 = k^k$, so $k = O(\log n / \log\log
n)$. Therefore, the example shows that dual fitting with
local charging cannot hope to get $O(\log n / \log\log n)$
ratio or better.

\paragraph{Remark} Notice this example is similar in spirit
to the $\Omega(\log n/ \log\log n)$ example for Hochbaum's
algorithm for UFL, constructed by Mahdian {\etal}
~\cite{JainMMSV03}.

%% ch6 conclusion
\chapter{Conclusion} \label{ch: conclusion} 

In thisb dissertation we studied the fault-tolerant facility
placement problem ({\FTFP}), a generalization of the
well-known uncapacitated facility location problem
({\UFL}). We demonstrated that the known LP-rounding
algorithms for {\UFL} can be adapted to {\FTFP} while
preserving the approximation ratio. To accomplish this
reduction, we developed two techniques, namely demand
reduction and adaptive partition, which could be of more
general interest. Our results demonstrated that {\FTFP}
seems easier to approximate, compared to {\FTFL}.

We also studied the primal-dual and dual-fitting approach,
and provided a possible explanation on the difficult to
obtain a constant approximation ratio using those techniques.

We hope our work in this dissertation will help other
researchers interested in the fault-tolerant variant of the
facility location problems to develop more insight into the
difficulty and possible solutions when clients demand more
than one facility and we still need to keep total cost under
control.

In anticipating future research, we tend to agree with the
authors, Byrka {\etal}, with their remark on {\UFL} and
{\FTFL}, that both problems are likely to have approximation
algorithms with ratio matching the $1.463$ lower bound. And
from our demand reduction technique, it is almost surely
that {\FTFP} shall have a $1.463$-approximation algorithm,
provided that {\FTFL} can be approximated to meet the lower
bound.

\bibliographystyle{plain}
\bibliography{facility}

\appendix
\chapter{Technical Background}

\section{Linear Programming and Integer Programming}
\label{sec: ILP}

In this section we give a short introduction on Linear
Programming and Integer Programming with an emphasis on
their applicability in design and analysis of approximation
algorithms for {\NP}-optimization problems.

Most {\NP}-optimization problems have a natural integer
program in which we use variables to represent parameters in
the solution we seek, and write the constraints imposed by
feasiblity of the solution, the objective function is
obtained by the cost function of the solution, specified by
the problem. For example, in the Vertex Cover problem, we
are given a graph $G=(V,E)$ and we are to find a subset $W$
of $V$, such that every edge $e\in E$ has at least one
endpoint in $W$, and we want the set $W$ to have minimum
size. To formulate this problem as an integer program, we
use $x_v \in \{0,1\}$ to denote whether a node $v\in V$ is
in $W$ or not. The constraint is that for every edge
$e=(u,v)$, we have $x_u + x_v \geq 1$. The objective is to
minimize $\sum_{v\in V} x_v$. The integer program for Vertex
Cover is written as
\begin{align*}
  &\text{minimize } \sum_{v\in V} x_v\\
  &\text{subject to } x_u + x_v \geq 1  \qquad \forall (u,v) \in
  E\\
  &x_v \in \{0, 1\} \qquad \forall v \in V\\
\end{align*}
In general an integer program cannot be solved exactly in
polynomial time, as integer programming is
{\NP}-hard. However, if we relax the integral constraint and
allow the variables to take fractional value, we then obtain
a Linear Program (LP) and LP is polynomially solvable, for
example, using the ellipsoid method or the interior point
method. Thus we can first solve the LP optimally, obtaining
a fractional optimal solution to the LP. The value of the
fractional optimal solution is then a lower bound on the
value of the optimal integral solution, assuming a
minimization problem. Our next step is then to round the
fractional solution appropriately, so that we maintain the
feasibility while keeping the cost from increasing too
much. The exact rounding procedure is problem specific and
we shall not delve into the details here. The rounding
relevant to the FTFP problem in this thesis is presented in
detail in Chapter~\ref{ch: lp-rounding}.

We now give a brief introduction of linear programming,
see~\cite{Chvatal83} for an introductory book on this topic.
A general Linear Program can be written as
\begin{align}
  \label{eqn:lp_primal}
  \text{minimize } & \sum_{j=1}^n c_j x_j & \\ \notag
  \text{subject to } & \sum_{j=1}^n a_{ij} x_j \geq b_i,
   & \text{for } i = 1, \ldots, m\\ \notag
   & x_j \geq 0 & \text{for } j = 1, \ldots, n\\ \notag
\end{align}
For the LP above, we can take its dual as
\begin{align}
  \label{eqn:lp_dual}
  \text{maximize } & \sum_{i=1}^m b_i y_i &\\ \notag
  \text{subject to } & \sum_{i=1}^m a_{ij} y_i \leq c_j
   & \text{for } j = 1, \ldots, n\\ \notag
   & y_i \geq 0 & \text{for } i = 1,\ldots,m\\ \notag
\end{align}
The LP (\ref{eqn:lp_primal}) is called the primal program
and the LP (\ref{eqn:lp_dual}) is called the dual program.
The weak duality theorem tells us that, for every feasible
solution $\bfx$ for the primal (\ref{eqn:lp_primal}) and
$\bfy$ for the dual (\ref{eqn:lp_dual}), we have that
$\textbf{c}^T \bfx \geq \textbf{b}^T \bfy$. The strong
duality theorem tells us that if both the primal
(\ref{eqn:lp_primal}) and the dual (\ref{eqn:lp_dual}) are
feasible, then both of them have optimal solution
$\bfx^\ast$ and $\bfy^\ast$ and their objective function
values equal, that is $\textbf{c}^T \bfx^\ast = \textbf{b}^T
\bfy^\ast$. Moreover, the complementary slackness conditions
assert that two feasible solutions $\bfx$ and $\bfy$ are
both optimal to LP (\ref{eqn:lp_primal}) and
(\ref{eqn:lp_dual}) respectively, if and only if, for every
primal variable $x_j$, either $x_j = 0$ or the corresponding
constraint in the dual is tight, that is $\sum_{i=1}^m
a_{ij} y_i = c_j$. And for every dual variable $y_i$, either
$y_i = 0$ or the corresponding constraint in the primal is
tight, that is $\sum_{j=1}^n a_{ij} x_j = b_i$. The
complementary slackness conditions provide a simple way to
validate the optimality when one is presented with two
solutions, proposed to be optimal for the primal and dual
program respectively.

The complementary slackness conditions play a crucial role
in the design and analysis of approximation algorithms. For
example, suppose we have an algorithm that computes a
feasible integral solution $\bfx$ to the primal program
(\ref{eqn:lp_primal}) and a feasible integral solution to
the dual program (\ref{eqn:lp_dual}). Moreover, we know that
the two solutions satisfy a relaxed version of the
complementary slackness conditions: for some numbers
$\alpha$ and $\beta$, we have
\begin{align*}
  \text{either } y_i = 0  \quad \text{or } \quad b_i \leq \sum_{j}
  a_{ij} x_j \leq \alpha\, b_i \qquad \text{for } i = 1,
  \ldots, m.\\
  \text{either } x_j = 0  \quad \text{or } \quad \beta\, c_j \leq
  \sum_{i}a_{ij}y_i \leq c_j \qquad \text{for } j = 1,
  \ldots, n.\\
\end{align*}
Then the integral solution $\bfx$ has cost no more than
$\alpha/\beta$ times the optimal value. In particular, we
have $\sum_{j} c_j x_j \leq \alpha/\beta \sum_{i} b_i y_i$
and the value for a feasible dual solution, namely $\sum_{i}
b_i y_i$, is a lower bound on the optimal value of the
primal program.

As an application of the complementary slackness conditions,
we look at their use in the design and analysis of
algorithms for the facility location problems. Recall that
we define the neighborhood $N(j)$ of a client $j$ as the set
of facilities with $x_{ij}^\ast > 0$, where
$\bfx^\ast,\bfy^\ast$ is some fractional optimal fractional
solution and $\bfalpha^\ast, \bfbeta^\ast$ is some optimal
fractional dual solution. The complementary slackness
conditions provide an upper bound on the maximum distance
from a facility $i \in N(j)$ to a client $j$, since one dual
constraint says $\alpha_j - \beta_{ij} \leq d_{ij}$ and if
the primal solution has $x_{ij}^\ast > 0$, then the
inequality is actually an equality and we have
$\alpha_j^\ast - \beta_{ij}^\ast = d_{ij}$. Together with
$\beta_{ij}^\ast \geq 0$, we have $\alpha_j^\ast \geq
d_{ij}$ for every $i$ such that $x_{ij}^\ast > 0$.

The idea of using relaxed complementary slackness conditions
in designing approximation algorithms for the uncapacitated
facility location problem is demonstrated by Jain and
Vazirani~\cite{JainV01}. They proposed an algorithm that
outputs an integral solution $(\bfx, \bfy)$ to the primal
program (\ref{eqn:fac_primal}) and a feasible (possibly
fractional) solution $(\bfalpha,\bfbeta)$ to the dual
program (\ref{eqn:fac_dual})~\footnote{For the uncapacitated
  facility location problem we have all $r_j = 1$ for
  $j\in\clientset$.}. Moreover, the two solutions satisfy the
conditions that
\begin{align*}
  &\text{either } \sum_{j} \beta_{ij} = f_i  \quad \text{ or
} \quad y_i = 0.\\
  &\text{either } 1/3\, d_{ij} \leq \alpha_j - \beta_{ij}
  \leq d_{ij} \quad \text{ or } \quad x_{ij} = 0.\\
\end{align*}
The solution $(\bfx,\bfy)$ then is an $3$-approximation to
the optimal solution.

\section{Proof of Inequality (\ref{eqn: echs ineq direct
    cost, step 1})}
\label{sec: ECHSinequality}

In Sections~\ref{sec: 1.736-approximation} and \ref{sec:
  1.575-approximation} we use the following inequality
%
\begin{align}
  \label{eq:min expected distance}
  \bard_1 g_1 + \bard_2 g_2 (1-g_1) +
  \ldots &+ \bard_k g_k (1-g_1) (1-g_2) \ldots (1-g_k)\\ \notag
  &\leq \frac{1}{\sum_{s=1}^k g_s} \left(\textstyle\sum_{s=1}^k \bard_s g_s\right)\left(\textstyle\sum_{t=1}^k g_t \textstyle\prod_{z=1}^{t-1} (1-g_z)\right).
\end{align}
%
for $0 < \bard_1\leq \bard_2 \leq \ldots \leq \bard_k$, and
$0 < g_1,...,g_s \le 1$.

\medskip

We give here a new proof of this inequality, much simpler
than the existing proof in \cite{ChudakS04}, and also
simpler than the argument by Sviridenko~\cite{Svi02}.  We
derive this inequality from the following generalized
version of the Chebyshev Sum Inequality:
%
\begin{equation}
  \label{eq:cheby}
  \textstyle{\sum_{i}} p_i \textstyle{\sum_j} p_j a_j b_j \leq \textstyle{\sum_i} p_i a_i \textstyle{\sum_j} p_j b_j,
\end{equation}
%
where each summation runs from $1$ to $l$ and the sequences
$(a_i)$, $(b_i)$ and $(p_i)$ satisfy the following
conditions: $p_i\geq 0, a_i \geq 0, b_i \geq 0$ for all $i$,
$a_1\leq a_2 \leq \ldots \leq a_l$, and $b_1 \geq b_2 \geq
\ldots \geq b_l$.

Given inequality (\ref{eq:cheby}), we can obtain our
inequality (\ref{eq:min expected distance}) by simple
substitution
%
\begin{equation*}
  p_i \leftarrow g_i, a_i \leftarrow \bard_i, b_i \leftarrow
  \Pi_{s=1}^{i-1} (1-g_s),
\end{equation*}
%
for $i = 1,...,k$.

For the sake of completeness, we include the proof of
inequality (\ref{eq:cheby}), due to Hardy, Littlewood and
Polya~\cite{HardyLP88}. The idea is to evaluate the
following sum:
%
\begin{align*}
  S &= \textstyle{\sum_i} p_i \textstyle{\sum_j} p_j a_j b_j - \textstyle{\sum_i} p_i a_i \textstyle{\sum_j} p_j b_j
	\\
  & = \textstyle{\sum_i \sum_j} p_i p_j a_j b_j - \textstyle{\sum_i \sum_j} p_i a_i p_j b_j
	\\
  & = \textstyle{\sum_j \sum_i} p_j p_i a_i b_i - \textstyle{\sum_j \sum _i} p_j a_j p_i b_i
	\\
	&= \half \cdot \textstyle{\sum_i \sum_j} (p_i p_j a_j b_j - p_i a_i p_j b_j + p_j p_i a_i
  							b_i - p_j a_j p_i b_i)
\\
  &= \half \cdot \textstyle{\sum_i \sum_j} p_i p_j (a_i - a_j)(b_i - b_j) \leq 0.
\end{align*}
The last inequality holds because $(a_i-a_j)(b_i-b_j) \leq
0$, since the sequences $(a_i)$ and $(b_i)$ are ordered
oppositely.


\end{document}

% marek Tue Jul  3 10:21:05 PDT 2012
% marek Sun Jul  1 14:57:39 PDT 2012
% lyan Sat Jun 30 2012, 22:01:27
% marek Sat Jun 30 10:08:59 PDT 2012
% lyan Fri Jun 29 19:54:18 PDT 2012
% marek Thu Jun 28 09:21:14 PDT 2012
% lyan Thu Jun 28 00:11:28 PDT 2012
% marek Wed Jun 27 11:24:07 PDT 2012
% lyan Wed Jun 27 2012, 10:08:21
% marek Tue Jun 26 14:48:45 PDT 2012
% lyan Mon Jun 25 2012, 22:23:13
% marek Sun Jun 24 16:46:23 PDT 2012
% marek Wed Jun 20 04:42:40 PDT 2012
% lyan, Sun Jun 17 2012, 09:49:22
% marek Sat Apr  7 16:42:21 PDT 2012
% marek Thu Apr  5 11:39:58 PDT 2012
% marek Wed Apr  4 11:28:20 PDT 2012
% lyan, 04/01/12 10:20 PM
% lyan, Mon Mar 26 2012, 09:10:54
% lyan, Tue Mar 20 2012, 23:28:17
% lyan, 03/18/12 12:28 PM
% marek Sat Mar 17 13:42:32 PDT 2012
% marek, Wed Mar  7 21:28:24 PST 2012
% marek Mon Mar 12 12:08:25 PDT 2012



\end{document}

%% reference
%% http://www.maths.qmul.ac.uk/~fv/books/mw/mwbook.pdf
%%
%% TODO: pdflatex has font issue with \beta k, ligature?

%% 05/15/2013, first draft