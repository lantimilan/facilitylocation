
%%%%%%%%%%%%%%%%%%%%%%%%%%%%%%%%%%%%%%%%%%%%%%%%%%%%%%%%%%%%%%%%%%%%%%%%%%%%%%%%
% APPENDIX %%%%%%%%%%%%%%%%%%%%%%%%%%%%%%%%%%%%%%%%%%%%%%%%%%%%%%%%%%%%%%%%%%%%%
%%%%%%%%%%%%%%%%%%%%%%%%%%%%%%%%%%%%%%%%%%%%%%%%%%%%%%%%%%%%%%%%%%%%%%%%%%%%%%%%


\section{Proof of Lemma~\ref{lem: EBGS target connection cost}}\label{sec: proof of lemma 15}

Lemma~\ref{lem: EBGS target connection cost} provides a bound on the
expected connection cost of a demand $\nu$ when Algorithm~{\EBGS} does not open
any facilities in $\wbarN(\nu)$, namely
%
\begin{equation}
  \Exp[C_{\nu} \mid \neg \Lambda^{\nu}] \leq
  \clsdist(\nu) +  2\fardist(\nu),
			\label{eqn: lemma ebgs target connection cost}
\end{equation}
%
We show a stronger inequality that 
\begin{equation}
  \Exp[C_{\nu} \mid \neg \Lambda^{\nu}] \leq
  \clsdist(\nu) + \clsmax(\nu) + \fardist(\nu)
			\label{eqn: lemma ebgs indirect connection cost},
\end{equation}
which then implies (\ref{eqn: lemma ebgs target connection cost})
because $\clsmax(\nu) \leq \fardist(\nu)$.  The proof of (\ref{eqn:
  lemma ebgs indirect connection cost}) is similar to that in
\cite{ByrkaA10}. For the sake of completeness, we provide it here,
formulated in our terminology and notation.

Assume that the event $\neg \Lambda^{\nu}$ is true, that is Algorithm~{\EBGS}
does not open any facility in $\wbarN(\nu)$.
Let $\kappa$ be the primary demand that $\nu$ was assigned to. Also let
%K
\begin{equation*}
K = \wbarclsnb(\kappa) \setminus \wbarN(\nu), \quad
V_{\cls} = \wbarclsnb(\kappa) \cap \wbarclsnb(\nu) \quad \textrm{and}\quad 
V_{\far} = \wbarclsnb(\kappa) \cap \wbarfarnb(\nu).
\end{equation*}
% 
Then $K, V_{\cls}, V_{\far}$ form a partition of
$\wbarclsnb(\kappa)$, that is, they are disjoint and their union is $\wbarclsnb(\kappa)$.
Moreover, we have that $K$ is not empty, because Algorithm~{\EBGS}
opens some facility in $\wbarclsnb(\kappa)$ and this facility cannot be in $V_{\cls}\cup V_{\far}$,
by our assumption. 
We also have that $V_{\cls}$ is not empty due to (PD'.\ref{PD1:assign:overlap}). 

Recall that $D(A,\eta) = \sum_{\mu\in A}d_{\mu\eta}\bary_{\mu}/\sum_{\mu\in A}\bary_{\mu}$
is the average distance between a demand $\eta$ and the facilities in a set $A$. We shall show that
%
\begin{equation}
	 D(K, \nu) \leq \clsdist(\kappa)+\clsmax(\kappa) + \fardist(\nu).
				\label{eqn: bound on D(K,nu)}
\end{equation}
%
This is sufficient, because, by the algorithm, $D(K,\nu)$ is exactly 
the expected connection cost for demand $\nu$ conditioned on
the event that none of $\nu$'s neighbors 
opens, that is the left-hand side of (\ref{eqn: lemma ebgs indirect connection cost}).
Further, (PD'.\ref{PD1:assign:cost}) states that 
$\clsdist(\kappa)+\clsmax(\kappa) \le \clsdist(\nu) + \clsmax(\nu)$, and thus
(\ref{eqn: bound on D(K,nu)})  implies (\ref{eqn: lemma ebgs indirect connection cost}).

\medskip

The proof of (\ref{eqn: bound on D(K,nu)}) is by analysis of several cases.
%

\medskip
\noindent
{\mycase{1}} $D(K, \kappa) \leq \clsdist(\kappa)$. For any
facility $\mu \in V_{\cls}$ (recall that $V_{\cls}\neq\emptyset$), 
we have $d_{\mu\kappa} \leq \clsmax(\kappa)$ 
and $d_{\mu\nu} \leq \clsmax(\nu) \leq \fardist(\nu)$. Therefore, using the
case assumption, we get
	$D(K,\nu) \leq D(K,\kappa) + d_{\mu\kappa} + d_{\mu\nu} 
				\leq \clsdist(\kappa) + \clsmax(\kappa) + \fardist(\nu)$.

\medskip
\noindent
{\mycase{2}} There exists a facility $\mu\in V_{\cls}$ such that
  $d_{\mu\kappa} \leq \clsdist(\kappa)$. Since $\mu\in V_{\cls}$, we infer
  that $d_{\mu\nu} \leq \clsmax(\nu) \leq \fardist(\nu)$.  Using
  $\clsmax(\kappa)$ to bound $D(K, \kappa)$, we have $D(K, \nu)
  \leq D(K, \kappa) + d_{\mu\kappa} + d_{\mu\nu} \leq
  \clsmax(\kappa) + \clsdist(\kappa) + \fardist(\nu)$.

\medskip
\noindent
{\mycase{3}} In this case we assume that neither of Cases~1 and 2 applies, that is
 $D(K, \kappa) > \clsdist(\kappa)$ and every $\mu \in V_{\cls}$ satisfies
 $d_{\mu\kappa} >  \clsdist(\kappa)$. This implies that
$D(K\cup V_{\cls}, \kappa) > \clsdist(\kappa) = D(\wbarclsnb(\kappa), \kappa)$.
Since sets $K$, $V_{\cls}$ and $V_{\far}$ form a partition of $\wbarclsnb(\kappa)$,
we obtain that in this case $V_{\far}$ is not
empty and $D(V_{\far}, \kappa) < \clsdist(\kappa)$. 
Let $\delta = \clsdist(\kappa) - D(V_{\far}, \kappa) > 0$. 
We now have two sub-cases:
%
\begin{description}
	
\item{\mycase{3.1}} {$D(V_{\far}, \nu) \leq \fardist(\nu) + \delta$}.
  Substituting $\delta$, this implies that $D(V_{\far}, \nu) +
  D(V_{\far},\kappa) \le \clsdist(\kappa) + \fardist(\nu)$.  From the
  definition of the average distance $D(V_{\far},\kappa)$ and
  $D(V_{\far}, \nu)$, we obtain that there exists some $\mu \in
  V_{\far}$ such that $d_{\mu\kappa} + d_{\mu\nu} \leq
  \clsdist(\kappa) + \fardist(\nu)$.  Thus $D(K, \nu) \leq D(K,
  \kappa) + d_{\mu\kappa} + d_{\mu\nu} \leq \clsmax(\kappa) +
  \clsdist(\kappa) + \fardist(\nu)$.

\item{\mycase{3.2}} {$D(V_{\far}, \nu) > \fardist(\nu) + \delta$}.
  The case assumption implies that $V_{\far}$ is a proper subset of
  $\wbarfarnb(\nu)$, that is $\wbarfarnb(\nu) \setminus V_{\far}
  \neq\emptyset$.  Let $\hat{y} = \gamma \sum_{\mu\in V_{\smallfar}}
  \bary_{\mu}$.  We can express $\fardist(\nu)$ using $\hat{y}$ as
  follows
%
\begin{equation*}
\fardist(\nu) = D(V_{\far},\nu) \frac{\hat{y}}{\gamma-1} +
    D(\wbarfarnb(\nu)\setminus V_{\far}, \nu) \frac{\gamma-1-\hat{y}}{\gamma-1}.
\end{equation*}
%
Then, using the case condition and simple algebra, we have
%
  \begin{align}
    \clsmax(\nu) &\leq D(\wbarfarnb(\nu) \setminus V_{\far}, \nu) 
			\notag
		\\
		&\leq \fardist(\nu) - \frac{\hat{y}\delta}{\gamma-1-\hat{y}} 
		\leq \fardist(\nu) - \frac{\hat{y}\delta}{1-\hat{y}},
			\label{eqn: case 3, bound on C_cls^max(nu)}
  \end{align}
%
where the last step follows from $1 < \gamma < 2$. 

On the other hand, since $K$, $V_{\cls}$, and $V_{\far}$ form a partition of $\wbarclsnb(\kappa)$,
we have
$\clsdist(\kappa) = (1-\hat{y}) D(K\cup V_{\cls}, \kappa) + \hat{y} D(V_{\far}, \kappa)$.
Then using the definition of $\delta$ we obtain
%
\begin{equation}
    D(K \cup V_{\cls}, \kappa) = \clsdist(\kappa) + \frac{\hat{y}\delta}{1-\hat{y}}.
				\label{eqn: formula for D(V_cls,kappa)}
\end{equation}
%
  Now we are essentially done. If there exists some $\mu \in V_{\cls}$ such
  that $d_{\mu\kappa} \leq \clsdist(\kappa) +
  \hat{y}\delta/(1-\hat{y})$, then	we have
%
  \begin{align*}
    D(K, \nu) &\leq D(K, \kappa) + d_{\mu\kappa} + d_{\mu\nu} \\
    &\leq \clsmax(\kappa) + \clsdist(\kappa) +
    			\frac{\hat{y}\delta}{1-\hat{y}}
    + \clsmax(\nu)\\
    &\leq \clsmax(\kappa) + \clsdist(\kappa) + \fardist(\nu),
  \end{align*}
%
where we used (\ref{eqn: case 3, bound on C_cls^max(nu)}) in the last step.
  Otherwise, from (\ref{eqn: formula for D(V_cls,kappa)}),
we must have $D(K, \kappa) \leq \clsdist(\kappa) +
  \hat{y}\delta/(1-\hat{y})$. Choosing any $\mu \in V_{\cls}$, it follows that
%
  \begin{align*}
    D(K, \nu) &\leq D(K, \kappa) + d_{\mu\kappa} + d_{\mu\nu} \\
    &\leq \clsdist(\kappa) + \frac{\hat{y}\delta}{1-\hat{y}} +
    		\clsmax(\kappa)  + \clsmax(\nu)\\
    &\leq \clsdist(\kappa) + \clsmax(\kappa) + \fardist(\nu),
  \end{align*}
%
again using (\ref{eqn: case 3, bound on C_cls^max(nu)}) in the last step.

\end{description}

This concludes the proof of (\ref{eqn: lemma ebgs target connection cost}).
As explained earlier, Lemma~\ref{lem: EBGS target connection cost} follows.

%%%%%%%%%%%%%%%%%%%%%%%%%%%%%%%%%%%%%%%%%%%%%%%%%%%%%%%%%%%%%%%%%%%%
%%%%%%%%%%%%%%%%%%%%%%%%%%%%%%%%%%%%%%%%%%%%%%%%%%%%%%%%%%%%%%%%%%%%
%%%%%%%%%%%%%%%%%%%%%%%%%%%%%%%%%%%%%%%%%%%%%%%%%%%%%%%%%%%%%%%%%%%%

\vfill

\section{Proof of Inequality  (\ref{eqn: echs ineq direct cost, step 1})}
\label{sec: ECHSinequality}

In Sections~\ref{sec: 1.736-approximation} and \ref{sec: 1.575-approximation}
we use the following inequality
%
\begin{align}
  \label{eq:min expected distance}
  \bard_1 g_1 + \bard_2 g_2 (1-g_1) +
  \ldots &+ \bard_k g_k (1-g_1) (1-g_2) \ldots (1-g_k)\\ \notag
  &\leq \frac{1}{\sum_{s=1}^k g_s} \left(\textstyle\sum_{s=1}^k \bard_s g_s\right)\left(\textstyle\sum_{t=1}^k g_t \textstyle\prod_{z=1}^{t-1} (1-g_z)\right).
\end{align}
%
for $0 < \bard_1\leq \bard_2 \leq \ldots \leq \bard_k$, and
$0 < g_1,...,g_s \le 1$.

\medskip

We give here a new proof of this inequality, much simpler
than the existing proof in \cite{ChudakS04}, and also simpler than the
argument by Sviridenko~\cite{Svi02}.  We derive this inequality from
the following generalized version of the Chebyshev Sum Inequality:
%
\begin{equation}
  \label{eq:cheby}
  \textstyle{\sum_{i}} p_i \textstyle{\sum_j} p_j a_j b_j \leq \textstyle{\sum_i} p_i a_i \textstyle{\sum_j} p_j b_j,
\end{equation}
%
where each summation runs from $1$ to $l$ and the sequences $(a_i)$,
$(b_i)$ and $(p_i)$ satisfy the following conditions: $p_i\geq 0, a_i
\geq 0, b_i \geq 0$ for all $i$, $a_1\leq a_2 \leq \ldots \leq a_l$,
and $b_1 \geq b_2 \geq \ldots \geq b_l$.

Given inequality (\ref{eq:cheby}), we can obtain our inequality
(\ref{eq:min expected distance}) by simple substitution
%
\begin{equation*}
  p_i \leftarrow g_i, a_i \leftarrow \bard_i, b_i \leftarrow
  \Pi_{s=1}^{i-1} (1-g_s),
\end{equation*}
%
for $i = 1,...,k$.

\ignore{
For the sake of completeness, we include the proof of inequality (\ref{eq:cheby}), 
due to Hardy, Littlewood and Polya~\cite{HardyLP88}. The idea is to evaluate the 
following sum:
%
\begin{align*}
  S &= \textstyle{\sum_i} p_i \textstyle{\sum_j} p_j a_j b_j - \textstyle{\sum_i} p_i a_i \textstyle{\sum_j} p_j b_j
	\\
  & = \textstyle{\sum_i \sum_j} p_i p_j a_j b_j - \textstyle{\sum_i \sum_j} p_i a_i p_j b_j
	\\
  & = \textstyle{\sum_j \sum_i} p_j p_i a_i b_i - \textstyle{\sum_j \sum _i} p_j a_j p_i b_i
	\\
	&= \half \cdot \textstyle{\sum_i \sum_j} (p_i p_j a_j b_j - p_i a_i p_j b_j + p_j p_i a_i
  							b_i - p_j a_j p_i b_i)
\\
  &= \half \cdot \textstyle{\sum_i \sum_j} p_i p_j (a_i - a_j)(b_i - b_j) \leq 0.
\end{align*}
The last inequality holds because $(a_i-a_j)(b_i-b_j) \leq 0$, since the sequences
$(a_i)$ and $(b_i)$ are ordered oppositely.
}

\end{document}

% marek Tue Jul  3 10:21:05 PDT 2012
% marek Sun Jul  1 14:57:39 PDT 2012
% lyan Sat Jun 30 2012, 22:01:27
% marek Sat Jun 30 10:08:59 PDT 2012
% lyan Fri Jun 29 19:54:18 PDT 2012
% marek Thu Jun 28 09:21:14 PDT 2012
% lyan Thu Jun 28 00:11:28 PDT 2012
% marek Wed Jun 27 11:24:07 PDT 2012
% lyan Wed Jun 27 2012, 10:08:21
% marek Tue Jun 26 14:48:45 PDT 2012
% lyan Mon Jun 25 2012, 22:23:13
% marek Sun Jun 24 16:46:23 PDT 2012
% marek Wed Jun 20 04:42:40 PDT 2012
% lyan, Sun Jun 17 2012, 09:49:22
% marek Sat Apr  7 16:42:21 PDT 2012
% marek Thu Apr  5 11:39:58 PDT 2012
% marek Wed Apr  4 11:28:20 PDT 2012
% lyan, 04/01/12 10:20 PM
% lyan, Mon Mar 26 2012, 09:10:54
% lyan, Tue Mar 20 2012, 23:28:17
% lyan, 03/18/12 12:28 PM
% marek Sat Mar 17 13:42:32 PDT 2012
% marek, Wed Mar  7 21:28:24 PST 2012
% marek Mon Mar 12 12:08:25 PDT 2012
