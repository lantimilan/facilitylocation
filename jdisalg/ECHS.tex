% marek Thu Nov 29 10:18:23 PST 2012, done with the journal version


\section{Algorithm~{\ECHS} with Ratio $1.736$}\label{sec: 1.736-approximation}

In this section we improve the approximation ratio to $1+2/e \approx
1.736$. The improvement comes from a slightly modified rounding
process and refined analysis.  Note that the facility opening cost of
Algorithm~{\EGUP} does not exceed that of the fractional optimum
solution, while the connection cost could be far from the optimum,
since we connect a non-primary demand to a facility in the neighborhood of
its assigned primary demand and then estimate the distance using the
triangle inequality. The basic idea to improve the estimate of the connection cost,
following the approach of Chudak and Shmoys~\cite{ChudakS04}, 
is to connect each non-primary demand to its
nearest neighbor when one is available, and to only use the facility opened by
its assigned primary demand when none of its neighbors is open.

%%%%%%%%%%

\paragraph{Algorithm~{\ECHS}.}
As before,
the algorithm starts by solving the linear program and applying the
adaptive partitioning algorithm  described in 
Section~\ref{sec: adaptive partitioning} to obtain a partitioned
solution $(\barbfx, \barbfy)$. Then we apply the rounding
process to compute an integral solution (see Pseudocode~\ref{alg:lpr3}).  

We start, as before, by opening exactly one facility $\phi(\kappa)$ in the 
neighborhood of each primary demand $\kappa$ (Line 2).  For any
non-primary demand $\nu$ assigned to $\kappa$, we refer to
$\phi(\kappa)$ as the \emph{target} facility of $\nu$.  In
Algorithm~{\EGUP}, $\nu$ was connected to $\phi(\kappa)$,
but in Algorithm~{\ECHS} we may be able to find an open
facility in $\nu$'s neighborhood and connect $\nu$ to this
facility.  Specifically, the two changes in the
algorithm are as follows:
%
\begin{description}
\item{(1)} Each facility $\mu$ that is not in the neighborhood of any
  primary demand is opened, independently, with probability
  $\bary_{\mu}$ (Lines 4--5). Notice that if $\bary_\mu>0$ then, due
  to completeness of the partitioned fractional solution, we have
  $\bary_{\mu}= \barx_{\mu\nu}$ for some demand $\nu$. This implies
  that $\bary_{\mu}\leq 1$, because $\barx_{\mu\nu}\le 1$, by
  (PS.\ref{PS:one}).
%
\item{(2)} When connecting demands to facilities, a primary demand
  $\kappa$ is connected to the only facility $\phi(\kappa)$ opened in
  its neighborhood, as before (Line 3).  For a non-primary demand
  $\nu$, if its neighborhood $\wbarN(\nu)$ has an open facility, we
  connect $\nu$ to the closest open facility in $\wbarN(\nu)$ (Line
  8). Otherwise, we connect $\nu$ to its target facility (Line 10).
%
\end{description}

%%%%%%%%%%%%%

\begin{algorithm}
  \caption{Algorithm~{\ECHS}:
    Constructing Integral Solution}
  \label{alg:lpr3}
  \begin{algorithmic}[1]
    \For{each $\kappa\in P$} 
    \State choose one $\phi(\kappa)\in \wbarN(\kappa)$,
    with each $\mu\in\wbarN(\kappa)$ chosen as $\phi(\kappa)$
    with probability $\bary_\mu$ 
    \State open $\phi(\kappa)$ and connect $\kappa$ to $\phi(\kappa)$
    \EndFor
    \For{each $\mu\in\facilityset - \bigcup_{\kappa\in P}\wbarN(\kappa)$} 
    \State open $\mu$ with probability $\bary_\mu$ (independently)
    \EndFor
    \For{each non-primary demand $\nu\in\demandset$}
    \If{any facility in $\wbarN(\nu)$ is open}
    \State{connect $\nu$ to the nearest open facility in $\wbarN(\nu)$}
    \Else
    \State connect $\nu$ to $\phi(\kappa)$ where $\kappa$ is $\nu$'s
     assigned primary demand
    \EndIf
    \EndFor
  \end{algorithmic}
\end{algorithm}

%%%%%%%%%%%%%%%%

\paragraph{Analysis.}
We shall first argue that the integral solution thus
constructed is feasible, and then we bound the total cost of
the solution. Regarding feasibility, the only constraint
that is not explicitly enforced by the algorithm is the
fault-tolerance requirement; namely that each client $j$ is
connected to $r_j$ different facilities. Let $\nu$ and
$\nu'$ be two different sibling demands of client $j$ and let
their assigned primary demands be $\kappa$ and $\kappa'$
respectively. Due to (SI.\ref{SI:primary
  disjoint}) we know $\kappa \neq \kappa'$. From
(SI.\ref{SI:siblings disjoint}) we have $\wbarN(\nu) \cap
\wbarN(\nu') = \emptyset$. From (SI.\ref{SI:primary
  disjoint}), we have $\wbarN(\nu) \cap \wbarN(\kappa') =
\emptyset$ and $\wbarN(\nu') \cap \wbarN(\kappa) =
\emptyset$. From (PD.\ref{PD:disjoint}) we have
$\wbarN(\kappa)\cap \wbarN(\kappa') = \emptyset$. It follows
that $(\wbarN(\nu) \cup \wbarN(\kappa)) \cap (\wbarN(\nu')
\cup \wbarN(\kappa')) = \emptyset$. Since the algorithm
connects $\nu$ to some facility in $\wbarN(\nu) \cup
\wbarN(\kappa)$ and $\nu'$ to some facility in $\wbarN(\nu')
\cup \wbarN(\kappa')$, $\nu$ and $\nu'$ will be connected to
different facilities.


%%%%%%%%%

\smallskip
We now show that the expected cost of the computed solution is bounded by
$(1+2/e) \cdot \LP^\ast$. By
(PD.\ref{PD:disjoint}), every facility may appear in at
most one primary demand's neighborhood, and the facilities
open in Line~4--5 of Pseudocode~\ref{alg:lpr3} do not appear
in any primary demand's neighborhood. Therefore, by
linearity of expectation, the expected facility cost of
Algorithm~{\ECHS} is 
%
\begin{equation*}
\Exp[F_{\smallECHS}] 
	= \sum_{\mu\in\facilityset} f_\mu \bary_{\mu} 
	= \sum_{i\in\sitesset} f_i\sum_{\mu\in i} \bary_{\mu} 
	= \sum_{i\in\sitesset} f_i y_i^\ast = F^\ast,
\end{equation*}
%
where the third equality follows from (PS.\ref{PS:yi}).

\smallskip

To bound the connection cost, we adapt an argument of Chudak
and Shmoys~\cite{ChudakS04}. Consider a demand $\nu$ and denote by $C_\nu$ the
random variable representing the connection cost for $\nu$.
Our goal now is to estimate $\Exp[C_\nu]$, the expected value of $C_\nu$.
Demand $\nu$ can either get connected directly to some facility in
$\wbarN(\nu)$ or indirectly to its target facility $\phi(\kappa)\in
\wbarN(\kappa)$, where $\kappa$ is the primary demand to
which $\nu$ is assigned. We will analyze these two cases separately.

In our analysis, in this section and the next one, we will use notation
$D(A,\sigma) \stackrel{\mathrm{def}}{=}\sum_{\mu\in A}
d_{\mu\sigma}\bary_{\mu}/\sum_{\mu\in A} \bary_{\mu}$ for the
average distance between a demand $\sigma$ and a set $A$ of facilities.
Note that, in particular, we have $\concost_\nu = D(\wbarN(\nu),\nu)$.

We first estimate the expected cost $d_{\phi(\kappa)\nu}$ of the indirect
connection. Let $\Lambda^\nu$ denote the event that some 
facility in $\wbarN(\nu)$ is opened. Then
%
\begin{equation}
	\Exp[C_\nu \mid\neg\Lambda^\nu] 
	=   \Exp[ d_{\phi(\kappa)\nu} \mid \neg\Lambda^\nu] 
	= 	D(\wbarN(\kappa) \setminus \wbarN(\nu), \nu),
			\label{eqn: expected indirect connection}
\end{equation}
%
Note that $\neg\Lambda^\nu$ implies that $\wbarN(\kappa) \setminus
\wbarN(\nu)\neq\emptyset$, since $\wbarN(\kappa)$ contains
exactly one open facility, namely $\phi(\kappa)$.

%%%%%%%%%%

\begin{lemma}
  \label{lem:echu indirect}
  Let $\nu$ be a demand assigned to a primary demand $\kappa$, and
assume that $\wbarN(\kappa) \setminus \wbarN(\nu)\neq\emptyset$.
Then $\Exp[ C_\nu \mid\neg\Lambda^\nu]  \leq
  		\concost_\nu+2\alpha_{\nu}^\ast$.
\end{lemma}

\begin{proof}
By (\ref{eqn: expected indirect connection}), we need to show that $D(\wbarN(\kappa)
  \setminus \wbarN(\nu), \nu) \leq \concost_\nu +
  2\alpha_{\nu}^\ast$. There are two cases to consider.

\begin{description}
%	
\item{\mycase{1}}
	 There exists some $\mu'\in \wbarN(\kappa) \cap
  \wbarN(\nu)$ such that $d_{\mu' \kappa} \leq \concost_\kappa$.
In this case, for every $\mu\in \wbarN(\kappa)\setminus \wbarN(\nu)$, we have
%
\begin{equation*}
d_{\mu \nu} \leq d_{\mu \kappa} + d_{\mu' \kappa} + d_{\mu' \nu}  
 	\le  \alpha^\ast_\kappa + \concost_\kappa + \alpha^\ast_{\nu}
  \leq \concost_\nu + 2\alpha_{\nu}^\ast,
\end{equation*}
%
using the triangle inequality, complementary slackness, and (PD.\ref{PD:assign:cost}).
By summing over all $\mu\in \wbarN(\kappa) \setminus \wbarN(\nu)$, it
follows that $D(\wbarN(\kappa) \setminus \wbarN(\nu), \nu) \leq
\concost_\nu + 2\alpha_{\nu}^\ast$.

\item{\mycase{2}}
 Every $\mu'\in \wbarN(\kappa)\cap \wbarN(\nu)$
has $d_{\mu'\kappa} > \concost_\kappa$. Since $\concost_{\kappa} = D(\wbarN(\kappa),\kappa)$,
this implies that
$D(\wbarN(\kappa) \setminus \wbarN(\nu),\kappa)\leq \concost_{\kappa}$. Therefore,
choosing an arbitrary $\mu'\in \wbarN(\kappa)\cap \wbarN(\nu)$,
we obtain
%
\begin{equation*}
  D(\wbarN(\kappa) \setminus \wbarN(\nu), \nu) 
	\leq  D(\wbarN(\kappa) \setminus \wbarN(\nu), \kappa) 
			+ d_{\mu' \kappa} + d_{\mu' \nu} 
	\leq  \concost_{\kappa} +
  \alpha_{\kappa}^\ast + \alpha_{\nu}^\ast
	\leq \concost_\nu + 2\alpha_{\nu}^\ast,
\end{equation*}
%
where we again use the triangle inequality,
complementary slackness, and  (PD.\ref{PD:assign:cost}).
%
\end{description}
%
Since the lemma holds in both cases, the proof is now complete.
\end{proof}

We now continue our estimation of the connection cost.  The next step
of our analysis is to show that 
%
\begin{equation}
	\Exp[C_\nu]\le \concost_{\nu} + \frac{2}{e}\alpha^\ast_\nu.
	\label{eqn: echs bound for connection cost}
\end{equation}
%
The argument is divided into three cases. The first, easy case is when
$\nu$ is a primary demand $\kappa$. According to the algorithm
(see Pseudocode~\ref{alg:lpr3}, Line~2), we have $C_\kappa = d_{\mu\kappa}$ with probability $\bary_{\mu}$, 
for $\mu\in \wbarN(\kappa)$. Therefore $\Exp[C_\kappa] = \concost_{\kappa}$, so
(\ref{eqn: echs bound for connection cost}) holds.

Next, we consider a non-primary demand $\nu$. Let $\kappa$
be the primary demand that $\nu$ is assigned to. We first
deal with the sub-case when $\wbarN(\kappa)\setminus
\wbarN(\nu) = \emptyset$, which is the same as
$\wbarN(\kappa) \subseteq \wbarN(\nu)$. Property (CO)
implies that $\barx_{\mu\nu} = \bary_{\mu} =
\barx_{\mu\kappa}$ for every $\mu \in \wbarN(\kappa)$, so we
have $\sum_{\mu\in\wbarN(\kappa)} \barx_{\mu\nu} =
\sum_{\mu\in\wbarN(\kappa)} \barx_{\mu\kappa} = 1$, due to
(PS.\ref{PS:one}). On the other hand, we have
$\sum_{\mu\in\wbarN(\nu)} \barx_{\mu\nu} = 1$, and
$\barx_{\mu\nu} > 0$ for all $\mu\in \wbarN(\nu)$. Therefore
$\wbarN(\kappa) = \wbarN(\nu)$ and $C_\nu$ has exactly the
same distribution as $C_\kappa$.  So this case reduces to
the first case, namely we have $\Exp[C_{\nu}] =
\concost_{\nu}$, and (\ref{eqn: echs bound for connection
  cost}) holds.

The last, and only non-trivial case is when $\wbarN(\kappa)\setminus
\wbarN(\nu)\neq\emptyset$. We handle this case in the following lemma.

%%%%%%

\begin{lemma}\label{lem: echs expected C_nu}
Assume that $\wbarN(\kappa) \setminus \wbarN(\nu) \neq \emptyset$.
Then the expected connection cost of $\nu$, conditioned on the event that at least one of 
its neighbor opens, satisfies
%
\begin{equation*}
  \Exp[C_\nu \mid \Lambda^\nu] \leq \concost_{\nu}.
\end{equation*}
\end{lemma}

\begin{proof}
The proof is similar to an analogous result in~\cite{ChudakS04,ByrkaA10}. 
We sketch it here for the sake of completeness. 
The idea is to consider a different random process that is
easier to analyze and whose expected connection cost is not better than that in
the algorithm.

We partition $\wbarN(\nu)$ into groups $G_1,...,G_k$, where two
different facilities $\mu$ and $\mu'$ are put in the same $G_s$ where
$s\in \{1,\ldots,k\}$, if they both belong to the same set
$\wbarN(\kappa)$ for some primary demand $\kappa$. If some $\mu$ is
not a neighbor of any primary demand, then it constitutes a singleton
group.  For each $s$, let $\bard_s = D(G_s,\nu)$ be the average
distance from $\nu$ to $G_s$.  Assume that $G_1,...,G_k$ are ordered
by nondecreasing average distance to $\nu$, that is $\bard_1 \le
\bard_2 \le ... \le \bard_k$.  For each group $G_s$, we select it,
independently, with probability $g_s = \sum_{\mu\in G_s}\bary_{\mu}$,
and then open exactly one facility in $G_s$, where each $\mu\in G_s$
is opened with probability $\bary_{\mu}/\sum_{\eta\in G_s}
\bary_{\eta}$.

So far, this process is the same as that in the algorithm (if restricted to $\wbarN(\nu)$).
However, we connect $\nu$ in a slightly different way, by choosing the smallest
$s$ for which $G_s$ was selected and connecting $\nu$ to the open facility in $G_s$.
This can only increase our expected connection cost, assuming that at least one
facility in $\wbarN(\nu)$ opens, so
%
\begin{align}
  \Exp[C_\nu \mid \Lambda^\nu] &\leq \frac{1}{\Prob[\Lambda^\nu]}
  \left( \bard_1 g_1 + \bard_2 g_2 (1-g_1) + \ldots + \bard_k g_k
    (1-g_1) (1-g_2) \ldots (1-g_k) \right)
			\notag
  \\
  &\leq \frac{1}{\Prob[\Lambda^\nu]}\cdot \frac{\textstyle \sum_{s=1}^k \bard_s g_s}{\textstyle \sum_{s=1}^k g_s}\cdot
\left({\textstyle \sum_{t=1}^k} g_t \textstyle \prod_{z=1}^{t-1} (1-g_z)\right)
			\label{eqn: echs ineq direct cost, step 1}
  \\
  &= \textstyle \sum_{s=1}^k \bard_s g_s
			\label{eqn: echs ineq direct cost, step 2}
	\\
			&= \concost_{\nu}.
				\label{eqn: echs ineq direct cost, step 3}
\end{align}
%
The proof for inequality (\ref{eqn: echs ineq direct cost, step 1}) 
is given in \ref{sec: ECHSinequality}, 
equality (\ref{eqn: echs ineq direct cost, step 2}) follows from
$\Prob[\Lambda^\nu] = 1 - \prod_{t=1}^k (1-g_t)
					= \sum_{t=1}^k g_t
                                        \prod_{z=1}^{t-1} (1 - g_z)$ and $\sum_{s=1}^k g_s = 1$,
and (\ref{eqn: echs ineq direct cost, step 3}) follows from the definition
of the distances $\bard_s$, probabilities $g_s$, and simple algebra.
\end{proof}

We are now ready to estimate the (unconditional) expected connection cost of $\nu$ as
follows:
%
\begin{align}
  \notag
  \Exp[C_\nu] &= \Exp[C_{\nu} \mid \Lambda^\nu] \cdot \Prob[\Lambda^\nu] 
	+ \Exp[C_{\nu} \mid \neg \Lambda^\nu] \cdot	\Prob[\neg \Lambda^\nu]
  \\
  &\leq \concost_{\nu} \cdot \Prob[\Lambda^\nu] 
		+ (\concost_{\nu} + 2\alpha_{\nu}^\ast)  \cdot \Prob[\neg \Lambda^\nu]
  \label{eqn: Cnu estimate 0}
  \\
  &= \concost_{\nu} 
	+  2\alpha_{\nu}^\ast \cdot \Prob[\neg \Lambda^\nu]
		\notag
  \\
  &= \concost_{\nu} + 2\alpha_{\nu}^\ast \cdot \textstyle \prod_{s=1}^k (1 - g_s)
		\notag
                \\
	&\leq \concost_{\nu} 
	+  2\alpha_{\nu}^\ast \cdot \textstyle e^{- \sum_{s=1}^k g_s}
        \label{eqn: Cnu estimate 1}
        \\
        &= \concost_{\nu} + 2\alpha_{\nu}^\ast \cdot \textstyle e^{-
          \sum_{\mu \in \wbarN(\nu)} \bary_{\mu}}
        \label{eqn: Cnu estimate 2}
	\\
	&= \concost_{\nu} + \frac{2}{e}\cdot\alpha_{\nu}^\ast.
	  \label{eqn: Cnu estimate last}
\end{align}
%
In the above derivation, inequality (\ref{eqn: Cnu estimate 0})
follows from Lemmas~\ref{lem:echu indirect} and \ref{lem: echs
  expected C_nu}, inequality (\ref{eqn: Cnu estimate 1}) follows from
$1-x \leq e^{-x}$ for any $x$, the equality (\ref{eqn: Cnu estimate
  2}) follows from $\sum_{s=1}^k g_s = \sum_{\mu \in \wbarN(\nu)} \bary_{\mu}$, and
equality (\ref{eqn: Cnu estimate last}) follows from $\sum_{\mu \in \wbarN(\nu)}
\bary_{\mu} = 1$.

Summing over all demands of a client $j$, we can now bound
the expected connection cost of client $j$:
%
\begin{equation*}
  \Exp[C_j] = \textstyle\sum_{\nu\in j} \Exp[C_\nu] 
\leq {\textstyle\sum_{\nu\in j} (\concost_{\nu} + \frac{2}{e}\cdot\alpha_{\nu}^\ast) }
  = { C_j^\ast + \frac{2}{e}\cdot r_j\alpha_j^\ast}.
\end{equation*}
%
Finally, summing over all clients $j$, we obtain our bound on
the expected connection cost,
%
\begin{equation*}
 \Exp[ C_{\smallECHS}] \le C^\ast + \frac{2}{e}\cdot\LP^\ast.
\end{equation*}
% 
Therefore, we have established that
our algorithm constructs a feasible integral solution with
an overall expected cost 
%
\begin{equation*}
  \label{eq:chudakall}
	 \Exp[ F_{\smallECHS} + C_{\smallECHS}]
	\le
  	F^\ast + C^\ast + \frac{2}{e}\cdot \LP^\ast = (1+2/e)\cdot \LP^\ast
  \leq (1+2/e)\cdot \OPT.
\end{equation*}
%
Summarizing, we obtain the main result of this section.

\begin{theorem}\label{thm:1736}
  Algorithm~{\ECHS} is a $(1+2/e)$-approximation algorithm for \FTFP.
\end{theorem}