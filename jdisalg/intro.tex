\section{Introduction}

In the \emph{Fault-Tolerant Facility Placement} problem
(FTFP), we are given a set $\sitesset$ of \emph{sites} at
which facilities can be built, and a set $\clientset$ of
\emph{clients} with some demands that need to be satisfied
by different facilities. A client $j\in\clientset$ has
demand $r_j$. Building one facility at a site
$i\in\sitesset$ incurs a cost $f_i$, and connecting one unit
of demand from client $j$ to a facility at site $i$ costs
$d_{ij}$. Throughout the paper we assume that the connection
costs (distances) $d_{ij}$ form a metric, that is, they are
symmetric and satisfy the triangle inequality. In a feasible
solution, some number of facilities, possibly zero, are
opened at each site $i$, and demands from each client are
connected to those open facilities, with the constraint that
demands from the same client have to be connected to
different facilities. Note that any two facilities at the
same site are considered different.

The FTFP problem is intended to model the fact that a client
in the real world may have more than one demand and each of
the demands needs to be satisfied by a distinct
facility. This requirement may be a result of performance
needs or fault-tolerance needs. For example, a web server
running some service may need to access multiple databases
so that it can fetch data in parallel, and be resilient to
the unfortunate days when one database needs to be upgraded
or goes offline while the web service has to be always
available. Those databases can either be set up in the same
location to simplify configuration and management, or
located at different places to guard against power outage or
natural disasters. In the supply chain domain, for a
distribution center, it is desirable that the center has
connection to multiple warehouses because those warehouses
carry different categories of commodities. In this
application, it is also possible for different warehouses to
be set up in the same neighborhood because of convenience of
transportation, while multiple warehouses are necessary,
because of the merchandise they carry are incomptabile, for
instance, toxic materials need to be separated from food.

From a theory perspective, the study of FTFP is motivated by
the discrepancy of approximation results for the classic
Uncapacitated Facility Location problem
(UFL)~\cite{ShmoysTA97} and the Fault-Tolerant Facility
Location problem (FTFL)~\cite{JainV03}.  It is easy to see
that if all $r_j=1$ then FTFP reduces to UFL.  If we add a
constraint that each site can have at most one facility
built on it, then the problem becomes equivalent to
FTFL. One implication of the one-facility-per-site
restriction in FTFL is that $\max_{j\in\clientset}r_j \leq
|\sitesset|$, while in FTFP the values of $r_j$'s can be
much bigger than $|\sitesset|$. The current best known
approximation result for FTFL does not match that for UFL
and the technique needed to address the fault-tolerant
requirement is sophisticated. As FTFP can be seen as more
generalized than UFL but has more relaxed constraints than
FTFP, the results we obtained on FTFP may shed light on how
the fault-tolerant constraint makes FTFL appear harder than
UFL.

The UFL problem has a long history; in particular, great
progress has been achieved in the past two decades in
developing techniques for designing constant-ratio
approximation algorithms for UFL.  Shmoys, Tardos and
Aardal~\cite{ShmoysTA97} proposed an approach based on
LP-rounding, that they used to achieve a ratio of 3.16.
This was then improved by Chudak~\cite{ChudakS04} to 1.736,
and later by Sviridenko~\cite{Svi02} to 1.582.  The best
known ``pure" LP-rounding algorithm is due to
Byrka~{\etal}~\cite{ByrkaGS10} with ratio 1.575.  Byrka and
Aardal~\cite{ByrkaA10} gave a hybrid algorithm that combines
LP-rounding and dual-fitting (based on \cite{JainMMSV03}),
achieving a ratio of 1.5.  Recently, Li~\cite{Li11} showed
that, with a more refined analysis and randomizing the
scaling parameter used in \cite{ByrkaA10}, the ratio can be
improved to 1.488. This is the best known approximation
result for UFL.  Other techniques include the primal-dual
algorithm with ratio 3 by Jain and Vazirani~\cite{JainV01},
the dual fitting method by Jain~{\etal}~\cite{JainMMSV03}
that gives ratio 1.61, and a local search heuristic by
Arya~{\etal}~\cite{AryaGKMMP04} with approximation ratio 3.
On the hardness side, UFL is easily shown to be {\NP}-hard,
and it is known that it is not possible to approximate UFL
in polynomial time with ratio less than $1.463$, provided
that $\NP\not\subseteq\DTIME(n^{O(\log\log
  n)})$~\cite{GuhaK98}. An observation by Sviridenko
strengthened the underlying assumption to $\PP\ne \NP$ (see
\cite{vygen05}).

FTFL was first introduced by Jain and
Vazirani~\cite{JainV03} and they adapted their primal-dual
algorithm for UFL to obtain a ratio of
$3\ln(\max_{j\in\clientset}r_j)$.  All subsequently
discovered constant-ratio approximation algorithms use
variations of LP-rounding.  The first such algorithm, by
Guha~{\etal}~\cite{GuhaMM01}, adapted the approach for UFL
from \cite{ShmoysTA97}.  Swamy and Shmoys~\cite{SwamyS08}
improved the ratio to $2.076$ using the idea of pipage
rounding introduced in \cite{Svi02}. Most recently,
Byrka~{\etal}~\cite{ByrkaSS10} improved the ratio to 1.7245
using dependent rounding and laminar clustering.

FTFP is a natural generalization of UFL. It was first
studied by Xu and Shen~\cite{XuS09}, who extended the
dual-fitting algorithm from~\cite{JainMMSV03} to give an
approximation algorithm with a ratio claimed to be
$1.861$. However their algorithm runs in polynomial time
only if $\max_{j\in\clientset} r_j$ is polynomial in
$O(|\sitesset|\cdot |\clientset|)$ and the analysis of the
performance guarantee in \cite{XuS09} is
flawed\footnote{Confirmed through private communication with
  the authors.}.  To date, the best approximation ratio for
FTFP in the literature is $3.16$, established by Yan and
Chrobak~\cite{YanC11}, while the only known lower bound is
the $1.463$ lower bound for UFL from~\cite{GuhaK98}, as UFL
is a special case of FTFP.  If all demand values $r_j$ are
equal, the problem can be solved by simple scaling and
applying LP-rounding algorithms for UFL. This does not
affect the approximation ratio, thus achieving ratio $1.575$
for this special case (see also \cite{LiaoShen11}).

\smallskip

The main result of this paper is an LP-rounding algorithm
for FTFP with approximation ratio 1.575, matching the best
ratio for UFL achieved via the LP-rounding method
\cite{ByrkaGS10} and significantly improving our earlier
bound in~\cite{YanC11}. In Section~\ref{sec: polynomial
  demands} we prove that, for the purpose of LP-based
approximations, the general FTFP problem can be reduced to
the restricted version where all demand values are
polynomial in the number of sites.  This \emph{demand
  reduction} trick itself gives us a ratio of $1.7245$,
since we can then treat an instance of FTFP as an instance
of FTFL by creating a sufficient (but polynomial) number of
facilities at each site, and then using the algorithm
from~\cite{ByrkaSS10} to solve the FTFL instance.

The reduction to polynomial demands suggests an approach
where clients' demands are split into unit demands. These
unit demands can be thought of as ``unit-demand clients'',
and a natural approach would be to adapt LP-rounding methods
from \cite{gupta08,ChudakS04,ByrkaGS10} to this new set of
unit-demand clients.  Roughly, these algorithms iteratively
pick a client that minimizes a certain cost function (that
varies for different algorithms) and open one facility in
the neighborhood of this client. The remaining clients are
then connected to these open facilities.  In order for this
to work, we also need to convert the optimal fractional
solution $(\bfx^\ast,\bfy^\ast)$ of the original instance
into a solution $(\barbfx,\barbfy)$ of the modified instance
which then can be used in the LP-rounding process. This can
be thought of as partitioning the fractional solution, as
each connection value $x^\ast_{ij}$ must be divided between
the $r_j$ unit demands of client $j$ in some way. In
Section~\ref{sec: adaptive partitioning} we formulate a set
of properties required for this partitioning to work. For
example, one property guarantees that we can connect demands
to facilities so that two demands from the same client are
connected to different facilities. Then we present our
\emph{adaptive partitioning} technique that computes a
partitioning with all the desired properties. Using adaptive
partitioning we were able to extend the algorithms for UFL
from \cite{gupta08,ChudakS04,ByrkaGS10} to FTFP. We
illustrate the fundamental ideas of our approach in
Section~\ref{sec: 3-approximation}, showing how they can be
used to design an LP-rounding algorithm with ratio $3$.  In
Section~\ref{sec: 1.736-approximation} we refine the
algorithm to improve the approximation ratio to
$1+2/e\approx 1.736$.  Finally, in Section~\ref{sec:
  1.575-approximation}, we improve it even further to
$1.575$ -- the main result of this paper.

Summarizing, our contributions are two-fold: One, we show
that the existing LP-rounding algorithms for UFL can be
extended to a much more general problem FTFP, retaining the
approximation ratio. We believe that, should even better
LP-rounding algorithms be developed for UFL in the future,
using our demand reduction and adaptive partitioning
methods, it should be possible to extend them to FTFP.  In
fact, an improvement of the ratio should be achieved by
randomizing the scaling parameter $\gamma$ used in our
algorithm, as Li showed in~\cite{Li11} for UFL. However, our
current algorithm will not give the same ratio of $1.488$
because Li's result also makes use of the dual-fitting
technique~\cite{MahdianMSV01}.

Two, our ratio of $1.575$ is significantly better than the
best currently known ratio of $1.7245$ for the
closely-related FTFL problem. This suggests that in the
fault-tolerant scenario, the capability of creating
additional copies of facilities on the existing sites makes
the problem easier from the point of view of approximation.
