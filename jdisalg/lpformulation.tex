\section{The LP Formulation}\label{sec: the lp formulation}

The FTFP problem has a natural Integer Programming (IP)
formulation. Let $y_i$ represent the number of facilities
built at site $i$ and let $x_{ij}$ represent the number of
connections from client $j$ to facilities at site $i$. If we
relax the integrality constraints, we obtain the following LP:

%%%%%%%%%%%
\begin{alignat}{3}
  \textrm{minimize} \quad \cost(\bfx,\bfy) &= \textstyle{\sum_{i\in \sitesset}f_iy_i 
								+ \sum_{i\in \sitesset, j\in \clientset}d_{ij}x_{ij}}\label{eqn:fac_primal}\hspace{-1.5in}&&
									\\ \notag
  \textrm{subject to}\quad y_i - x_{ij} &\geq 0 			&\quad 		&\forall i\in \sitesset, j\in \clientset 
									\\ \notag
     \textstyle{\sum_{i\in \sitesset} x_{ij}} &\geq r_j  &			&\forall j\in \clientset
 									\\ \notag
  	  x_{ij} \geq 0, y_i &\geq 0 						& 			&\forall i\in \sitesset, j\in \clientset 
  									\\ \notag
\end{alignat}

%%%%%%%%%%%%

\noindent
The dual program is:

\begin{alignat}{3}
  \textrm{maximize}\quad \textstyle{\sum_{j\in \clientset}} r_j\alpha_j&\label{eqn:fac_dual}  
     						\\ \notag
  \textrm{subject to} \quad \textstyle{
    \sum_{j\in \clientset}\beta_{ij}} &\leq f_i  &\quad\quad			&\forall i \in \sitesset  
							\\ \notag
  \alpha_{j} - \beta_{ij} 	&\leq  d_{ij}       &                 & \forall i\in \sitesset, j\in \clientset 
							\\ \notag
  \alpha_j \geq 0, \beta_{ij} &\geq 0           &            & \forall i\in \sitesset, j\in \clientset
  							\\ \notag
\end{alignat}

In each of our algorithms we will fix some optimal
solutions of the LPs (\ref{eqn:fac_primal}) and (\ref{eqn:fac_dual})
that we will denote by $(\bfx^\ast, \bfy^\ast)$ and
$(\bfalpha^\ast,\bfbeta^\ast)$, respectively.

With $(\bfx^\ast, \bfy^\ast)$ fixed, we can define the
optimal facility cost as $F^\ast=\sum_{i\in\sitesset} f_i
y_i^\ast$ and the optimal connection cost as $C^\ast =
\sum_{i\in\sitesset,j\in\clientset} d_{ij}x_{ij}^\ast$.
Then $\LP^\ast = \cost(\bfx^\ast,\bfy^\ast) = F^\ast+C^\ast$
is the joint optimal value of (\ref{eqn:fac_primal}) and
(\ref{eqn:fac_dual}).  We can also associate with each
client $j$ its fractional connection cost $C^\ast_j =
\sum_{i\in\sitesset} d_{ij}x_{ij}^\ast$.  Clearly, $C^\ast =
\sum_{j\in\clientset} C^\ast_j$.  Throughout the paper we
will use notation $\OPT$ for the optimal integral solution
of (\ref{eqn:fac_primal}).  $\OPT$ is the value we wish to
approximate, but, since $\OPT\ge\LP^\ast$, we can instead use
$\LP^\ast$ to estimate the approximation ratio of our
algorithms.

%%%%%%%%%

\paragraph{Completeness and facility splitting.}
Define $(\bfx^\ast, \bfy^\ast)$ to be \emph{complete} if
$x_{ij}^\ast>0$ implies that $x_{ij}^\ast=y_i^\ast$ for all $i,j$. In
other words, each connection either uses a site fully or not at all.
As shown by Chudak and Shmoys~\cite{ChudakS04}, we can modify the
given instance by adding at most $|\clientset|$ sites to obtain an
equivalent instance that has a complete optimal solution, where
``equivalent" means that the values of $F^\ast$, $C^\ast$ and
$\LP^\ast$, as well as $\OPT$, are not affected. Roughly, the argument
is this: We notice that, without loss of generality, for each client
$k$ there exists at most one site $i$ such that $0 < x_{ik}^\ast <
y_i^\ast$.  We can then perform the following \emph{facility
  splitting} operation on $i$: introduce a new site $i'$, let
$y^\ast_{i'} = y^\ast_i - x^\ast_{ik}$, redefine $y^\ast_i$ to be
$x^\ast_{ik}$, and then for each client $j$ redistribute $x^\ast_{ij}$
so that $i$ retains as much connection value as possible and $i'$
receives the rest. Specifically, we set
%
\begin{align*}
  &y^\ast_{i'} \;\assign\; y^\ast_i - x^\ast_{ik},\;   y^\ast_{i} \;\assign\; x^\ast_{ik}, \quad \text{ and }\\
  &x^\ast_{i'j} \;\assign\;\max( x^\ast_{ij} - x^\ast_{ik}, 0 ),\;	 x^\ast_{ij} \;\assign\; \min( x^\ast_{ij} , x^\ast_{ik}) 
			\quad	\textrm{for all}\ j \neq k.
\end{align*}
%
This operation eliminates the partial connection between $k$
and $i$ and does not create any new partial
connections. Each client can split at most one site and
hence we shall have at most $|\clientset|$ more sites.

By the above paragraph,  without loss of generality we can
assume that the optimal fractional solution $(\bfx^\ast, \bfy^\ast)$
is complete. This assumption will in fact greatly simplify some of
the arguments in the paper. Additionally, we will frequently use the facility
splitting operation described above in our algorithms to obtain fractional solutions with
desirable properties.
