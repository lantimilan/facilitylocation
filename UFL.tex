\documentclass{article}[11pt]

\usepackage{fullpage, amsmath}
\title{Notes on the Uncapacitated Facility Location Problem}
\author{lyan}
\begin{document}
\maketitle

\section{A Different Filtering for STA'97}
In STA'97 paper, the neighboring facilities of a client $j$ was cut at
a total fration of constant $\alpha$. Then the job is to estimate the
worst distance from any facility in this prefix. It was shown that
$d_j^{\max} * (1-\alpha) \leq C_j^\ast$, so we have a bound on
$d_j^{\max}$, in the STA'97 paper it is called $g_j$.

In the subsequent rounding part, each client $j$ is connected to some
facility no more than $3d_j^{\max}$ away and we also need to scale all
$y_i^\ast$ by $1/\alpha$ as we take only a prefix of $\alpha$. In the
end we have $F^A = 1/\alpha * F^\ast$ and $C^A = 3 C^\ast /
(1-\alpha)$, take $\alpha = 1/4$ gives a ratio of $4$.

\paragraph{A Different Filtering} It seems also natural to cut at a
distance $d$ such that $d \leq C_j^\ast / \alpha$ for some constant
$\alpha \leq 1$. Using argument like Markov's inequality, we can show
that the fraction in the suffix is no more than $\alpha$, it follows
that the fraction in the prefix is no less than $1-\alpha$. Using the
same rounding idea we get $F^A \leq F^\ast / (1-\alpha)$ and $C^A \leq
3C^\ast / \alpha$. Take $\alpha = 3/4$ we again have a ratio of $4$.

Now the second approach gives a deterministic bound on $d_j^{\max}$,
which might be advantageous as all the rounding algorithms are
spending major effort in bounding $d_j^{\max}$. However, does this
really imply a promising algorithm?

\end{document}
