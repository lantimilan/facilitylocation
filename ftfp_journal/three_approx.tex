\section{Algorithm~{\EGUP} with Ratio $3$}
\label{sec: 3-approximation}

With the partitioned FTFP instance and its associated fractional
solution in place, we now begin to introduce our rounding algorithms.
The algorithm we describe in this section achieves ratio $3$. Although
this is still quite far from our best ratio $1.575$ that we derive
later, we include this algorithm in the paper to illustrate, in a
relatively simple setting, how the properties of our partitioned
fractional solution are used in rounding it to an integral solution
with cost not too far away from an optimal solution.  The rounding
approach we use here is an extension of the corresponding method for
UFL described in~\cite{gupta08}.

\paragraph{Algorithm~{\EGUP.}}
At a high level, we would open exactly one facility for each
primary demand $\kappa$, and each non-primary demand is
connected to the facility opened for the primary demand it
was assigned to.

More precisely, we apply a rounding process, guided by the
fractional values $(\bary_{\mu})$ and $(\barx_{\mu\nu})$,
that produces an integral solution. This integral solution
is obtained by choosing a subset of facilities in
$\facilityset$ to open, and for each demand in $\demandset$,
specifying an open facility that this demand will be
connected to.  For each primary demand $\kappa\in P$, we
want to open one facility $\phi(\kappa) \in
\wbarN(\kappa)$. To this end, we use randomization: for each
$\mu\in\wbarN(\kappa)$, we choose $\phi(\kappa) = \mu$ with
probability $\barx_{\mu\kappa}$, ensuring that exactly one
$\mu \in \wbarN(\kappa)$ is chosen. Note that
$\sum_{\mu\in\wbarN(\kappa)}\barx_{\mu\kappa}=1$, so this
distribution is well-defined.  We open this facility
$\phi(\kappa)$ and connect to $\phi(\kappa)$ all demands
that are assigned to $\kappa$.

In our description above, the algorithm is presented as a
randomized algorithm. It can be de-randomized using the
method of conditional expectations, which is commonly used
in approximation algorithms for facility location problems
and standard enough that presenting it here would be
redundant. Readers less familiar with this field are
recommended to consult \cite{ChudakS04}, where the method of
conditional expectations is applied in a context very
similar to ours.

%%%%%%%%%

\paragraph{Analysis.}
We now bound the expected facility cost and connection cost
by establishing the two lemmas below.

%%%%%

\begin{lemma}\label{lemma:3fac}
The expectation of facility cost $F_{\smallEGUP}$ of our solution is
  at most $F^\ast$.
\end{lemma}

\begin{proof}
  By Property~(PD.\ref{PD:disjoint}), the neighborhoods of
  primary demands are disjoint. Also, for any primary demand
  $\kappa\in P$, the probability that a facility
  $\mu\in\wbarN(\kappa)$ is chosen as the open facility
  $\phi(\kappa)$ is $\barx_{\mu\kappa}$. Hence the expected
  total facility cost is
%
\begin{align*}
    \Exp[F_{\smallEGUP}]
	&= \textstyle{\sum_{\kappa\in P}\sum_{\mu\in\wbarN(\kappa)}} f_{\mu} \barx_{\mu\kappa}
	\\
	&= \textstyle{\sum_{\kappa\in P}\sum_{\mu\in\facilityset}} f_{\mu} \barx_{\mu\kappa} 
	\\
	&= \textstyle{\sum_{i\in\sitesset}} f_i \textstyle{\sum_{\mu\in i}\sum_{\kappa\in P}} \barx_{\mu\kappa} 
	\\
	&\leq \textstyle{\sum_{i\in\sitesset}} f_i y_i^\ast 
	= F^\ast,
\end{align*}
%
where the inequality follows from Property~(PD.\ref{PD:yi}).
\end{proof}

%%%%%%%

\begin{lemma}\label{lemma:3dist}
The expectation of connection cost $C_{\smallEGUP}$ of our solution
is at most  $C^\ast+2\cdot\LP^\ast$.
\end{lemma}

\begin{proof}
  For a primary demand $\kappa$, its expected connection cost is
  $C_{\kappa}^{\avg}$ because we choose facility $\mu$ with
  probability $\barx_{\mu\kappa}$.

  Consider a non-primary demand $\nu$ assigned to a primary demand
  $\kappa\in P$. Let $\mu$ be any facility in $\wbarN(\nu) \cap
  \wbarN(\kappa)$.  Since $\mu$ is in both $\wbarN(\nu)$ and
  $\wbarN(\kappa)$, we have $d_{\mu\nu} \leq \alpha_{\nu}^\ast$ and
  $d_{\mu\kappa} \leq \alpha_{\kappa}^\ast$ (This follows from the
  complementary slackness conditions since
  $\alpha_{\nu}^\ast=\beta_{\mu\nu}^\ast + d_{\mu\nu}$ for each
  $\mu\in \wbarN(\nu)$.). Thus, applying the triangle inequality, for
  any fixed choice of facility $\phi(\kappa)$ we have
%
\begin{equation*}
    d_{\phi(\kappa)\nu} \leq d_{\phi(\kappa)\kappa}+d_{\mu\kappa}+d_{\mu\nu}
    \leq d_{\phi(\kappa)\kappa} + \alpha_{\kappa}^\ast + \alpha_{\nu}^\ast.
\end{equation*}
%
Therefore the expected distance from $\nu$ to its facility $\phi(\kappa)$ is 
%
\begin{align*}
  \Exp[  d_{\phi(\kappa)\nu}   ] &\le \concost_{\kappa} + \alpha_{\kappa}^\ast + \alpha_{\nu}^\ast 
\\
  &\leq \concost_{\nu} + \alpha_{\nu}^\ast + \alpha_{\nu}^\ast
   = \concost_{\nu} + 2\alpha_{\nu}^\ast,
  \end{align*}
%
  where the second inequality follows from Property~(PD.\ref{PD:assign:cost}).  
From the definition of $\concost_{\nu}$ and Property~(PS.\ref{PS:xij}), for any $j\in \clientset$ 
we have
%
\begin{align*}
\sum_{\nu\in j} \concost_{\nu} &= \sum_{\nu\in j}\sum_{\mu\in\facilityset}d_{\mu\nu}\barx_{\mu\nu}
			\\
 			&= \sum_{i\in\sitesset} d_{ij}\sum_{\nu\in j}\sum_{\mu\in i}\barx_{\mu\nu}
			\\
			&= \sum_{i\in\sitesset} d_{ij}x^\ast_{ij} 
			= C^\ast_j.
\end{align*}
% 
Thus, summing over all demands, the expected total connection cost is
%
\begin{align*}
    \Exp[C_{\smallEGUP}] &\le 
			\textstyle{\sum_{j\in\clientset} \sum_{\nu\in j}} (\concost_{\nu} + 2\alpha_{\nu}^\ast) 
			\\
    	& = \textstyle{\sum_{j\in\clientset}} (C_j^\ast + 2r_j\alpha_j^\ast)
 		= C^\ast + 2\cdot\LP^\ast,
\end{align*}
%
completing the proof of the lemma.
\end{proof}

%%%%%%%%

\begin{theorem}
Algorithm~{\EGUP} is a $3$-approximation algorithm.
\end{theorem}

\begin{proof}
  By Property~(SI.\ref{SI:primary disjoint}), different
  demands from the same client are assigned to different
  primary demands, and by PD(\ref{PD:disjoint}) each primary
  demand opens a different facility. This ensures that our
  solution is feasible, namely each client $j$ is connected
  to $r_j$ different facilities (some possibly located on
  the same site).  As for the total cost,
  Lemma~\ref{lemma:3fac} and Lemma~\ref{lemma:3dist} imply
  that the total cost is at most
  $F^\ast+C^\ast+2\cdot\LP^\ast = 3\cdot\LP^\ast \leq
  3\cdot\OPT$.
\end{proof}

%%%%%%%%%
