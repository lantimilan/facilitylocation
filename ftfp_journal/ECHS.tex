\section{Algorithm~{\ECHS} with Ratio $1.736$}\label{sec: 1.736-approximation}

In this section we improve the approximation ratio to $1+2/e
\approx 1.736$. The improvement comes from a slightly modified
rounding process and refined analysis.  Note that the
facility opening cost of Algorithm~{\EGUP} does not exceed
that of the fractional optimum solution, while the
connection cost is quite far from the optimum, due to the
cost of indirect connections (that is, connections from
non-primary demands).  The basic idea behind the
improvement, following the approach of Chudak and
Shmoys~\cite{ChudakS04}, is to balance these bounds by
opening more facilities and using direct connections when available.

%%%%%%%%%%

\paragraph{Algorithm~{\ECHS}.}
As before,
the algorithm starts by solving the linear program and applying the
adaptive partitioning algorithm  described in 
Section~\ref{sec: adaptive partitioning} to obtain a partitioned
solution $(\barbfx, \barbfy)$. Then we apply the rounding
process to compute an integral solution (see Pseudocode~\ref{alg:lpr3}).  
For convenience, we will use
the term \emph{facility cluster} for the neighborhood of a
primary demand. Facilities that do not belong to these
clusters are said to be \emph{non-clustered}.  We start, as before, by
opening exactly one facility $\phi(\kappa)$ in the facility
cluster of each primary demand $\kappa$ (Line 2).  For any
non-primary demand $\nu$ assigned to $\kappa$, we refer to
$\phi(\kappa)$ as the \emph{target} facility of $\nu$.  In
Algorithm~{\EGUP}, $\nu$ was connected to $\phi(\kappa)$,
but in Algorithm~{\ECHS} we may be able to find an open
facility in $\nu$'s neighborhood and connect $\nu$ to this
facility.  Specifically, the two changes in the
algorithm are as follows:
%
\begin{description}
	\item{(1)} Each non-clustered facility $\mu$ is opened,
independently, with probability $\bary_{\mu}$ (Lines
4--5). Notice that if $\bary_\mu>0$ then,
due to completeness of the partitioned
fractional solution, we have $\bary_{\mu}= \barx_{\mu\nu}$
for some demand $\nu$. This implies that $\bary_{\mu}\leq 1$,
because $\barx_{\mu\nu}\le 1$, by (PS.\ref{PS:one}).
%
	\item{(2)} When connecting demands to facilities, a primary demand
$\kappa$ is connected to the only facility $\phi(\kappa)$
opened in its neighborhood, as before (Line 3).  For a
non-primary demand $\nu$, if its neighborhood $\wbarN(\nu)$ has an open
facility, we connect $\nu$ to the closest open facility in $\wbarN(\nu)$
(Line 8). Otherwise, we connect $\nu$ to
its target facility (Line 10).
%
\end{description}

%%%%%%%%%%%%%

\begin{algorithm}
  \caption{Algorithm~{\ECHS}:
    Constructing Integral Solution}
  \label{alg:lpr3}
  \begin{algorithmic}[1]
    \For{each $\kappa\in P$} 
    \State choose one $\phi(\kappa)\in \wbarN(\kappa)$,
    with each $\mu\in\wbarN(\kappa)$ chosen as $\phi(\kappa)$
    with probability $\bary_\mu$
    \State open $\phi(\kappa)$ and connect $\kappa$ to $\phi(\kappa)$
    \EndFor
    \For{each $\mu\in\facilityset - \bigcup_{\kappa\in P}\wbarN(\kappa)$} 
    \State open $\mu$ with probability $\bary_\mu$ (independently)
    \EndFor
    \For{each non-primary demand $\nu\in\demandset$}
    \If{any facility in $\wbarN(\nu)$ is open}
    \State{connect $\nu$ to the nearest open facility in $\wbarN(\nu)$}
    \Else
    \State connect $\nu$ to $\phi(\kappa)$ where $\kappa$ is $\nu$'s
     primary demand
    \EndIf
    \EndFor
  \end{algorithmic}
\end{algorithm}

%%%%%%%%%%%%%%%%

\paragraph{Analysis.}
We shall first argue that the integral solution thus
constructed is feasible, and then we bound the total cost of
the solution. Regarding feasibility, the only constraint
that is not explicitly enforced by the algorithm is the
fault-tolerance requirement; namely that each client $j$ is
connected to $r_j$ different facilities. Let $\nu$ and
$\nu'$ be two different sibling demands of client $j$ and let
their assigned primary demands be $\kappa$ and $\kappa'$
respectively. Due to (SI.\ref{SI:primary
  disjoint}) we know $\kappa \neq \kappa'$. From
(SI.\ref{SI:siblings disjoint}) we have $\wbarN(\nu) \cap
\wbarN(\nu') = \emptyset$. From (SI.\ref{SI:primary
  disjoint}), we have $\wbarN(\nu) \cap \wbarN(\kappa') =
\emptyset$ and $\wbarN(\nu') \cap \wbarN(\kappa) =
\emptyset$. From (PD.\ref{PD:disjoint}) we have
$\wbarN(\kappa)\cap \wbarN(\kappa') = \emptyset$. It follows
that $(\wbarN(\nu) \cup \wbarN(\kappa)) \cap (\wbarN(\nu')
\cup \wbarN(\kappa')) = \emptyset$. Since the algorithm
connects $\nu$ to some facility in $\wbarN(\nu) \cup
\wbarN(\kappa)$ and $\nu'$ to some facility in $\wbarN(\nu')
\cup \wbarN(\kappa')$, $\nu$ and $\nu'$ will be connected to
different facilities.


%%%%%%%%%

\smallskip
We now show that the expected cost of the computed solution is bounded by
$(1+2/e) \cdot \LP^\ast$. By
(PD.\ref{PD:disjoint}), every facility may appear in at
most one primary demand's neighborhood, and the facilities
open in Line~4--5 of Pseudocode~\ref{alg:lpr3} do not appear
in any primary demand's neighborhood. Therefore, by
linearity of expectation, the expected facility cost of
Algorithm~{\ECHS} is 
%
\begin{equation*}
\Exp[F_{\smallECHS}] 
	= \sum_{\mu\in\facilityset} f_\mu \bary_{\mu} 
	= \sum_{i\in\sitesset} f_i\sum_{\mu\in i} \bary_{\mu} 
	= \sum_{i\in\sitesset} f_i y_i^\ast = F^\ast,
\end{equation*}
%
where the third equality follows from (PS.\ref{PS:yi}).

\smallskip

To bound the connection cost, we adapt an argument of Chudak
and Shmoys~\cite{ChudakS04}. Consider a demand $\nu$. This
demand can either get connected directly to some facility in
$\wbarN(\nu)$ or indirectly to its target facility $\phi(\kappa)\in
\wbarN(\kappa)$, where $\kappa$ is the primary demand to
which $\nu$ is assigned.

We now estimate the expected cost $d_{\phi(\kappa)\nu}$ of the indirect
connection. Let $\Lambda_\nu$ denote the event that none of the
facilities in $\wbarN(\nu)$ is opened. Then
%
\begin{equation*}
	\Exp[ d_{\phi(\kappa)\nu} | \Lambda_\nu] 
			= D(\wbarN(\kappa) \setminus \wbarN(\nu), \nu),
\end{equation*}
%
where $D(A,\sigma) \stackrel{\mathrm{def}}{=}\sum_{\mu\in A}
d_{\mu\sigma}\bary_{\mu}/\sum_{\mu\in A} \bary_{\mu}$, for
any set $A$ of facilities and a demand $\sigma$.
Note that $\concost_\nu = D(\wbarN(\nu),\nu)$,
and that $\Lambda_\nu$ implies that $\wbarN(\kappa) \setminus \wbarN(\nu)\neq\emptyset$,
since $\wbarN(\kappa)$ contains at least one open facility, namely $\phi(\kappa)$.

%%%%%%%%%%

\begin{lemma}
  \label{lem:echu indirect}
  Let $\nu$ be a demand assigned to a primary demand $\kappa$, and
assume that $\wbarN(\kappa) \setminus \wbarN(\nu)\neq\emptyset$.
Then $\Exp[ d_{\phi(\kappa)\nu} | \Lambda_\nu]  \leq
  		\concost_\nu+2\alpha_{\nu}^\ast$.
\end{lemma}

\begin{proof}
  By the discussion above, we need to show that $D(\wbarN(\kappa)
  \setminus \wbarN(\nu), \nu) \leq \concost(\nu) +
  2\alpha_{\nu}^\ast$. There are two cases to consider.

\begin{description}
%	
\item{\mycase{1}}
	 There exists some $\mu'\in \wbarN(\kappa) \cap
  \wbarN(\nu)$ such that $d_{\mu' \kappa} \leq \concost_\kappa$.
In this case, for every $\mu\in \wbarN(\kappa)\setminus \wbarN(\nu)$, we have
%
\begin{equation*}
d_{\mu \nu} \leq d_{\mu \kappa} + d_{\mu' \kappa} + d_{\mu' \nu}  
 	\le  \alpha^\ast_\kappa + \concost_\kappa + \alpha^\ast_{\nu}
  \leq \concost_\nu + 2\alpha_{\nu}^\ast,
\end{equation*}
%
using the triangle inequality, complementary slackness, and (PD.\ref{PD:assign:cost}).
By summing over all $\mu\in \wbarN(\kappa) \setminus \wbarN(\nu)$, it
follows that $D(\wbarN(\kappa) \setminus \wbarN(\nu), \nu) \leq
\concost_\nu + 2\alpha_{\nu}^\ast$.

\item{\mycase{2}}
 Every $\mu'\in \wbarN(\kappa)\cap \wbarN(\nu)$
has $d_{\mu'\kappa} > \concost_\kappa$. Then we
have $D(\wbarN(\kappa) \setminus \wbarN(\nu),\kappa)\leq
D(\wbarN(\kappa),\kappa)=\concost_{\kappa}$. Therefore,
choosing an arbitrary $\mu'\in \wbarN(\kappa)\cap \wbarN(\nu)$,
we obtain
%
\begin{equation*}
  D(\wbarN(\kappa) \setminus \wbarN(\nu), \nu) 
	\leq  D(\wbarN(\kappa) \setminus \wbarN(\nu), \kappa) 
			+ d_{\mu' \kappa} + d_{\mu' \nu} 
	\leq  \concost_{\kappa} +
  \alpha_{\kappa}^\ast + \alpha_{\nu}^\ast
	\leq \concost_\nu + 2\alpha_{\nu}^\ast,
\end{equation*}
%
where we again use the triangle inequality,
complementary slackness, and  (PD.\ref{PD:assign:cost}).
%
\end{description}
%
Since the lemma holds in both cases, the proof is now complete.
\end{proof}

We now continue our estimation of the connection cost. We note first
that for a primary demand $\kappa$, its expected connection cost
is $C^\avg_{\kappa} = \sum_{\mu\in\wbarN(\kappa)}
d_{\mu\kappa}\barx_{\mu\kappa}$ as in the previous section.

Next, we consider a non-primary demand $\nu$.
Denote by $C_\nu$ the random variable representing the connection
cost for $\nu$. We first deal with the case when
$\wbarN(\kappa)\setminus \wbarN(\nu)\neq\emptyset$, which is the
same as $\wbarN(\kappa) \subseteq \wbarN(\nu)$. This
actually implies that $\wbarN(\kappa) = \wbarN(\nu)$ because
Property (CO) implies that $\barx_{\mu\nu} = \bary_{\mu} =
\barx_{\mu\kappa}$ for every $\mu \in \wbarN(\kappa)$ and
therefore $\sum_{\mu\in\wbarN(\kappa)} \barx_{\mu\nu} =
1$. Thus $\Exp[C_{\nu}] = \concost_{\nu}$. 

Now we bound the expected connection cost of $\nu$ when
$\wbarN(\kappa)\setminus \wbarN(\nu)$ is not empty.
Let $\wbarN(\nu) = \braced{\mu_1,\ldots,\mu_l}$ and let $d_s
= d_{\mu_s\nu}$ and $y_s = \bary_{\mu_s}$
for $s = 1,\ldots,l$. By reordering, we can
assume that $d_1 \le d_2 \le \ldots \le d_l$. 
By Pseudocode~\ref{alg:lpr3}, the connection cost is
no more than that obtained through the random process that
opens each $\mu_s\in \wbarN(\nu)$ independently with
probability $\barx_{\mu_s \nu} = y_{s}$ (because
$\barx_{\mu_s \nu} > 0$ and by (CO)), and connects $\nu$ to the
nearest such open facility, if any of them opens; otherwise
$\nu$ is connected indirectly to its target facility
$\phi(\kappa)$. The intuition is that we only use a facility
$\mu_s$ if none of $\mu_1,\ldots,\mu_{s-1}$ is
open. Unconditionally we know that $\mu_s$ opens with
probability $y_{s}$. The only way that some of
$\mu_1,\ldots,\mu_{s-1}$ can affect that probability is that
they belong to $\wbarN(\kappa)$ for some primary demand
$\kappa$ and it also happens to be that $\mu_s \in
\wbarN(\kappa)$ as well. However in this case, the condition
that they are closed actually implies that the conditional
probability of $\mu_s$ being open is larger than
$\bary_{\mu_s}$. (For a detailed proof,
see~\cite{ChudakS04}.)

Putting all together, we estimate the (unconditional) expected 
connection cost of $\nu$ as follows:
%
\begin{align*}
  \Exp[C_\nu] &\leq 
	\sum_{r=1}^l d_ry_r\prod_{s=1}^{r-1}(1-y_s)
		+  \Exp[C_\nu |\Lambda_s]\textstyle{\prod_{s=1}^{l}} (1-y_{s})
		\\
  &\leq \Big(1 - \prod_{s=1}^l (1-y_s)\Big) \cdot \sum_{r=1}^l d_r y_r
	+  \Exp[ C_\nu |\Lambda_s ]\textstyle{\prod_{s=1}^{l}} (1-y_{s})
	\\
  &\leq (1-\frac{1}{e}) {\textstyle\sum_{r=1}^l} d_ry_r 
	+ \frac{1}{e} \Exp[ C_\nu |\Lambda_s ] \leq (1-\frac{1}{e}) \concost_{\nu} 
	+	\frac{1}{e}	(\concost_{\nu} + 2\alpha_{\nu}^\ast) = \concost_{\nu} + \frac{2}{e}\alpha_{\nu}^\ast,
\end{align*}
%
where the second inequality is shown in the appendix
(see also \cite{ChudakS04}) and the last inequality follows from
Lemma~\ref{lem:echu indirect}. Notice that the
completeness property  allows us to write
$\bary_{\mu}$ instead of $\barx_{\mu\nu}$.

Summing over all demands of a client $j$, we bound the
expected connection cost of client $j$:
%
\begin{equation*}
  \Exp[C_j] \leq {\textstyle\sum_{\nu\in j} (\concost_{\nu} + \frac{2}{e}\alpha_{\nu}^\ast) }
  = \textstyle{ C_j^\ast + \frac{2}{e}r_j\alpha_j^\ast}.
\end{equation*}
%
Summing over all clients $j$, the expected connection
cost is
%
\begin{equation*}
	 \Exp[ C_{\smallECHS}] \le C^\ast +
\frac{2}{e}\LP^\ast.
\end{equation*}
% 
Therefore, we have established that
our algorithm constructs a feasible integral solution with
an overall expected cost 
%
\begin{equation*}
  \label{eq:chudakall}
	 \Exp[ F_{\smallECHS} + C_{\smallECHS}]
	\le
  	F^\ast + C^\ast + \frac{2}{e}\cdot \LP^\ast = (1+2/e)\cdot \LP^\ast
  \leq (1+2/e)\cdot \OPT.
\end{equation*}
%
Summarizing, we obtain the main result of this section.

\begin{theorem}\label{thm:1736}
  Algorithm~{\ECHS} is a $(1+2/e)$-approximation algorithm for \FTFP.
\end{theorem}