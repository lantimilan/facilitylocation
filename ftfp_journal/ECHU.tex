\section{Algorithm~{\ECHS} with Ratio $1.736$}\label{sec: 1.736-approximation}

In this section we improve the approximation ratio to $1+2/e
\approx 1.736$. The new algorithm, named Algorithm~{\ECHS},
starts with the same partitioned fractional solution
$(\barbfx, \barbfy)$ as Algorithm~{\EGUP}. The improvement
in the approximation ratio comes from a slightly modified
rounding process and much refined analysis.  Note that the
facility opening cost of Algorithm~{\EGUP} does not exceed
that of the fractional optimum solution, while the
connection cost is quite far from the optimum, due to the
cost of indirect connections (that is, connections from
non-primary demands).  The basic idea behind the
improvement, following the approach of Chudak and
Shmoys~\cite{ChudakS04}, is to balance these bounds by
opening more facilities
and using direct connections when available, so as to reduce
the number of indirect connections. 

We now describe the algorithm (see
Pseudocode~\ref{alg:lpr3}).  For convenience, we will use
the term \emph{facility cluster} for the neighborhood of a
primary demand. Facilities that do not belong to these
clusters are said to be \emph{non-clustered}.  As before, we
open exactly one facility $\phi(\kappa)$ in the facility
cluster of each primary demand $\kappa$ (Line 2).  For any
non-primary demand $\nu$ assigned to $\kappa$, we refer to
$\phi(\kappa)$ as the \emph{target} facility of $\nu$.  In
Algorithm~{\EGUP}, $\nu$ was connected to $\phi(\kappa)$,
but in Algorithm~{\ECHS} we may be able to find an open
facility in $\nu$'s neighborhood and connect $\nu$ to this
facility.  Specifically, the two changes in the
algorithm are as follows:
%
\begin{description}
	\item{(1)} Each non-clustered facility $\mu$ is opened,
independently, with probability $\bary_{\mu}$ (Lines
4--5). Notice that if $\bary_\mu>0$ then,
due to completeness of the partitioned
fractional solution, that is Property~(CO), we have $\bary_{\mu}=$
for some demand $\nu$. This implies that $\bary_{\mu}\leq 1$,
because $\bary_{\mu}\le 1$, by (PS.\ref{PS:one}).
%
	\item{(2)} When connecting demands to facilities, a primary demand
$\kappa$ is connected to the only facility $\phi(\kappa)$
opened in its neighborhood, as before (Line 3).  For a
non-primary demand $\nu$, if its neighborhood has an open
facility, we connect $\nu$ to the closest open facility in
its neighborhood (Line 8). Otherwise, we connect $\nu$ to
its target facility (Line 10).
%
\end{description}

%%%%%%%%%%%%%

\begin{algorithm}
  \caption{Algorithm~{\ECHS}:
    Constructing Integral Solution}
  \label{alg:lpr3}
  \begin{algorithmic}[1]
    \For{each $\kappa\in P$} 
    \State choose one $\phi(\kappa)\in \wbarN(\kappa)$,
    with each $\mu\in\wbarN(\kappa)$ chosen as $\phi(\kappa)$
    with probability $\bary_\mu$
    \State open $\phi(\kappa)$ and connect $\kappa$ to $\phi(\kappa)$
    \EndFor
    \For{each $\mu\in\facilityset - \bigcup_{\kappa\in P}\wbarN(\kappa)$} 
    \State open $\mu$ with probability $\bary_\mu$ (independently)
    \EndFor
    \For{each non-primary demand $\nu\in\demandset$}
    \If{any facility in $\wbarN(\nu)$ is open}
    \State{connect $\nu$ to the nearest open facility in $\wbarN(\nu)$}
    \Else
    \State connect $\nu$ to $\phi(\kappa)$ where $\kappa$ is $\nu$'s
     primary demand
    \EndIf
    \EndFor
  \end{algorithmic}
\end{algorithm}

%%%%%%%%%%%%%%%%

\paragraph{Analysis.}
We shall first argue that the integral solution thus
constructed is feasible, and then we bound the total cost of
the solution. Regarding feasibility, the only constraint
that is not explicitly enforced by the algorithm is the
fault-tolerance requirement; namely that each client $j$ is
connected to $r_j$ different facilities. Let $\nu$ and
$\nu'$ be two different sibling demands of client $j$ and
their primary demands be $\kappa$ and $\kappa'$
respectively. Due to the implication in (SI.\ref{SI:primary
  disjoint}) we know $\kappa \neq \kappa'$. From
(SI.\ref{SI:siblings disjoint}) we have $\wbarN(\nu) \cap
\wbarN(\nu') = \emptyset$. From (SI.\ref{SI:primary
  disjoint}), we have $\wbarN(\nu) \cap \wbarN(\kappa') =
\emptyset$ and $\wbarN(\nu') \cap \wbarN(\kappa) =
\emptyset$. From (PD.\ref{PD:disjoint}) we have
$\wbarN(\kappa)\cap \wbarN(\kappa') = \emptyset$. It follows
that $(\wbarN(\nu) \cup \wbarN(\kappa)) \cap (\wbarN(\nu')
\cup \wbarN(\kappa')) = \emptyset$. Since the algorithm
connects $\nu$ to some facility in $\wbarN(\nu) \cup
\wbarN(\kappa)$ and $\nu'$ to some facility in $\wbarN(\nu')
\cup \wbarN(\kappa')$, $\nu$ and $\nu'$ will be connected to
different facilities.


%%%%%%%%%

\medskip
We now show the expected cost of the computed solution is bounded by
$(1+2/e) \cdot \LP^\ast$. By
(PD.\ref{PD:disjoint}), every facility may appear in at
most one primary demand's neighborhood, and the facilities
open in Line~4--5 of Pseudocode~\ref{alg:lpr3} do not appear
in any primary demand's neighborhood. Therefore, by
linearity of expectation, the expected facility cost of
algorithm~{\ECHS} is $\sum_{\mu\in\facilityset}
f_\mu \bary_{\mu} = \sum_{i\in\sitesset} f_i\sum_{\mu\in i}
\bary_{\mu} = \sum_{i\in\sitesset} f_i y_i^\ast = F^\ast$,
where the second equality follows from (PS.\ref{PS:yi}).

To bound the connection cost, we adapt an argument of Chudak
and Shmoys~\cite{ChudakS04}. Consider a demand $\nu$. This
demand can either get connected directly to some facility in
$\wbarN(\nu)$ or indirectly to facility $\phi(\kappa)\in
\wbarN(\kappa)$, where $\kappa$ is the primary demand to
which $\nu$ is assigned.

We introduce one more notation. Given a facility set $A$,
let $d(A,\nu) \stackrel{\mathrm{def}}{=}\sum_{\mu\in A}
d_{\mu\nu}\bary_{\mu}/\sum_{\mu\in A} \bary_{\mu}$. By this
definition and Pseudocode~\ref{alg:lpr3}, we note that
$d(\wbarN(\kappa) \setminus \wbarN(\nu), \nu)$ is the
conditional expected distance from $\nu$ to its target
facility $\phi(\kappa)$ in the event that none in
$\wbarN(\nu)$ opens. We now estimate this conditional
expected connection cost by showing the following lemma.
\begin{lemma}
  \label{lem:echu indirect}
  Let $\nu$ be a non-primary demand of client $j$. Under the
  event that no facility in $\wbarN(\nu)$ opens, the
  expected connection cost $\nu$ to its target facility is
  bounded by $\concost(\nu)+2\alpha_{\nu}^\ast$.
\end{lemma}
\begin{proof}
  Let $\kappa$ be the primary demand that $\nu$ is assigned
  to and $\phi(\kappa)$ be the only facility opened by
  $\kappa$ since the algorithm in Pseudocode~\ref{alg:lpr3}
  opens exactly one facility in $\wbarN(\kappa)$. Then
  $\phi(\kappa)$ is the target facility of $\nu$. Because
  $\nu$ was assigned to $\kappa$, there exists a facility
  $\mu\in \wbarN(\nu)\cap \wbarN(\kappa)$.  For any fixed
  choice of $\phi(\kappa)$, the cost of connecting $\nu$ to
  $\phi(\kappa)$ is $\bard \stackrel{\mathrm{def}}{=}
  d_{\phi(\kappa)\nu} \le d_{\phi(\kappa)\kappa}+
  d_{\mu\kappa}+d_{\mu\nu} \le
  d_{\phi(\kappa)\kappa}+\alpha_{\kappa}^\ast +
  \alpha_{\nu}^\ast$, where the last inequality follows
  from the complementary slackness conditions. Suppose
  $\kappa$ is a demand of client $p$, we can bound the
  expectation of $\bard$ as follows:
%
\begin{equation}
	\label{eqn: 1.76 connection cost}
  \Exp[\bard] \;\leq\; \Exp[d_{\phi(\kappa) \kappa}]  + \alpha_{\kappa}^\ast + \alpha_{\nu}^\ast
  \;\leq\; \concost_{\kappa} + \alpha_{\kappa}^\ast + \alpha_{\nu}^\ast
  \;\leq\; \concost_{\nu} + \alpha_{\nu}^\ast + \alpha_{\nu}^\ast\;=\; \concost_{\nu} + 2\alpha_{\nu}^\ast.
\end{equation}
%
The third inequality follows from (PD.\ref{PD:assign:cost})
and the last inequality is from Lemma~\ref{lem: tcc
  optimal}. Now we argue the second inequality. Notice that
the expectation is computed for facility set
$X(\kappa,\nu)=\wbarN(\kappa) \setminus \wbarN(\nu)$ with
each facility $\mu$ chosen from $X(\kappa,\nu)$ with
probability proportional to $\bary_{\mu}$. Moreover, we have
that $\concost_{\kappa}=d(\wbarN(\kappa),\kappa)$ by
definition. We claim that $\Exp[d_{\phi(\kappa) \nu}] =
d(X(\kappa,\nu),\nu) \leq \concost_{\kappa} +
\alpha_{\kappa}^\ast + \alpha_{\nu}^\ast$. To prove this
claim, we consider two cases.

\mycase{1} The first case is when there exists some $\mu'\in
\wbarN(\kappa) \cap \wbarN(\nu)$ such that $d_{\mu' \kappa}
\leq d(\wbarN(\kappa), \kappa)$. In this case, denoting
$d(\kappa, \nu) \stackrel{\mathrm{def}}{=}\min_{\mu\in\facilityset} (d_{\mu\kappa} +
d_{\mu\nu})$, we have
\begin{equation*}
  d(\kappa,\nu)\leq
  d_{\mu'\kappa} + d_{\mu'\nu} \leq d(\wbarN(\kappa),\kappa) +
  \alpha_{\nu}^\ast=\concost_{\kappa} + \alpha_{\nu}^\ast. 
\end{equation*}
With the triangle inequality, it follows that for every
$\mu\in X(\kappa,\nu)$, we have
\begin{equation*}
  d_{\mu\nu} \leq d_{\mu\kappa} + d(\kappa,\nu) \leq
  \alpha_{\kappa}^\ast + \concost_{\kappa} + \alpha_{\nu}^\ast.
\end{equation*}

\mycase{2} In the second case, every $\mu'\in
\wbarN(\nu)\cap \wbarN(\kappa)$ has $d_{\mu'\kappa} >
d(\wbarN(\kappa),\kappa)$. Then we have
$d(X(\kappa,\nu),\kappa)\leq
d(\wbarN(\kappa),\kappa)=\concost_{\kappa}$, therefore
\begin{equation*}
  d(X(\kappa,\nu), \nu) \leq d(\wbarN(\kappa), \kappa) +
  d_{\mu\kappa} + d_{\mu\nu} \leq \concost_{\kappa} +
  \alpha_{\kappa}^\ast + \alpha_{\nu}^\ast,  
\end{equation*}
where $\mu$ is any facility in $\wbarN(\kappa)\cap \wbarN(\nu)$.
This completes the justification of (\ref{eqn: 1.76
  connection cost}).
\end{proof}

For a primary demand $\kappa$, its expected connection cost
is $C^\avg_{\kappa} = \sum_{\mu\in\wbarN(\kappa)}
d_{\mu\kappa}\barx_{\mu\kappa}$ as in the previous
section. We now estimate the expected connection cost of a
non-primary demand $\nu$. Let $\wbarN(\nu) =
\braced{\mu_1,\ldots,\mu_l}$ and let $d_s = d_{\mu_s\nu}$
for $s = 1,\ldots,l$. By reordering, we can assume that $d_1
\le d_2 \le \ldots \le d_l$. By the algorithm
Pseudocode~\ref{alg:lpr3}, the connection cost is no more
than that obtained through the random process that opens
each $\mu_s\in \wbarN(\nu)$ independently with probability
$\barx_{\mu_s \nu} =\bary_{\mu_s}$ (because $\barx_{\mu_s
  \nu} > 0$ and (CO)), and connects $\nu$ to the nearest
such open facility, if any of them opens; otherwise $\nu$ is
connected indirectly to $\phi(\kappa)$ at cost $\bard$. The
intuition is that we only use a facility $\mu_s$ if none of
$\mu_1,\ldots,\mu_{s-1}$ is open. Unconditionally we know
that $\mu_s$ opens with probability $\bary_{\mu_s}$. The
only way that some of $\mu_1,\ldots,\mu_{s-1}$ can affect
that probability is that they belong to $\wbarN(\kappa)$ for
some primary demand $\kappa$ and it also happens to be that
$\mu_s \in \wbarN(\kappa)$ as well. However in this case,
the condition that they are closed actually implies the
conditional probability of $\mu_s$ being open is larger than
$\bary_{\mu_s}$. For a detailed proof, see~\cite{ChudakS04}.

Denoting by $C_\nu$ the connection cost for $\nu$, we thus have
%
\begin{align*}
  \Exp[C_\nu] &\leq d_1 \bary_{\mu_1} + d_2 \bary_{\mu_2}(1-\bary_{\mu_1}) + \ldots
 		+  d_l \bary_{\mu_l}\textstyle{\prod_{s=1}^{l-1}}(1-\bary_{\mu_s}) 
		+  \Exp[\bard]\textstyle{\prod_{s=1}^{l}} (1-\bary_{\mu_s})
		\\
  &\leq (1-\textstyle{\prod_{i=1}^l} (1-\bary_{\mu_i}))
  	\sum_{i=1}^l d_i\bary_{\mu_i} + {\textstyle\prod_{i=1}^l} (1-\bar  y_{\mu_i})\Exp[\bard]
	\\
  &\leq (1-\frac{1}{e}) {\textstyle\sum_{i=1}^l} d_i\bary_i 
	+ \frac{1}{e} \Exp[\bard] \leq (1-\frac{1}{e}) \concost_{\nu} 
	+	\frac{1}{e}	(\concost_{\nu} + 2\alpha_{\nu}^\ast) = \concost_{\nu} + \frac{2}{e}\alpha_{\nu}^\ast,
\end{align*}
%
where the second inequality was shown in
\cite{ChudakS04}. (In the appendix we give a simpler
proof of this inequality using only elementary techniques.)  Notice that the
completeness of the fractional solution allows us to write
$\bary_{\mu}$ instead of $\barx_{\mu\nu}$.

Summing over all demands of a client $j$, we obtain the
expected connection cost of client $j$:
%
\begin{equation*}
  \Exp[C_j] \leq {\textstyle\sum_{\nu\in j} (\concost_{\nu} + \frac{2}{e}\alpha_{\nu}^\ast) }
  = \textstyle{ C_j^\ast + \frac{2}{e}r_j\alpha_j^\ast}.
\end{equation*}
%
Summing over all clients $j$ gives the expected connection
cost being bounded by $C^\ast +
\frac{2}{e}\LP^\ast$. Therefore, we have established that
our algorithm constructs a feasible integral solution with
an overall expected cost bounded by
%
\begin{equation*}
  \label{eq:chudakall}
  	F^\ast + C^\ast + \frac{2}{e}\cdot \LP^\ast = (1+2/e)\cdot \LP^\ast
  \leq (1+2/e)\cdot \OPT.
\end{equation*}

Summarizing, we obtain the result of this section.

\begin{theorem}\label{thm:1736}
  Algorithm~{\ECHS} is a $(1+2/e)$-approximation algorithm for \FTFP.
\end{theorem}