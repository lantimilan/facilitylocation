%% NEW VERSION

\section{Algorithm~{\EBGS} with Ratio $1.575$}\label{sec: 1.575-approximation}

In this section we give our main result, a $1.575$-approximation algorithm
for $\FTFP$, where $1.575$ is a solution of the equation
 $\min_{\gamma\geq 1}\max\{\gamma,
  1+2/e^\gamma, \frac{1/e+1/e^\gamma}{1-1/\gamma}\}$}, rounded to three
decimal digits. This matches the ratio of the best known LP-rounding
algorithm for UFL by Byrka~{\etal}~\cite{ByrkaGS10}. Recall that in
Section~\ref{sec: 1.736-approximation}, we showed how to compute an integral
solution with facility cost bounded by $F^\ast$ and
connection cost bounded by $C^\ast + 2/e\cdot\LP^\ast$. A
natural idea is to balance these two costs, by reducing the connection
cost, at the expense of slightly increasing the facility cost. If it works,
this should result in reducing the approximation ratio.

Our approach can be thought of as a combination of the ideas in~\cite{ByrkaGS10}
with the techniques of demand reduction and adaptive partitioning that we
introduced earlier. However, our adaptive partitioning technique needs to 
be carefully modified, because now we will be using a more refined neighborhood 
structure. The neighborhood of each demand will be divided into two parts, 
the close and far neighborhood, and our properties of the partitioned fractional
solution $(\barbfx,\barbfy)$ need to guarantee certain close
neighborhoods overlap while other neighborhoods are
disjoint. It turns out that the properties required are much
more delicate but nonetheless, we show a modified partition
algorithm actually delivers a partitioned fractional
solution that satisfies all the properties. The rounding
stage that construct an integral solution is a relatively
straightforward generalization of \cite{ByrkaGS10}, as is
the analysis of approximation ratio.

We begin with a list of properties that our partitioned fractional
solution $(\barbfx,\barbfy)$ needs to satisfy. The neighborhood
$\wbarN(\nu)$ of each demand $\nu$ is partitioned into two parts,
called the \emph{close neighborhood} $\wbarclsnb(\nu)$ and the
\emph{far neighborhood} $\wbarfarnb(\nu)$, whose formal definitions
are given below as Property~PS(\ref{PS1:gamma}). Their respective
average connection costs are defined by
$\clsdist(\nu)=\gamma\sum_{\mu\in\wbarclsnb(\nu)}
d_{\mu\nu}\barx_{\mu\nu}$ and
$\fardist(\nu)=\frac{\gamma}{\gamma-1}\sum_{\mu\in\wbarfarnb(\nu)}
d_{\mu\nu}\barx_{\mu\nu}$. Note that by PS~(\ref{PS1:gamma}) we have
$\sum_{\mu\in\wbarclsnb(\nu)} \barx_{\mu\nu} = 1/\gamma$ and
$\sum_{\mu\in\wbarfarnb(\nu)} \barx_{\mu\nu} = 1-1/\gamma$. We will
also use notation $\clsmax(\nu)=\max_{\mu\in\wbarclsnb(\nu)}
d_{\mu\nu}$ for the maximum distance from $\nu$ to its close
neighborhood.
\begin{description}
	
      \renewcommand{\theenumii}{(\alph{enumii})}
      \renewcommand{\labelenumii}{\theenumii}

\item{(PS)} \emph{Partitioned solution}.
Vector $(\barbfx,\barbfy)$ is a partition of $(\bfx^\ast,\bfy^\ast)$, with unit-value
  demands, that is:

	\begin{enumerate}
		%
	\item \label{PS1:one} 
          $\sum_{\mu\in\facilityset} \barx_{\mu\nu} = 1$ for each demand $\nu\in\demandset$. 
		%
	\item \label{PS1:xij} $\sum_{\mu\in i, \nu\in j} \barx_{\mu\nu}
          = x^\ast_{ij}$ for each site $i\in\sitesset$ and client $j\in\clientset$.
		%
	\item \label{PS1:yi}
          $\sum_{\mu\in i} \bary_{\mu} = y^\ast_i$ for each site $i\in\sitesset$.
		%
  \item \label{PS1:gamma}
	For each demand $\nu$, its neighborhood is divided into \emph{close} and
	\emph{far} neighborhood, that is $\wbarN(\nu) = \wbarclsnb(\nu) \cup \wbarfarnb(\nu)$, where
(n1) $\wbarclsnb(\nu) \cap \wbarfarnb(\nu) = \emptyset$,
(n2) $\sum_{\mu\in\wbarclsnb(\nu)} \barx_{\mu\nu} =1/\gamma$,
(n3) $\sum_{i\in\wbarfarnb(\nu)} \barx_{\mu\nu} =1-1/\gamma$,
and 
(n4) if $\mu\in \wbarclsnb(\nu)$ and $\mu'\in \wbarfarnb(\nu)$ then $d_{\mu\nu}\le d_{\mu'\nu}$.   

	\end{enumerate}
		
\item{(CO)} \emph{Completeness.}
	Solution   $(\barbfx,\barbfy)$ is complete, that is $\barx_{\mu\nu}\neq 0$ implies
				$\barx_{\mu\nu} = \bary_{\mu}$, for all $\mu\in\facilityset, \nu\in\demandset$.

\item{(PD)} \emph{Primary demands.}
	Primary demands satisfy the following conditions:

	\begin{enumerate}
		
	\item\label{PD1:disjoint}  For any two different primary demands $\kappa,\kappa'\in P$ we have
				$\wbarclsnb(\kappa)\cap \wbarclsnb(\kappa') = \emptyset$.

	\item \label{PD1:yi} For each site $i\in\sitesset$, 
		$ \sum_{\kappa\in P}\sum_{\mu\in i\cap\wbarclsnb(\mu)}\barx_{\mu\kappa} \leq y_i^\ast$.
		
	\item \label{PD1:assign} Each demand $\nu\in\demandset$ is assigned
        to one primary demand $\kappa\in P$ such that

  			\begin{enumerate}
	
				\item \label{PD1:assign:overlap} $\wbarclsnb(\nu) \cap \wbarclsnb(\kappa) \neq \emptyset$, and
				%
				\item \label{PD1:assign:cost}
          $\clsdist(\nu)+\clsmax(\nu) \geq
          \clsdist(\kappa)+\clsmax(\kappa)$.
          %
			\end{enumerate}

	\end{enumerate}
	
\item{(SI)} \emph{Siblings}. For any pair $\nu,\nu'$ of different siblings we have
  \begin{enumerate}

	\item \label{SI1:siblings disjoint}
		  $\wbarN(\nu)\cap \wbarN(\nu') = \emptyset$.
		
	\item \label{SI1:primary disjoint} If $\nu$ is assigned to a primary demand $\kappa$ then
 		$\wbarN(\nu')\cap \wbarclsnb(\kappa) = \emptyset$. In particular, by Property~PD(\ref{PD1:assign:overlap}),
		this implies that different sibling demands are assigned to different primary demands.

	\end{enumerate}
	
\end{description}

To obtain a fractional solution with the above properties,
we employ a modified adaptive partitioning algorithm with
two phases, a partitioning phase and an augmenting phase. As
usual we split clients into demands and create facilities on
sites in Phase 1, with each demand possessing some
connection values. In Phase 2 we augment each demand to
having a total connection value equal $1$.

Phase 1 runs in iterations. Consier any client $j$.  Let
$\wtildeN(j)$ be the set of facilities $\mu$ such that
$\tildex_{\mu j}>0$. Order facilities in $\wtildeN(j)$ such
that $d_{1 j} \leq d_{2 j} \leq \ldots \leq d_{|\wtildeN(j)|
  j}$ and $\sum_{s=1}^l \tildex_{s j} = 1/\gamma$ for some
integer $l$ (we split the $l^{th}$ facility if necessary to
make the sum equal to $1/\gamma$). Then the set of
facilities $1,2,\ldots,l$ is defined as
$\wtildeN_{\gamma}(j)$. $\tcc(j)$ now refers to
$d(\wtildeN(j), j)$ and $\dmax(j)$ refers to $\max_{\mu \in
  \wtildeN_{\gamma}(j)} d_{\mu j}$. In each iteration, we
find a client $p$ with minimum value of $\tcc(p) +
\dmax(p)$. Now we have two cases:

\smallskip
\noindent
\mycase{1} If $\wtildeN_{\gamma}(p)$ overlaps $\wbarclsnb(\kappa)$ for
any existing primary demand $\kappa$, we simply assign $\nu$ to
$\kappa$. As before, if there are multiple such $\kappa$, we pick any
of them. We also fix $\barx_{\mu\kappa} \assign \tildex_{\mu p},
\tildex_{\mu p}\assign 0$ for each $\mu \in \wtildeN(p)\cap
\wbarclsnb(\kappa)$. As before, although we check for overlap between
$\wtildeN_{\gamma}(p)$ and $\wbarclsnb(\kappa)$, the facilities we
actually move into $\wbarN(\nu)$ include all facilities in the
intersection of $\wtildeN(p)$, a bigger set, with
$\wbarclsnb(\kappa)$. At this point the total connection value in
$\wbarN(\nu)$ might be smaller than $1/\gamma$ (it cannot be bigger)
and we shall augment $\nu$ with additional facilities later to make a
neighborhood with total connection value of $1$ (We augment to make
sum of $\barx_{\mu\nu}$ for all $\mu\in\wbarN(\nu)$ equal to $1$.).

\smallskip
\noindent
\mycase{2} The best client $p$ has $\wtildeN_{\gamma}(p)$ disjoint
from $\wbarclsnb(\kappa)$, for all existing primary demands $\kappa$.
In this case we make $\nu$ a primary demand. We then fix
$\barx_{\mu\kappa}\assign \tildex_{\mu p}$ for $\mu \in
\wtildeN_{\gamma}(p)$ and set the corresponding $\tildex_{\mu p}$ to
$0$.  Note that the total connection value in $\wbarclsnb(\kappa)$ is
$1/\gamma$.  The set $\wtildeN_{\gamma}(p)$ turns out to coincide with
$\wbarclsnb(\kappa)$ as we only add farther away facilities when
augmenting a primary demand thereafter. Thus $\wbarclsnb(\kappa)$ is
defined when it is created. We also define $\tcc(\kappa) = \tcc(p)$
and
$\clsdist(\kappa)=\gamma\sum_{\mu\in\wbarclsnb(\kappa)}d_{\mu\kappa}\barx_{\mu\kappa},
\clsmax(\kappa)=\max_{\mu\in\wbarclsnb(\kappa)}d_{\mu\kappa}$.

Once all clients are exhausted, that is, each has $r_j$
demands created and assigned to some primary demand, Phase 1
concludes. We then do an augmenting phase 2. For each demand
with total connection value less than $1$, we use our
$\AugmentToUnit()$ procedure to add additional facilities to
its neighborhood to make its total connection value equal
$1$. We remark here that the augment step does not change
the close neighborhood of a primary demand, as it already
contains all the nearest facilities with total connection
value $1/\gamma$.  For non-primary demands, the issue is
more subtle, because $\wbarN(\nu)$ has taken all overlapping
facilities in $\wbarclsnb(\kappa)\cap \wtildeN(j)$, which
might be close to $\kappa$ but far from $j$. It seems that
facilities added in the augment step might actually be
closer to $\nu$ than some of the overlapping facilities
already in $\wbarN(\nu)$. As a result facilities added in
the augment step might appear in $\nu$'s close neighborhood
$\wbarclsnb(\nu)$, yet they are not in $\wbarclsnb(\kappa)$
of the primary demand $\kappa$ that $\nu$ is assigned
to. This could be detrimental to ensuring
Property~PD(\ref{PD:assign:overlap}), which requires the
close neighborhood of demand $\nu$ and that of its primary
demand $\kappa$ need overlap.

We now argue that the fractional solution $(\barbfx,\barbfy)$ does
satisfy all the stated properties. The (PS) and (CO) properties are
directly enforced by the adaptive partition algorithm, as well as
PD(\ref{PD1:disjoint}) and SI(\ref{SI1:siblings disjoint}). Proof of
other properties are similar to those in Section~\ref{sec: adaptive
  partitioning} except PD(\ref{PD:assign:overlap}), which we now prove
here. Consider an iteration when we create demand $\nu$ for client $p$
and assign it to $\kappa$. We have that $\wtildeN_{\gamma}(p)$ having
a nonempty intersection with $\wbarclsnb(\kappa)$. Let
$B(p)=\wtildeN_{\gamma}(p)\cap \wbarclsnb(\kappa)$, we claim that
$B(p)$ must be a subset of $\wbarclsnb(\nu)$, after $\wbarN(\nu)$ is
finalized with a total connection value of $1$. To see this, first
observe that $B(p)$ is a subset of $\wbarN(\nu)$, which in turn is a
subset of $\wtildeN(p)$, after taking into account of facility
split. Here $\wtildeN(p)$ refers to the neighborhood of client $p$
just before $\nu$ was created. Consider an arbitrary set of facilities
$A$, and define $\dmax(A, \nu)$ as the minimum distance $\tau$ such
that $\sum_{\mu\in A \suchthat d_{\mu\nu} \leq \tau}\;\bary_{\mu} \geq
1/\gamma$, then adding additional facilities into $A$ cannot make
$\dmax(A, \nu)$ larger. It follows that $\dmax(\wbarclsnb(\nu), \nu)
\geq \dmax(\wtildeN(p), \nu)$ because $\wbarclsnb(\nu)$ is a subset of
$\wtildeN(p)$. Since we have $d_{\mu \nu} = d_{\mu p}$ by definition,
it is easy to see that every $\mu \in B(p)$ satisfies $d_{\mu \nu}
\leq \dmax(\wtildeN(p), \nu) \leq \dmax(\wbarclsnb(\nu), \nu)$ and
hence they all belong to $\wbarclsnb(\nu)$. We need to be a bit more
careful here when we have a tie in $d_{\mu\nu}$ but we can assume ties
are always broken in favor of facilities in $B(p)$ when deciding
$\wbarclsnb(\nu)$. Finally, since $B(p)$ is nonempty, we have that the
close neighborhood of a demand $\nu$ and its primary demand $\kappa$
must overlap.

%%%%%%%%%%%%%%%%%
\paragraph{Algorithm EBGS}
Given the partitioned fractional solution $(\barbfx,
\barbfy)$ with the desired properties, we then start opening
facilities and making connections to obtain an integral
solution. As before we open exactly one facility in each
cluster (the close neighborhood of a primary demand), but
now each facility $\mu$ is chosen with probability
$\gamma\bary_{\mu}$. The non-clusterd facilities $\mu$,
those that do not belong to $\wbarN_{\cls}(\kappa)$ for any
primary demand $\kappa$, are opened independently with
probability $\gamma\bary_{\mu}$ each. This implies that the
expected facility cost of our algorithm is bounded by
$\gamma F^\ast$, using essentially the same argument as in
the previous section (with the the factor $\gamma$
accounting for using probabilities $\gamma \bary_{\mu}$
instead of $\bary_{\mu}$).

For connections, each primary demand $\kappa$ will connect
to the only facility $\phi(\kappa)$ open in its cluster
$\wbarclsnb(\kappa)$.  For each non-primary demand $\nu$, if
there is an open facility in $\wbarN(\nu)$ then we connect
$\nu$ to the nearest such facility. Otherwise, we connect
$\nu$ to $\phi(\kappa)$, where $\kappa$ is the primary
demand that $\nu$ is assigned to. This facility
$\phi(\kappa)$ will be called the \emph{target facility} of
$\nu$.

%%%%%%%%%%%

\paragraph{Analysis.}
The feasibility of our integral solution follows from
Property~SI(\ref{SI1:siblings disjoint}), SI(\ref{SI1:primary
  disjoint}), and PD(\ref{PD1:disjoint}), as these properties together
ensure that each facility is accessible to at most one demand among
sibling demands of the same client, regardless whether a demand
connects to its neighbor or its target facility.

We now bound the
cost. Properties~PD(\ref{PD1:assign:overlap}) and
PD(\ref{PD1:assign:cost}) allow us to bound the expected
distance from a demand $\nu$ to its target facility by
$\clsdist(\nu)+\clsmax(\nu)+\fardist(\nu)$, in the event
that none of $\nu$'s neighbors opens, using a similar
argument as Lemma 2.2 in~\cite{ByrkaGS10}~\footnote{The full
  proof of the lemma appears in~\cite{ByrkaA10} as
  Lemma~3.3.}. We are then able to show that the expected
connection cost for demand $\nu$ using an argument similar
to~\cite{ByrkaGS10}: For each demand $\nu$, with probability
no less than $1-1/e$, $\nu$ has some facility open in its
close neighborhood, and with probability no less than
$1-1/e^\gamma$, $\nu$ has some facility open in its overall
neighborhood, and with probability no more than
$1/e^\gamma$, $\nu$ will connect to its target facility and
we have bounded the distance for this case.
%
\begin{align*}
  \Exp[C_{\nu}] &\leq \clsdist(\nu)(1-1/e) +
  \fardist(\nu)(1/e-1/e^\gamma) + (\clsdist(\nu)+\clsmax(\nu)+\fardist(\nu))1/e^\gamma \\
  &\leq \clsdist(\nu)(1-1/e) +
  \fardist(\nu)(1/e-1/e^\gamma) + (\clsdist(\nu)+2\fardist(\nu))1/e^\gamma\\
  &\leq
  \concost(\nu)((1-\rho_{\nu})(\frac{1/e+1/e^\gamma}{1-1/\gamma})
  + \rho_{\nu}(1+2/e^\gamma)) \\
  &\leq \concost(\nu) \cdot
  \max\{\frac{1/e+1/e^\gamma}{1-1/\gamma},
  1+\frac{2}{e^\gamma}\},
\end{align*}
%
where $\rho_{\nu}=\clsdist(\nu)/\concost(\nu)$. It is easy
to see that $\rho_{\nu}$ is between 0 and 1 for every demand
$\nu$.  Since $\sum_{\nu\in j} C^{\avg}(\nu) = \sum_{\nu\in
  j}\sum_{\mu\in\facilityset} d_{\mu\nu}\barx_{\mu\nu} =
\sum_{i\in\sitesset} d_{ij}x_{ij}^\ast = C_j^\ast$, summing
over all clients $j$ we have total connection cost bounded
by $C^\ast \max\{\frac{1/e+1/e^\gamma}{1-1/\gamma},
1+\frac{2}{e^\gamma}\}$. The expected facility cost is
bounded by $\gamma F^\ast$, as argued earlier. Hence the
total cost is bounded by $\max\{\gamma,
\frac{1/e+1/e^\gamma}{1-1/\gamma},
1+\frac{2}{e^\gamma}\}\cdot \LP^\ast$. Picking
$\gamma=1.575$ we obtain the desired ratio.
