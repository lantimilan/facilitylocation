%% NEW VERSION

\section{Algorithm~{\EBGS} with Ratio $1.575$}\label{sec: 1.575-approximation}

In this section we give our main result, a $1.575$-approximation algorithm
for $\FTFP$, where $1.575$ is the value of
 $\min_{\gamma\geq 1}\max\{\gamma,
  1+2/e^\gamma, \frac{1/e+1/e^\gamma}{1-1/\gamma}\}$, rounded to three
decimal digits. This matches the ratio of the best known LP-rounding
algorithm for UFL by Byrka~{\etal}~\cite{ByrkaGS10}. Recall that in
Section~\ref{sec: 1.736-approximation} we showed how to compute an integral
solution with facility cost bounded by $F^\ast$ and
connection cost bounded by $C^\ast + 2/e\cdot\LP^\ast$. A
natural idea is to balance these two costs, by reducing the connection
cost, at the expense of slightly increasing the facility cost. If it works,
this should result in reducing the approximation ratio.

Our approach can be thought of as a combination of the ideas in~\cite{ByrkaGS10}
with the techniques of demand reduction and adaptive partitioning that we
introduced earlier. However, our adaptive partitioning technique needs to 
be carefully modified because now we will be using a more intricate neighborhood 
structure, with the neighborhood of each demand divided into two parts, 
the close and far neighborhood, and with some conditions on which pairs of neighborhoods
need to overlap and which need to be disjoint. The final rounding
stage that construct an integral solution is a relatively
straightforward generalization of the rounding method in \cite{ByrkaGS10}.

We begin by describing properties that our partitioned fractional
solution $(\barbfx,\barbfy)$ needs to satisfy. The neighborhood
$\wbarN(\nu)$ of each demand $\nu$ will be divided into two disjoint parts.
The first part, called the \emph{close neighborhood} $\wbarclsnb(\nu)$,
contains the facilities in $\wbarN(\nu)$ nearest to $\nu$ with the total
connection value equal $1/\gamma$. The second part,
called the \emph{far neighborhood} $\wbarfarnb(\nu)$, contains the 
remaining facilities in $\wbarN(\nu)$. The formal definitions of
these sets  are given below in Property~(NB). 
The respective average connection costs from $\nu$ for these sets are defined by
$\clsdist(\nu)=\gamma\sum_{\mu\in\wbarclsnb(\nu)}
d_{\mu\nu}\barx_{\mu\nu}$ and
$\fardist(\nu)=\frac{\gamma}{\gamma-1}\sum_{\mu\in\wbarfarnb(\nu)}
d_{\mu\nu}\barx_{\mu\nu}$. We will
also use notation $\clsmax(\nu)=\max_{\mu\in\wbarclsnb(\nu)}
d_{\mu\nu}$ for the maximum distance from $\nu$ to its close neighborhood.

Our partitioned solution $(\barbfx,\barbfy)$ must satisfy the
same partitioning and completeness properties as before, namely 
properties (PS) and (CO) in Section~\ref{sec: adaptive partitioning}.
In addition, it must satisfy new neighborhood property (NB) and
modified properties (PD) and (SI), listed below.

\begin{description}
	
      \renewcommand{\theenumii}{(\alph{enumii})}
      \renewcommand{\labelenumii}{\theenumii}

\item{(NB)} \label{NB}
	For each demand $\nu$, its neighborhood is divided into \emph{close} and
	\emph{far} neighborhood, that is $\wbarN(\nu) = \wbarclsnb(\nu) \cup \wbarfarnb(\nu)$, where
	%
	\begin{itemize}
	\item $\wbarclsnb(\nu) \cap \wbarfarnb(\nu) = \emptyset$,
	\item $\sum_{\mu\in\wbarclsnb(\nu)} \barx_{\mu\nu} =1/\gamma$, and 
	\item if $\mu\in \wbarclsnb(\nu)$ and $\mu'\in \wbarfarnb(\nu)$ 
				then $d_{\mu\nu}\le d_{\mu'\nu}$.   
	\end{itemize}
	%
	Note that the second condition, together with (PS.\ref{PS:one}), implies
	that $\sum_{\mu\in\wbarfarnb(\nu)} \barx_{\mu\nu} =1-1/\gamma$.

\item{(PD')} \emph{Primary demands.}
	Primary demands satisfy the following conditions:

	\begin{enumerate}
		
	\item\label{PD1:disjoint}  For any two different primary demands $\kappa,\kappa'\in P$ we have
				$\wbarclsnb(\kappa)\cap \wbarclsnb(\kappa') = \emptyset$.

	\item \label{PD1:yi} For each site $i\in\sitesset$, 
		$ \sum_{\kappa\in P}\sum_{\mu\in i\cap\wbarclsnb(\kappa)}\barx_{\mu\kappa} \leq y_i^\ast$.
		
	\item \label{PD1:assign} Each demand $\nu\in\demandset$ is assigned
        to one primary demand $\kappa\in P$ such that

  			\begin{enumerate}
	
				\item \label{PD1:assign:overlap} $\wbarclsnb(\nu) \cap \wbarclsnb(\kappa) \neq \emptyset$, and
				%
				\item \label{PD1:assign:cost}
          $\clsdist(\nu)+\clsmax(\nu) \geq
          \clsdist(\kappa)+\clsmax(\kappa)$.
          %
			\end{enumerate}

	\end{enumerate}
	
\item{(SI')} \emph{Siblings}. For any pair $\nu,\nu'$ of different siblings we have
  \begin{enumerate}

	\item \label{SI1:siblings disjoint}
		  $\wbarN(\nu)\cap \wbarN(\nu') = \emptyset$.
		
	\item \label{SI1:primary disjoint} If $\nu$ is assigned to a primary demand $\kappa$ then
 		$\wbarN(\nu')\cap \wbarclsnb(\kappa) = \emptyset$. In particular, by Property~(PD.\ref{PD1:assign:overlap}),
		this implies that different sibling demands are assigned to different primary demands.

	\end{enumerate}
	
\end{description}

%%%%%%%%%%%%%%%%%

\paragraph{Modified adaptive partitioning.}
To obtain a fractional solution with the above properties,
we employ a modified adaptive partitioning algorithm. As
in Section~\ref{sec: adaptive partitioning}, we have two phases.
In Phase~1 we split clients into demands and create facilities on
sites, while in Phase~2 we augment each demand's
connection values so that its total value is $1$.

Phase~1 runs in iterations. Consider any client $j$.  As before,
$\wtildeN(j)$ is the neighborhood of $j$ with respect to the yet
unpartitioned solution, namely the set of facilities $\mu$ such that
$\tildex_{\mu j}>0$. Order the facilities in this set as
$\wtildeN(j) = \braced{\mu_1,...,\mu_q}$ in order of non-decreasing
distance from $j$, that is
$d_{\mu_1 j} \leq d_{\mu_2 j} \leq \ldots \leq d_{\mu_q j}$, where
$q = |\wtildeN(j)|$. Without loss of generality, there is an index
$l$ for which $\sum_{s=1}^l \tildex_{s j} = 1/\gamma$, since we can
always split one facility to have this property. Then we define
$\wtildeN_{\gamma}(j) = \braced{\mu_1,...,\mu_l}$. We also use notation
%
\begin{equation*}
\tcc_\gamma(j) =  d(\wtildeN_\gamma(j), j) 
			\quad\textrm{ and }\quad
 \dmax_\gamma(j) = \max_{\mu \in \wtildeN_{\gamma}(j)} d_{\mu j}. 
\end{equation*}
%

{\Large\bf
LI, these were not correctly defined earlier, but I think this is what they
should be. You should use subscript $\gamma$ to distinguish these notations
from those used earlier.
}

In each iteration, we find a not yet exhausted client $p$ that minimizes the
value of $\tcc_\gamma(p) + \dmax_\gamma(p)$. Now we have two cases:

\begin{description}

\item{\mycase{1}:} $\wtildeN_{\gamma}(p) \cap \wbarclsnb(\kappa)\neq\emptyset$,
for some existing primary demand $\kappa$.
In this case we assign $\nu$ to
$\kappa$. As before, if there are multiple such $\kappa$, we pick any
of them. We also fix $\barx_{\mu\kappa} \assign \tildex_{\mu p},
\tildex_{\mu p}\assign 0$ for each $\mu \in \wtildeN(p)\cap
\wbarclsnb(\kappa)$. As before, although we check for overlap between
$\wtildeN_{\gamma}(p)$ and $\wbarclsnb(\kappa)$, the facilities we
actually move into $\wbarN(\nu)$ include all facilities in the
intersection of $\wtildeN(p)$, a bigger set, with
$\wbarclsnb(\kappa)$. At this point the total connection value in
$\wbarN(\nu)$ might be smaller than $1/\gamma$ (it cannot be bigger)
and we shall augment $\nu$ with additional facilities later to make a
neighborhood with total connection value of $1$ (We augment to make
sum of $\barx_{\mu\nu}$ for all $\mu\in\wbarN(\nu)$ equal to $1$.).

{\Large\bf
LI: The last two sentences make no sense. Who cares if
the neighborhood of nu is smaller than $1/\gamma$. Should it be the 
close neighborhood? If so, why are you saying you will make it 1?
}

\item{\mycase{2}:}
$\wtildeN_{\gamma}(p) \cap \wbarclsnb(\kappa) = \emptyset$,
for all existing primary demands $\kappa$.
In this case we make $\nu$ a primary demand. We then fix
$\barx_{\mu\kappa}\assign \tildex_{\mu p}$ for $\mu \in
\wtildeN_{\gamma}(p)$ and set the corresponding $\tildex_{\mu p}$ to
$0$.  Note that the total connection value in $\wbarclsnb(\kappa)$ is
now exactly
$1/\gamma$.  The set $\wtildeN_{\gamma}(p)$ turns out to coincide with
$\wbarclsnb(\kappa)$ as we only add farther facilities when
augmenting a primary demand thereafter. Thus $\wbarclsnb(\kappa)$ is
defined when it is created. We also define $\tcc(\kappa) = \tcc(p)$
and
$\clsdist(\kappa)=\gamma\sum_{\mu\in\wbarclsnb(\kappa)}d_{\mu\kappa}\barx_{\mu\kappa},
\clsmax(\kappa)=\max_{\mu\in\wbarclsnb(\kappa)}d_{\mu\kappa}$.

\end{description}

Once all clients are exhausted, that is, each client $j$ has $r_j$ demands created, 
Phase~1 concludes. We then do Phase~2, the augmentation phase.
For each demand with total connection value less than $1$, we use our
$\AugmentToUnit()$ procedure to add additional facilities to
its neighborhood to make its total connection value equal $1$. 
This completes the description of the partitioning algorithm.

\smallskip

We now argue that the fractional solution $(\barbfx,\barbfy)$ 
satisfies all the stated properties. Properties~(PS), (CO),
(PD.\ref{PD1:disjoint}) and (SI.\ref{SI1:siblings disjoint})  are
directly enforced by the adaptive partitioning algorithm. The
proofs for other properties are similar to those in 
Section~\ref{sec: adaptive partitioning}, with the
exception of (PD.\ref{PD:assign:overlap}), which we justify below.

The argument for (PD.\ref{PD:assign:overlap}) is a bit subtle, because
of possible complications arising in the augmentation phase.
This phase does not change close neighborhoods of primary demands, as each primary demand
already contains all the nearest facilities with total connection
value $1/\gamma$. 
For non-primary demands, however, $\wbarN(\nu)$, for $\nu\in j$, takes all 
facilities in $\wbarclsnb(\kappa)\cap \wtildeN(j)$, which
might be close to $\kappa$ but far from $j$. 
It seems that
facilities added in the augment step might actually be
closer to $\nu$ than some of the facilities
already in $\wbarN(\nu)$. As a result, facilities added in
the augment step might appear in 
$\wbarclsnb(\nu)$, yet they are not in $\wbarclsnb(\kappa)$,
the close neighborhood
of the primary demand $\kappa$ that $\nu$ is assigned to.  
Nevertheless, we show that Property~(PD.\ref{PD:assign:overlap}) holds.

Consider an iteration when we create a demand $\nu\in p$
and assign it to $\kappa$. Then the set
$B(p)=\wtildeN_{\gamma}(p)\cap \wbarclsnb(\kappa)$ is not empty.
We claim that
$B(p)$ must be a subset of $\wbarclsnb(\nu)$ after $\wbarN(\nu)$ is
finalized with a total connection value of $1$. To see this, first
observe that $B(p)$ is a subset of $\wbarN(\nu)$, which in turn is a
subset of $\wtildeN(p)$, after taking into account the facility
split. Here $\wtildeN(p)$ refers to the neighborhood of client $p$
just before $\nu$ was created. For an arbitrary set of facilities
$A$ define $\dmax(A, \nu)$ as the minimum distance $\tau$ such
that $\sum_{\mu\in A \suchthat d_{\mu\nu} \leq \tau}\;\bary_{\mu} \geq
1/\gamma$.
Adding additional facilities into $A$ cannot make
$\dmax(A, \nu)$ larger, so it follows that $\dmax(\wbarclsnb(\nu), \nu)
\geq \dmax(\wtildeN(p), \nu)$, because $\wbarclsnb(\nu)$ is a subset of
$\wtildeN(p)$. Since we have $d_{\mu \nu} = d_{\mu p}$ by definition,
it is easy to see that every $\mu \in B(p)$ satisfies $d_{\mu \nu}
\leq \dmax(\wtildeN(p), \nu) \leq \dmax(\wbarclsnb(\nu), \nu)$ and
hence they all belong to $\wbarclsnb(\nu)$. We need to be a bit more
careful here when we have a tie in $d_{\mu\nu}$ but we can assume ties
are always broken in favor of facilities in $B(p)$ when defining
$\wbarclsnb(\nu)$. Finally, since $B(p)\neq\emptyset$, we have that the
close neighborhood of a demand $\nu$ and its primary demand $\kappa$
must overlap.

%%%%%%%%

\paragraph{Algorithm~{\EBGS}.}
The complete algorithm starts with solving the linear program and
computing the partitioning described earlier in this section.
Given the partitioned fractional solution $(\barbfx,
\barbfy)$ with the desired properties, we then start opening
facilities and making connections to obtain an integral
solution. As before, we open exactly one facility in each
cluster (the close neighborhood of a primary demand), but
now each facility $\mu$ is chosen with probability
$\gamma\bary_{\mu}$. The non-clusterd facilities $\mu$,
those that do not belong to $\wbarN_{\cls}(\kappa)$ for any
primary demand $\kappa$, are opened independently with
probability $\gamma\bary_{\mu}$ each. 

Next, we connect demands to facilities.
Each primary demand $\kappa$ will connect
to the only facility $\phi(\kappa)$ open in its cluster
$\wbarclsnb(\kappa)$.  For each non-primary demand $\nu$, if
there is an open facility in $\wbarN(\nu)$ then we connect
$\nu$ to the nearest such facility. Otherwise, we connect
$\nu$ to its \emph{target facility} $\phi(\kappa)$, where $\kappa$ is the primary
demand that $\nu$ is assigned to. 

%%%%%%%%%%%

\paragraph{Analysis.}
The feasibility of our integral solution follows from
Properties~(SI.\ref{SI1:siblings disjoint}), (SI.\ref{SI1:primary
  disjoint}), and (PD.\ref{PD1:disjoint}), as these properties together
ensure that each facility is accessible to at most one demand among
sibling demands of the same client, regardless whether a demand
connects to its neighbor or its target facility.

The expected facility cost of our algorithm is bounded by
$\gamma F^\ast$, using essentially the same argument as in
the previous section (with the the factor $\gamma$
accounting for using probabilities $\gamma \bary_{\mu}$
instead of $\bary_{\mu}$).

We now bound the connection
cost. Properties~(PD.\ref{PD1:assign:overlap}) and
(PD.\ref{PD1:assign:cost}) allow us to bound the expected
distance from a demand $\nu$ to its target facility by
$\clsdist(\nu)+\clsmax(\nu)+\fardist(\nu)$, in the event
that none of $\nu$'s neighbors opens, using a similar
argument as Lemma 2.2 in~\cite{ByrkaGS10}~\footnote{The full
  proof of the lemma appears in~\cite{ByrkaA10} as
  Lemma~3.3.}. We are then able to estimate the expected
connection cost for demand $\nu$ using an argument similar
to~\cite{ByrkaGS10}: with probability
no less than $1-1/e$, $\nu$ has some facility open in its
close neighborhood, with probability no less than
$1-1/e^\gamma$, $\nu$ has some facility open in its overall
neighborhood, and with probability no more than
$1/e^\gamma$, $\nu$ will connect to its target facility.
This gives us the bound
%
\begin{align*}
  \Exp[C_{\nu}] &\leq \clsdist(\nu)(1-1/e) +
  \fardist(\nu)(1/e-1/e^\gamma) + (\clsdist(\nu)+\clsmax(\nu)+\fardist(\nu))1/e^\gamma \\
  &\leq \clsdist(\nu)(1-1/e) +
  \fardist(\nu)(1/e-1/e^\gamma) + (\clsdist(\nu)+2\fardist(\nu))1/e^\gamma\\
  &\leq
  \concost(\nu)((1-\rho_{\nu})(\frac{1/e+1/e^\gamma}{1-1/\gamma})
  + \rho_{\nu}(1+2/e^\gamma)) \\
  &\leq \concost(\nu) \cdot
  \max\Big\{\frac{1/e+1/e^\gamma}{1-1/\gamma},
  1+\frac{2}{e^\gamma}\Big\},
\end{align*}
%
where $\rho_{\nu}=\clsdist(\nu)/\concost(\nu)$. It is easy
to see that $\rho_{\nu}$ is between 0 and 1.
Since $\sum_{\nu\in j} C^{\avg}(\nu) = \sum_{\nu\in
  j}\sum_{\mu\in\facilityset} d_{\mu\nu}\barx_{\mu\nu} =
\sum_{i\in\sitesset} d_{ij}x_{ij}^\ast = C_j^\ast$, summing
over all clients $j$ we have total connection cost bounded
by $C^\ast \max\{\frac{1/e+1/e^\gamma}{1-1/\gamma},
1+\frac{2}{e^\gamma}\}$. The expected facility cost is
bounded by $\gamma F^\ast$, as argued earlier. Hence the
total cost is bounded by $\max\{\gamma,
\frac{1/e+1/e^\gamma}{1-1/\gamma},
1+\frac{2}{e^\gamma}\}\cdot \LP^\ast$. Picking
$\gamma=1.575$ we obtain the desired ratio.


\begin{theorem}\label{thm:ebgs}
  Algorithm~{\EBGS} is a $1.575$-approximation algorithm for \FTFP.
\end{theorem}



