%% NEW VERSION
% marek Thu Nov 29 12:16:01 PST 2012
%          still a lot of work needed. 
% marek Tue Dec  4 12:54:24 PST 2012

\section{Algorithm~{\EBGS} with Ratio $1.575$}\label{sec: 1.575-approximation}

In this section we give our main result, a $1.575$-approximation
algorithm for $\FTFP$, where $1.575$ is the value of $\min_{\gamma\geq
  1}\max\{\gamma, 1+2/e^\gamma, \frac{1/e+1/e^\gamma}{1-1/\gamma}\}$,
rounded to three decimal digits. This matches the ratio of the best
known LP-rounding algorithm for UFL by
Byrka~{\etal}~\cite{ByrkaGS10}. 

Recall that in Section~\ref{sec: 1.736-approximation} we showed how to compute an integral solution
with facility cost bounded by $F^\ast$ and connection cost bounded by
$C^\ast + 2/e\cdot\LP^\ast$. A natural idea is to balance these two
costs, by reducing the connection cost at the expense of a slight
increase in the facility cost. One way to do this would be to increase the
probability of opening facilities, from $\bary_{\mu}$ (used in Algorithm~{\ECHS})
to, say, $\gamma\bary_{\mu}$, for some $\gamma > 1$. This would increase the
expected facility cost by a factor of $\gamma$ but, as it turns out,
it reduces the probability of an indirect connection of a non-primary demand
to $1/e^\gamma$ (from the previous value $1/e$). For any primary demand $\kappa$, as we
must open exactly one facility $\phi(\kappa)$ in its neighborhood, we choose this
facility from the subset of $\wbarN(\kappa)$ consisting of the facilities that
are nearest to $\kappa$ and whose total connection value is $1/\gamma$.
This is the essence of the approach in~\cite{ByrkaGS10}, expressed in our terminology.

Our approach can be thought of as a combination of the above ideas
with the techniques of demand reduction and
adaptive partitioning that we introduced earlier. However, our
adaptive partitioning technique needs to be carefully modified,
because now we will be using a more intricate neighborhood structure,
with the neighborhood of each demand divided into two disjoint parts,
and with some restrictions on how parts from different demands can overlap.

We begin by describing properties that our partitioned fractional
solution $(\barbfx,\barbfy)$ needs to satisfy. As mentioned earlier, the neighborhood
$\wbarN(\nu)$ of each demand $\nu$ will be divided into two disjoint
parts.  The first part, called the \emph{close neighborhood} and denoted
$\wbarclsnb(\nu)$, contains the facilities in $\wbarN(\nu)$ nearest to
$\nu$ with the total connection value equal $1/\gamma$, that is
$\sum_{\mu\in\wbarclsnb(\nu)} \barx_{\mu\nu} = 1/\gamma$.
The second
part, called the \emph{far neighborhood} and denoted $\wbarfarnb(\nu)$, contains
the remaining facilities in $\wbarN(\nu)$ (so $\sum_{\mu\in\wbarfarnb(\nu)} = 1-1/\gamma$).
We restate these definitions formally
below in Property~(NB).  The respective average
connection costs from $\nu$ for these sets are defined by
%
\begin{equation*}
\clsdist(\nu)=\gamma\sum_{\mu\in\wbarclsnb(\nu)}
			d_{\mu\nu}\barx_{\mu\nu} \quad\text{and}\quad
\fardist(\nu)=\frac{\gamma}{\gamma-1}\sum_{\mu\in\wbarfarnb(\nu)}
d_{\mu\nu}\barx_{\mu\nu}. 
\end{equation*}
%
We will also use
notation $\clsmax(\nu)=\max_{\mu\in\wbarclsnb(\nu)} d_{\mu\nu}$ for
the maximum distance from $\nu$ to its close neighborhood.

Our partitioned solution $(\barbfx,\barbfy)$ must satisfy the same
partitioning and completeness properties as before, namely properties
(PS) and (CO) in Section~\ref{sec: adaptive partitioning}.  In
addition, it must satisfy a new neighborhood property (NB) and modified
properties (PD') and (SI'), listed below.

\begin{description}
	
      \renewcommand{\theenumii}{(\alph{enumii})}
      \renewcommand{\labelenumii}{\theenumii}

\item{(NB)} \label{NB}
	\emph{Neighborhoods.}
	For each demand $\nu$, its neighborhood is divided into \emph{close} and
	\emph{far} neighborhood, that is $\wbarN(\nu) = \wbarclsnb(\nu) \cup \wbarfarnb(\nu)$, where
	%
	\begin{itemize}
	\item $\wbarclsnb(\nu) \cap \wbarfarnb(\nu) = \emptyset$,
	\item $\sum_{\mu\in\wbarclsnb(\nu)} \barx_{\mu\nu} =1/\gamma$, and 
	\item if $\mu\in \wbarclsnb(\nu)$ and $\mu'\in \wbarfarnb(\nu)$ 
				then $d_{\mu\nu}\le d_{\mu'\nu}$.   
	\end{itemize}
	%
	Note that the second condition, together with (PS.\ref{PS:one}), implies
	that $\sum_{\mu\in\wbarfarnb(\nu)} \barx_{\mu\nu} =1-1/\gamma$.

\item{(PD')} \emph{Primary demands.}
	Primary demands satisfy the following conditions:

	\begin{enumerate}
		
	\item\label{PD1:disjoint}  For any two different primary demands $\kappa,\kappa'\in P$ we have
				$\wbarclsnb(\kappa)\cap \wbarclsnb(\kappa') = \emptyset$.

	\item \label{PD1:yi} For each site $i\in\sitesset$, 
		$ \sum_{\kappa\in P}\sum_{\mu\in
                  i\cap\wbarclsnb(\kappa)}\barx_{\mu\kappa} \leq
                y_i^\ast$. In the summation, as before, we overload notation $i$ to stand for the set of
						facilities opened on site $i$.
		
	\item \label{PD1:assign} Each demand $\nu\in\demandset$ is assigned
        to one primary demand $\kappa\in P$ such that

  			\begin{enumerate}
	
				\item \label{PD1:assign:overlap} $\wbarclsnb(\nu) \cap \wbarclsnb(\kappa) \neq \emptyset$, and
				%
				\item \label{PD1:assign:cost}
          $\clsdist(\nu)+\clsmax(\nu) \geq
          \clsdist(\kappa)+\clsmax(\kappa)$.
          %
			\end{enumerate}

	\end{enumerate}
	
\item{(SI')} \emph{Siblings}. For any pair $\nu,\nu'$ of different siblings we have
  \begin{enumerate}

	\item \label{SI1:siblings disjoint}
		  $\wbarN(\nu)\cap \wbarN(\nu') = \emptyset$.
		
	\item \label{SI1:primary disjoint} If $\nu$ is assigned to a primary demand $\kappa$ then
 		$\wbarN(\nu')\cap \wbarclsnb(\kappa) = \emptyset$. In particular, by Property~(PD'.\ref{PD1:assign:overlap}),
		this implies that different sibling demands are assigned to different primary demands.

	\end{enumerate}
	
\end{description}

%%%%%%%%%%%%%%%%%

\paragraph{Modified adaptive partitioning.}
To obtain a fractional solution with the above properties,
we employ a modified adaptive partitioning algorithm. As
in Section~\ref{sec: adaptive partitioning}, we have two phases.
In Phase~1 we split clients into demands and create facilities on
sites, while in Phase~2 we augment each demand's
connection values $\barx_{\mu\nu}$ so that its total value is $1$.

Similar to the algorithm in Section~\ref{sec: adaptive partitioning},
Phase~1 runs in iterations. Fix some iteration and consider any client $j$.  As before,
$\wtildeN(j)$ is the neighborhood of $j$ with respect to the yet
unpartitioned solution, namely the set of facilities $\mu$ such that
$\tildex_{\mu j}>0$. Order the facilities in this set as
$\wtildeN(j) = \braced{\mu_1,...,\mu_q}$ with non-decreasing
distance from $j$, that is
$d_{\mu_1 j} \leq d_{\mu_2 j} \leq \ldots \leq d_{\mu_q j}$, where
$q = |\wtildeN(j)|$. Without loss of generality, there is an index
$l$ for which $\sum_{s=1}^l \tildex_{\mu_s j} = 1/\gamma$, since we can
always split one facility to achieve this property. Then we define
$\wtildeN_{\gamma}(j) = \braced{\mu_1,...,\mu_l}$. We also use notation
%
\begin{equation*}
\tcc_\gamma(j) =  D(\wtildeN_\gamma(j), j) = \gamma\sum_{\mu\in\wtildeN_{\gamma}(j)} d_{\mu j} \tildex_{\mu j}
			\quad\textrm{ and }\quad
 \dmax_\gamma(j) = \max_{\mu \in \wtildeN_{\gamma}(j)} d_{\mu j}. 
\end{equation*}
%

When the iteration starts, we first find a not-yet-exhausted client
$p$ that minimizes the value of $\tcc_\gamma(p) + \dmax_\gamma(p)$ and
we create a new demand $\nu$ for $p$.  Now we have two cases:
%
\begin{description}
%
\item{\mycase{1}} $\wtildeN_{\gamma}(p) \cap
  \wbarclsnb(\kappa)\neq\emptyset$, for some existing primary demand
  $\kappa\in P$.  In this case we assign $\nu$ to $\kappa$. As before, if
  there are multiple such $\kappa$, we pick any of them. We also fix
  $\barx_{\mu \nu} \assign \tildex_{\mu p}$ and $\tildex_{\mu p}\assign 0$
  for each $\mu \in \wtildeN(p)\cap \wbarclsnb(\kappa)$. Note that, as before,
  although we check for overlap between $\wtildeN_{\gamma}(p)$ and
  $\wbarclsnb(\kappa)$, the facilities we actually move into
  $\wbarN(\nu)$ include all facilities in the intersection of
  $\wtildeN(p)$, a bigger set, with $\wbarclsnb(\kappa)$.

  At this time, the total connection value between $\nu$ and $\wbarN(\nu)$ is
  at most $1/\gamma$ since the total connection value of $\kappa$ to
  $\wbarclsnb(\kappa)$ is $1/\gamma$. Later in Phase 2 we will add
  additional facilities from $\wtildeN(p)$ to $\wbarN(\nu)$ to make $\nu$'s total connection value to this set
  equal to $1$. 

%
\item{\mycase{2}} $\wtildeN_{\gamma}(p) \cap \wbarclsnb(\kappa) =
  \emptyset$, for all existing primary demands $\kappa\in P$.  In this case
  we make $\nu$ a primary demand (that is, add it to $P$). 
  We then move the facilities from $\wtildeN_{\gamma}(p)$ to $\wbarN(\nu)$, that is
  for $\mu \in \wtildeN_{\gamma}(p)$
	we set $\barx_{\mu \nu}\assign \tildex_{\mu p}$  and   $\tildex_{\mu p}\set 0$.  

  It is easy to see that the total connection value of $\nu$ to $\wbarN(\nu)$
  is exactly $1/\gamma$, so we now have $\wbarclsnb(\nu) = \wbarN(\nu)$.
Moreover, facilities remaining in
  $\wtildeN(p)$ are all farther away from $\nu$ than those in
  $\wbarN(\nu)$. As we add only facilities from $\wtildeN(p)$ to $\wbarN(\nu)$
  in Phase~2, $\wbarclsnb(\nu)$ will remain unchanged from now on.
%
\end{description}
%
Once all clients are exhausted, that is, each client $j$ has $r_j$
demands created, Phase~1 concludes. We then run Phase~2, the
augmenting phase, following the same steps as in Section~\ref{sec: adaptive partitioning}.
For each client $j$ and each demand $\nu\in j$ with total connection
value less than $1$, we use our $\AugmentToUnit()$ procedure to add
additional facilities (possibly split, if necessary) from
$\wtildeN(j)$ to $\wbarN(\nu)$ to make the total connection value of $\wbarN(\nu)$
equal $1$. As before, a facility is removed from
$\wtildeN(j)$ once it is added to $\wbarN(\nu)$.  

This completes the description of the partitioning algorithm. Summarizing, for 
each client $j$ we have created $r_j$ demands on the same point as $j$ and
we created a number of facilities at each site $i$. 
For each facility $\mu\in i$ we defined its fractional opening value $\bary_\mu$, $0\le \bary_\mu\le 1$,
and for each demand $\nu\in j$ we defined its fractional connection value $\barx_{\mu\nu}\in \braced{0,\bary_\mu}$.
The connections with $\barx_{\mu\nu} > 0$ define the neighborhood $\wbarN(\nu)$, that we divided into two parts,
$\wbarclsnb(\nu)$ and $\wbarfarnb(\nu)$. It remains to show that this partitioning
satisfies all the desired properties.

%%%%%%%

\medskip
\paragraph{Correctness of partitioning.}
We now argue that our partitioned fractional solution $(\barbfx,\barbfy)$
satisfies all the stated properties. Properties~(PS), (CO) and (NB) are
directly enforced by the algorithm.

(PD'.\ref{PD1:disjoint}) holds because for each primary demand $\kappa\in p$, 
$\wbarclsnb(\kappa)$ is the same set as $\wtildeN_{\gamma}(p)$ at the time when
$\kappa$ was created, and $\wtildeN_{\gamma}(p)$ is removed from $\wtildeN(p)$
right after this step. Further, the partitioning algorithm makes $\kappa$ a primary demand
only if $\wtildeN_{\gamma}(p)$ is disjoint from close neighborhoods
$\wbarclsnb(\kappa')$ of all existing primary demands $\kappa'$.

The justification of (PD'.\ref{PD1:yi}) is similar to (PD.\ref{PD:yi}) from
Section~\ref{sec: adaptive partitioning}. All close neighborhoods of
primary demands are disjoint, due to (PD'.\ref{PD1:disjoint}), so
each facility $\mu \in i$ can appear in at most one
$\wbarclsnb(\kappa)$, for some $\kappa\in P$. This and condition (CO) imply together
that $\bary_{\mu} = \barx_{\mu\kappa}$. As a result, the
summation on the left-hand side is not larger than 
$\sum_{\mu\in i}\bary_{\mu} = y_i^\ast$. 

Regarding (PD'.\ref{PD1:assign:overlap}), at first glance this
property seems to follow from the algorithm, as we only assign a
demand $\nu$ to a primary demand $\kappa$ when $\wbarN(\nu)$ overlaps
with $\wbarclsnb(\kappa)$.  However, it is more subtle, as we need
$\wbarclsnb(\nu)$ to overlap with $\wbarclsnb(\kappa)$ and at the end
$\wbarclsnb(\kappa)$ may contain facilities added to $\wbarN(\nu)$ in
the augmentation phase (maybe even only such facilities). We postpone
the proof of this property to Lemma~\ref{lem: PD1: primary overlap}.
The proof of (PD'.\ref{PD1:assign:cost}) is similar to that of
Lemma~\ref{lem: PD:assign:cost holds}, and we defer it to
Lemma~\ref{lem: PD1: primary optimal}.

(SI'.\ref{SI1:siblings disjoint}) follows directly from the
algorithm. The proof of (SI'.\ref{SI1:primary disjoint}) is similar to
that of Lemma~\ref{lem: property SI:primary disjoint holds}, by noting
that for a primary demand $\kappa$, $\wbarclsnb(\kappa)$ contains the
same set of facilities as the $\wbarN(\kappa)$ at the end of Phase
1. Moreover, if demand $\nu\in j$ is assigned to primary demand
$\kappa$, then all facilities in $\wtildeN(j)$ that are also in
$\wbarclsnb(\kappa)$ are added to $\wbarN(\kappa)$ in Phase 1. As a
result, if $\nu \in j$ is a non-primary demand and assigned to primary
demand $\kappa$, then facilities added to $\wbarN(\nu')$ for any
$\nu'\in j$ in Phase 2 do not appear in $\wbarclsnb(\kappa)$.

%%%%%%%%%%%%%%%

\begin{lemma} \label{lem: PD1: primary overlap}
  Property (PD'.\ref{PD1:assign:overlap}) holds.
\end{lemma}
\begin{proof}
Consider an iteration when we create a demand $\nu\in p$ and assign it
to $\kappa$. Then the set $B=\wtildeN_{\gamma}(p)\cap
\wbarclsnb(\kappa)$ is not empty.  We claim that $B$ must be a subset
of $\wbarclsnb(\nu)$ after $\wbarN(\nu)$ is finalized with a total
connection value of $1$. To see this, first observe that $B$ is a
subset of $\wbarN(\nu)$, which in turn is a subset of $\wtildeN(p)$,
after taking into account the facility split. (Here $\wtildeN(p)$
refers to the neighborhood of client $p$ just before $\nu$ was created
and $\wbarN(\nu)$ is the final value of this set.)  For an arbitrary
set of facilities $A$ define $\dmax_\gamma(A, \nu)$ as the minimum
$\tau$ such that $\sum_{\mu\in A \suchthat d_{\mu\nu} \leq
  \tau}\;\bary_{\mu} \geq 1/\gamma$.  Adding additional facilities
into $A$ cannot make $\dmax_\gamma(A, \nu)$ larger, so it follows that
$\dmax_\gamma (\wbarclsnb(\nu), \nu) \geq \dmax_\gamma(\wtildeN(p),
\nu)$, because $\wbarclsnb(\nu)$ is a subset of $\wtildeN(p)$. Since,
by definition, for all $\mu$ we have $d_{\mu \nu} = d_{\mu p}$, it is
easy to see that every $\mu \in B$ satisfies $d_{\mu \nu} \leq
\dmax_\gamma(\wtildeN(p), \nu) \leq \dmax_\gamma(\wbarclsnb(\nu),
\nu)$ and hence $B\subseteq \wbarclsnb(\nu)$. We need to be a bit more
careful here when we have a tie in $d_{\mu\nu}$ but we can assume ties
are always broken in favor of facilities in $B$ when defining
$\wbarclsnb(\nu)$. Finally, since $B\neq\emptyset$, we have that
$\wbarclsnb(\nu)\cap \wbarclsnb(\kappa)\neq\emptyset$, which is
Property~(PD'.\ref{PD1:assign:overlap}).
\end{proof}

\begin{lemma}\label{lem: PD1: primary optimal}
  Property (PD'.\ref{PD:assign:cost}) holds.
\end{lemma}
\begin{proof}
  Suppose $\nu \in j$ is assigned to primary demand $\kappa \in p$, it
  is then true that $\clsdist(\kappa) + \clsmax(\kappa)$ is no more
  than $\tcc_\gamma^\kappa(j) + \dmax_{\gamma}^\kappa(j)$ when demand
  $\kappa$ was created, because for primary demands we have
  $\clsdist(\kappa) = \tcc_\gamma^\kappa (p)$ and $\clsmax(\kappa) =
  \tcc_{\gamma}^\kappa(p)$ and the algorithm chooses client $p$ with
  minimum $\tcc_{\gamma}(p) + \tcc_{\gamma}(p)$ in the modified
  partitioning algorithm.

  Using an argument similar to that in the proof of Lemma~\ref{lem:
    tcc optimal}, our modified partitioning algorithm guarantees
  $\tcc_{\gamma}^{\kappa}(j) \leq \clsdist(\nu)$ and
  $\dmax_{\gamma}^{\kappa}(j) \leq \dmax(\nu)$.

  Therefore, we have
  \begin{equation*}
    \clsdist(\kappa) + \clsmax(\kappa) = \tcc_{\gamma}^{\kappa}(p) +
    \dmax_{\gamma}^{\kappa}(p) \leq \tcc_{\gamma}^{\kappa}(j) +
    \dmax_{\gamma}^{\kappa}(j) \leq \clsdist(\nu) + \dmax(\nu)
  \end{equation*}
\end{proof}
%%%%%%%%
Now we have completed the proof of all properties.

\paragraph{Algorithm~{\EBGS}.}
The complete algorithm starts with solving the linear program and
computing the partitioning described earlier in this section.
Given the partitioned fractional solution $(\barbfx,
\barbfy)$ with the desired properties, we then start opening
facilities and making connections to obtain an integral
solution. To this end, for each primary demand $\kappa\in P$,
we open exactly one facility $\phi(\kappa)$ in $\wbarclsnb(\kappa)$,
where each $\mu\in\wbarclsnb(\kappa)$ is chosen as $\phi(\kappa)$ with probability
$\gamma\bary_{\mu}$. For all facilities
 $\mu\in\facilityset - \bigcup_{\kappa\in P}\wbarclsnb(\kappa)$,
we open them independently, each with
probability $\gamma\bary_{\mu}$. 

Next, we connect demands to facilities.
Each primary demand $\kappa\in P$ will connect
to the only open facility $\phi(\kappa)$ in $\wbarclsnb(\kappa)$.  
For each non-primary demand $\nu\notin P$, if
there is an open facility in $\wbarN(\nu)$ then we connect
$\nu$ to the nearest such facility. Otherwise, we connect
$\nu$ to its \emph{target facility} $\phi(\kappa)$, where $\kappa$ is the primary
demand that $\nu$ is assigned to. 

%%%%%%%%%%%

\paragraph{Analysis.}
The feasibility of our integral solution follows from
Properties~(SI'.\ref{SI1:siblings disjoint}), (SI'.\ref{SI1:primary
  disjoint}), and (PD'.\ref{PD1:disjoint}), as these properties together
ensure that each facility is accessible to at most one demand among
all sibling demands of the same client, regardless whether a demand
connects to its neighbor or its target facility.

We now estimate the expected facility cost and connection cost of
Algorithm {\EBGS}.

The expected facility cost of our algorithm is bounded by the
following lemma.

\begin{lemma} \label{lem: EBGS facility cost}
  The expected facility cost of the algorithm $F_{\smallEBGS}$ is no more
  than $\gamma F^\ast$.
\end{lemma}
\begin{proof}
  Consider facilities in a primary demand $\kappa$'s close
  neighborhood $\wbarclsnb(\kappa)$. One of them will be opened, and
  each facility $\mu$ is chosen with probability $\gamma
  \bary_{\mu}$. For facilities outside any $\wbarclsnb(\kappa)$, each
  is opened independently with probability $\gamma \bary_{\mu}$. By
  linearity of expectation, we have the expected facility cost is
  bouned by $\sum_{\mu \in \facilityset} f_\mu \gamma \bary_{\mu} =
  \gamma \sum_{i\in \sitesset} f_i \sum_{\mu\in i} \bary_{\mu} =
  \gamma \sum_{i \in \sitesset} f_i y_i^\ast = \gamma F^\ast$ where we
  used (PS.\ref{PS:yi}).

\end{proof}

%%%%%%%%%%%
\margincomment{this needs to be expanded, as in the previous
  section. add an explicit lemma, etc.}  
%%%%%%%%%%%
We now bound the connection cost by the following lemma.
\begin{lemma}\label{lem: EBGS connection cost}
  For a fixed constant $1 < \gamma < 2$, the expected connection cost
  of the algorithm $C_{\smallEBGS}$ is no more than
  $C^\ast\max\{\frac{1/e+1/e^\gamma}{1-1/\gamma},
  1+\frac{2}{e^\gamma}\}$.
\end{lemma}

\begin{proof}
  Properties~(PD'.\ref{PD1:assign:overlap}) and
  (PD'.\ref{PD1:assign:cost}) allow us to bound the expected distance
  from a demand $\nu$ to its target facility by
  $\clsdist(\nu)+\clsmax(\nu)+\fardist(\nu)$, in the event that none
  of $\nu$'s neighbors opens, using a similar argument as Lemma 2.2
  in~\cite{ByrkaGS10}~\footnote{The full proof of the lemma appears
    in~\cite{ByrkaA10} as Lemma~3.3.}. A proof is available in
  Appendix~\ref{app:EBGS connection cost}.

  We are then able to estimate the expected connection cost for demand
  $\nu$ using an argument similar to~\cite{ByrkaGS10}: with
  probability no less than $1-1/e$, $\nu$ has some facility open in
  its close neighborhood, with probability no less than
  $1-1/e^\gamma$, $\nu$ has some facility open in its overall
  neighborhood, and with probability no more than $1/e^\gamma$, $\nu$
  will connect to its target facility.  This gives us the bound
%
\begin{align*}
  \Exp[C_{\nu}] &\leq 
	\textstyle 
	\clsdist(\nu)(1-\frac{1}{e}) + \fardist(\nu)(\frac{1}{e}-\frac{1}{e^\gamma}) 
					+ (\clsdist(\nu)+\clsmax(\nu)+\fardist(\nu))\frac{1}{e^\gamma}
\\
  &\leq 
	\textstyle
	\clsdist(\nu)(1-\frac{1}{e}) + \fardist(\nu)(\frac{1}{e}-\frac{1}{e^\gamma})
	 			+ (\clsdist(\nu)+2\fardist(\nu))\frac{1}{e^\gamma}
\\
  &\leq
	\textstyle
  \concost(\nu)((1-\rho_{\nu})(\frac{1/e+1/e^\gamma}{1-1/\gamma})
  + \rho_{\nu}(1+\frac{2}{e^\gamma}) 
\\
  &\leq 
	\textstyle
	\concost(\nu) \cdot \max\Big\{\frac{1/e+1/e^\gamma}{1-1/\gamma},
  								1+\frac{2}{e^\gamma}\Big\},
\end{align*}
%
where $\rho_{\nu}=\clsdist(\nu)/\concost(\nu)$. It is easy
to see that $\rho_{\nu}$ is between 0 and 1.
Since $\sum_{\nu\in j} C^{\avg}(\nu) = \sum_{\nu\in
  j}\sum_{\mu\in\facilityset} d_{\mu\nu}\barx_{\mu\nu} =
\sum_{i\in\sitesset} d_{ij}x_{ij}^\ast = C_j^\ast$, summing
over all clients $j$ we have total connection cost bounded
by $C^\ast \max\{\frac{1/e+1/e^\gamma}{1-1/\gamma},
1+\frac{2}{e^\gamma}\}$. 
\end{proof}

Recall that the expected facility cost is bounded by $\gamma F^\ast$,
as argued earlier. Hence the total cost is bounded by $\max\{\gamma,
\frac{1/e+1/e^\gamma}{1-1/\gamma}, 1+\frac{2}{e^\gamma}\}\cdot
\LP^\ast$. Picking $\gamma=1.575$ we obtain the desired ratio.


\begin{theorem}\label{thm:ebgs}
  Algorithm~{\EBGS} is a $1.575$-approximation algorithm for \FTFP.
\end{theorem}



