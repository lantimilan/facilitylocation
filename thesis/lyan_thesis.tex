\documentclass{article}

\title{Approximation Algorithms for Facility Location Problems\\A
  Dissertation for PhD program of Computer Science\\U of California
  Riverside} \author{Li Yan}

\usepackage{fullpage, amsmath}

\begin{document}
\maketitle

\section{Introduction}
We start with the classic uncapacitated facility location problem
(UFL) and review the basics of the approximation algorithm developed
in the past two decades. Then proceed to our generalized problem, the
fault-tolerant facility placement problem (FTFP) and the related
results.

\section{The Problem}
An instance is a set of facilities $\mathcal F$ and clients $\mathcal
C$, each $i\in \mathcal F$ has cost $f_i$ and each $i\in\mathcal F,
j\in\mathcal C$ has $d_{ij}$. Each $j\in\mathcal C$ has demand
$r_j$. The problem is to find, for each $i\in\mathcal F$, the number
of facilities to open, and to serve each client $j$ with $r_j$
different facilities. The total cost is $\sum_{i\in\mathcal F} f_i y_i
+ \sum_{i,j} d_{ij} x_{ij}$.

\section{FTFP with LP-rounding}
We have shown that, starting with a complete solution $(x^\ast,
y^\ast)$, we can split into $(x_{\mu\nu}, y_\mu)$ between each
facility $\mu$ and each demand $\nu$. We were then able to apply all
known LP-rounding algorithms to obtain the same approximation ratio,
while satisfying all fault-tolerance requirement, that is, sibling
demands from same client are connected to different facilities.

\section{FTFP with Primal-dual}
The first result is to show an upper and lower bound of $H_n$ if we
use same greedy algorithm and analyze using $\alpha_j^l$ as the
average cost of the star and take dual solution as $\bar \alpha_j =
\sum_{l=1}^{r_j} \alpha_j^l$, so that the shrinking factor $\gamma$ is
$H_n$ to make $\bar \alpha_j$ dual feasible, $\sum_{j\in\mathcal C}
(\bar \alpha_j - d_{ij})^+ \leq f_i$ for all $i \in \mathcal F$.

The second result is trying to mitigate the dual-fitting analysis and
trying to find an $O(1)$ ratio. The proposed charing scheme is to
assign share of $f_i$ proportional to $r_j$ among all participating
clients of the star (could use $\bar r_j$, the residual demand,
although the analysis would be more complicated).

One key technical result is the following:
\begin{align*}
  &1\cdot \left( \frac{1}{n} \right)\\
  &\frac{1}{2} \cdot \left( \frac{1}{n} + \frac{1}{n-1} \right)\\
  &\frac{1}{3} \cdot \left( \frac{1}{n} + \frac{1}{n-1} + \frac{1}{n-2} \right)\\
  &\vdots\\
  &\frac{1}{n-1} \cdot \left( \frac{1}{n} + \frac{1}{n-1} + \ldots + \frac{1}{2} \right)\\
  &\frac{1}{n} \cdot \left( \frac{1}{n} + \frac{1}{n-1} + \frac{1}{n-2} + \ldots + \frac{1}{2} + 1 \right)\\
  &= \sum_{i=1}^n 1/i^2 = \pi^2/6\\
\end{align*}

The sum might be daunting to evaluate, the true story is that I
numerically computed up to $n=10^7$ and found it to be $1.664934$ and
it magically coincides with $pi^2/6$. Later Neal helped show that it
is indeed $\sum_{i=1}^n 1/i^2$.

\end{document}
