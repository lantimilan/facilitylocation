\documentclass{article}

\title{On primal-dual and dual-fitting of the FTFP problem}
\author{UCR theory lab}
\usepackage{fullpage, amsmath}
\begin{document}
\maketitle

\section{$H_n$-approximation of Star-greedy on FTFP}
In this section we show that the star-greedy algorithm which
repeatedly picking the best star (the one with minimum
average cost) gives an approximation ratio of $H_n = \ln(n)$
where $n=|\mathcal{C}|$ is the number of clients.

When we run the star-greedy algorithm, for every client $j$,
we associate each demand of $j$ with a number $\alpha_j^l$,
which is the average cost of the star when $l^{th}$ demand
of $j$ is connected. Now we let $\alpha_j = \alpha_j^{r_j}$,
that is, take $\alpha_j$ to be the finishing $\alpha_j^l$,
and order clients by increasing $\alpha_j$. That is,
\begin{equation*}
  \alpha_1 \leq \alpha_2 \leq \ldots \leq \alpha_n
\end{equation*}

Due to the algorithm, for every $j=1,\ldots,n$, we have
\begin{equation*}
  \sum_{l=j}^n (\alpha_j - d_{il})_+ \leq f_i
\end{equation*}
for every facility $i$.  The reason is that, when the last
demand of $j$ is connected, all clients $j+1,\ldots,n$ are
still active so their total contribution cannot exceed
$f_i$.

Now we take a closer look at the numbers $\{\alpha_j\}$. We
know that the algorithm's total cost is exactly
$\sum_{j=1}^n \sum_{l=1}^{r_j} \alpha_j^l$, which is no more
than $\sum_{j=1}^n r_j \alpha_j$ since we take $\alpha_j$ to
be $\alpha_j^{r_j}$. Now if we can show that $\sum_{j=1}^n
r_j \alpha_j$ is no more than $\gamma \cdot \textrm{OPT}$,
where $\textrm{OPT}$ is the cost of an integral optimal
solution to the given FTFP instance, then we claim our
algorithm returns an integral solution within a factor of
$\gamma$.

We show that $\sum_{j=1}^n r_j \alpha_j$ is within a factor
of $\gamma$ from $\textrm{OPT}$ by showing that
$\{\alpha_j/\gamma\}$ is a feasible dual solution to the
following program, which is the dual program of the primal
LP for FTFP.
\begin{align*}
  \max &\sum_j r_j\alpha_j\\
  \textrm{subject to: }& \sum_{j=1}^n (\alpha_j - d_{ij})_+
  \leq f_i \textrm{ for every facility i}\\
\end{align*}

To find the minimum $\gamma$ that would shrink
$\{\alpha_j\}$ to a feasible dual solution, we need to find
a worst case instance to maximize $\gamma$, also it is clear
that the worst case instance must contain a star whose
feasibility requirement would achieves the value of
$\gamma$, and this star would be the worst star in that
instance.

As a first step we can drop the $\max\{0, \cdot\}$, because
we can always find a new star by dropping those $j$ with
$\alpha_j - d_{ij}$ term negative, and that new star would
still be a worst case star. Suppose a worst case star has
$k$ clients, and is with facility $i$, then we have
\begin{equation*}
  \sum_{j=1}^k \alpha_j - d_{ij} \leq f_i
\end{equation*}
Here we rename clients in the new star to be $1,\ldots,k$,
although among them, they are still ordered by their
$\alpha_j$.

Now our goal is to find a supremum of the following program:
\begin{align*}
  \max & \frac{\sum_{j=1}^k \alpha_j}{f_i + \sum_{j=1}^k d_{ij}}\\
  \textrm{subject to: } & \sum_{l=j}^k (\alpha_l -
  d_{il})_+\leq f_i \textrm{ for } j=1,\ldots,n\\
\end{align*}

Since we are dealing with a particular star, we can abstract
away $i$, to obtain the following program:
\begin{align}
  \label{eq:star}
  \max & \frac{\sum_{j=1}^k \alpha_j}{f + \sum_{j=1}^k
    d_j}\\ \notag
  \textrm{subject to: } & \sum_{l=j}^k (\alpha_j - d_{l})_+
  \leq f \textrm{ for } j=1,\ldots,n\\ \notag
\end{align}

Now we claim we can drop the $\max\{0, \cdot\}$ operator
because this would relax the constraint in (\ref{eq:star})
and can only make objective value larger (since we are
maximizing), so the real optimal is upper bounded by the
relaxed optimal. This allows us to work on
\begin{align}
  \label{eq:frlp}
  \max & \frac{\sum_{j=1}^k \alpha_j}{f + \sum_{j=1}^k
    d_j}\\ \notag
  \textrm{subject to: } & \sum_{l=j}^k (\alpha_j - d_{l})
  \leq f \textrm{ for } j=1,\ldots,n\\ \notag
\end{align}

For each $j=1,\ldots,n$, the constraint above simply can be
rewritten as
\begin{equation}
  (k-j+1) \alpha_j \leq f + \sum_{l=j}^k d_l \leq f +
  \sum_{l=1}^k d_l.
\end{equation}
The first inequality is a rewrite of the constraint in
(\ref{eq:frlp}) and the second is straightforward.

Therefore we have $\alpha_j \leq (1/(k-j+1)) (f +
\sum_{j=1}^k d_j)$, and it easily follows that
\begin{equation}
  \sum_{j=1}^n \alpha_j \leq (1/k + 1/(k-1) + \ldots + 1) =
  H_k \leq H_n = \ln(n)
\end{equation}

\section{Dual-fitting Analysis on FTFP}
This section gives a simple example that the dual-fitting analysis of
a greedy algorithm which repeatedly picking the most cost-effective
star (the star with minimum average cost) is unlikely to give the same
ratio as that for the UFL problem.

The example consists of $1$ facility with cost $f_1 = n$, and $n$
clients with demands $r_1=r_2=\ldots=r_{n-1} = 1$ and $r_n = n$. All
$d_{ij} = 0$. Now running the star-greedy algorithm, we will first
pick a star with all $n$ clients and we open $1$ copy of facility
$f_1$. We then have only client $n$ with residual demand $r_n' = n-1$,
and we have no other option but to open facility $f_1$ for another
$n-1$ copies.

Now the dual-fitting based analysis will associate each demand with a
dual variable $\alpha_j^1, \ldots, \alpha_j^{r_j}$ and the proposed
dual solution is $\bar\alpha_j = \sum_{l=1}^{r_j} \alpha_j^{l} / r_j$
and try to find a minimum $\gamma$ such that $\{\bar\alpha_j/\gamma\}$
is a feasible dual, that is
\begin{equation}
  \sum_{j=1}^n (\bar\alpha_j/\gamma - d_{1j})_+ \leq f_1 = n
\end{equation}
which is
\begin{equation}
  \sum_{j=1}^n \bar\alpha_j/\gamma  \leq n
\end{equation}
since all $d_{ij}=0$ and $f_1 = n$.

From the greedy algorithm, we have $\alpha_j^1 = 1$ for
$j=1,\ldots,n$, and $\alpha_n^l = n$ for $l=2,\ldots,n$. Therefore
$\bar\alpha_j = 1$ for $j=1,\ldots,n-1$ and $\bar\alpha_n =
(1+(n-1)n)/n = n-1$. The shrinking factor $\gamma$ we seek thus
satisfies
\begin{equation}
  \sum_{j=1}^{n-1} (\bar\alpha_j/\gamma - 0)_+ + (\bar\alpha_n/\gamma -0)_+
  \leq f_1 = n,
\end{equation}
which is
\begin{equation}
  (n-1)(1/\gamma)  + (n/\gamma) \leq n
\end{equation}
Simple algebra will show that $\gamma$ can be made arbitrarily close
to $2$ when $n$ is large. On the other hand we know that the same
greedy algorithm with dual-fitting analysis gives a ratio of $1.81$
for the UFL problem where all $r_j=1$.

This example does not actually rule out the possibility to prove a
constant ratio of the star-greedy algorithm on FTFP. In fact greedy
gets exactly the same solution as the optimal integral solution for
this example. All it says is that the dual-fitting analysis on greedy
algorithm, when applied to the FTFP or FTFL problem, cannot possibly
give a ratio much better than $2$. And this partly explains why
generalizing primal-dual or dual-fitting algorithms from UFL to
fault-tolerant problems like FTFL or FTFP is not successful when
$r_j$'s are not equal, that is demands are not uniform. Intuitively
there seems to be some issue fundamental to the dual-fitting approach
as the proposed dual solution $\bar\alpha_j$'s can be very different
between each other so shrinking all of them by a common factor
$\gamma$ might not give a strong upper bound on the approximation
ratio. It is also quite possible that the example may be strengthened
to show that dual-fitting cannot achieve a worse yet ratio.

\end{document}
