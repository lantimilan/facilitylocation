\documentclass{article}

\title{On primal-dual and dual-fitting of the FTFP problem}
\author{UCR theory lab}
\usepackage{fullpage}
\begin{document}
\maketitle

\section{Dual-fitting Analysis on FTFP}
This section gives a simple example that the dual-fitting analysis of
a greedy algorithm which repeatedly picking the most cost-effective
star (the star with minimum average cost) is unlikely to give the same
ratio as that for the UFL problem.

The example consists of $1$ facility with cost $f_1 = n$, and $n$
clients with demands $r_1=r_2=\ldots=r_{n-1} = 1$ and $r_n = n$. All
$d_{ij} = 0$. Now running the star-greedy algorithm, we will first
pick a star with all $n$ clients and we open $1$ copy of facility
$f_1$. We then have only client $n$ with residual demand $r_n' = n-1$,
and we have no other option but to open facility $f_1$ for another
$n-1$ copies.

Now the dual-fitting based analysis will associate each demand with a
dual variable $\alpha_j^1, \ldots, \alpha_j^{r_j}$ and the proposed
dual solution is $\bar\alpha_j = \sum_{l=1}^{r_j} \alpha_j^{l} / r_j$
and try to find a minimum $\gamma$ such that $\{\bar\alpha_j/\gamma\}$
is a feasible dual, that is
\begin{equation}
  \sum_{j=1}^n (\bar\alpha_j/\gamma - d_{1j})_+ \leq f_1 = n
\end{equation}
which is
\begin{equation}
  \sum_{j=1}^n \bar\alpha_j/\gamma  \leq n
\end{equation}
since all $d_{ij}=0$ and $f_1 = n$.

From the greedy algorithm, we have $\alpha_j^1 = 1$ for
$j=1,\ldots,n$, and $\alpha_n^l = n$ for $l=2,\ldots,n$. Therefore
$\bar\alpha_j = 1$ for $j=1,\ldots,n-1$ and $\bar\alpha_n =
(1+(n-1)n)/n = n-1$. The shrinking factor $\gamma$ we seek thus
satisfies
\begin{equation}
  \sum_{j=1}^{n-1} (\bar\alpha_j/\gamma - 0)_+ + (\bar\alpha_n/\gamma -0)_+
  \leq f_1 = n,
\end{equation}
which is
\begin{equation}
  (n-1)(1/\gamma)  + (n/\gamma) \leq n
\end{equation}
Simple algebra will show that $\gamma$ can be made arbitrarily close
to $2$ when $n$ is large. On the other hand we know that the same
greedy algorithm with dual-fitting analysis gives a ratio of $1.81$
for the UFL problem where all $r_j=1$.

This example does not actually rule out the possibility to prove a
constant ratio of the star-greedy algorithm on FTFP. In fact greedy
gets exactly the same solution as the optimal integral solution for
this example. All it says is that the dual-fitting analysis on greedy
algorithm, when applied to the FTFP or FTFL problem, cannot possibly
give a ratio much better than $2$. And this partly explains why
generalizing primal-dual or dual-fitting algorithms from UFL to
fault-tolerant problems like FTFL or FTFP is not successful when
$r_j$'s are not equal, that is demands are not uniform. Intuitively
there seems to be some issue fundamental to the dual-fitting approach
as the proposed dual solution $\bar\alpha_j$'s can be very different
between each other so shrinking all of them by a common factor
$\gamma$ might not give a strong upper bound on the approximation
ratio. It is also quite possible that the example may be strengthened
to show that dual-fitting cannot achieve a worse yet ratio.

\end{document}
