% start journal version
%%%%%%%%%%%%%%%%%%%%%%%%%%%%%%%%%%%%%%%
% Tue Aug 28 2012, 09:55:36
% marek Sat Jul 14 13:24:51 PDT 2012
% marek Fri Jul 13 15:51:33 PDT 2012
% lyan Thu Jul 12 2012, 00:06:30
% lyan Wed Jul 04 2012, 22:29:33
%
\documentclass[11pt]{article}

\usepackage{fullpage,amssymb,amsthm,enumerate}
\usepackage[nosumlimits]{amsmath}
\usepackage[nothing]{algorithm}
\usepackage{algorithmicx}
%\usepackage[noend]{algorithmic}
\usepackage[noend]{algpseudocode}
%\usepackage[firstinits=true]{biblatex}

%\algsetup{indent=2em}
\floatname{algorithm}{Pseudocode}
\renewcommand{\algorithmicrequire}{\textbf{Input:}}
\renewcommand{\algorithmicensure}{\textbf{Output:}}


%%%%%%%%%%%%%%%%%%%%%%%%%%%%%%%%%%%%%%%%%%%%%%%%%%%%%%%%%%

% non-math stuff

\newcommand{\myparagraph}[1]{{\smallskip\noindent{\bf #1}}}
\newcommand{\emparagraph}[1]{{\smallskip\noindent{\it #1}}}
\newcommand{\etal}{{\it et al.}}
\newcommand{\myif}{{\mbox{\rm\ if \ }}}
\newcommand{\mycase}[1]{\mbox{{\underline{Case #1}}:\/}}

\newcommand{\margincomment}[1]%
    {{%
      \marginpar{{\tiny\begin{minipage}{0.5in}
                       \begin{flushleft}
                          {#1}
                       \end{flushleft}
                       \end{minipage}
                }}
    }}


%%%%%%%%%%%%%%%%%%%%%%%%%%%%%%%%%%%%%%%%%%%%%%%%%%%%%%%%%%

% various letters

\newcommand{\hatc}{{\hat c}}
\newcommand{\hatC}{{\hat C}}
\newcommand{\hatr}{{\hat r}}
\newcommand{\hatx}{{\hat x}}
\newcommand{\haty}{{\hat y}}
\newcommand{\dotx}{{\dot x}}
\newcommand{\doty}{{\dot y}}
\newcommand{\dotr}{{\dot r}}
\newcommand{\boldx}{{\mathbf x}}

\newcommand{\doubledone}{{\bar 1}}
\newcommand{\doubledtwo}{{\bar 2}}
\newcommand{\barc}{{\bar c}}
\newcommand{\bart}{{\bar t}}

\newcommand{\barx}{{\bar x}}
\newcommand{\bary}{{\bar y}}
\newcommand{\barz}{{\bar z}}
\newcommand{\barr}{{\bar r}}
\newcommand{\barX}{{\bar X}}
\newcommand{\barY}{{\bar Y}}
\newcommand{\barZ}{{\bar Z}}
\newcommand{\bara}{{\bar a}}
\newcommand{\bard}{{\bar d}}
\newcommand{\barm}{{\bar m}}
\newcommand{\barA}{{\bar A}}
\newcommand{\barB}{{\bar B}}
\newcommand{\barC}{{\bar C}}
\newcommand{\barG}{{\bar G}}
\newcommand{\barE}{{\bar E}}
\newcommand{\barV}{{\bar V}}

\newcommand{\wbarC}{{\overline{C}}}
\newcommand{\wbarD}{{\overline{D}}}
\newcommand{\wbarN}{{\overline{N}}}
\newcommand{\wbarX}{{\overline{X}}}


\newcommand{\barbeta}{{\bar\beta}}
\newcommand{\bargamma}{{\bar\gamma}}
\newcommand{\apomega}{{\bar\omega}}

\newcommand{\bfr}{\boldsymbol{r}}
\newcommand{\bfv}{{\bf v}}
\newcommand{\bfx}{\boldsymbol{x}}
\newcommand{\bfy}{\boldsymbol{y}}
\newcommand{\bfz}{{\bf z}}
\newcommand{\bfQ}{{\bf Q}}
\newcommand{\bfR}{{\bf R}}
\newcommand{\bfS}{{\bf S}}
\newcommand{\bfT}{{\bf T}}
\newcommand{\bfV}{{\bf V}}
\newcommand{\bfone}{{\bf 1}}
\newcommand{\bfalpha}{\boldsymbol{\alpha}}
\newcommand{\bfbeta}{\boldsymbol{\beta}}

\newcommand{\calA}{{\cal A}}
\newcommand{\calB}{{\cal B}}
\newcommand{\calC}{{\cal C}}
\newcommand{\calD}{{\cal D}}
\newcommand{\calE}{{\cal E}}
\newcommand{\calG}{{\cal G}}
\newcommand{\calH}{{\cal H}}
\newcommand{\calJ}{{\cal J}}
\newcommand{\calK}{{\cal K}}
\newcommand{\calL}{{\cal L}}
\newcommand{\calM}{{\cal M}}
\newcommand{\calN}{{\cal N}}
\newcommand{\calS}{{\cal S}}
\newcommand{\calU}{{\cal U}}
\newcommand{\calX}{{\cal X}}
\newcommand{\calT}{{\cal T}}

\newcommand{\hatcalI}{{\hat{\cal I}}}
\newcommand{\barcalI}{{\bar{\cal I}}}
\newcommand{\dotcalI}{{\dot{\cal I}}}

\newcommand{\vecS}{{\bar S}}
\newcommand{\vecT}{{\bar T}}
\newcommand{\vecone}{{\bf 1}}
\newcommand{\tildec}{{\tilde c}}
\newcommand{\tilded}{{\tilde d}}
\newcommand{\tildeD}{{\tilde D}}
\newcommand{\tildeC}{{\widetilde C}}
\newcommand{\tildeZ}{{\tilde Z}}
\newcommand{\tilder}{{\widetilde r}}
\newcommand{\tildex}{{\widetilde x}}
\newcommand{\wtildeN}{{\widetilde N}}
\newcommand{\tildebfr}{\widetilde{\boldsymbol{r}}}
\newcommand{\tildebfx}{\widetilde{\boldsymbol{x}}}
\newcommand{\tildebfy}{\widetilde{\boldsymbol{y}}}

\newcommand{\barbfx}{\bar{\boldsymbol{x}}}
\newcommand{\barbfy}{\bar{\boldsymbol{y}}}
\newcommand{\hatbfx}{\hat{\boldsymbol{x}}}
\newcommand{\hatbfy}{\hat{\boldsymbol{y}}}
\newcommand{\dotbfx}{\dot{\boldsymbol{x}}}
\newcommand{\dotbfy}{\dot{\boldsymbol{y}}}

\newcommand{\wbarcalC}{{\overline{\calC}}}
\newcommand{\wbarcalD}{{\overline{\calD}}}
\newcommand{\eps}{{\epsilon}}

%%%%%%%%%%%%%%%%%%%%%%%%%%%%%%%%%%%%%%%%%%%%%%%%%%%%%%%%%%

\newcommand{\half}{{\mbox{$\frac{1}{2}$}}}
\newcommand{\threehalfs}{{\mbox{$\frac{3}{2}$}}}
\newcommand{\threefourths}{{\mbox{$\frac{3}{4}$}}}
\newcommand{\fivehalfs}{{\mbox{$\frac{5}{2}$}}}
\newcommand{\onethird}{{\mbox{$\frac{1}{3}$}}}
\newcommand{\twothirds}{{\mbox{$\frac{2}{3}$}}}
\newcommand{\fourthirds}{{\mbox{$\frac{4}{3}$}}}
\newcommand{\fivethirds}{{\mbox{$\frac{5}{3}$}}}
\newcommand{\fivefourths}{{\mbox{$\frac{5}{4}$}}}
\newcommand{\onefourth}{{\mbox{$\frac{1}{4}$}}}
\newcommand{\onefifth}{{\mbox{$\frac{1}{5}$}}}
\newcommand{\twofifths}{{\mbox{$\frac{2}{5}$}}}
\newcommand{\threefifths}{{\mbox{$\frac{3}{5}$}}}
\newcommand{\fourfifths}{{\mbox{$\frac{4}{5}$}}}
\newcommand{\ninefifths}{{\mbox{$\frac{9}{5}$}}}
\newcommand{\sevensixths}{{\mbox{$\frac{7}{6}$}}}
\newcommand{\oneeighth}{{\mbox{$\frac{1}{8}$}}}
\newcommand{\threeeighths}{{\mbox{$\frac{3}{8}$}}}
\newcommand{\fiveeighths}{{\mbox{$\frac{5}{8}$}}}
\newcommand{\seveneighths}{{\mbox{$\frac{7}{8}$}}}
\newcommand{\onetenth}{{\mbox{$\frac{1}{10}$}}}
\newcommand{\seventenths}{{\mbox{$\frac{7}{10}$}}}
\newcommand{\ninetenths}{{\mbox{$\frac{9}{10}$}}}
\newcommand{\twonineths}{{\mbox{$\frac{2}{9}$}}}
\newcommand{\fivenineths}{{\mbox{$\frac{5}{9}$}}}
\newcommand{\elevennineths}{{\mbox{$\frac{11}{9}$}}}
\newcommand{\threetwentieths}{{\mbox{$\frac{3}{20}$}}}
\newcommand{\twentyfivenineteenths}{{\mbox{$\frac{25}{19}$}}}

\newcommand{\sqrttwo}{\sqrt{2}}

%%%%%%%%%%%%%%%%%%%%%%%%%%%%%%%%%%%%%%%%%%%%%%%%%%%%%%%%%%

% various delimiters

\newcommand{\braced}[1]{{ \left\{ #1 \right\} }}
\newcommand{\angled}[1]{{ \left\langle #1 \right\rangle }}
\newcommand{\brackd}[1]{{ \left[ #1 \right] }}
\newcommand{\parend}[1]{{ \left( #1 \right) }}
\newcommand{\barred}[1]{{ \left| #1 \right| }}
\newcommand{\dbarred}[1]{{ \left\| #1 \right\| }}
\newcommand{\floor}[1]{{ \lfloor #1 \rfloor }}
\newcommand{\ceiling}[1]{{ \lceil #1 \rceil }}

%%%%%%%%%%%%%%%%%%%%%%%%%%%%%%%%%%%%%%%%%%%%%%%%%%%%%%%%%%

% some math symbols

\newcommand{\set}{\,{\leftarrow}\,}
\newcommand{\suchthat}{{\,:\,}}
\newcommand{\cost}{{\it cost}}
\newcommand{\yield}{{\it yield}}
\newcommand{\opt}{{\it opt}}

\newcommand{\algA}{{\bf A}}
\newcommand{\LHS}{{\rm LHS}}
\newcommand{\RHS}{{\rm RHS}}
\newcommand{\reals}{{\bf R}}
\newcommand{\posreals}{{\bf R}^+}

\newcommand{\assign}{{\,\leftarrow\,}}

\newcommand{\absvalue}[1]{{\barred{#1}}}
\newcommand{\posvalue}[1]{{\brackd{#1}^+}}

\newcommand{\NP}{{\mbox{\sf NP}}}
\newcommand{\PP}{{\mbox{\sf P}}}
\newcommand{\DTIME}{{\mbox{\sf DTIME}}}

\newcommand{\letbox}[1]{{\makebox[11pt]{{\small {$#1$}}}}}
\newcommand{\optstring}[1]{{ \frame{\;\raisebox{0pt}[12pt][5pt]{#1}\;} }}

\newcommand{\leftend}{{\diamond}}
\newcommand{\rightend}{{\diamond}}

%\newcommand{\argmin}{{\mbox{\rm argmin}}}
\DeclareMathOperator*{\argmin}{arg\,min}

\newcommand\litem[1]{\item{\bfseries #1\enspace}}
\newcommand{\ceil}[1] {\lceil #1 \rceil}
\newcommand{\naive}{na\"{\i}ve}
\newcommand{\LP}{\mbox{\rm LP}}
\newcommand{\OPT}{\mbox{\rm OPT}}
\newcommand{\ALG}{\mbox{\rm ALG}}
\newcommand{\LPR}[1]{{\mbox{\rm LPR#1}}}
\newcommand{\smallLPR}[1]{{\mbox{\tiny\rm LPR#1}}}
% algorithm names
\newcommand{\ESTA}{\mbox{\rm ESTA}} % 4approx
\newcommand{\EGUP}{\mbox{\rm EGUP}} % 3approx
\newcommand{\ECHS}{\mbox{\rm ECHS}} % 1.736
\newcommand{\EBGS}{\mbox{\rm EBGS}} % 1.575
\newcommand{\GUP}{\mbox{\rm GUP}}
\newcommand{\smallESTA}{\mbox{\tiny\rm ESTA}}
\newcommand{\smallEGUP}{\mbox{\tiny\rm EGUP}}
\newcommand{\smallECHS}{\mbox{\tiny\rm ECHS}}
\newcommand{\smallEBGS}{\mbox{\tiny\rm EBGS}}

\newcommand{\SOL}[1]{{{\mbox{\rm SOL}}_{#1}}}
\newcommand{\FTFP}{\mbox{\rm FTFP}}
\newcommand{\FTFL}{\mbox{\rm FTFL}}
\newcommand{\calI}{\mathcal{I}}
\newcommand{\avg}{{\mbox{\scriptsize\rm avg}}}

\newcommand{\dmax}{\text{dmax}}
\newcommand{\davg}{\text{davg}}
\newcommand{\favg}{f_{\text{avg}}}
\newcommand{\conn}{\text{conn}}
\newcommand{\cls}{\text{cls}}
\newcommand{\far}{\text{far}}

\newcommand{\sitesset}{\mathbb{F}}
\newcommand{\clientset}{\mathbb{C}}
\newcommand{\facilityset}{\overline{\sitesset}}
\newcommand{\demandset}{\overline{\clientset}}

%\newcommand{\dist}{{\mbox{dist}}}
\newcommand{\concost}{C^{\avg}}
\newcommand{\faccost}{F^{\avg}}
\newcommand{\tcc}{\textrm{tcc}}
\newcommand{\clsdist}{C_{\cls}^{\avg}}
\newcommand{\fardist}{C_{\far}^{\avg}}
\newcommand{\clsmax}{C_{\cls}^{\max}}
\newcommand{\clsnb}{N_{\cls}}
\newcommand{\farnb}{N_{\far}}
\newcommand{\wbarclsnb}{\wbarN_{\cls}}
\newcommand{\wbarfarnb}{\wbarN_{\far}}

\newcommand{\Exp}{\mbox{\rm Exp}}

\newcommand{\FacilityDistSort}{{\textsc{FacilityDistSort}}}
\newcommand{\NearestUnitChunk}{{\textsc{NearestUnitChunk}}}
\newcommand{\AugmentToUnit}{{\textsc{AugmentToUnit}}}
\newcommand{\connsum}{{\textrm{conn}}}

%%%%%%%%%%%%%%%%%%%%%%%%%%%%%%%%%%%%%%%%%%%%%%%%%%%%%%%%%%

% theorem and such

\newtheorem{fact}[theorem]{Fact}
\newtheorem{observation}[theorem]{Observation}

%%%%%%%%%%%%%%%%%%%%%%%%%%%%%%%%%%%%%%%%%%%%%%%%%%%%%%%%%%

\newcommand{\ignore}[1]{}

%%%%%%%%%%%%%%%%%%%%%%%%%%%%%%%%%%%%%%%%%%%%%%%%%%%%%%%%%%%%%%%%%%%%%%%%%%%%%%%
%%%%%%%%%%%%%%%%%%%%%%%%%%%%%%%%%%%%%%%%%%%%%%%%%%%%%%%%%%%%%%%%%%%%%%%%%%%%%%%
%%%%%%%%%%%%%%%%%%%%%%%%%%%%%%%%%%%%%%%%%%%%%%%%%%%%%%%%%%%%%%%%%%%%%%%%%%%%%%%

\title{LP-rounding Algorithms for the Fault-Tolerant\\
 		Facility Placement Problem}

\author{Li Yan and Marek Chrobak\\
  Department of Computer Science\\
 University of California at Riverside}

\date{}

\begin{document}
\maketitle

\thispagestyle{empty}
\begin{abstract} 
  The Fault-Tolerant Facility Placement problem (FTFP) is a
  generalization of the classic Uncapacitated Facility
  Location Problem (UFL). In FTFP we are given a set of
  facility sites and a set of clients. Opening a facility at
  site $i$ costs $f_i$ and connecting client $j$ to a
  facility at site $i$ costs $d_{ij}$. We assume that the
  connection costs (distances) $d_{ij}$ satisfy the triangle
  inequality. Multiple facilities can be opened at any
  site. Each client $j$ has a demand $r_j$, which means that
  it needs to be connected to $r_j$ different facilities
  (some of which could be located on the same site). The
  goal is to minimize the sum of facility opening cost and
  connection cost.

  The main result of this paper is a $1.575$-approximation
  algorithm for FTFP, based on LP-rounding. The algorithm
  first reduces the demands to values polynomial in the
  number of sites. Then it uses a technique that we
  call adaptive partitioning, which partitions the instance
  by splitting clients into unit demands and creating a
  number of (not yet opened) facilities at each site. It
  also partitions the optimal fractional solution to produce
  a fractional solution for this new instance.  The
  partitioned instance satisfies a number of properties that
  allow us to exploit existing LP-rounding methods for UFL to
  round our partitioned solution to an integral
  solution, preserving the approximation ratio.  In
  particular, our $1.575$-approximation algorithm is based
  on the ideas from the $1.575$-approximation algorithm for
  UFL by Byrka~\etal, with changes necessary to satisfy
  the fault-tolerance requirement.
\end{abstract}

\pagebreak
\setcounter{page}{1}
%%%%%%%%%%%%%%%%%%%%%%%%%%%%%%%%%%%%%%%%%%%%%%%%%%%%%%%%%%%%%%%%%%%%%%%%%%%%%%%
%%% INTRODUCTION %%%%%%%%%%%%%%%%%%%%%%%%%%%%%%%%%%%%%%%%%%%%%%%%%%%%%%%%%%%%%%
%%%%%%%%%%%%%%%%%%%%%%%%%%%%%%%%%%%%%%%%%%%%%%%%%%%%%%%%%%%%%%%%%%%%%%%%%%%%%%%

\section{Introduction}

In the \emph{Fault-Tolerant Facility Placement} problem
(FTFP), we are given a set $\sitesset$ of \emph{sites} at
which facilities can be built, and a set $\clientset$ of
\emph{clients} with some demands that need to be satisfied
by different facilities. A client $j$ has demand
$r_j$. Building one facility at a site $i$ incurs a cost
$f_i$, and connecting one unit of demand from client $j$ to
a facility at site $i\in\sitesset$ costs $d_{ij}$. Throughout the
paper we assume that the connection costs (distances)
$d_{ij}$ form a metric, that is, they are
symmetric and satisfy the triangle inequality. In a feasible solution, some
number of facilities, possibly zero, are opened at each site
$i$, and demands from each client are connected to those
open facilities, with the constraint that demands from the
same client have to be connected to different
facilities. Note that facilities at the same site are considered different.

It is easy to see that if all $r_j=1$ then FTFP reduces to
the classic Uncapacitated Facility Location problem (UFL).
If we add a constraint that each site can have at most one
facility built, then the problem becomes equivalent to the
Fault-Tolerant Facility Location problem (FTFL). One
implication of the one-facility-per-site restriction in FTFL
is that $\max_{j\in\clientset}r_j \leq |\sitesset|$, while
in FTFP the values of $r_j$'s can be much bigger than
$|\sitesset|$.

The UFL problem has a long history; in particular, great
progress has been achieved in the past two decades in
developing techniques for designing constant-ratio
approximation algorithms for UFL.  Shmoys, Tardos and
Aardal~\cite{ShmoysTA97} proposed an approach based on
LP-rounding, that they used to achieve a ratio of 3.16.
This was then improved by Chudak~\cite{ChudakS04} to 1.736,
and later by Sviridenko~\cite{Svi02} to 1.582.
The best known ``pure" LP-rounding algorithm is due to
Byrka~{\etal}~\cite{ByrkaGS10} with ratio 1.575. 
Byrka and Aardal~\cite{ByrkaA10} gave a hybrid algorithm that combines LP-rounding
and dual-fitting (based on \cite{JainMMSV03}), achieving a ratio of 1.5.  Recently,
Li~\cite{Li11} showed that, with a more refined analysis and
randomizing the scaling parameter used in \cite{ByrkaA10}, the ratio can be improved
to 1.488. This is the best known approximation result for UFL.  
Other techniques include the primal-dual algorithm with ratio 3 by
Jain and Vazirani~\cite{JainV01}, the dual fitting method by
Jain~{\etal}~\cite{JainMMSV03} that gives ratio 1.61, and a
local search heuristic by Arya~{\etal}~\cite{AryaGKMMP04}
with approximation ratio 3.  On the hardness side, UFL is
easily shown to be {\NP}-hard, and it is known that it is
not possible to approximate UFL in polynomial time with
ratio less than $1.463$, provided that
$\NP\not\subseteq\DTIME(n^{O(\log\log
  n)})$~\cite{GuhaK98}. An observation by Sviridenko
strengthened the underlying assumption to $\PP\ne \NP$ (see \cite{vygen05}).

FTFL was first introduced by Jain and
Vazirani~\cite{JainV03} and they adapted their primal-dual
algorithm for UFL to obtain a ratio of
$3\ln(\max_{j\in\clientset}r_j)$.  All subsequently
discovered constant-ratio approximation algorithms use
variations of LP-rounding.  The first such algorithm, by
Guha~{\etal}~\cite{GuhaMM01}, adapted the approach for UFL
from \cite{ShmoysTA97}.  Swamy and Shmoys~\cite{SwamyS08}
improved the ratio to $2.076$ using the idea of pipage
rounding introduced in \cite{Svi02}. Most recently,
Byrka~{\etal}~\cite{ByrkaSS10} improved the ratio to 1.7245
using dependent rounding and laminar clustering.

FTFP is a natural generalization of UFL. It was first
studied by Xu and Shen~\cite{XuS09}, who extended the
dual-fitting algorithm from~\cite{JainMMSV03} to give an
approximation algorithm with a ratio claimed to be
$1.861$. However their algorithm runs in polynomial time
only if $\max_{j\in\clientset} r_j$ is polynomial in
$O(|\sitesset|\cdot |\clientset|)$ and the analysis of the
performance guarantee in \cite{XuS09} is flawed\footnote{Confirmed through
  private communication with the authors.}.  To date, the
best approximation ratio for FTFP in the literature is $4$,
established by Yan and Chrobak~\cite{YanC11}, while the only
known lower bound is the $1.463$ lower bound for UFL
from~\cite{GuhaK98}, as UFL is a special case of FTFP.

\smallskip

The main result of this paper is an LP-rounding algorithm
for FTFP with approximation ratio 1.575, matching the best
ratio for UFL achieved via the LP-rounding method
\cite{ByrkaGS10} and significantly improving our earlier
bound in~\cite{YanC11}. In Section~\ref{sec: polynomial
  demands} we prove that, for the purpose of LP-based
approximations, the general FTFP problem can be reduced to
the restricted version where all demand values are
polynomial in the number of sites.  This \emph{demand
  reduction} trick itself gives us a ratio of $1.7245$,
since we can then treat an instance of FTFP as an instance
of FTFL, by creating a sufficient (but polynomial) number of
facilities at each site and using the algorithm
from~\cite{ByrkaSS10}.

The reduction to polynomial demands suggests an approach
where clients' demands are split into unit demands. These
unit demands can be thought of as ``unit-demand clients'',
and a natural approach would be to adapt LP-rounding methods
from \cite{gupta08,ChudakS04,ByrkaGS10} to this new set of
unit-demand clients.  Roughly, these algorithms iteratively
pick a client that minimizes a certain cost function (that
varies for different algorithms) and open one facility in
the neighborhood of this client. The remaining clients are
then connected to these open facilities.  In order for this
to work, we also need to convert the optimal fractional
solution $(\bfx^\ast,\bfy^\ast)$ of the original instance
into a solution $(\barbfx,\barbfy)$ of the modified instance
which then can be used in the LP-rounding process. This can
be thought of as partitioning the fractional solution, as
each connection value $x^\ast_{ij}$ must be somehow divided
between the $r_j$ unit demands of client $j$. The
partitioned fractional solution needs to possess certain
properties to apply rounding algorithms similar to those for
UFL. In Section~\ref{sec: adaptive partitioning} we present
our \emph{adaptive partitioning} technique by first stating
a set of desired properties, and then present a natural way
of partitioning the instance (and the optimal fractional
solution). We also prove that the partitioned solution
satisfies all prescribed properties. Using adaptive
partitioning we were able to extend the algorithms for UFL
from \cite{gupta08,ChudakS04,ByrkaGS10} to FTFP. We
illustrate the fundamental ideas of our approach in
Section~\ref{sec: 3-approximation}, showing how they can be
used to design an LP-rounding algorithm with ratio $3$. The
algorithm relies heavily on the properties of the instance
and its associated fractional solution produced by adaptive
partitioning; in particular, these properties allow us to
assign facilities to demands so that two demands from the
same client are assigned to different facilities.  In
Section~\ref{sec: 1.736-approximation} we refine the
algorithm to improve the approximation ratio to
$1+2/e\approx 1.736$.  Finally, in Section~\ref{sec:
  1.575-approximation}, we improve it even further to
$1.575$ -- the main result of this paper.

Summarizing, our contributions are two-fold: One, we show
that the existing LP-rounding algorithms for UFL can be
extended to a much more general problem FTFP, retaining the
approximation ratio. We believe that, should even better
LP-rounding algorithms be developed for UFL in the future,
using our demand reduction and adaptive partitioning
methods, it should be possible to extend them to FTFP.
In fact, some improvement of the ratio
should be achieved by randomizing the scaling parameter
$\gamma$ used in our algorithm, as Li showed in \cite{Li11}
for UFL.  (Since the ratio $1.488$ for UFL in~\cite{Li11}
uses also dual-fitting
algorithms~\cite{MahdianYZ06}, we would not obtain the same
ratio for FTFP yet using only LP-rounding.)

Two, our ratio of $1.575$ is significantly better than the
best currently known ratio of $1.7245$ for the
closely-related FTFL problem. This suggests that in the
fault-tolerant scenario the capability of creating
additional copies of facilities on the existing sites makes
the problem easier from the point of view of approximation.

%%%%%%%%%%%%%%%%%%%%%%%%%%%%%%%%%%%%%%%%%%%%%%%%%%%%%%%%%%%%%%%%%%%%%%%%%%%%%%%
%%% LP ForMULATION %%%%%%%%%%%%%%%%%%%%%%%%%%%%%%%%%%%%%%%%%%%%%%%%%%%%%%%%%%%%
%%%%%%%%%%%%%%%%%%%%%%%%%%%%%%%%%%%%%%%%%%%%%%%%%%%%%%%%%%%%%%%%%%%%%%%%%%%%%%%

\section{The LP Formulation}\label{sec: the lp formulation}

The FTFP problem has a natural Integer Programming (IP)
formulation. Let $y_i$ represent the number of facilities
built at site $i$ and let $x_{ij}$ represent the number of
connections from client $j$ to facilities at site $i$. If we
relax the integrality constraints, we obtain the following LP:

%%%%%%%%%%%
\begin{alignat}{3}
  \textrm{minimize} \quad \cost(\bfx,\bfy) &= \textstyle{\sum_{i\in \sitesset}f_iy_i 
								+ \sum_{i\in \sitesset, j\in \clientset}d_{ij}x_{ij}}\label{eqn:fac_primal}\hspace{-1.5in}&&
									\\ \notag
  \textrm{subject to}\quad y_i - x_{ij} &\geq 0 			&\quad 		&\forall i\in \sitesset, j\in \clientset 
									\\ \notag
     \textstyle{\sum_{i\in \sitesset} x_{ij}} &\geq r_j  &			&\forall j\in \clientset
 									\\ \notag
  	  x_{ij} \geq 0, y_i &\geq 0 						& 			&\forall i\in \sitesset, j\in \clientset 
  									\\ \notag
\end{alignat}

%%%%%%%%%%%%

\noindent
The dual program is:

\begin{alignat}{3}
  \textrm{maximize}\quad \textstyle{\sum_{j\in \clientset}} r_j\alpha_j&\label{eqn:fac_dual}  
     						\\ \notag
  \textrm{subject to} \quad \textstyle{
    \sum_{j\in \clientset}\beta_{ij}} &\leq f_i  &\quad\quad			&\forall i \in \sitesset  
							\\ \notag
  \alpha_{j} - \beta_{ij} 	&\leq  d_{ij}       &                 & \forall i\in \sitesset, j\in \clientset 
							\\ \notag
  \alpha_j \geq 0, \beta_{ij} &\geq 0           &            & \forall i\in \sitesset, j\in \clientset
  							\\ \notag
\end{alignat}

In each of our algorithms we will fix some optimal
solutions of the LPs (\ref{eqn:fac_primal}) and (\ref{eqn:fac_dual})
that we will denote by $(\bfx^\ast, \bfy^\ast)$ and
$(\bfalpha^\ast,\bfbeta^\ast)$, respectively.

With $(\bfx^\ast, \bfy^\ast)$ fixed, we can define the
optimal facility cost as $F^\ast=\sum_{i\in\sitesset} f_i
y_i^\ast$ and the optimal connection cost as $C^\ast =
\sum_{i\in\sitesset,j\in\clientset} d_{ij}x_{ij}^\ast$.
Then $\LP^\ast = \cost(\bfx^\ast,\bfy^\ast) = F^\ast+C^\ast$
is the joint optimal value of (\ref{eqn:fac_primal}) and
(\ref{eqn:fac_dual}).  We can also associate with each
client $j$ its fractional connection cost $C^\ast_j =
\sum_{i\in\sitesset} d_{ij}x_{ij}^\ast$.  Clearly, $C^\ast =
\sum_{j\in\clientset} C^\ast_j$.  Throughout the paper we
will use notation $\OPT$ for the optimal integral solution
of (\ref{eqn:fac_primal}).  $\OPT$ is the value we wish to
approximate, but, since $\OPT\ge\LP^\ast$, we can instead use
$\LP^\ast$ to estimate the approximation ratio of our
algorithms.

%%%%%%%%%

\paragraph{Completeness and facility splitting.}
Define $(\bfx^\ast, \bfy^\ast)$ to be \emph{complete} if
$x_{ij}^\ast>0$ implies that $x_{ij}^\ast=y_i^\ast$ for all
$i,j$. In other words, each connection either uses a site
fully or not at all.  As shown by Chudak and
Shmoys~\cite{ChudakS04}, we can modify the given instance by
adding at most $|\clientset|$ sites to obtain an equivalent
instance that has a complete optimal solution, where
``equivalent" means that the values of $F^\ast$, $C^\ast$ and
$\LP^\ast$ are not affected. Roughly, the
argument is this: We notice that, without loss of
generality, for each client $k$ there exists at most one
site $i$ such that $0 < x_{ik}^\ast < y_i^\ast$.  We can
then perform the following \emph{facility splitting}
operation on $i$: introduce a new site $i'$, let
$y^\ast_{i'} = y^\ast_i - x^\ast_{ik}$, redefine $y^\ast_i$
to be $x^\ast_{ik}$, and then for each client $j$
redistribute $x^\ast_{ij}$ so that $i$ retains as much
connection value as possible and $i'$ receives the
rest. Specifically, we set
%
\begin{alignat*}{2}
  y^\ast_{i'} \;&\assign\; y^\ast_i - x^\ast_{ik},  & y^\ast_{i} \;&\assign\; x^\ast_{ik},\quad\intertext{and}
  x^\ast_{i'j} \;&\assign\;\max( x^\ast_{ij} - x^\ast_{ik}, 0 ),\quad	& x^\ast_{ij} \;&\assign\; \min( x^\ast_{ij} , x^\ast_{ik}) 
			\quad	\textrm{for all}\ j \neq k.
\end{alignat*}
%
This operation eliminates the partial connection between $k$
and $i$ and does not create any new partial
connections. Each client can split at most one site and
hence we shall have at most $|\clientset|$ more sites.

By the above paragraph,  without loss of generality we can
assume that the optimal fractional solution $(\bfx^\ast, \bfy^\ast)$
is complete. This assumption will in fact greatly simplify some of
the arguments in the paper. Additionally, we will frequently use the facility
splitting operation in our algorithms to obtain fractional solutions with
desirable properties.


%%%%%%%%%%%%%%%%%%%%%%%%%%%%%%%%%%%%%%%%%%%%%%%%%%%%%%%%%%%%%%%%%%%%%%%%%%%%%%%
%%% REDUCTION TO POLYNOMIAL DEMANDS %%%%%%%%%%%%%%%%%%%%%%%%%%%%%%%%%%%%%%%%%%%
%%%%%%%%%%%%%%%%%%%%%%%%%%%%%%%%%%%%%%%%%%%%%%%%%%%%%%%%%%%%%%%%%%%%%%%%%%%%%%%

\section{Reduction to Polynomial Demands}
\label{sec: polynomial demands}

This section presents a \emph{demand reduction} trick that
reduces the problem for arbitrary demands to a special case
where demands are bounded by $|\sitesset|$, the number of
sites.  (The formal statement is a little more technical --
see Theorem~\ref{thm: reduction to polynomial}.)  Our
algorithms in the sections that follow process individual
demands of each client one by one, and thus they critically
rely on the demands being bounded polynomially in terms of
$|\sitesset|$ and $|\clientset|$ to keep running time
polynomial.

The reduction is based on an optimal fractional solution
$(\bfx^\ast,\bfy^\ast)$ of LP (\ref{eqn:fac_primal}). From
optimality of the solution, we can assume that
$\sum_{i\in\sitesset} x^\ast_{ij} = r_j$ for all
$j\in\clientset$.  As explained in Section~\ref{sec: the lp
  formulation}, we can assume that $(\bfx^\ast,\bfy^\ast)$
is complete, that is $x^\ast_{ij} > 0$ implies $x^\ast_{ij}
= y^\ast_i$ for all $i,j$.  We split this solution into two
parts, namely $(\bfx^\ast,\bfy^\ast) = (\hatbfx,\hatbfy)+
(\dotbfx,\dotbfy)$, where
%
\begin{align*}
\haty_i &= \floor{y_i^\ast}, \quad
			\hatx_{ij} = \floor{x_{ij}^\ast} \;\textrm{and}
			\\
\doty_i &= y_i^\ast - \floor{y_i^\ast}, \quad
 	\dotx_{ij} = x_{ij}^\ast -  \floor{x_{ij}^\ast}
\end{align*}
%
for all $i,j$. Now we construct two
FTFP instances $\hatcalI$ and $\dotcalI$ with the same
parameters as the original instance, except that the demand of each client $j$ is
$\hatr_j = \sum_{i\in\sitesset} \hatx_{ij}$ in instance $\hatcalI$ and
$\dotr_j = \sum_{i\in\sitesset} \dotx_{ij} = r_j - \hatr_j$ in instance $\dotcalI$. 
It is obvious that if we have integral solutions to both $\hatcalI$
and $\dotcalI$ then, when added together, they form an integral
solution to the original instance.  Moreover, we have the
following lemma.

%%%%%%%%%%

\begin{lemma}\label{lem: polynomial demands partition}
{\rm (i)}
  $(\hatbfx, \hatbfy)$ is a feasible integral solution to
  instance $\hatcalI$.

\noindent
{\rm (ii)}
  $(\dotbfx, \dotbfy)$ is a feasible fractional
  solution to instance $\dotcalI$.

\noindent
{\rm (iii)}
$\dotr_j\leq |\sitesset|$ for every client $j$.

\end{lemma}

\begin{proof}
(i) For feasibility, we need to verify that the constraints of LP~(\ref{eqn:fac_primal})
are satisfied. Directly from the definition, we have $\hatr_j = \sum_{i\in\sitesset} \hatx_{ij}$.
For any $i$ and $j$, by the feasibility of $(\bfx^\ast,\bfy^\ast)$ we have
$\hatx_{ij} = \floor{x_{ij}^\ast} \le \floor{y^\ast_i} = \haty_i$.

(ii) From the definition, we have  $\dotr_j = \sum_{i\in\sitesset} \dotx_{ij}$.
It remains to show that $\doty_i \geq \dotx_{ij}$ for all $i,j$. 
If $x_{ij}^\ast=0$, then $\dotx_{ij}=0$ and we are done. 
Otherwise, by completeness, we have $x_{ij}^\ast=y_i^\ast$. 
Then  $\doty_i = y_i^\ast - \floor{y_i^\ast} = x_{ij}^\ast - \floor{x_{ij}^\ast} =\dotx_{ij}$. 

(iii) From the definition of $\dotx_{ij}$ we have
  $\dotx_{ij} < 1$.  Then the bound follows from the definition of $\dotr_j$.
\end{proof}

Notice that our construction relies on the completeness assumption; in fact, it is
easy to give an example where $(\dotbfx, \dotbfy)$ would not be feasible if we
used a non-complete optimal solution $(\bfx^\ast,\bfy^\ast)$.
Note also that the solutions $(\hatbfx,\hatbfy)$ and $(\dotbfx, \dotbfy)$ are in fact
optimal for their corresponding instances, for if a better solution to $\hatcalI$ or
$\dotcalI$ existed, it could
give us a solution to $\calI$ with a smaller objective value.

%%%%%%%%%%%%%%%

\begin{theorem}\label{thm: reduction to polynomial}
  Suppose that there is a polynomial-time algorithm $\calA$
  that, for any instance of {\FTFP} with maximum demand
  bounded by $|\sitesset|$, computes an integral solution
  that approximates the fractional optimum of this instance
  within factor $\rho\geq 1$.  Then there is a
  $\rho$-approximation algorithm $\calA'$ for {\FTFP}.
\end{theorem}

%%%%%%%%%%%%%%%

\begin{proof}
  Given an {\FTFP} instance with arbitrary demands, Algorithm~$\calA'$ works
as follows: it solves the LP~(\ref{eqn:fac_primal}) to obtain a
  fractional optimal solution $(\bfx^\ast,\bfy^\ast)$, then constructs
  instances $\hatcalI$ and $\dotcalI$ described above,  applies
  algorithm~$\calA$ to $\dotcalI$, and finally combines (by adding
  the values) the integral solution $(\hatbfx, \hatbfy)$ of
  $\hatcalI$ and the integral solution of $\dotcalI$ produced
  by $\calA$. This clearly produces a feasible integral
  solution for the original instance $\calI$.
The solution produced by $\calA$ has cost at most
$\rho\cdot\cost(\dotbfx,\dotbfy)$, because $(\dotbfx,\dotbfy)$
is feasible for $\dotcalI$. Thus the cost of $\calA'$ is at most
% 
 \begin{align*}
 \cost(\hatbfx, \hatbfy) + \rho\cdot\cost(\dotbfx,\dotbfy)
	\le
 \rho(\cost(\hatbfx, \hatbfy) + \cost(\dotbfx,\dotbfy))
		= \rho\cdot\LP^\ast \le \rho\cdot\OPT,
  \end{align*}
%
where the first inequality follows from $\rho\geq 1$. This completes
the proof.
\end{proof}

%%%%%%%%%%%%%%%%%%%%%%%%%%%%%%%%%%%%%%%%%%%%%%%%%%%%%%%%%%%%%%%%%%%%%%%%%%%%%%%
%%% ADAPTIVE PARTITION  %%%%%%%%%%%%%%%%%%%%%%%%%%%%%%%%%%%%%%%%%%%%%%%%%%%%%%%
%%%%%%%%%%%%%%%%%%%%%%%%%%%%%%%%%%%%%%%%%%%%%%%%%%%%%%%%%%%%%%%%%%%%%%%%%%%%%%%

\section{Adaptive Partitioning}
\label{sec: adaptive partitioning}

In this section we develop our second technique, which we
call \emph{adaptive partitioning}. Given an FTFP instance
and an optimal fractional solution $(\bfx^\ast, \bfy^\ast)$
to LP~(\ref{eqn:fac_primal}), we split each client $j$ into
$r_j$ individual \emph{unit demand points} (or just
\emph{demands}), and we split each site $i$ into no more
than $|\sitesset|+R|\clientset|$ \emph{facility points} (or
\emph{facilities}), where $R=\max_{j\in\clientset}r_j$. We
denote the demand set by $\demandset$ and the facility set
by $\facilityset$, respectively.  We will also partition
$(\bfx^\ast,\bfy^\ast)$ into a fractional solution
$(\barbfx,\barbfy)$ for the split instance.  We will
typically use symbols $\nu$ and $\mu$ to index demands and
facilities respectively, that is $\barbfx =
(\barx_{\mu\nu})$ and $\barbfy = (\bary_{\mu})$.  As before,
the \emph{neighborhood of a demand} $\nu$ is
$\wbarN(\nu)=\braced{\mu\in\facilityset \suchthat
  \barx_{\mu\nu}>0}$.  We will use notation $\nu\in j$ to
mean that $\nu$ is a demand of client $j$; similarly,
$\mu\in i$ means that facility $\mu$ is on site
$i$. Different demands of the same client (that is,
$\nu,\nu'\in j$) are called \emph{siblings}.  Further, we
use the convention that $f_\mu = f_i$ for $\mu\in i$,
$\alpha_\nu^\ast = \alpha_j^\ast$ for $\nu\in j$ and
$d_{\mu\nu} = d_{\mu j} = d_{ij}$ for $\mu\in i$ and $\nu\in
j$.  We define $\concost_{\nu}
=\sum_{\mu\in\wbarN(\nu)}d_{\mu\nu}\barx_{\mu\nu} =
\sum_{\mu\in\facilityset}d_{\mu\nu}\barx_{\mu\nu}$.  Similar
to $C^\ast_j$, one can think of $\concost_{\nu}$ as the
average connection cost of demand $\nu$, if we chose a
connection to facility $\mu$ with probability
$\barx_{\mu\nu}$. In our partitioned fractional solution we
guarantee for every $\nu$ that $\sum_{\mu\in\facilityset}
\barx_{\mu\nu}=1$.

Some demands in $\demandset$ will be designated as
\emph{primary demands} and the set of primary demands will
be denoted by $P$. In addition, we will use the overlap
structure between demand neighborhoods to define a mapping
that assigns each demand $\nu\in\demandset$ to some primary
demand $\kappa\in P$. As shown in the rounding algorithms in
later sections, for each primary demand we guarantee exactly
one open facility in its neighborhood, while for a
non-primary demand, there is constant probability that none
of its neighbors open. In this case we estimate its
connection cost by the distance to the facility opened in
its assigned primary demand's neighborhood. For this reason
the connection cost of a primary demand must be ``small''
compared to the non-primary demands assigned to it. We also
need sibling demands assigned to different primary demands
for fault-tolerant requirements. Specifically, this
partitioning will be constructed to satisfy a number of
properties that are detailed below.
%

OLD VERSION ......

\begin{enumerate}
      \renewcommand{\theenumi}{P\arabic{enumi}}
      \renewcommand{\labelenumi}{(\theenumi)}
      \renewcommand{\theenumii}{(\alph{enumii})}
      \renewcommand{\labelenumii}{\theenumii}

\item Vectors $\barbfx,\barbfy$ satisfy the following properties:
%
  \begin{enumerate}
	\item \label{P1:one} 
          $\sum_{\mu\in\facilityset} \barx_{\mu\nu} = 1$ for each demand $\nu\in\demandset$. 
	\item \label{P1:xij} $\sum_{\mu\in i, \nu\in j} \barx_{\mu\nu}
          = x^\ast_{ij}$ for each site $i\in\sitesset$ and client $j\in\clientset$.
	\item \label{P1:yi}
          $\sum_{\mu\in i} \bary_{\mu} = y^\ast_i$ for each site $i\in\sitesset$.
	\item \label{P1:complete}
          $(\barbfx,\barbfy)$ is complete, that is $\barx_{\mu\nu}\neq 0$ implies
				$\barx_{\mu\nu} = \bary_{\mu}$, for all 
						$\mu\in\facilityset, \nu\in\demandset$.
	\item \label{P1:siblings disjoint}
		 If $\nu,\nu'\in\demandset$ are different siblings
			then $\wbarN(\nu)\cap \wbarN(\nu') = \emptyset$.
  \end{enumerate}
%
  The first three conditions say that $(\barbfx,\barbfy)$ is
  a partition of $(\bfx^\ast,\bfy^\ast)$, with unit-value
  demands. 

\item Primary demands satisfy the following properties:
%
\begin{enumerate}
%
\item \label{P2:yi} For each site $i\in\sitesset$, $ \sum_{\mu\in i}\sum_{\kappa\in P}\barx_{\mu\kappa} \leq y_i^\ast$.
	(Recall that $\barx_{\mu\kappa}\neq 0$ iff $\mu\in \wbarN(\kappa)$.)
%
	\item\label{P2:disjoint}  For any two different primary demands $\kappa,\kappa'\in P$ we have
	$\wbarN(\kappa)\cap \wbarN(\kappa') = \emptyset$.
%
	\item \label{P2:assign} Each demand $\nu\in\demandset$ is assigned
        to one primary demand $\kappa\in P$ with  $\wbarN(\nu) \cap \wbarN(\kappa) \neq \emptyset$.
%
      \item \label{P2:diff} 
Let $\nu,\nu'$ be different siblings and let $\kappa$ be the primary demand that $\nu$ is assigned to.
Then $\wbarN(\nu')\cap \wbarN(\kappa) = \emptyset$. In particular, by Property~\ref{P2:assign},
this implies that different sibling demands are assigned to different primary demands.
%
      \item \label{P2:cost} If a demand $\nu\in\demandset$
        is assigned to a primary demand $\kappa\in P$, then
        $\concost_{\nu}+\alpha_{\nu}^\ast \geq
        \concost_{\kappa}+\alpha_{\kappa}^\ast$.
\end{enumerate}

\end{enumerate}



NEW VERSION ......

\begin{description}
	
      \renewcommand{\theenumii}{(\alph{enumii})}
      \renewcommand{\labelenumii}{\theenumii}

\item{(PS)} \emph{Partitioned solution}.
Vector $(\barbfx,\barbfy)$ is a partition of $(\bfx^\ast,\bfy^\ast)$, with unit-value
  demands, that is:

	\begin{enumerate}
		%
	\item \label{PS1:one} 
          $\sum_{\mu\in\facilityset} \barx_{\mu\nu} = 1$ for each demand $\nu\in\demandset$. 
		%
	\item \label{PS1:xij} $\sum_{\mu\in i, \nu\in j} \barx_{\mu\nu}
          = x^\ast_{ij}$ for each site $i\in\sitesset$ and client $j\in\clientset$.
		%
	\item \label{PS1:yi}
          $\sum_{\mu\in i} \bary_{\mu} = y^\ast_i$ for each site $i\in\sitesset$.
		%
	\end{enumerate}
		
\item{(CO)} \emph{Completeness.}
	Solution   $(\barbfx,\barbfy)$ is complete, that is $\barx_{\mu\nu}\neq 0$ implies
				$\barx_{\mu\nu} = \bary_{\mu}$, for all $\mu\in\facilityset, \nu\in\demandset$.

\item{(PD)} \emph{Primary demands.}
	Primary demands satisfy the following conditions:

	\begin{enumerate}
		
	\item\label{PD:disjoint}  For any two different primary demands $\kappa,\kappa'\in P$ we have
				$\wbarN(\kappa)\cap \wbarN(\kappa') = \emptyset$.

	\item \label{PD:yi} For each site $i\in\sitesset$, 
		$ \sum_{\mu\in i}\sum_{\kappa\in P}\barx_{\mu\kappa} \leq y_i^\ast$.
		(Recall that $\barx_{\mu\kappa}\neq 0$ iff $\mu\in \wbarN(\kappa)$.)
		
	\item \label{PD:assign} Each demand $\nu\in\demandset$ is assigned
        to one primary demand $\kappa\in P$ such that

  			\begin{enumerate}
	
				\item \label{PD:assign:overlap} $\wbarN(\nu) \cap \wbarN(\kappa) \neq \emptyset$, and
				%
				\item \label{PD:assign:cost} $\concost_{\nu}+\alpha_{\nu}^\ast \geq
        			\concost_{\kappa}+\alpha_{\kappa}^\ast$.

			\end{enumerate}

	\end{enumerate}
	
\item{(SI)} \emph{Siblings}. For any pair $\nu,\nu'$ of different siblings we have
  \begin{enumerate}

	\item \label{SI:siblings disjoint}
		  $\wbarN(\nu)\cap \wbarN(\nu') = \emptyset$.
		
	\item \label{SI:primary disjoint} If $\nu$ is assigned to a primary demand $\kappa$ then
 		$\wbarN(\nu')\cap \wbarN(\kappa) = \emptyset$. In particular, by Property~PD(\ref{PD:assign:overlap}),
		this implies that different sibling demands are assigned to different primary demands.

	\end{enumerate}
	
\end{description}

As we shall demonstrate in later sections, these properties
allow us to extend known UFL rounding algorithms to obtain
an integral solution to our FTFP problem with a matching
approximation ratio. Our partitioning is ``adaptive" in the
sense that it is constructed one demand at a time, and the
connection values for the demands of a client depend on the choice of
earlier demands, of this or other clients, and their connection values.

%%%%%%%%%%%%%%%%

\paragraph{Implementation of Adaptive Partitioning}
We now describe an algorithm for partitioning the instance
and the fractional solution so that the properties (PS),
(CO), (PD), and (SI) are satisfied.  Recall that
$\facilityset$ and $\demandset$, respectively, denote the
sets of facilities and demands that will be created in this
stage, and $(\barbfx,\barbfy)$ is the partitioned solution
to be computed. At a high level, for each site $i$ and
client $j$ we would split the $x_{ij}^\ast$ value and give
each piece to exactly one of its $r_j$ demands. In the end
we require each demand $\nu$ has its connection values with
all facilities sum up to $1$.

The adaptive partitioning algorithm consists of two phases:
Phase 1 is called the partition phase and Phase 2 is called
the augmenting phase. Phase 1 is done in iterations, where
in each iteration we find the best client and create a new
demand $\nu$ out of it. This demand either becomes a primary
demand itself, or it is assigned to some existing primary
demand. We call a client $j$ \emph{exhausted} when all its
$r_j$ demands have been created and assigned to some primary
demands. Phase 1 completes when all clients are
exhausted. We then do an augmenting phase 2 to ensure every
demand has a neighborhood with total connection values equal
to $1$.

For each site $i$ we will initially create one ``big"
facility $\mu$ with initial value $\bary_\mu = y^\ast_i$.
While we partition the instance, creating new demands and
connections, this facility may end up being split into more
facilities to preserve completeness of the fractional
solution. Also, we will gradually decrease the fractional
connection vector for each client $j$, to account for the
demands created for $j$ and their connection values.  These
decreased connection values will be stored in an auxiliary
vector $\tildebfx$. The intuition is that $\tildebfx$
represents the part of $\bfx^\ast$ that still has not been
partitioned into demands and future demands can use
$\tildebfx$ for their connections. For technical reasons,
$\tildebfx$ will be indexed by facilities (rather than
sites) and clients, that is $\tildebfx = (\tildex_{\mu j})$.
At the beginning, we set $\tildex_{\mu j}\assign
x_{ij}^\ast$ for each $j\in\clientset$, where $\mu\in i$ is
the single facility created initially at site $i$.  At each
step, whenever we create a new demand $\nu$ for a client
$j$, we will define its values $\barx_{\mu\nu}$ and
appropriately reduce the values $\tildex_{\mu j}$, for all
facilities $\mu$. We will deal with two types of
neighborhoods, with respect to $\tildebfx$ and $\barbfx$,
that is $\wtildeN(j)=\{\mu\in\facilityset
\suchthat\tildex_{\mu j} > 0\}$ for $i\in\clientset$ and
$\wbarN(\nu)=\{\mu\in\facilityset \suchthat \barx_{\mu\nu}
>0\}$ for $\nu\in\demandset$.  During this process we
preserve the completeness of the fractional solutions
$\tildebfx$ and $\barbfx$. More precisely, the following
properties will hold for every facility $\mu$ after every
iteration:
%
\begin{description}
	
	\item{(c1)} For each demand $\nu$ either $\barx_{\mu\nu}=0$ or
			$\barx_{\mu\nu}=\bary_{\mu}$. This is the same
      condition as condition (CO), yet we repeat here as
      (c1) needs to hold after every iteration, while
      condition (CO) only applies to the final partitioned
      fractional solution $(\barbfx, \barbfy)$.

	\item{(c2)} For each client $j$,
			either $\tildex_{\mu j}=0$ or $\tildex_{\mu j}=\bary_{\mu}$.
			
\end{description}

A full description of the algorithm is given in
Pseudocode~\ref{alg:lpr2}.  Initially, the set $U$ of
non-exhausted clients contains all clients, the set
$\demandset$ of demands is empty, the set $\facilityset$ of
facilities consists of one facility $\mu$ on each site $i$
with $\bary_\mu = y^\ast_i$, and the set $P$ of primary
demands is empty (Lines 1--4).  In one iteration of the
while loop (Line 5), in Lines~6--8, for each client $j$ we
compute a quantity called $\tcc(j)$ (tentative connection
cost), that represents the average distance from $j$ to the
set $\wtildeN_1(j)$ of the nearest facilities $\mu$ whose
total connection value to $j$ (the sum of $\tildex_{\mu
  j}$'s) equals $1$.  This set is computed by Procedure
$\NearestUnitChunk()$ (see Pseudocode~\ref{alg:helper},
Lines~1--9), which adds facilities to $\wtildeN_1(j)$ in
order of increasing distance, until the total connection
value is exactly $1$. (The procedure actually uses the
$\bary_\mu$ values, which are equal to the connection values
by completeness.)  This may require splitting the last added
facility and adjusting the connection values so that
conditions (c1) and (c2) are preserved.

%%%%%%%%%%%

\begin{algorithm}[ht]
  \caption{Algorithm: Adaptive Partitioning}
  \label{alg:lpr2}
  \begin{algorithmic}[1]
    \Require $\sitesset$, $\clientset$, $(\bfx^\ast,\bfy^\ast)$
    \Ensure  $\facilityset$,  $\demandset$, $(\barbfx, \barbfy)$ 
    \Comment Unspecified $\barx_{\mu \nu}$'s and $\tildex_{\mu j}$'s are assumed to be $0$

    \State $\tildebfr \assign \bfr, U\assign \clientset, \facilityset\assign \emptyset,
    \demandset\assign \emptyset, P\assign \emptyset$
    \Comment{Phase 1}

    \For{each site $i\in\sitesset$} 
    \State create a facility $\mu$ at $i$ and add $\mu$ to $\facilityset$
    \State $\bary_\mu \assign y_i^\ast$ and $\tildex_{\mu j}\assign
    x_{ij}^\ast$ for each $j\in\clientset$ 
    \EndFor

    \While{$U\neq \emptyset$}
    \For{each $j\in U$}
    \State $\wtildeN_1(j) \assign {\NearestUnitChunk}(j, \facilityset, \tildebfx, \barbfx, \barbfy)$ \Comment see Pseudocode~\ref{alg:helper}
    \State $\tcc(j)\assign \sum_{\mu\in \wtildeN_1(j)} d_{{\mu}j}\cdot \tildex_{\mu j}$
    \EndFor
 
    \State $p \assign {\argmin}_{j\in U}\{ \tcc(j)+\alpha_j^\ast \}$
    \State create a new demand $\nu$ for client $p$

    \If{$\wtildeN_1 (p)\cap \wbarN(\kappa) \neq \emptyset$
      for some primary demand $\kappa\in P$}
    \State assign $\nu$ to $\kappa$
    \State $\barx_{\mu \nu}\assign \tildex_{\mu p}$ and $\tildex_{\mu p}\assign 0$ for each $\mu \in \wtildeN(p) \cap \wbarN(\kappa)$
    \Else 
    \State make $\nu$ primary, $P \assign P \cup \{\nu\}$, assign $\nu$ to itself
    \State set $\barx_{\mu\nu} \assign \tildex_{\mu p}$ and $\tildex_{\mu p}\assign 0$ for each $\mu\in \wtildeN_1(p)$

    \EndIf
    \State $\demandset\assign \demandset\cup \{\nu\},
    \tilder_p \assign \tilder_p -1$
	\State \textbf{if} {$\tilder_p=0$} \textbf{then} $U\assign U \setminus \{p\}$
    \EndWhile

    \For{each client $j\in\clientset$} \Comment{Phase 2}
    \For{each demand $\nu\in j$}    \Comment{each client $j$ has $r_j$ demands}
    \State \textbf{if} $\sum_{\mu\in \wbarN(\nu)}\barx_{\mu\nu}<1$
    \textbf{then} $\AugmentToUnit(\nu, j, \facilityset, \tildebfx, \barbfx, \barbfy)$ \Comment see Pseudocode~\ref{alg:helper}
    \EndFor
    \EndFor
  \end{algorithmic}
\end{algorithm}
%%%%%%%%%%%%%%%%%%%%%%%%%%%%
%% subroutine: NearestUnitChunk and AugmentToUnit
%%%%%%%%%%%%%%%%%%%%%%%%%%%%%
\begin{algorithm}[ht]
  \caption{Helper functions used in Pseudocode~\ref{alg:lpr2}}
  \label{alg:helper}
  \begin{algorithmic}[1]
    \Function{\NearestUnitChunk}{$j, \facilityset, \tildebfx, \barbfx,\barbfy$}		
						\Comment upon return, $\sum_{\mu\in\wtildeN_1(j)} \tildex_{\mu j} = 1$
    \State Let $\wtildeN(j) = \{\mu_1,...,\mu_{n_j}\}$ where $d_{\mu_1 j} \leq d_{\mu_2 j} \leq \ldots \leq d_{\mu_{n_j j}}$
    \State Let $l$ be such that $\sum_{k=1}^{l} \bary_{\mu_k} \geq 1$ and $\sum_{k=1}^{l -1} \bary_{\mu_{k}} < 1$
    \State Create a new facility $\sigma$ at the same site as $\mu_l$ and add it to $\facilityset$
			\Comment split $\mu_l$
    \State Set $\bary_{\sigma}\assign \sum_{k=1}^{l} \bary_{\mu_{k}}-1$
					and $\bary_{\mu_l} \assign \bary_{\mu_l} - \bary_{\sigma}$
    \State For each $\nu\in\demandset$ with $\barx_{\mu_{l}\nu}>0$
 			set $\barx_{\mu_{l}\nu} \assign \bary_{\mu_l}$ and $\barx_{\sigma \nu} \assign \bary_{\sigma}$
    \State For each $j'\in\clientset$ with $\tildex_{\mu_{l} j'}>0$ (including $j$)
			set $\tildex_{\mu_l j'} \assign \bary_{\mu_l}$ and $\tildex_{\sigma j'} \assign \bary_\sigma$
	\State (All other new connection values are set to $0$)
    \State \Return $\wtildeN_1(j) = \{\mu_{1},\ldots,\mu_{l-1}, \mu_{l}\}$    				
    \EndFunction

    \Function{\AugmentToUnit}{$\nu, j, \facilityset, \tildebfx, \barbfx, \barbfy$}
    					\Comment $\nu$ is a demand of client $j$
    \While{$\sum_{\mu\in \facilityset} \barx_{\mu\nu} <1$}
    					\Comment upon return, $\sum_{\mu\in\wbarN(\nu)} \barx_{\mu\nu} = 1$
    \State Let $\eta$ be any facility such that $\tildex_{\eta j} > 0$
    \If{$1-\sum_{\mu\in \facilityset} \barx_{\mu\nu} \geq \tildex_{\eta j}$}
    \State $\barx_{\eta\nu} \assign \tildex_{\eta j}, \tildex_{\eta j} \assign 0$
    \Else
    \State Create a new facility $\sigma$ at the same site as $\eta$ and add it to $\facilityset$
    					\Comment split $\eta$
    \State Let $\bary_\sigma \assign 1-\sum_{\mu\in \facilityset} \barx_{\mu\nu}, \bary_{\mu^\ast} \assign \bary_{\eta} - \bary_{\sigma}$
    \State Set $\barx_{\sigma\nu}\assign \bary_{\sigma},\; \barx_{\eta \nu} \assign  0,\; \tildex_{\eta j} \assign \bary_{\eta}, \; \tildex_{\sigma j} \assign 0$
    \State For each $\nu' \neq \nu$ with $\barx_{\eta \nu'}>0$ set $\barx_{\eta \nu'} \assign \bary_{\eta},\; \barx_{\sigma \nu'} \assign \bary_{\sigma}$
    \State For each $j' \neq j$ with $\tildex_{\eta j'}>0$ set $\tildex_{\eta j'} \assign \bary_{\eta}, \tildex_{\sigma j'} \assign \bary_{\sigma}$
    \EndIf
    \EndWhile
    \EndFunction
  \end{algorithmic}
\end{algorithm}

%%%%%%%%%%%%%%


The next step is to pick a client $p$ with minimum
$\tcc(p)+\alpha_p^\ast$ and create a demand $\nu$ for $p$
(Lines~9--10). If $\wtildeN_1(p)$ overlaps the neighborhood
of some existing primary demand $\kappa$ (if there are
multiple such $\kappa$'s, pick any of them), we assign $\nu$
to $\kappa$, and $\nu$ acquires all the connection values
$\tildex_{\mu p}$ between client $p$ and facility $\mu$ in
$\wtildeN(p)\cap \wbarN(\kappa)$ (Lines~11--13). Note that
although we check for overlap with $\wtildeN_1(p)$, we then
move all facilities in the intersection with $\wtildeN(p)$,
a bigger set, into $\wbarN(\nu)$.  The other case is when
$\wtildeN_1(p)$ is disjoint from the neighborhoods of all
existing primary demands. Then, in Lines~15--16, $\nu$
becomes itself a primary demand and we assign $\nu$ to
itself. It also inherits the connection values to all
facilities $\mu\in\wtildeN_1(p)$ from $p$ (recall that
$\tildex_{\mu p} = \bary_{\mu}$), with all other
$\barx_{\mu\nu}$ values set to $0$.

At this point all primary demands satisfy
Property~PS(\ref{PS1:one}), but this may not be true for
non-primary demands. For those demands we still may need to
adjust the $\barx_{\mu\nu}$ values so that $\connsum(\nu)
\stackrel{\mathrm{def}}{=}
\sum_{\mu\in\facilityset}\barx_{\mu \nu}$ equals $1$. This
is accomplished by Procedure $\AugmentToUnit()$ (definition
in Pseudocode~\ref{alg:helper}, Lines~10--20) that allocates
to $\nu\in j$ some of the remaining connection values
$\tildex_{\mu j}$ of client $j$ (Lines 19--21).
$\AugmentToUnit()$ will repeatedly pick any $\mu$ with
$\tildex_{\mu j} >0$.  If $\tildex_{\mu j} \leq
1-\connsum(\nu)$, then the connection value $\tildex_{\mu
  j}$ is reassigned to $\nu$. Otherwise, $\tildex_{\mu j} >
1-\connsum(\nu)$, in which case we split $\mu$ so that
connecting $\nu$ to one of the created copies of $\mu$ will
make $\connsum(\nu)$ equal $1$, and we'll be done.

\smallskip

Notice that we start with $|\sitesset|$ facilities and in each
iteration each client causes at most one split. We have a total of no more
than $R|\clientset|$ iterations as in each iteration we create
one demand. (Recall that $R = \max_jr_j$.)
So the total number of facilities we created
will be at most $|\sitesset|+ R|\clientset|^2$, which is
polynomial in $|\sitesset|+|\clientset|$ due to our earlier bound on $R$.

%%%%%%

\medskip

\emparagraph{Correctness.}  We now show that all the
required properties (PS), (CO), (PD) and (SI) are satisfied
by the above construction.

Properties~(PS) and (CO) follow directly from the
algorithm. (CO) is implied by the completeness condition
(c1) that the algorithm maintains after each
iteration. PS(\ref{PS1:one}) follows from the result of
calling Procedure~$\AugmentToUnit()$ in Line~21. To see
PS(\ref{PS1:xij}), at each step the algorithm maintains an
invariant that, for every $i\in\sitesset$ and
$j\in\clientset$, we have $\sum_{\mu\in i}\sum_{\nu \in j}
\barx_{\mu \nu} + \sum_{\mu\in i} \tildex_{\mu j} =
x_{ij}^\ast$. In the end, we will create $r_j$ demands for
each client $j$, with each demand $\nu\in j$ satisfying
PS(\ref{PS1:one}), and thus $\sum_{\nu\in
  j}\sum_{\mu\in\facilityset}\barx_{\mu\nu}=r_j$.  This
implies that $\tildex_{\mu j}=0$ for every facility
$\mu\in\facilityset$, and PS(\ref{PS1:xij}) follows.
PS(\ref{PS1:yi}) follows from that every time we split a
facility $\mu$ into $\mu'$ and $\mu''$, the sum of
$\bary_{\mu'}$ and $\bary_{\mu''}$ is equal to the old value
$\bary_{\mu}$.

Now we deal with properties in group
(PD). PD(\ref{PD:disjoint}) follows directly from the
algorithm since we only create a new primary demand if its
neighborhood is disjoint from all existing primary demands
and its neighborhood is fixed to $\wtildeN_1(p)$
immediately. PD(\ref{PD:yi}) follows from the fact that
neighborhoods of primary demands are disjoint and property
(CO). One way to see this is that the double summation is
the same as summing over a subset of facilities split from
site $i$. PD(\ref{PD:assign:overlap}) follows from the way
assignment is done by the algorithm.  When demand $\nu$ of
client $p$ is assigned to a primary demand $\kappa$ in
Lines~11--13 of Pseudocode~\ref{alg:lpr2}, we move all
facilities in $\wtildeN(p)\cap \wbarN(\kappa)$ (the
intersection is nonempty) into $\wbarN(\nu)$, and we never
remove a facility from $\wbarN(\nu)$.  We postpone the proof
of PD(\ref{PD:assign:cost}) to Lemma~\ref{lem:
  PD:assign:cost holds}.

Finally we argue properties in group (SI)
hold. SI(\ref{SI:siblings disjoint}) is easy, since each
facility $\mu$ is added to at most one sibling's
neighborhood by setting $\barx_{\mu\nu}$ to $\bary_\mu$
while other siblings $\nu'$ have $\barx_{\mu\nu'}=0$. Note
that right after a demand $\nu\in p$ is created, its
neighborhood is disjoint from the neighborhood of $p$, that
is $\wbarN(\nu)\cap \wtildeN(p) = \emptyset$, by
Lines~11--13 of the algorithm. Thus all demands of $p$
created later will have neighborhoods disjoint from
$\wbarN(\nu)$. Furthermore, Procedure~$\AugmentToUnit()$
preserves this property, because when it adds an existing
facility to $\wbarN(\nu)$ then it removes it from
$\wtildeN(p)$, and in case of splitting, one resulting
facility is added to $\wbarN(\nu)$ and the other to
$\wtildeN(p)$. For the proof of SI(\ref{SI:primary
  disjoint}), we defer to Lemma~\ref{lem: property SI:primary
  disjoint holds}.

It remains to show Properties~PD(\ref{PD:assign:cost}) and
SI(\ref{SI:primary disjoint}). We show them in the lemmas
below, thus completing the description of our adaptive
partition process.

%%%%%%%%%%%

%%\begin{lemma}\label{lem: property P2:yi holds}
%%Property~\ref{P2:yi} holds after the Adaptive Partitioning stage.
%%\end{lemma}
%%
%%\begin{proof}
%%Let $Q$ denote the set of facilities $\mu\in i$ that are in a neighborhood of some primary demand.
%%Then, from the completeness of $(\barbfx,\barbfy)$ (\ref{P1:complete}),
%%the disjointness of primary demands' neighborhoods (\ref{P2:disjoint}),
%%and Property~\ref{P1:yi}, we have
%%$\sum_{\mu\in i} \sum_{\kappa \in P} \barx_{\mu\kappa} = \sum_{\mu \in Q} \bary_{\mu}
%% 	\le \sum_{\mu \in i} \bary_{\mu} = y_i^\ast$.
%%\end{proof}

%%%%%%%
\begin{lemma}\label{lem: property SI:primary disjoint holds}
  Property~SI(\ref{SI:primary disjoint}) holds after the
  Adaptive Partitioning stage.
\end{lemma}
\begin{proof}
  Let $\nu_1,\ldots,\nu_{r_j}$ be the demands of a client
  $j\in\clientset$, listed in the order of creation, and,
  for each $q=1,2,\ldots,r_j$, denote by $\kappa_q$ the
  primary demand that $\nu_q$ is assigned to.  We claim
  that, for every $q=1,\ldots,r_j$, at the time when $\nu_q$
  is about to be created, we have
%
\begin{eqnarray*}
\wtildeN(j) \cap \bigcup_{s=1}^{q-1} \wbarN(\nu_s) \;=\;
\emptyset\ (*)\quad\textrm{and} \quad
\wtildeN(j) \cap \bigcup_{s=1}^{q-1} \wbarN(\kappa_s) &=&
\emptyset\ (**).
\end{eqnarray*}
%
Indeed, this follows by an easy induction on $q$, because
when we create $\nu_q$, the neighborhood $\wbarN(\nu_q)$
inherits all facilities in $\wtildeN(j)\cap
\wbarN(\kappa_q)$, which are then removed from
$\wtildeN(j)$. (Equation (*) above was in fact already
indirectly shown when we proved Property~SI(\ref{SI:siblings
  disjoint}).) This implies that, for $s < q$,
$\wbarN(\nu_q)$ is disjoint from $\wbarN(\kappa_s)$ and
$\wbarN(\kappa_q)$ is disjoint from $\wbarN(\nu_s)$. In the
augmenting phase 2, when we add facilities using
\AugmentToUnit() in Line~19--21 of
Pseudocode~\ref{alg:lpr2}, the neighborhoods of all primary
demands, including $\wbarN(\kappa_1), \ldots,
\wbarN(\kappa_{r_j})$, have already been fixed. Further,
these added facilities in phase 2 do not appear in any
neighborhood $\wbarN(\kappa_1), \ldots,
\wbarN(\kappa_{r_j})$ and each of these facilities is added
to the neighborhood of exactly one sibling.  Therefore all
the disjointness conditions are preserved after the augment
phase, proving Property~SI(\ref{SI:primary disjoint}).
\end{proof}

%%%%%%%

We need one lemma before proving our last property
PD(\ref{PD:assign:cost}).  For a client $j$ and a demand
$\nu$, we use notation $\tcc_{\nu}(j)$ for the value of
$\tcc(j)$ at the time when $\nu$ was created. (It is not
necessary that $\nu\in j$ but we assume that $j$ is not
exhausted at that time.)


\begin{lemma}\label{lem: tcc optimal}
  Let $\eta$ and $\nu$ be two demands, with $\eta$ created
  earlier than $\nu$, and let $j\in\clientset$ be a client
  that is not exhausted when $\nu$ is created. Then we have
\begin{description}
	\item{(a)} $\tcc_\eta(j) \le \tcc_{\nu}(j)$, and 
	\item{(b)} if $\nu\in j$ then $\tcc_\eta(j) \le \concost_{\nu}$.
\end{description}
\end{lemma}

\begin{proof}
  We focus on the time when $\eta$ is about to be created,
  after the call to $\NearestUnitChunk()$ in
  Pseudocode~\ref{alg:lpr2}, Line~7.  Let $\wtildeN(j) =
  \{\mu_1,...,\mu_q\}$ with all facilities $\mu_s$ ordered
  according to nondecreasing distance from $j$.  Consider
  the following linear program:
%
\begin{alignat*}{1}
	\textrm{minimize} \quad & \sum_s d_{\mu_s j}z_s
			\\
	\textrm{subject to} \quad & \sum_s z_s  \ge 1
			\\
 	0 &\le z_s \le \tildex_{\mu_s j} \quad \textrm{for all}\ s
\end{alignat*}
%
  This is a fractional
  minimum knapsack covering problem (with knapsack size equal $1$) and its fractional
  solution is the greedy solution, which is exactly
  $\tcc_\eta(j)$.  On the other hand, we claim that
  $\tcc_\nu(j)$ and $\concost_{\nu}$ both represent feasible
  solutions to this linear program. Indeed, each of these
  quantities involves some later values $\tildex_{\mu j}$,
  where $\mu$ could be one of the facilities $\mu_s$ or a
  new facility resulted from splitting. For each distance
  $d_{\mu_s j}$, however, the sum of all $\tildex_{\mu j}$,
  for all facilities that were split from $\mu_s$, cannot exceed
 the value $\tildex_{\mu_s j}$ at the time when
  $\eta$ was created, because splitting facilities preserves this sum and
 creating new demands for $j$ can only decrease it.
Therefore both quantities
  $\tcc_\nu(j)$ and $\concost_{\nu}$ correspond to some
  choice of the $z_s$ variables (adding up to $1$), and the
  lemma follows.
\end{proof}

%%%%%%%

\begin{lemma}\label{lem: PD:assign:cost holds}
Property~PD(\ref{PD:assign:cost}) holds after the Adaptive Partitioning stage.
\end{lemma}

\begin{proof}
Suppose that demand $\nu\in j$ is assigned to some primary demand $\kappa\in p$.
Then
%
\begin{eqnarray*}
 \concost_{\kappa} + \alpha_{\kappa}^\ast \;=\; \tcc_\kappa(p) + \alpha^\ast_p
 					\;\le\; \tcc_\kappa(j) + \alpha^\ast_j   
					\;\le\; \concost_{\nu} + \alpha^\ast_\nu.
\end{eqnarray*}
%
We now justify this derivation. By definition we have
$\alpha_{\kappa}^\ast = \alpha^\ast_p$.  Further, by the
algorithm, if $\kappa$ is a primary demand of client $p$,
then $\concost_{\kappa}$ is equal to $\tcc(p)$ computed when
$\kappa$ is created, which is exactly $\tcc_\kappa(p)$. Thus
the first equation is true. The first inequality follows
from the choice of $p$ as the primary demand in Line~9
Pseudocode~\ref{alg:lpr2}. The last inequality holds
because $\alpha^\ast_j = \alpha^\ast_\nu$ (due to $\nu\in
j$), and because $\tcc_\kappa(j) \le \concost_{\nu}$, where
this inequality is tight for $\nu = \kappa$ (and $j=p$) and
it follows from Lemma~\ref{lem: tcc optimal} if
$\nu\neq\kappa$.
\end{proof}


%%%%%%%%%%%%%%%%%%%%%%%%%%%%%%%%%%%%%%%%%%%%%%%%%%%%%%%%%%%%%%%%%%%%%%%%%%%%%%%
%%% 3-APPROXIMATION ALGORITHM %%%%%%%%%%%%%%%%%%%%%%%%%%%%%%%%%%%%%%%%%%%%%%%%%
%%%%%%%%%%%%%%%%%%%%%%%%%%%%%%%%%%%%%%%%%%%%%%%%%%%%%%%%%%%%%%%%%%%%%%%%%%%%%%%

\section{Algorithm~{\EGUP} with Ratio $3$}
\label{sec: 3-approximation}

The algorithm we describe in this section achieves ratio
$3$. Although this is still quite far from our best ratio $1.575$ that
we derive later, we include this algorithm in the paper to show,
in a relatively simple setting, how the properties of our partitioned fractional
solution are used in rounding it to an integral solution.
The rounding approach we use here is an extension of the corresponding
method for UFL described by~\cite{gupta08}, that we now briefly review. 

We begin with a brief overview of the algorithm
in~\cite{gupta08}, rephrased in a way that is consistent
with our approach.  Recall that in UFL we have $r_j=1$ for
all $j\in\clientset$. Let $(\bfx^\ast,\bfy^\ast)$ and
$(\bfalpha^\ast,\bfbeta^\ast)$ be some fixed optimal optimal
solutions to LPs~(\ref{eqn:fac_primal}) and
(\ref{eqn:fac_dual}), respectively, where
$(\bfx^\ast,\bfy^\ast)$ is assumed to be complete. The
algorithm then proceeds in two phases. In the first phase we
select a set of clients that we call \emph{primary clients}
and each client will be assigned to one primary client. To
this end, in each iteration, among all yet non-assigned
clients, we select a client $p$ with minimum value
$C_p^\ast+\alpha_p^\ast$. (Recall that
$C_j^\ast=\sum_{i\in\sitesset} d_{ij}x_{ij}^\ast$ represents
the fractional connection cost of a client $j$ in the
optimal solution.) Let $N(p)$ denote the neighborhood of
client $p$, defined as the set of facilities $i$ with
$x_{ip}^\ast>0$.  If $N(p)$ overlaps some $N(p')$, where
$p'$ is some existing primary client, then we assign $p$ to
$p'$. Otherwise we make $p$ a new primary client and assign
$p$ to itself. This process is repeated until all clients
are assigned. In the next phase we open facilities and
establish connections. First, for each primary client $p$ we
open exactly one facility $\phi(p)\in N(p)$, with each
facility $i\in N(p)$ chosen as $\phi(p)$ with probability
$y_i^\ast$. (Note that, by completeness and optimality, we
have $\sum_{i\in N(j)} y_i^\ast = \sum_{i\in N(j)}
x_{ij}^\ast= \sum_{i\in \sitesset} x_{ij}^\ast = 1$ for each
client $j$, as $x_{ij}^\ast=0$ for every $i\notin N(j)$.)
For any client $j$, if $j$ is assigned to a primary client
$p$, we connect $j$ to $\phi(p)$.

The obtained solution has expected facility cost (sum of
$f_i$'s, for opened facilities $i$) bounded by
$F^\ast$, due to the choice of the probability distribution
and because the neighborhoods of primary clients are
disjoint. To estimate the connection cost, let client $j$ be
assigned to a primary client $p$ and let $i$ be a facility
in $N(p)\cap N(j)$ (which is not empty, by the algorithm).
Then the expected connection cost for $j$ is
%
\begin{eqnarray*}
\Exp[d_{\phi(p)j}] \leq \Exp[d_{\phi(p)p}] + d_{ip} + d_{ij} 
					\leq C_p^\ast + \alpha_p^\ast + \alpha_j^\ast 
					\leq C_j^\ast + \alpha_j^\ast + \alpha_j^\ast 
					= C_j^\ast + 2\alpha_j^\ast. 
\end{eqnarray*}
%
The first inequality is due to the triangle
inequality, the second inequality follows from the complementary
slackness conditions (which imply that $\alpha_j^\ast = d_{ij} +
\beta_{ij}^\ast \geq d_{ij}$ for $i\in N(j)$), and the third one follows
from the choice of $p$ as a primary client (which implies that
$C_p^\ast + \alpha_p^\ast \le C_j^\ast + \alpha_j^\ast)$. 
Hence the total cost is at most 
%
\begin{eqnarray*}
	F^\ast + \sum_j (C^\ast_j + 2\alpha^\ast_j)
		= F^\ast + C^\ast+ 2\cdot\LP^\ast 
		= 3\cdot\LP^\ast 
		\leq 3\cdot\OPT.
\end{eqnarray*}
%
In FTFP, the demand values $r_j$ could be arbitrary
non-negative integers but, by Theorem~\ref{thm: reduction to
  polynomial}, to obtain a $3$-approximation algorithm, we
can assume that every $r_j$ is at most $|\sitesset|$, as
long as our algorithm approximates the optimal fractional
solution within a factor of $3$.  With this assumption, it
is natural to split each client into unit demands and then
deal with those demands separately.  The general approach is
then to iteratively connect those individual demands, in a
manner similar to the one described above. However,
simple-minded partition methods, like dividing each value
$x^\ast_{ij}$ evenly among the demands of $j$, are unlikely
to work, because they result in unmanageable overlapping
between neighborhoods of primary and non-primary demands.
To extend the above method to FTFP, the partitioning must
satisfy certain structural properties. For example, to
guarantee that siblings (demands of the same client) are
connected to different facilities, we need them to be
assigned to different primary demands. This in turn implies
that the neighborhoods of siblings should be disjoint, for
otherwise they could both overlap the neighborhood of the
same primary demand and then they could end up being both
assigned to this demand. For these reasons, the existing
techniques developed for FTFL, like the structured fractional solutions
used by Guha~{\etal}~\cite{GuhaMM01}, are not useful for our
purpose. Although superficially similar, the solutions in
\cite{GuhaMM01} are constructed statically and not even used in the
algorithm, only in the analysis. 

As we have shown in Section~{\ref{sec: adaptive partitioning}},
by a judicious choice of splitting clients into demands as
well as splitting their associated fractional connection
values $x_{ij}^\ast$, we can obtain a new fractional solution
$(\barbfx,\barbfy)$ satisfying various structural properties, including
those mentioned above. We shall
show now how we can exploit these properties to round that split fractional
solution, constructing an integral solution with expected
facility cost no more than $F^\ast$ and connection cost no
more than $C^\ast+2\cdot\LP^\ast$, which then implies
$3$-approximation. 

%%%%%%%

\paragraph{Algorithm~{\EGUP.}}
We start our algorithm with a partitioned fractional
solution $(\barbfx,\barbfy)$, computed from an optimal
fractional solution $(\bfx^\ast, \bfy^\ast)$ to
LP~(\ref{eqn:fac_primal}), using the algorithm in
Section~\ref{sec: adaptive partitioning}. Then we apply a
rounding process, guided by the fractional values
$(\bary_{\mu})$ and $(\barx_{\mu\nu})$, that produces an
integral solution. This integral solution is obtained by
choosing a subset of facilities in $\facilityset$ to open,
and for each demand in $\demandset$, specifying an open
facility that this demand will be connected to.  For each
primary demand $\kappa\in P$, we want to open one facility
$\phi(\kappa) \in \wbarN(\kappa)$. To this end, we use
randomization: for each $\mu\in\wbarN(\kappa)$, we choose
$\phi(\kappa) = \mu$ with probability $\barx_{\mu\kappa}$,
ensuring that exactly one $\mu \in \wbarN(\kappa)$ is
chosen. Note that
$\sum_{\mu\in\wbarN(\kappa)}\barx_{\mu\kappa}=1$, so this distribution
is well-defined.  We open
this facility $\phi(\kappa)$ and connect to $\phi(\kappa)$
all demands that are assigned to $\kappa$.  

In our description above, the algorithm is presented as a randomized
algorithm. It can be de-randomized using the method of conditional
expectations, which is commonly used in approximation algorithms
for facility location problems and standard enough that presenting
it here would be redundant. Readers less familiar with this field
are recommended to consult \cite{ChudakS04}, where the method of
conditional expectations is applied in a context very similar to ours.

%%%%%%%%%

\paragraph{Analysis.}
We now bound the expected facility cost and connection cost
by establishing the two lemmas below.

%%%%%

\begin{lemma}\label{lemma:3fac}
The expectation of facility cost $F_{\smallEGUP}$ of our solution is
  at most $F^\ast$.
\end{lemma}

\begin{proof}
By Property~\ref{P2:disjoint}, the neighborhoods of primary demands are disjoint. Also,
  for any primary demand $\kappa\in P$, the probability that
  a facility $\mu\in\wbarN(\kappa)$ is chosen as the open
  facility $\phi(\kappa)$ is $\barx_{\mu\kappa}$. Hence the
  expected total facility cost is
%
\begin{align*}
    \Exp[F_{\smallEGUP}]
	&= \textstyle{\sum_{\kappa\in P}\sum_{\mu\in\wbarN(\kappa)}} f_{\mu} \barx_{\mu\kappa}
	\\
	&= \textstyle{\sum_{\kappa\in P}\sum_{\mu\in\facilityset}} f_{\mu} \barx_{\mu\kappa} 
	\\
	&= \textstyle{\sum_{i\in\sitesset}} f_i \textstyle{\sum_{\mu\in i}\sum_{\kappa\in P}} \barx_{\mu\kappa} 
	\\
	&\leq \textstyle{\sum_{i\in\sitesset}} f_i y_i^\ast 
	= F^\ast,
\end{align*}
%
where the inequality follows from Property~\ref{P2:yi}.
\end{proof}

%%%%%%%

\begin{lemma}\label{lemma:3dist}
The expectation of connection cost $C_{\smallEGUP}$ of our solution
is at most  $C^\ast+2\cdot\LP^\ast$.
\end{lemma}

\begin{proof}
  Consider a demand $\nu$ assigned to a primary demand
  $\kappa\in P$. Let $\mu$ be any facility in $\wbarN(\nu)
  \cap \wbarN(\kappa)$.  Since $\mu$ is in both
  $\wbarN(\nu)$ and $\wbarN(\kappa)$, we have $d_{\mu\nu}
  \leq \alpha_{\nu}^\ast$ and $d_{\mu\kappa} \leq
  \alpha_{\kappa}^\ast$ (this follows from the complementary
  slackness conditions since
  $\alpha_{\nu}^\ast=\beta_{\mu\nu}^\ast + d_{\mu\nu}$ for
  each $\mu\in \wbarN(\nu)$). Thus, applying the triangle inequality, for any fixed choice of
  $\phi(\kappa)$ we have
%
\begin{equation*}
    d_{\phi(\kappa)\nu} \leq d_{\phi(\kappa)\kappa}+d_{\mu\kappa}+d_{\mu\nu}
    \leq d_{\phi(\kappa)\kappa} + \alpha_{\kappa}^\ast + \alpha_{\nu}^\ast.
\end{equation*}
%
Therefore the expected distance from $\nu$ to its facility $\phi(\kappa)$ is 
%
\begin{align*}
  \Exp[  d_{\phi(\kappa)\nu}   ] &\le \concost_{\kappa} + \alpha_{\kappa}^\ast + \alpha_{\nu}^\ast 
\\
  &\leq \concost_{\nu} + \alpha_{\nu}^\ast + \alpha_{\nu}^\ast
   = \concost_{\nu} + 2\alpha_{\nu}^\ast,
  \end{align*}
%
  where the second inequality follows from Property~\ref{P2:cost}.  
From the definition of $\concost_{\nu}$ and Property~\ref{P1:xij}, for any $j\in \clientset$ 
we have
%
\begin{align*}
\sum_{\nu\in j} \concost_{\nu} &= \sum_{\nu\in j}\sum_{\mu\in\facilityset}d_{\mu\nu}\barx_{\mu\nu}
			\\
 			&= \sum_{i\in\sitesset} d_{ij}\sum_{\nu\in j}\sum_{\mu\in i}\barx_{\mu\nu}
			\\
			&= \sum_{i\in\sitesset} d_{ij}x^\ast_{ij} 
			= C^\ast_j.
\end{align*}
% 
Thus, summing over all demands, the expected total connection cost is
%
\begin{align*}
    \Exp[C_{\smallEGUP}] &\le 
			\textstyle{\sum_{j\in\clientset} \sum_{\nu\in j}} (\concost_{\nu} + 2\alpha_{\nu}^\ast) 
			\\
    	& = \textstyle{\sum_{j\in\clientset}} (C_j^\ast + 2r_j\alpha_j^\ast)
 		= C^\ast + 2\cdot\LP^\ast,
\end{align*}
%
completing the proof of the lemma.
\end{proof}

%%%%%%%%

\begin{theorem}
Algorithm~{\EGUP} is a $3$-approximation algorithm.
\end{theorem}

\begin{proof}
  By Property~\ref{P2:diff}, different demands from the same
  client are assigned to different primary demands, and by
  \ref{P2:disjoint} each primary demand opens a different
  facility. This ensures that our solution is feasible,
  namely each client $j$ is connected to $r_j$ different
  facilities (some possibly located on the same site).  As
  for the total cost, Lemma~\ref{lemma:3fac} and
  Lemma~\ref{lemma:3dist} imply that the total cost is at
  most $F^\ast+C^\ast+2\cdot\LP^\ast = 3\cdot\LP^\ast \leq
  3\cdot\OPT$.
\end{proof}

%%%%%%%%%


This completes the presentation and analysis of our $3$-approximation algorithm.

%%%%%%%%%%%%%%%%%%%%%%%%%%%%%%%%%%%%%%%%%%%%%%%%%%%%%%%%%%%%%%%%%%%%%%%%%%%%%%%
%%% 1.736-APPROXIMATION ALGORITHM %%%%%%%%%%%%%%%%%%%%%%%%%%%%%%%%%%%%%%%%%%%%%
%%%%%%%%%%%%%%%%%%%%%%%%%%%%%%%%%%%%%%%%%%%%%%%%%%%%%%%%%%%%%%%%%%%%%%%%%%%%%%%

\section{Algorithm~{\ECHU} with Ratio $1.736$}\label{sec: 1.736-approximation}

In this section we improve the approximation ratio to $1+2/e
\approx 1.736$. The new algorithm, named Algorithm~{\ECHU},
starts with the same partitioned fracitional solution
$(\barbfx, \barbfy)$ as Algorithm~{\EGUP}. The improvement
on approximation ratio comes from a more slightly modified
rounding process and much refined analysis.  Note that the
facility opening cost of Algorithm~{\EGUP} does not exceed
that of the fractional optimum solution, while the
connection cost is quite far from the optimum, due to the
cost of indirect connections (that is, connections from
non-primary demands).  The basic idea behind the
improvement, following the approach of Chudak and
Shmoys'~\cite{ChudakS04}, is then to open more facilities
and use direct connections when available so as to reduce
the number of indirect connections. A {\naive}
generalization of the method from \cite{ChudakS04} may
produce infeasible solutions, with different demands from
the same client being connected to the same facility. This
could happen, for example, when one demand of a client
connects to a facility in its neighborhood while another
uses the same facility via an indirect connection. We show,
however, that the fractional solution we obtained from
Adaptive Partitioning in the last section possesses
additional properties that allows us to prevent such
conflicts.

The details of the modified rounding process is given in
Pseudocode~\ref{alg:lpr3}.  We will use the term
\emph{facility cluster} for the neighborhood of a primary
client. Facilities that do not belong to these clusters will
be called \emph{non-clustered}.

As before, we open exactly one facility $\phi(\kappa)$ in
the facility cluster of a primary demand $\kappa$, with the
same probability distribution as before (Line 2).  For any
non-primary demand $\nu$ assigned to $\kappa$, we refer to
$\phi(\kappa)$ as the \emph{target} facility of $\nu$.  In
Algorithm~{\EGUP}, $\nu$ was connected to $\phi(\kappa)$,
but in Algorithm~{\ECHU} we may open a facility in $\nu$'s
neighborhood and connect $\nu$ to this facility.
Specifically, the two changes in the algorithm are as
follows: (1) Each non-clustered facility $\mu$ is opened,
independently, with probability $\bary_{\mu}$ (Lines
4--5). Notice that due to completeness of the partitioned
fractional solution, we have $\bary_{\mu}=\barx_{\mu\nu}$
for some demand $\nu$, and $\sum_{\mu\in\facilityset}
\barx_{\mu\nu} = 1$, therefore $\bary_{\mu}\leq 1$.  (2)
When connecting demands to facilities, a primary demand
$\kappa$ is connected to the only facility $\phi(\kappa)$
opened in its neighborhood, as before (Line 3).  For a
non-primary demand $\nu$, if its neighborhood has an open
facility, we connect $\nu$ to the closest open facility in
its neighborhood (Line 8). Otherwise, we connect $\nu$ to
its target facility (Line 10).

%%%%%%%%%%%%%

\begin{algorithm}
  \caption{Algorithm~{\ECHU}, Stage~2:
    Constructing Integral Solution}
  \label{alg:lpr3}
  \begin{algorithmic}[1]
    \For{each $\kappa\in P$} 
    \State choose one $\phi(\kappa)\in \wbarN(\kappa)$,
    with each $\mu\in\wbarN(\kappa)$ chosen as $\phi(\kappa)$
    with probability $\bary_\mu$
    \State open $\phi(\kappa)$ and connect $\kappa$ to $\phi(\kappa)$
    \EndFor
    \For{each $\mu\in\demandset - \bigcup_{\kappa\in P}\wbarN(\kappa)$} 
    \State open $\mu$ with probability $\bary_\mu$ (independently)
    \EndFor
    \For{each non-primary demand $\nu\in\demandset$}
    \If{any facility in $\wbarN(\nu)$ is open}
    \State{connect $\nu$ to the nearest open facility in $\wbarN(\nu)$}
    \Else
    \State connect $\nu$ to $\phi(\kappa)$ where $\kappa$ is $\nu$'s
     primary demand
    \EndIf
    \EndFor
  \end{algorithmic}
\end{algorithm}

%%%%%%%%%%%%%%%%

\paragraph{Analysis.}
We shall first argue that the integral solution thus
constructed is feasible, and then we bound the total cost of
the solution. Regarding feasibility, the only constraint that is
not explicitly enforced by the algorithm is the fault-tolerance
requirement; namely that each client $j$ is connected to $r_j$
different facilities. Thus it is sufficient to prove the following lemma.

%%%%%%%%%

\begin{lemma}\label{lem: lpr3 feasible}
For each client $j\in\clientset$, all demands from $j$ are connected to
different facilities.
\end{lemma}


\begin{proof}
  Let $\nu_1,\nu_2$ be two sibling demands and
  $\kappa_1,\kappa_2$ be the primary demands they are assigned to, respectively. 
  Property~\ref{P1:siblings disjoint} and \ref{P2:diff} imply that
%
  \begin{equation*}
    (\wbarN(\kappa_1) \cup \wbarN(\nu_1)) \cap \wbarN(\nu_2) = \emptyset.
  \end{equation*}
%
  Therefore each facility is accessible to at most one
  demand among siblings that are created from the same
  client.
\end{proof}

%%%%%%%%%

\medskip

We now show the solution has a total cost bounded by
$(1+2/e) \cdot \LP^\ast$ in expectation. By
Property~\ref{P2:disjoint}, every facility may appear in at
most one primary demand's neighborhood, and the facilities
open in Line~4--5 of Pseudocode~\ref{alg:lpr3} do not appear
in any primary demand's neighborhood. Therefore, by
linearity of expectation, the expected facility cost of
algorithm~{\ECHU} is $\sum_{\mu\in\facilityset}
f_\mu \bary_{\mu} = \sum_{i\in\sitesset} f_i\sum_{\mu\in i}
\bary_{\mu} = \sum_{i\in\sitesset} f_i y_i^\ast = F^\ast$,
where the second equality follows from Property~\ref{P1:yi}.

To bound the connection cost, we adapt an argument of Chudak
and Shmoys'~\cite{ChudakS04}. Consider a demand $\nu$. This demand can either
get connected directly to some facility in $\wbarN(\nu)$ or indirectly to
facility $\phi(\kappa)\in \wbarN(\kappa)$, where $\kappa$ is the primary
demand to which $\nu$ is assigned. 

We first estimate the expected cost of the indirect
connection of $\nu$, when $\nu$ is connected to its target
facility $\phi(\kappa)$, where $\kappa$ is the primary
demand that $\nu$ is assigned to and $\phi(\kappa)$ is the
only facility opened by $\kappa$. We have the following
lemma.
\begin{lemma}
  \label{lem:echu indirect}
  The expected cost of an indirect connection of a demand
  $\nu$ to its target facility is bounded by
  $\concost(\nu)+2\alpha_{\nu}^\ast$.
\end{lemma}
\begin{proof}
  Let $\kappa$ be the primary demand that $\nu$ is assigned
  to and $\phi(\kappa)$ be the only facility opened by
  $\kappa$. Then $\phi(\kappa)$ is the target facility of
  $\nu$. By the definition of $\kappa$, there exists a
  facility $\mu\in \wbarN(\nu)\cap \wbarN(\kappa)$.  For any
  fixed value of $\phi(\kappa)$, the cost of connecting
  $\nu$ to $\phi(\kappa)$ is $\bard = d(\phi(\kappa),\nu)
  \le d(\phi(\kappa),\kappa)+ d_{\mu\kappa}+d_{\mu\nu} \le
  d(\phi(\kappa),\kappa)+\alpha_{\kappa}^\ast +
  \alpha_{\nu'}^\ast$, where the last inequality follows
  from the complementary slackness conditions. We can bound
  the expectation of $\bard$ as follows:
%
\begin{equation}
	\label{eqn: 1.76 connection cost}
  \Exp[\bard] = \Exp[d(\phi(\kappa), \kappa)]  + \alpha_{\kappa}^\ast + \alpha_{\nu}^\ast
  \leq \tcc(\kappa) + \alpha_{\kappa}^\ast + \alpha_{\nu}^\ast
  \leq \tcc(\nu) + 2\alpha_{\nu}^\ast\;\leq\; \concost_{\nu} + 2\alpha_{\nu}^\ast.
\end{equation}
%
The second inequality follows from Property~\ref{P2:cost}
and the last inequality is from Lemma~\ref{lem: tcc optimal}.
To see the first inequality, notice that the
expectation is computed for facility set
$X(\kappa)=\wbarN(\kappa) \setminus \wbarN(\nu)$ with each facility
$\mu$ chosen with probability proportional to
$\bary_{\mu}$. Given a facility set $A$, let
$d(A,\nu)=\sum_{\mu\in A} d_{\mu\nu}\bary_{\mu}/\sum_{\mu\in
  A} \bary_{\mu}$. Notice that
$\tcc(\kappa)=d(\wbarN(\kappa),\kappa)$ by definition. We
claim that $d(X(\kappa),\nu) \leq \tcc(\kappa) +
\alpha_{\kappa}^\ast + \alpha_{\nu}^\ast$. To prove this
claim, we consider two cases. 

\mycase{1} The first case is when there exists some $\mu'\in
\wbarN(\kappa) \cap \wbarN(\nu)$ such that $d_{\mu' \kappa}
\leq d(\wbarN(\kappa), \kappa)$. In this case, denoting
$d_{\kappa \nu}=\min_{\mu\in\facilityset} (d_{\mu\kappa} +
d_{\mu\nu})$, we have
\begin{equation*}
  d(\kappa,\nu)\leq
  d_{\mu'\kappa} + d_{\mu'\nu} \leq d(\wbarN(\kappa),\kappa) +
  \alpha_{\nu}^\ast=\tcc(\kappa)+\alpha_{\nu}^\ast. 
\end{equation*}
It follows that for every $\mu\in X(\kappa)$, we have
\begin{equation*}
  d(\mu,\nu) \leq d_{\mu\kappa} + d(\kappa,\nu) \leq
  \alpha_{\kappa}^\ast + \tcc(\kappa) + \alpha_{\nu}^\ast.
\end{equation*}

\mycase{2} In the second case, every $\mu'\in
\wbarN(\nu)\cap \wbarN(\kappa)$ has $d_{\mu'\kappa} >
d(\wbarN(\kappa),\kappa)$. Then we have
$d(X(\kappa),\kappa)\leq
d(\wbarN(\kappa),\kappa)=\tcc(\kappa)$, therefore
\begin{equation*}
  d(X(\kappa), \nu) \leq d(\wbarN(\kappa), \kappa) +
  d_{\mu\kappa} + d_{\mu\nu} \leq \tcc(\kappa) +
  \alpha_{\kappa}^\ast + \alpha_{\nu}^\ast.  
\end{equation*}
This completes the justification of (\ref{eqn: 1.76
  connection cost}).
\end{proof}

We now estimate the expected connection cost of demand
$\nu$. Let $\wbarN(\nu) = \braced{\mu_1,\ldots,\mu_l}$ and
let $d_s = d_{\mu_s\nu}$ for $s = 1,\ldots,l$. By
reordering, we can assume that $d_1 \le d_2 \le \ldots \le
d_l$. By the algorithm the connection cost is no more than
that obtained through the random process that opens each
$\mu_s\in \wbarN(\nu)$ independently with probability
$\barx_{\mu_s \nu} =\bary_{\mu_s}$ (because the solution is
complete) and connects $\nu$ to the nearest such open
facility, if any of them opens; otherwise $\nu$ is connected
indirectly to $\phi(\kappa)$ at cost $\bard$. The intuition
is that we only use a facility $\mu_s$ if none of
$\mu_1,\ldots,\mu_{s-1}$ is open. Suppose $\mu_s$ is a
facility in $\wbarN(\kappa)$ for some primary demand
$\kappa$. If none of $\mu_1,\ldots,\mu_{s-1}$ is in
$\wbarN(\kappa)$, then $\mu_s$ opens with probability
$\bary_{\mu_s}$. If some of them do appear in
$\wbarN(\kappa)$, the fact that they are closed can only
increase the probability that $\mu_s$ opens, by the choice
of $\phi(\kappa)$.  For a detailed proof,
see~\cite{ChudakS04}.

Denoting by $C_\nu$ the connection cost for $\nu$, we thus have
%
\begin{align*}
  \Exp[C_\nu] &\leq d_1 \bary_{\mu_1} + d_2 \bary_{\mu_2}(1-\bary_{\mu_1}) + \ldots
 		+  d_l \bary_{\mu_l}\textstyle{\prod_{s=1}^{l-1}}(1-\bary_{\mu_s}) 
		+  \Exp[\bard]\textstyle{\prod_{s=1}^{l}} (1-\bary_{\mu_s})
		\\
  &\leq (1-\textstyle{\prod_{i=1}^l} (1-\bary_{\mu_i}))
  	\sum_{i=1}^l d_i\bary_{\mu_i} + {\textstyle\prod_{i=1}^l} (1-\bar  y_{\mu_i})\Exp[\bard]
	\\
  &\leq (1-\frac{1}{e}) {\textstyle\sum_{i=1}^l} d_i\bary_i 
	+ \frac{1}{e} \Exp[\bard] \leq (1-\frac{1}{e}) \concost_{\nu} 
	+	\frac{1}{e}	(\concost_{\nu} + 2\alpha_{\nu}^\ast) = \concost_{\nu} + \frac{2}{e}\alpha_{\nu}^\ast,
\end{align*}
%
where the second inequality was shown in
\cite{ChudakS04}. (In the appendix we give a simpler
proof of this inequality using only elementary techniques.)  Notice that the
completeness of the fractional solution allows us to write
$\bary_{\mu}$ instead of $\barx_{\mu\nu}$.

Summing over all demands of a client $j$, we obtain the
expected connection cost of client $j$:
%
\begin{equation*}
  \Exp[C_j] \leq {\textstyle\sum_{\nu\in j} (\concost_{\nu} + \frac{2}{e}\alpha_{\nu}^\ast) }
  = \textstyle{ C_j^\ast + \frac{2}{e}r_j\alpha_j^\ast}.
\end{equation*}
%
Summing over all clients $j$ gives the expected connection
cost being bounded by $C^\ast +
\frac{2}{e}\LP^\ast$. Therefore, we have established that
our algorithm constructs a feasible integral solution with
an overall expected cost bounded by
%
\begin{equation*}
  \label{eq:chudakall}
  	F^\ast + C^\ast + \frac{2}{e}\cdot \LP^\ast = (1+2/e)\cdot \LP^\ast
  \leq (1+2/e)\cdot \OPT.
\end{equation*}

Summarizing, we obtain the result of this section.

\begin{theorem}\label{thm:1736}
  Algorithm~{\ECHU} is a $(1+2/e)$-approximation algorithm for \FTFP.
\end{theorem}

%%%%%%%%%%%%%%%%%%%%%%%%%%%%%%%%%%%%%%%%%%%%%%%%%%%%%
%%% Byrka 2010 1.575
%%%%%%%%%%%%%%%%%%%%%%%%%%%%%%%%%%%%%%%%%%%%%%%%%%%%%

\section{Algorithm~{\EBGS} with Ratio $1.575$}\label{sec: 1.575-approximation}

%%% motivation of better approximation
%%% review of Byrka's idea
%%% refined partition
%%% rounding
%%% analysis of approx ratio
In this section we give our main result, a $1.575$-approximation algorithm
for $\FTFP$
%
\footnote{$1.575$ is $\min_{\gamma\geq 1}\max\{\gamma,
  1+2/e^\gamma, \frac{1/e+1/e^\gamma}{1-1/\gamma}\}$.}. 
%
This ratio matches the ratio of the best known LP-rounding
algorithm for UFL~\cite{ByrkaGS10}. Our {\EBGS} algorithm
consists of two stages, with stage 1 being a more refined
partitioning of a fractional optimal solution, which allows
us to apply a similar rounding process and analysis
as~\cite{ByrkaGS10}.

Recall that in Section~\ref{sec: 1.736-approximation}, we
have an integral solution with facility cost bounded by
$F^\ast$ and connection cost bounded by $C^\ast +
2/e\cdot\LP^\ast$. A natural idea is to look for a way to
reduce the connection cost, at the expense of increasing the
facility cost by a small fraction.

To help motivate our improved algorithm, we start by giving
an overview of the related BGS algorithm for UFL in
\cite{ByrkaGS10}. To help introduce our notation, we present
the BGS algorithm in a slightly different but equivalent
way. The BGS algorithm uses some constant $\gamma\ge 1$, to
be fixed later. As usual, it first computes a primal
fractional optimal solution $(\bfx^\ast,\bfy^\ast)$ to
LP~(\ref{eqn:fac_primal}). Notice that for the UFL problem
we have all $r_j=1$. By splitting facilities and adjusting
appropriately facility opening and connection values, we can
assume that solution $(\bfx^\ast, \bfy^\ast)$ satisfies the
enhanced complete property: For each client $j$ and facility
$i$, $x_{ij}^\ast>0$ implies
$x_{ij}^\ast=y_i^\ast$. Moreover, for each client $j$, if we
order facilities $i$ in its neighborhood with nondecreasing
$d_{ij}$, then there exists some $l$ such that $\sum_{i=1}^l
x_{ij}^\ast = 1/\gamma$. As before we also have $\sum_{i\in
  N(j)} x_{ij}^\ast = 1$ from optimality, where
$N(j)=\{i\in\sitesset \suchthat x_{ij}^\ast > 0\}$ is client
$j$'s neighborhood. This enhanced complete property allow us
to introduce two more definitions.

For each client $j$, its neighborhood $N(j)$ is partitioned
into two parts, the close and the far neighborhood, that is
$N(j) = \clsnb(j) \cup \farnb(j)$, where (i) $\clsnb(j) \cap
\farnb(j) = \emptyset$, (ii) $\sum_{i\in\clsnb(j)}
x_{ij}^\ast=1/\gamma$, (iii) $\sum_{i\in\farnb(j)}
x_{ij}^\ast=1-1/\gamma$, and (iv) if $i\in \clsnb(j)$ and
$i'\in \farnb(j)$ then $d_{ij}\le d_{i'j}$.  Intuitively,
$\clsnb(j)$ (also called \emph{$j$'s cluster}) consists of
the first $1/\gamma$ ``chunk" of the facilities nearest to
$j$, while $\farnb(j)$ contains the remaining facilities.
We define the corresponding connection costs:
$\clsdist(j)={\gamma}\sum_{i\in\clsnb(j)}d_{ij}x_{ij}^\ast$
and
$\fardist(j)=\frac{\gamma}{\gamma-1}\sum_{i\in\farnb(j)}d_{ij}x_{ij}^\ast$
as the average distance from client $j$ to facilities in
$\clsnb(j)$ and $\farnb(j)$ respectively, and
$\clsmax(j)=\max_{i\in\clsnb(j)}d_{ij}$ is the maximum
distance from $j$ to any facility in $\clsnb(j)$. The
overall average connection cost for $j$ is
$\concost(j)=\sum_{i\in N(j)} d_{ij}x_{ij}^\ast$.

The BGS algorithm then runs in iterations. In each
iteration, it picks a client $p$ with minimum value of
$\clsdist(p)+\clsmax(p)$. If $\clsnb(p)$ overlaps with that
of some existing primary client $p'$, then $p$ is assigned
to $p'$. Otherwise $p$ becomes a primary client and assigned
to itself.  Once each client has been assigned to a primary
client, the algorithm starts opening facilities. For each
primary client $p$ it opens one facility $\phi(p)$ in $p$'s
cluster $\clsnb(p)$, with each facility $i\in \clsnb(p)$
chosen as $\phi(p)$ with probability $\gamma y_i^\ast$. In
addition, each non-clustered facility $i$ opens with
probability $\gamma y_i^\ast$, independently. For
connections, each client $j$ will first try to connect to
the nearest open facility in $N(j)$. If none in $N(j)$
opens, then $j$ gets connected to $\phi(p)$, called its
\emph{target} facility, where $p$ is the primary client to
which $j$ is assigned.

It was shown in~\cite{ByrkaA10} that the above rounding
process implies that $\clsnb(j)$ has at least one open
facility with probability no less than $1-1/e$ and that
$N(j)$ has an open facility with probability at least
$1-1/e^\gamma$. Hence with probability at most $1/e^\gamma$,
a client uses an indirect connection to its target facility,
whose cost is bounded by a technical lemma
in~\cite{ByrkaA10}.

%%%%%%%%%%%%

\begin{lemma}~\cite{ByrkaA10}
  \label{lem:redirect}
  Define $d(j,A)=\sum_{i\in A} d_{ij}y_i^\ast/\sum_{i\in A}
  y_i$, let $\gamma <2$ be a constant, and let $p$ be the
  primary client that $j$ was assigned to. Then either
  $N(p)-(\clsnb(j)\cup \farnb(j))=\emptyset$ or
  $d(j,N(p) \setminus (\clsnb(j)\cup \farnb(j))) \leq \clsdist(j) +
  \clsmax(j) + \fardist(j)$.
\end{lemma}

Given Lemma~\ref{lem:redirect}, the expected connection cost
for a client $j$ is then
\begin{align*}
  \Exp[C_j] &\leq \clsdist(j)(1-1/e) + \fardist(j)(1/e-1/e^\gamma) + (\clsdist(j)
  + \clsmax(j)+\fardist(j))1/e^\gamma \\
  &\leq \clsdist(j)(1-1/e) +
  \fardist(j)(1/e-1/e^\gamma) + (\clsdist(j) + 2\fardist(j))1/e^\gamma\\
  &\leq
  \concost(j)((1-\rho_j)(\frac{1/e+1/e^\gamma}{1-1/\gamma}) +
  \rho_j(1+2/e^\gamma)),
\end{align*}
where we define $\rho_j = \clsdist(j)/\concost(j)$. It is
easy to see that $\concost(j) = \sum_{i\in\sitesset}
d_{ij}x_{ij}^\ast = \frac{1}{\gamma}\clsdist(j) +
(1-\frac{1}{\gamma})\fardist(j)$. Then $\fardist(j)=(\gamma
\concost(j)-\clsdist(j))/(\gamma-1)$. The approximation
ratio is then bounded by the expression
$\max\{\gamma,\frac{1/e+1/e^\gamma}{1-1/\gamma},1+2/e^\gamma\}$,
which is minimized for $\gamma=1.575$.


\smallskip

We now describe our Algorithm~{\EBGS} for the FTFP
problem. As before, our algorithm first partitions the
instance and then applies the rounding approach similar to
the BGS algorithm~\cite{ByrkaGS10} described above. However,
to apply the rounding approach similar to~\cite{ByrkaGS10},
we need to introduce the close and far neighborhood of each
demand, and our properties of the partitioned fractional
solution need to guarantee certain close neighborhoods
overlap while other neighborhoods are disjoint. It turns out
that the properties required are much more delicate but
nonetheless, we show a modified partition algorithm actually
delivers a partitioned fractional solution that satisfies
all the properties. The rounding stage that construct an
integral solution is relatively straightforward, as is the
analysis of approximation ratio.

%%%%%%%%%%%%

\emparagraph{Stage~1: Adaptive Partitioning.} The
partitioning stage of Algorithm~{\EBGS} is similar to that
in Section~\ref{sec: 1.736-approximation}, with changes
inspired by the BGS algorithm in~\cite{ByrkaGS10} to obtain
an improved ratio. In Stage~1, we also begin with
$(\bfx^\ast, \bfy^\ast)$ that is complete, that is
$x_{ij}^\ast=y_i^\ast$ if $x_{ij}^\ast>0$. Then we run our
partition algorithm to obtain a fractional solution
$(\barbfx, \barbfy)$ for the modified instance where clients
are divided into unit-value demands and sites are divided
into facilities. We adapt all notation for the modified
instance from Sections~\ref{sec: 3-approximation}
and~\ref{sec: 1.736-approximation}. In particular, the
neighborhood of a demand and its average connection cost is
defined as before:
$\wbarN(\nu)=\{\mu\in\facilityset\suchthat
\barx_{\mu\nu}>0\}$ and
$\concost(\nu)=\sum_{\mu\in\wbarN(\nu)}d_{\mu\nu}\barx_{\mu\nu}$,
for each $\nu\in\demandset$.

In our algorithm, the neighborhood $\wbarN(\nu)$ of each
demand $\nu$ will be partitioned into two parts, called the
\emph{close neighborhood} $\wbarclsnb(\nu)$ and the
\emph{far neighborhood} $\wbarfarnb(\nu)$, whose formal
definitions are given below as
Property~\ref{PP1:gamma}. Their respective average
connection costs are defined by
$\clsdist(\nu)=\gamma\sum_{\mu\in\wbarclsnb(\nu)}
d_{\mu\nu}\barx_{\mu\nu}$ and
$\fardist(\nu)=\frac{\gamma}{\gamma-1}\sum_{\mu\in\wbarfarnb(\nu)}
d_{\mu\nu}\barx_{\mu\nu}$.  We will also use notation
$\clsmax(\nu)=\max_{\mu\in\wbarclsnb(\nu)} d_{\mu\nu}$ for
the maximum distance from $\nu$ to its close neighborhood.

\smallskip 

To better motivate our approach and for ease of
presentation, we give only the desired properties of the
partitioned instance and the fractional solution $(\barbfx,
\barbfy)$ produced by our algorithm. The details on how to
construct such a fractional solution, along with a proof of
all the properties are deferred to the end of this section.

%
\begin{enumerate}
      \renewcommand{\theenumi}{P\arabic{enumi}}
      \renewcommand{\labelenumi}{(\theenumi)}
      \renewcommand{\theenumii}{(\alph{enumii})}
      \renewcommand{\labelenumii}{\theenumii}

\item Vectors $\barbfx,\barbfy$ satisfy the following properties:
%
  \begin{enumerate}
	\item \label{PP1:one} For each demand $\nu$, $\sum_{\mu\in\facilityset}
    \barx_{\mu\nu} = 1$.
	\item \label{PP1:xij} $\sum_{\mu\in i, \nu\in j} \barx_{\mu\nu}
          = x_{ij}^\ast$ for each site $i\in\sitesset$ and client $j\in\clientset$.
	\item \label{PP1:yi}
          $\sum_{\mu\in i} \bary_{\mu} = y_i^\ast$ for each site $i\in\sitesset$.
	\item \label{PP1:eq}
          If $\barx_{\mu\nu}\neq 0$ then
				$\barx_{\mu\nu} = \bary_{\mu}$, for each facility 
						$\mu\in\facilityset$ and demand
            $\nu\in\demandset$ (completeness).

  \item \label{PP1:gamma} 
	For each demand $\nu$, its neighborhood is divided into \emph{close} and
	\emph{far} neighborhood, that is $\wbarN(\nu) = \wbarclsnb(\nu) \cup \wbarfarnb(\nu)$, where
(n1) $\wbarclsnb(\nu) \cap \wbarfarnb(\nu) = \emptyset$,
(n2) $\sum_{\mu\in\wbarclsnb(\nu)} \barx_{\mu\nu} =1/\gamma$,
(n3) $\sum_{i\in\wbarfarnb(\nu)} \barx_{\mu\nu} =1-1/\gamma$,
and 
(n4) if $\mu\in \wbarclsnb(\nu)$ and $\mu'\in \wbarfarnb(\nu)$ then $d_{\mu\nu}\le d_{\mu'\nu}$.   

  \end{enumerate}

\item Primary demands satisfy the following properties:
%
\begin{enumerate}
\item \label{PP2:yi} $\sum_{\kappa\in P} \sum_{\mu\in i\cap
    \wbarclsnb(\kappa)}\barx_{\mu\kappa} \leq y_i^\ast$, for
  each site $i\in\sitesset$. Note that in the sum
  we do not count contributions from
  the far neighborhood of primary demands.
\item \label{PP2:assign} Each demand $\nu\in\demandset$ is
  assigned to exactly one primary demand $\kappa\in P$ with
  an overlapping close neighborhood, that is
  $\wbarclsnb(\nu) \cap \wbarclsnb(\kappa) \neq \emptyset$.
\item \label{PP2:diff} Different demands from the same
  client are assigned to different primary demands.

\item \label{PP2:disjoint} For any two different sibling
  demands $\nu$ and $\nu'$, if $\nu$ is assigned to a
  primary demand $\kappa$ then $(\wbarN(\nu) \cup
  \wbarclsnb(\kappa)) \cap \wbarN(\nu') = \emptyset$.  This
  condition ensures that no conflict can arise between
  sibling demands when they connect to facilities.

\item \label{PP2:cost} If a demand $\nu\in\demandset$ is
  assigned to a primary demand $\kappa\in P$, then
  $\clsdist(\nu)+\clsmax(\nu) \geq
  \clsdist(\kappa)+\clsmax(\kappa)$.
\end{enumerate}

\end{enumerate}

\emparagraph{Stage~2: Computing Integral Solution.}  Given
the partitioned fractional solution $(\barbfx, \barbfy)$
with the desired properties, we then start opening
facilities and making connections to obtain an integral
solution. As before we open exactly one facility in each
cluster (the close neighborhood of a primary demand), but
now each facility $\mu$ is chosen with probability
$\gamma\bary_{\mu}$. The non-clusterd facilities $\mu$,
those that do not belong to $\wbarN_{\cls}(\kappa)$ for any
primary demand $\kappa$, are opened independently with
probability $\gamma\bary_{\mu}$ each. This implies that the
expected facility cost of our algorithm is bounded by
$\gamma F^\ast$, using essentially the same argument as in
the previous section (with the the factor $\gamma$
accounting for using probabilities $\gamma \bary_{\mu}$
instead of $\bary_{\mu}$).

For connections, each primary demand $\kappa$ will connect
to the only facility $\phi(\kappa)$ open in its cluster
$\wbarclsnb(\kappa)$.  For each non-primary demand $\nu$, if
there is an open facility in $\wbarN(\nu)$ then we connect
$\nu$ to the nearest such facility. Otherwise, we connect
$\nu$ to $\phi(\kappa)$, where $\kappa$ is the primary
demand that $\nu$ is assigned to. This facility
$\phi(\kappa)$ will be called the \emph{target facility} of
$\nu$.

%%%%%%%%%%%

\paragraph{Analysis.}
The feasibility of our integral solution follows from
Property~\ref{PP2:diff}, that each sibling demand is
assigned to a different primary demand, and the disjoint-neighborhood property
Property~\ref{PP2:disjoint}. These properties ensure that
each facility is accessible to at most one demand among all demands of the same client.

We now bound the cost. Properties~\ref{PP2:assign}
and~\ref{PP2:cost} allow us to bound the expected distance
from a demand $\nu$ to its target facility by
$\clsdist(\nu)+\clsmax(\nu)+\fardist(\nu)$, in the event
that none of $\nu$'s neighbors opens, using a similar
argument as Lemma 2.2 in~\cite{ByrkaGS10}~\footnote{The full
  proof of the lemma appears in~\cite{ByrkaA10} as
  Lemma~3.3.}. We are then able to show that the expected
connection cost for demand $\nu$ using an argument similar to~\cite{ByrkaGS10}.
%
\begin{align*}
  \Exp[C_{\nu}] &\leq \clsdist(\nu)(1-1/e) +
  \fardist(\nu)(1/e-1/e^\gamma) + (\clsdist(\nu)+\clsmax(\nu)+\fardist(\nu))1/e^\gamma \\
  &\leq \clsdist(\nu)(1-1/e) +
  \fardist(\nu)(1/e-1/e^\gamma) + (\clsdist(\nu)+2\fardist(\nu))1/e^\gamma\\
  &\leq
  \concost(\nu)((1-\rho_{\nu})(\frac{1/e+1/e^\gamma}{1-1/\gamma})
  + \rho_{\nu}(1+2/e^\gamma)) \\
  &\leq \concost(\nu) \cdot
  \max\{\frac{1/e+1/e^\gamma}{1-1/\gamma},
  1+\frac{2}{e^\gamma}\},
\end{align*}
%
where $\rho_{\nu}=\clsdist(\nu)/\concost(\nu)$. It is easy
to see that $\rho_{\nu}$ is between 0 and 1 for every demand
$\nu$.  Since $\sum_{\nu\in j} C^{\avg}(\nu) = \sum_{\nu\in
  j}\sum_{\mu\in\facilityset} d_{\mu\nu}\barx_{\mu\nu} =
\sum_{i\in\sitesset} d_{ij}x_{ij}^\ast = C_j^\ast$, summing
over all clients $j$ we have total connection cost bounded
by $C^\ast \max\{\frac{1/e+1/e^\gamma}{1-1/\gamma},
1+\frac{2}{e^\gamma}\}$. The expected facility cost is
bounded by $\gamma F^\ast$, as argued earlier. Hence the
total cost is bounded by $\max\{\gamma,
\frac{1/e+1/e^\gamma}{1-1/\gamma},
1+\frac{2}{e^\gamma}\}\cdot \LP^\ast$. Picking
$\gamma=1.575$ we obtain the desired ratio.

%%%%%%%

\emparagraph{Stage~1 Implementation.}  We now return to
Stage~1. Our adaptive partitioning scheme in the {\EBGS}
algorithm is very similar to that in the {\EGUP} algorithm,
except that we now cut a chunk of $1/\gamma$ when creating a
primary demand, instead of $1$. In the augment step we will
add additional facilities to the neighborhood of all demands
to make them each have a neighborhood with total connection
value of $1$.

Our scheme still runs in iterations. Consier any client $j$.
Let $\wtildeN(j)$ be the set of facilities $\mu$ such that
$\tildex_{\mu j}>0$. Order facilities in $\wtildeN(j)$ such
that $d_{1 j} \leq d_{2 j} \leq \ldots \leq d_{|\wtildeN(j)|
  j}$ and $\sum_{s=1}^l \tildex_{s j} = 1/\gamma$ for some
integer $l$ (we split the $l^{th}$ facility if necessary to
make the sum equal to $1/\gamma$). Then the set of
facilities $1,2,\ldots,l$ is defined as
$\wtildeN_{\gamma}(j)$.  The value of $\tcc(j)$ is now
redefined as the average distance from $j$ to
$\wtildeN_{\gamma}(j)$. Also let $\dmax(j)$ be $\max_{\mu\in
  \wtildeN_{\gamma}(j)}d_{\mu j}$. In each iteration, we
find a client $p$ with minimum value $\tcc(p)+\dmax(p)$. We
then create a demand $\nu$ for client $p$. Now we have two
cases:

\smallskip
\noindent
\mycase{1} If $\wtildeN_{\gamma}(p)$ overlaps
$\wbarclsnb(\kappa)$ for any existing primary
demand $\kappa$, we simply assign $\nu$ to $\kappa$. As
before, if there are multiple such $\kappa$, we pick any of
them. We also fix $\barx_{\mu\kappa} \assign \tildex_{\mu
  p}, \tildex_{\mu p}\assign 0$ for each $\mu \in
\wtildeN(p)\cap \wbarclsnb(\kappa)$. At this point
the total connection value in $\wbarN(\nu)$ might be smaller
than $1/\gamma$ (it cannot be bigger) and we shall augment
$\nu$ with additional facilities later to make a
neighborhood with total connection value of $1$ (We
augment to make sum of $\barx_{\mu\nu}$ for all
$\mu\in\wbarN(\nu)$ equal to $1$.).

\smallskip
\noindent
\mycase{2}
The best client $p$ has $\wtildeN_{\gamma}(p)$
disjoint from $\wbarclsnb(\kappa)$, for all primary demands $\kappa$.
In this case we
make $\nu$ a primary demand. We then fix
$\barx_{\mu\kappa}\assign \tildex_{\mu p}$ for $\mu \in
\wtildeN_{\gamma}(p)$ and set the corresponding
$\tildex_{\mu p}$ to $0$.  Note that the total connection value in
$\wbarclsnb(\kappa)$ is $1/\gamma$.  The set
$\wtildeN_{\gamma}(p)$ turns out to coincide with
$\wbarclsnb(\kappa)$ as we only add farther away facilities
when augmenting a primary demand thereafter. Thus
$\wbarclsnb(\kappa)$ is defined when it is created. We also
define $\tcc(\kappa) = \tcc(p)$ and
$\clsdist(\kappa)=\gamma\sum_{\mu\in\wbarclsnb(\kappa)}d_{\mu\kappa}\barx_{\mu\kappa},
\clsmax(\kappa)=\max_{\mu\in\wbarclsnb(\kappa)}d_{\mu\kappa}$.

Once all clients are exhausted, that is, each has $r_j$
demands created and assigned to some primary demand, we do
an augment step. For each demand with total connection value
less than $1$, we use our $\AugmentToUnit()$ procedure to
add additional facilities to its neighborhood to make its
total connection value equal $1$. We remark here that the
augment step does not change the close neighborhood of a
primary demand, as it already contains all the nearest
facilities with total connection value $1/\gamma$.  For
non-primary demands, the issue is more subtle, because
$\wbarN(\nu)$ has taken all overlapping facilities in
$\wbarclsnb(\kappa)\cap \wtildeN(j)$, which might be close
to $\kappa$ but far from $j$. It seems that facilities added
in the augment step might actually be closer to $\nu$ than
some of the overlapping facilities already in
$\wbarN(\nu)$. As a result facilities added in the augment
step might appear in $\nu$'s close neighborhood
$\wbarclsnb(\nu)$, yet they are not in $\wbarclsnb(\kappa)$
of the primary demand $\kappa$ that $\nu$ is assigned
to. This could be detrimental to ensuring
Property~\ref{PP2:assign}, which requires the close
neighborhood of demand $\nu$ and that of its primary demand
$\kappa$ need overlap. In the previous analysis of
approximation ratio, our bound on the expected connection
cost depends on the intersection of $\wbarclsnb(\nu)$ and
$\wbarclsnb(\kappa)$ being nonempty. In the following,
however, we give a proof on all properties~\ref{PP1:one} to
\ref{PP1:gamma} and \ref{PP2:yi} to \ref{PP2:cost}.

\emparagraph{Proof of the Properties.}  The proof of
Property~\ref{PP1:one}, \ref{PP1:xij}, \ref{PP1:yi},
\ref{PP1:eq}, \ref{PP1:gamma} are very similar to that in
the {\EGUP} algorithm's partition stage. These properties
are directly enforced by the partition algorithm, by
splitting a facility whenever some of the properties are
violated. Property~\ref{PP2:yi} follows from the way we
adjust $\tildex_{\mu j}$ and $\barx_{\mu \nu}$ values when
splitting a facility, and the fact that the close
neighborhood of primary demands are
disjoint. Property~\ref{PP2:cost} follows from our choice of
primary clients, and the proof is very similar to the proof
of Lemma~\ref{lem: property P2:cost holds}. The main
observation is that for any client $j$, the value $\tcc(j)$
can only get worse (larger) as the partition stage proceeds
in iterations.

We now show Property~\ref{PP2:diff} and~\ref{PP2:disjoint}
hold. First we start with a simple observation, that in the
end of our partition algorithm, neighborhoods of sibling
demands are disjoint. This follows from the fact that when
the algorithm adds a facility to the neighborhood of a
demand, it never adds the same facility to that of a sibling
demand. An immediate implication is that close neighborhoods
of sibling demands are disjoint as well. Now consider an
iteration when we create a demand $\nu$ for some client
$j$. If $\nu$ is set to be a new primary demand, then
$\wbarclsnb(\nu)$ is necessarily disjoint from the close
neighborhood of its siblings. Therefore no sibling will get
assigned to $\nu$, and by the algorithm $\nu$ is assigned to
itself. Otherwise $\nu$ is assigned to some existing primary
demand $\kappa$. According to the algorithm, all facilities
in $\wbarclsnb(\kappa) \cap \wtildeN(j)$ are moved into
$\wbarN(\nu)$ and excluded from $\wtildeN(j)$. It follows
that after that iteration, we have $\wtildeN(j)$ and
$\wbarclsnb(\kappa)$ being disjoint. Therefore later demands
cannot possibly have a neighborhood overlapping
$\wbarclsnb(\kappa)$. At this point, earlier sibling demands
each has a neighborhood being a subset of
$\wbarclsnb(\kappa')$ where $\kappa'$ is the primary demand
it was assigned. Since we know $\wbarclsnb(\kappa)\cap
\wbarclsnb(\kappa')=\emptyset$ from the partition algorithm,
we have that no sibling demand has a neighborhood
overlapping $\wbarclsnb(\kappa)$ at the end of this
iteration. Although we add more facilities to the
neighborhoods of sibling demands in the augment step, those
facilities cannot appear in $\wbarclsnb(\kappa)$ either,
because all overlapping facilities have already been moved
from $\wtildeN(j)$ into $\wbarN(\nu)$. As a result, at most
one sibling demand can have a neighborhood overlapping
$\wbarclsnb(\kappa)$. Thus we have
Property~\ref{PP2:disjoint} established. Since the algorithm
only assign a demand to a primary demand with common
facility, this proves Property~\ref{PP2:diff}.

We now show our last property~\ref{PP2:assign}. Consider an
iteration when we create demand $\nu$ for client $p$ and
assign it to $\kappa$. We have that $\wtildeN_{\gamma}(p)$
having a nonempty intersection with
$\wbarclsnb(\kappa)$. Let $B(p)=\wtildeN_{\gamma}(p)\cap
\wbarclsnb(\kappa)$, we claim that $B(p)$ must be a subset
of $\wbarclsnb(\nu)$, after $\wbarN(\nu)$ is finalized with
a total connection value of $1$. To see this, first observe
that $B(p)$ is a subset of $\wbarN(\nu)$, which in turn is a
subset of $\wtildeN(p)$, after taking into account of
facility split. Consider an arbitrary set of facilities $A$,
and define $\dmax(\nu, A)$ as the minimum distance $\tau$
such that $\sum_{\mu\in A \suchthat d_{\mu\nu} \leq
  \tau}\;\bary_{\mu} \geq 1/\gamma$, then adding additional
facilities into $A$ cannot make $\dmax(\nu, A)$ larger. It
follows that $\dmax(\nu, \wbarclsnb(\nu)) \geq \dmax(\nu,
\wtildeN(p))$. Since we have $d_{\mu \nu} = d_{\mu p}$ by
definition, it is easy to see that every $\mu \in B(p)$
satisfies $d_{\mu \nu} \leq \dmax(\nu, \wtildeN(p)) \leq
\dmax(\nu, \wbarclsnb(\nu))$ and hence they all belong to
$\wbarclsnb(\nu)$. We need to be a bit more careful here
when we have a tie in $d_{\mu\nu}$ but we can assume ties
are always broken in favor of facilities in $B(p)$ when
deciding $\wbarclsnb(\nu)$. Finally, since $B(p)$ is
nonempty, we have that the close neighborhood of a demand
$\nu$ and its primary demand $\kappa$ must overlap. This
proves Property~\ref{PP2:assign}.

The above concludes our discussion on the Stage~1
partitioning process and we have thus established that
our~{\EBGS} algorithm is a $1.575$-approximation algorithm.


%%%%%%%%%%%%%%%%%%%%%%

\section{Discussions}

In this paper we show a sequence of LP-rounding approximation algorithms
for FTFP, with the best algorithm achieving  ratio $1.575$. The two techniques we introduced,
namely the demand reduction and adaptive partitioning, are very flexible. Should any new
LP-rounding algorithms be discovered for UFL, we believe that with our approach they can be
adapted to FTFP as well, preserving the approximation ratio. In fact, by randomizing the
scaling parameter $\gamma$, as Li showed in \cite{Li11}, we
could further improve the ratio to below $1.575$, although we would not be
able to match the $1.488$ bound for UFL in~\cite{Li11}, because this would
also require appropriately extending dual-fitting algorithms~\cite{MahdianYZ06}
to FTFP, which we have so far been unable to do.

One of the main open problems in this area is whether FTFL can be approximated with the
same ratio as UFL, and our is was partly motivated by this question. The techniques we
introduced are not directly applicable to FTFL, however, mainly because our partitioning
approach involves facility splitting that could result in several sibling demands being served
by facilities on the same site. Nonetheless we hope our construction might inspire new
algorithm for the FTFL algorithm with a matching ratio with
LP-based algorithms for the UFL problem as well.

\bibliographystyle{plain}
\bibliography{facility}
%\printbibliography

%%%%%%%%%%%%%%%%%%%%%%%%%%%%%%%%%%%%%%%%%%%%%%%%%%%%%%%%%%%%%%%%%%%%%%%%%%%%%%%%
%%%%%%%%%%%%%%%%%%%%%%%%%%%%%%%%%%%%%%%%%%%%%%%%%%%%%%%%%%%%%%%%%%%%%%%%%%%%%%%%
%%%%%%%%%%%%%%%%%%%%%%%%%%%%%%%%%%%%%%%%%%%%%%%%%%%%%%%%%%%%%%%%%%%%%%%%%%%%%%%%

\appendix
\section{An Elementary Proof of the Expected Connection Cost}


In the $1+2/e=1.736$-approximation, we need to show the following inequality
%
\begin{equation}
  \label{eq:dist}
  d_1 y_1 + d_2 y_2 (1-y_1) + d_3 y_3 (1-y_1)(1-y_2) + \ldots + d_l y_l \Pi_{s=1}^{l-1} (1-y_s) \leq (d_1 y_1 + d_2 y_2 + \ldots + d_l y_l) (1 - \Pi_{s=1}^l (1-y_s))
\end{equation}
%
for $d_1\leq d_2 \leq \ldots \leq d_l$ and $\sum_{s=1}^l y_s = 1, y_s \geq 0$.

In this section we give a new proof of this inequality, much
simpler than the existing proof in
\cite{ChudakS04}~\footnote{It is known that Sviridenko has
  proved a similar lemma using Chebyshev's sum inequality,
  but the cited reference~\cite{Svi02} contains a different
  proof from ours here.}.  We derive this inequality from
the following generalized version of the Chebyshev Sum
Inequality:
%
\begin{equation}
  \label{eq:cheby}
  \sum_{i} p_i \sum_j p_j a_j b_j \leq \sum_i p_i a_i \sum_j p_j b_j,
\end{equation}
%
where each summation below runs from $1$ to $l$ and the sequences 
$(a_i)$, $(b_i)$ and $(p_i)$ satisfy the following conditions:
$p_i\geq 0, a_i \geq 0, b_i \geq 0$ for all $i$, $a_1\leq a_2 \leq
\ldots \leq a_l$, and $b_1 \geq b_2 \geq \ldots \geq b_l$.

Given inequality (\ref{eq:cheby}), we can obtain our inequality
(\ref{eq:dist}) by simple substitution
%
\begin{equation}
  p_i \leftarrow y_i, a_i \leftarrow d_i, b_i \leftarrow
  \Pi_{s=1}^{i-1} (1-y_s)
\end{equation}

For the sake of completeness, we include the proof of inequality (\ref{eq:cheby}), 
due to Hardy, Littlewood and Polya~\cite{HardyLP88}. The idea is to evaluate the 
following sum:
%
\begin{align*}
  S &= \sum_i p_i \sum_j p_j a_j b_j - \sum_i p_i a_i \sum_j p_j b_j
	\\
  & = \sum_i \sum_j p_i p_j a_j b_j - \sum_i \sum_j p_i a_i p_j b_j
	\\
  & = \sum_j \sum_i p_j p_i a_i b_i - \sum_j \sum _i p_j a_j p_i b_i
	\\
	&= \half \cdot \sum_i \sum_j (p_i p_j a_j b_j - p_i a_i p_j b_j + p_j p_i a_i
  							b_i - p_j a_j p_i b_i)
\\
  &= \half \cdot \sum_i \sum_j p_i p_j (a_i - a_j)(b_i - b_j) \leq 0.
\end{align*}
The last inequality holds because $(a_i-a_j)(b_i-b_j) \leq 0$, since the sequences
$(a_i)$ and $(b_i)$ are ordered oppositely.

\end{document}

% marek Tue Jul  3 10:21:05 PDT 2012
% marek Sun Jul  1 14:57:39 PDT 2012
% lyan Sat Jun 30 2012, 22:01:27
% marek Sat Jun 30 10:08:59 PDT 2012
% lyan Fri Jun 29 19:54:18 PDT 2012
% marek Thu Jun 28 09:21:14 PDT 2012
% lyan Thu Jun 28 00:11:28 PDT 2012
% marek Wed Jun 27 11:24:07 PDT 2012
% lyan Wed Jun 27 2012, 10:08:21
% marek Tue Jun 26 14:48:45 PDT 2012
% lyan Mon Jun 25 2012, 22:23:13
% marek Sun Jun 24 16:46:23 PDT 2012
% marek Wed Jun 20 04:42:40 PDT 2012
% lyan, Sun Jun 17 2012, 09:49:22
% marek Sat Apr  7 16:42:21 PDT 2012
% marek Thu Apr  5 11:39:58 PDT 2012
% marek Wed Apr  4 11:28:20 PDT 2012
% lyan, 04/01/12 10:20 PM
% lyan, Mon Mar 26 2012, 09:10:54
% lyan, Tue Mar 20 2012, 23:28:17
% lyan, 03/18/12 12:28 PM
% marek Sat Mar 17 13:42:32 PDT 2012
% marek, Wed Mar  7 21:28:24 PST 2012
% marek Mon Mar 12 12:08:25 PDT 2012
